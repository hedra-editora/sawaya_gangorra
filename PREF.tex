%!TEX root=./LIVRO.tex

\chapter*{}
%\chapterspecial{O que é o Transtorno Bipolar do Humor}{}{}

\epigraph{%
O transtorno bipolar é uma doença mental grave, caracterizada por mudanças 
bruscas no humor e nos níveis de energia do indivíduo. Há uma fase de elevação 
do humor conhecida como mania ou hipomania (uma forma um pouco mais leve de mania), 
bem como um período de depressão. Pode haver também uma fase mista, na qual a mania 
e a depressão aparecem juntas, causando um estado de agitação extrema. A passagem 
de uma fase de humor para outra é chamada de ciclo e cada ciclo pode perdurar por alguns meses.
Durante os episódios de mania, o paciente pode se sentir extraordinariamente alegre, 
enérgico e agitado. Muitos indivíduos em mania podem ser imprudentes e manifestar comportamentos 
impulsivos, como apostas, gastos excessivos, uso de drogas e álcool em excesso, bem como 
apresentar um comportamento sexual de risco. Alguns podem ficar acordados por longos períodos de 
tempo, sentindo que precisam de pouquíssimas horas de sono graças aos níveis extremos de energia. 
Durante a mania severa, o indivíduo pode também ser acometido por um quadro psicótico.
}{\emph{Adaptado de} Pedersen, T. \emph{Bipolar Disorder}.\\ Psych Central. (2016)}

\thispagestyle{empty}


%No decurso da fase depressiva, o indivíduo pode se sentir extremamente desesperançoso, triste, 
%desinteressado de qualquer atividade, ansioso e profundamente fatigado. Pode também 
%haver casos de despersonalização e ideações suicidas. Quando o episódio é grave, o indivíduo pode 
%experienciar psicoses, como delírios e, em casos mais raros, alucinações. 
%O episódio depressivo pode se estender por algumas semanas 
%e, se não for tratado, pode perdurar por meses.
%
%Os pacientes bipolares que experienciam um estado misto, apresentam, como já mencionado, uma condição na qual os 
%sintomas tanto da mania como da depressão ocorrem simultaneamente. Um indivíduo 
%sofrendo com um episódio misto pode se sentir desesperado, ansioso, agitado, 
%acometido por pensamentos de grandeza e impulsos controladores. Tal estado 
%representa um risco alto de suicídio e de abuso de substâncias.
%
%As mudanças de humor podem ser bruscas, mas os episódios são, geralmente, duradouros. 
%As fases depressivas alongam-se por pelo menos 15 dias (e podem se 
%estender por anos na ausência de tratamento); a mania dura no mínimo 
%uma semana, e a hipomania perdura por ao menos quatro dias. 
%Todas essas fases são intercaladas por períodos de normalidade.
%
%Estima-se que o transtorno bipolar afete cerca de 4\% da população adulta no 
%mundo; no Brasil, o número de pacientes com esse diagnóstico chega a até 6 milhões de indivíduos.

%(\emph{adaptado de} Pedersen, T. (2016). \emph{Bipolar Disorder}. Psych Central. 
%https://psychcentral.com/encyclopedia/bipolar-disorder/)


\chapterspecial{Apresentação}{Diário de guerra}{Beatriz Bracher}

{Compartilhar a vivência da bipolaridade} é o tema e o motivo explícito
de \emph{Gangorra}{, de Regina Sawaya. O livro tem forma de diário: às
vezes os dias são seguidos, outras vezes há saltos de meses. Após cada
data, lemos o que foi escrito naquele dia, ou o que foi escrito a partir
daquele dia e segue até alguma data não mencionada.}

{Regina narra seu dia a dia, sua energia ou falta de energia, os
compromissos a que não conseguiu ir, ou a que foi com muito esforço; a
raiva com a obesidade; o cansaço; o orgulho de ter conseguido acordar
antes do meio-dia; o prazer em pintar, ir a uma exposição; o telefonema
de um irmão, o não telefonema de pessoa alguma; a solidão; a falta de
tônus psíquico para ir ao cinema com uma amiga; a demora em se vestir;
as compras que fez, as comidas que comeu, as bebidas que tomou; a
arrumação do apartamento; a gatinha; a empregada que disputa espaço, a
que toma conta; idas à igreja Messiânica, ao terreiro de Candomblé, ao
psiquiatra e ao analista; remédios, remédios e mais remédios: quando
tomar, quanto tomar, mudanças de prescrição, efeitos colaterais.}

{Misturada à narrativa do cotidiano, há a do passado. No começo do
livro, o que mais aparece é a história de sua bipolaridade, desencadeada
após a morte da irmã e, a partir de então, uma condição permanente de
sua vida. Mania e depressão se revezam, emergem e submergem, com
períodos de alívio e ``normalidade'' que tendem, inexoravelmente, a ser
cada vez mais curtos.}

{Aos poucos, o registro do cotidiano e do desenrolar da doença vai se
misturando com o relato da história da família e da juventude da autora.
Descendente de família paulista antiga e de imigrantes
libaneses, neta de médico de Campinas e de mascate vindo do Líbano,
filha de importante professor da USP e de uma ``}\emph{lady}{, quando
estava bem'', Regina é a sétima de dez irmãos. Viveu sua juventude nos
anos 60 e 70: escutou Jimi Hendrix, Janis Joplin, Jorge Ben, vendeu
artesanato na Praça da República, estudou Artes na FAAP, namorou o
Gaiarsa, entre muitos outros, trabalhou com estamparia na Moinho
Santista, bebeu, fumou e transou de tudo.}

\asterisc

Regina Sawaya é artista plástica, trabalha com cerâmica e pintura. É
arte-educadora, deu aula na Pinacoteca do Estado, na Escola Vera Cruz e
organizou seminários no Museu Lasar Segall. Regina Sawaya é uma
excelente escritora, escreveu \emph{Gangorra}.

\asterisc

O que há de único neste livro é a quantidade de energia e sensibilidade
que Regina consegue revelar em sua narrativa. É um touro revoltando-se
contra a leseira da depressão e a correnteza da mania.

A cada volta da mania, o desespero de ter-se deixado levar, o rombo na
conta do banco, os perigos a que se expôs, a percepção renovada e
trágica de que a alegria é sinal de perigo. A vitalidade a toma, assume
o comando, e Regina, eufórica, acorda cedo, arruma o apartamento, se
irrita com as pessoas, vai a exposições, compra e come muito.

Quando a depressão bate à porta, o corpo e a casa se desmilinguem, não
há ânimo para limpeza, para ``bom-dia, como vai?'', para ir a lugar
algum. Livros de arte, exposição, aula de gravura, tudo que a alimenta
não diz mais nada. Deprimida, Regina quer a solidão, privacidade, não
ser dependente, não acordar, quer o escuro. Precisa da ajuda da
empregada, dos acompanhantes terapêuticos, dos irmãos e da luz do sol.
Nem sempre tem o que precisa. E mesmo quando tem, tudo é ruim.

Lemos estes dois estados nas lembranças de seu passado e no cotidiano
que oscila entre um polo e outro ao longo dos dias que compõem o livro.
A bipolaridade é uma doença que se manifesta em comportamentos
característicos, praticamente iguais em todas as pessoas. Não há nada de
original, leseira e correnteza são o padrão, extremos da gangorra em que
milhares de pessoas vivem.

O original neste livro é a pessoa que está na gangorra, seu modo de
reagir e de se deixar levar. Principalmente a sua capacidade de
descrever a vida~que vai muito além da sua condição bipolar.

Talvez ``além'' não seja a palavra, já que a mania e a depressão são
constituintes da vida da autora, condicionam de maneira significativa
seus passos, as relações afetivas e profissionais. Portanto, não existe
uma Regina bipolar e outra sã. Não é uma Regina estável que escreve
sobre a vida e os sofrimentos de uma Regina desestabilizada.

Se a vida não está além dos ciclos de euforia e melancolia, também não
se resume a eles. O que há de cativante na autora do livro, marcada por
sua condição psíquica específica, é a riqueza de seu olhar sobre a vida
ao redor e sobre si mesma. É a inteligência e a extrema sensibilidade,
potentes e instáveis, que nos revelam seu mundo interior, as vivências
passadas e atuais.

Mais do que a formação de Regina, que combina um ambiente familiar
tradicional e intelectualizado com a onda libertária dos anos 60 e 70, é
a urgência da escrita o que nos cativa. \emph{Gangorra} se desenvolve
como uma onda, cada vez maior e mais forte, em busca do fora de si. O
interior da autora, ao ser escrito e lido, transforma-se em fora de si.
E, parece-me, essa convocação do leitor para sentir e compreender a
solidão exposta é uma das preciosidades do livro.

Faz parte desse interior partilhado conosco a aguda percepção para a
fisicalidade do mundo. A beleza dos detalhes do copo, do ouriço-do-mar e
da luz roxa sobre a árvore na esquina da Rebouças com a avenida Brasil.
Quando são descritas a sala da casa dos avós, a dança de umbanda e a
pequena chácara onde compra as ervas para o banho de descarrego, é
explícito o prazer do convívio com as coisas em seus detalhes, o deleite
com sua evocação. O vidro colorido do copo e a forma oval de sua boca, a
textura da areia da praia de São Sebastião, o brilho das estrelas do mar
vão compondo a paisagem dentro da qual as crises se sucedem.

Somos envolvidos pelo despudor da raiva contra o inimigo, que não será
vencido, e apresentados às estratégias sistemáticas de Regina para
contê-lo, ou, ao menos, para proteger-se de seus piores efeitos. Uma
destas estratégias são as listas que povoam o livro:


\begin{itemize}
\tightlist
\item
  ações para segurar a mania e a depressão;
\item
  o que comprar nas farmácias;
\item
  remédios para levar na viagem;
\item
  ervas para banho;
\item
  vezes que já foi para S. Sebastião e com quem;
\item
  coisas para se fazer em Recife;
\item
  o que comprou.
\end{itemize}

Por vezes parece que a sensibilidade da autora são como raios que saem
de um vaso de argila cozida e esmaltada, escura. Vemos as faíscas e seu
brilho forte, mas não sua fonte. É como um sol escondido que não
consegue se libertar para brilhar livremente. O livro não ameniza a
tragédia da doença, pelo contrário, mostra com muita verdade sua prisão
e a falta de perspectiva de cura. E porque Regina é uma pessoa tão
vibrante, com talento incomum para captar as várias formas da beleza se
manifestar, talvez sua bipolaridade seja ainda mais terrível e o
sentimento de crueldade, maior.

{A mania e a depressão estão e não estão no controle. Não é um demônio
que entra no corpo, um santo que se encarna, o corpo não é apenas a
casca da euforia e da depressão. Não no caso de Regina. É um demônio e
um santo que entram no corpo, e a resistência a eles pelo controle é o
elemento central do drama que se conta nestas páginas.~ Personagem
importante do livro, Del Porto, psiquiatra de Regina há 24 anos, lhe
aconselha: ``Não acredite na leitura depressiva da sua vida''. Este
livro existe, em grande parte, graças ao acolhimento do conselho pela
autora.}

\asterisc

Regina Sawaya é minha tia. Faço parte da ``sobrinhada'' do ramo de
Soninha e Fernão. Acho que 80\% das pessoas que aparecem no
livro são parte dessa família numerosa, que também é a minha.
Esforcei-me para imaginar como será \emph{Gangorra} para quem não
conhece nem conheceu Paulinho, Soninha, Rogério, Tuxa, Sylvio, Tina,
Beti, Zeca e Tomi. Evidentemente não foi possível.
%\hfill {Beatriz Bracher}
