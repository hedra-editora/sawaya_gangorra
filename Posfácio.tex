\chapter{Posfácio}

\epigraph{\textit{``Could it be madness -- This?''}}{Emily Dickinson}

Na descrição do seu quotidiano a autora relata, de forma vívida e
espontânea, os altos e baixos do estado de ânimo, próprios do chamado
``transtorno bipolar''... Essa seria uma síntese ``clínica'' do que o
livro apresenta, e no entanto ele é muito mais que isso. Revela-nos uma
escritora sensível, uma artista plástica que passa para o papel suas
cores sempre vibrantes, seja para descrever os abismos da melancolia,
seja para pintar um estado de espírito em que paixão e fúria se
confundem. Sua história de vida emerge, aos poucos, mesclada com o
relato dos acontecimentos diários, e revela uma cronista do quotidiano,
atenta a detalhes que compõem um conjunto de pontos (``pointillisme'',
fauvismo?), mais perceptível quando se olha o conjunto do texto.

Em seu livro ``Touched by fire'', a autora Kay Jamison, cientista e ela
própria portadora da condição ``bipolar'', estuda as relações entre arte
e bipolaridade. É, de fato, impossível negar a relação entre essa
condição singularmente cíclica (os altos e baixos do humor) e sua
semelhança com os ciclos de luz e sombra, morte e recomeço, que se
repetem na natureza e são capturados pelos poetas, pintores e músicos
tocados pela bipolaridade. Jamison indaga se os inúmeros artistas, por
ela estudados, produzem sua arte \emph{apesar} da ``bipolaridade'' ou
se, ao contrário, existe algo da vivência de longos períodos
melancólicos, interrompidos de tempos em tempos pela experiência
avassaladora da mania, que em conjunto levam a um tipo diferente de
``insight'', compaixão e expressão da condição humana. Inclino-me pela
segunda hipótese, a de que esses elementos podem, para algumas pessoas,
adicionar profundidade, ``fogo'' e maior fluidez à expressão artística.

{\bigskip\itshape\hfill {José Alberto Del Porto}}\medskip

