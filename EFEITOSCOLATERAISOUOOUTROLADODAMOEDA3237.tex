\chapterspecial{Efeitos colaterais ou o outro lado da moeda}{}{}
 

​

\begin{flushright}\textbf{2010, 28/01 a 17/03 (final de janeiro)}\end{flushright}


Esqueci o outro caderno na malinha que ficou no carro da Vera. Saco! Vou
continuar neste.

Tenho vivido um tempo terrível. Para evitar uma possível mania, o Del
Porto suspendeu completamente o Pristiq, no dia 24, domingo. Na
segunda"-feira passei muito mal, a cabeça latejando forte, tanta fraqueza
que tive que me apoiar na Juce para conseguir andar. Para ir à análise,
tive que tomar um energético. A~sessão depois das férias foi muito boa,
voltamos a falar da mania de 2008, da atitude do Celso, da do Del Porto,
assim como das acompanhantes. Acho que ainda voltaremos a esse assunto
muitas vezes. O~Celso, que também é psiquiatra, em vez de me indicar
outro psiquiatra para manias, já que analista não medica, me medicou.
Acho que ele quis ajudar, mas cometeu um erro, além de não ter me falado
claramente que eu estava em mania, outro erro crasso. Quem resolveu se
consultar com o Itiro fui eu sozinha e acho que ele deveria ter me
internado naquele momento, eu ia gastar muito menos dinheiro. Só há
pouco tempo o Celso me contou que ligou para o Itiro e falou para ele
que eu não queria ser internada. A~intenção foi boa, mas acho que não
concordo. Será que sou excessivamente crítica e não aceito os limites
das pessoas? Sou reclamenta?

O que me atrapalhou muito foi que, sob a orientação do Rudi, que era meu
guia espiritual de linha indiana, um espírito cujo retrato estava no
vidro da janela do meu quarto e na frente do qual eu meditava
diariamente, parei de tomar os remédios. Mas falando honestamente,
quando eu entro em mania, tomo todos os remédios, mas uns três dias
depois eu acho que estou ótima, curada e paro, com guru ou sem guru. Mas
em 98, quando tive uma mania e já fazia meditação, a Ciça, que era
professora de meditação, disse que eu deveria parar os remédios para a
energia que ela mandava não ter que atravessar a barreira química de
remédios. Só assim eu subiria de degrau, ela falou. Quem é que não fica
tentando subir de degrau, se desenvolver? Ela não entende nada de
bipolaridade e é uma puta irresponsável. Tive a maior briga com ela
quando saí da clínica.

Eu não sei como voltei à meditação anos depois, quando estava mais forte
devido à analise e aos remédios. É~impressionante a atração que tenho
por esse trabalho espiritual, e deu no que deu, neste desastre que já
dura quase dois anos de sofrimento. Em 98, tive uma crise de mania e fui
internada pela primeira vez na Clínica Conviver, por dez dias. Me deram
um sossega"-leão e eu desmaiei, acordei no dia seguinte imunda, com a
mesma roupa com que tinha chegado. A~Soninha mandou uma malinha de roupas
que não eram minhas, ela comprou, e isso só prejudicou a retomada da
minha identidade, apesar da generosidade dela em me ajudar. Mas eu
precisava retomar o meu eu, e roupa é algo pessoal, é símbolo.

Eu adorei quando ela mandou uma cesta imensa, linda, do Santa Luzia,
cheia de biscoitos, geleias e chocolates finíssimos. Reparti com todo
mundo. E~adorei também quando o Paulinho mandou um caixote cheio de
sorvetes, inclusive Häagen"-Dazs, que todo mundo devorou. A~Bete mandou
uma caneta linda, e o Sylvio um caderno de desenho, eu acho.

A internação, para falar a verdade, não foi tão difícil quanto eu
imaginava, arranjei até um namorado. Eu acho até que concordei logo,
porque o mais insuportável era a ansiedade dos irmãos em cima de mim.
Isso, sim, é enlouquecedor e muito pesado, desintegrador. Eu não
acreditei que o Rogério chegou atrasado à reunião com Dr. Mário Queiroz,
na qual já estavam presentes o Sylvio e a Soninha. Talvez, ele fazendo
uma dobradinha com o Sylvio, conseguisse evitar a internação. Na mania
de 1987, a segunda, eles conseguiram e se responsabilizaram por mim. Eu
fiquei morando na minha casa com um amigo que a dividia comigo, e uma
empregada ótima, a Maria, que ficou cuidando de mim. A~Maria veio em
janeiro e, em agosto, eu já estava coordenando uma Equipe de Cultura
(música, dança, artes plásticas, teatro) num projeto da Secretaria do
Menor. Fantástico, apesar de ser do Governo Quércia. O~Sylvio que me
arranjou esse emprego na Secretaria da Cultura, que me emprestou para a
Secretaria do Menor; a Secretária era a Alda Marco Antônio. Meu
psiquiatra era o Luiz Milan. Foi o Sylvio que mandou a Maria lá para
casa.

O lítio foi a primeira porrada química, a primeira de infinitas porradas
químicas que duram até hoje, aos 61 anos. É~um calvário, um longo
calvário, com poucos intervalos de descanso. E~tudo isso para quê? Não é
melhor um suicídio construído, uma demissão digna por absoluta exaustão?
Eu estou exausta, muito exausta.

O que adianta o talento para a pintura, para a cerâmica, para o trabalho
educacional, se não tenho um pingo de energia para fazê"-los?

Apesar de tudo, eu estou satisfeita porque lutei contra a absoluta falta
de energia desta noite e consegui escrever estas reflexões, o que me
ajuda muito. Pensar ajuda muito. A~viver. A~morrer.

Que dor, que angústia, que dor, que angústia. Que medo do final. O~suicídio se constrói com todo o sofrimento acumulado durante anos e
anos, lentamente. Com todo o sofrimento, como o de agora, a rejeição por
parte do Zeca, do Rogério, da Soninha e do Fernão em eu ir para a
fazenda no carnaval, por ser portadora de uma doença mental. A~dor
infinita de ter uma doença mental, logo agora que eu estava ficando bem,
renascendo, levantando da tumba às custas de um esforço enorme e de
D"-Ribose, umas pílulas de açúcar que me dão energia e eu descobri numa
loja para atletas que tem bem em frente à análise. Eu estava muito feliz
aproveitando as liquidações do \emph{shopping}. Eu estava curtindo
perceber como eu estava calma, comparando com a loucura de 2008, quando
eu estava na joalheria Vivara e briguei, a ponto de a dona da loja
ameaçar chamar o segurança. Naquele momento, eu tirei as joias que
estavam em mim, piquei o cheque em pedacinhos e fui embora. Graças a
Deus, porque me livrei de pagar R\$ 8.000,00, que obviamente eu não
teria. É~obvio que as compras foram todas absurdas.

Na mania de 2008, eu deixei de ser alternativa e passei a ser madame com
tudo. Afinal, eu tenho uma porção madame: eu sou filha da dona Sônia
Sawaya, aquela \emph{lady} que recebia com perfeição e ensinou a todos
nós tudo que existe de bom e de melhor.

Esforço.

Esforço.

Esforço.

Um puta esforço lembrar tudo isso, ressentir tudo isso. Ressentir,
ressentir e ressentir.

O Rogério criou agora um quadro de mania e possível internação,
justamente no momento em que eu estava conseguindo levantar da cama e
andar com as próprias pernas. E~estava muito feliz com isso. E~ele
passou esse quadro para o Zeca e para a Soninha. Deu no que deu.

Viva o Nelson Rodrigues!

Viva o Nelson Rodrigues!

Só Nelson Rodrigues e o Gaiarsa me entenderiam neste momento. E~o Freud,
o Winnicott, a Melanie Klein e o Bion.

Me senti ameaçada, foi muito desgastante. O tempo todo eu imaginava os
três chegando aqui e me levando para internar, como da última vez, o
Paulinho, o Rogério e o Zeca. E~ninguém acreditou no Del Porto, com o
qual eu tive uma consulta anteontem e falou que eu estava bem e deveria
ir para a fazenda, mas deixou uma receita de Zyprexa para uma
\emph{eventualidade}. Será que a família até hoje não sacou que quem
percebe as manias sou eu e liga para o Del Porto? E tem que ser assim,
porque afinal ele não mora comigo. É~este o grande risco dessa doença:
\emph{perceber na hora e ser medicada na hora}. Eu sou craque, já
consegui ficar sete anos sem ter mania! Arre!

Os meus irmãos tinham razão. Eu estava mesmo em mania no carnaval. Tive
certeza quando chegaram os extratos dos cartões de crédito.

\begin{center}\textbf{\asterisc{}}\end{center}


\textbf{}

\begin{flushright}\textbf{22/02/2010}\end{flushright}


Foi duro, foi duro e é duro.

Duríssimo.

Ter sido internada duas vezes por estar maníaca, de final de junho a
agosto. Del Porto me liberou antes da hora. Imperdoável. Afinal, ele me
conhece há 20 anos, como não soube discriminar? E aí, na segunda fase,
eu gastei muito, mas muito dinheiro. Os objetos não cabiam na minha
casa, tive que me desfazer do jogo de pratos antigos para caberem os
novos. Ridículo. Doloroso. Doloroso. Quando eu lembrava disso na
clínica, eu chorava. O~aparelho Vista Alegre completo entope o meu
armário da cozinha até hoje. Só usei uma vez numa hipomania do ano
passado. Fiz um cuscuz maravilhoso junto com a Cida, acompanhante, para
receber o Antônio, o Pierre, a Sara e o Zé. Tive o maior prazer em
cozinhar e em arrumar tudo, do aperitivo ao cafezinho. Tudo impecável.
Filha da dona Sônia. Irmã da dona Soninha. Florzinhas na mesa, taças de
champanhe e tudo mais.

Não sei por que, nos últimos anos, só consigo receber em hipomania ou
mania. Claro que em parte é porque nesses estados me sinto mais segura
de mim, mais interessante, com algo a oferecer para os outros. Quando
estou depressiva, sinto que os outros não acham a menor graça em mim,
fico muito quieta, muda, parece que nem existo. Como vou receber alguém
nesse estado de vazio? E se alguém me convida para alguma coisa, que
seja um cinema, é muito difícil eu conseguir ir, apesar de às vezes até
querer ir. Impotência total. Total mesmo. Mesmíssimo.

Não sei por que, na mania de 2008, comprei tantas coisas para casa:
louça Vista Alegre e uma mais simples, italiana, e copos bordô, Strauss,
lindos. Outro dia vi na loja que meia dúzia custa R\$780,00! Na mania, a
gente vai mandando ver, o preço não tem a menor importância, no meu caso
só o visual, o desenho, a cor. Todos os sonhos e desejos, antes
reprimidos, se realizam então. E, pra falar a verdade, os copos são
lindos, pelo menos até hoje enfeitam o meu armarinho de médico, de
vidro, que foi do papai, a parte de ferro eu pintei de azul. Estou
sentindo angústia só de lembrar de tudo isso. Aquela angústia que dá um
nó na garganta e a gente engole em seco. Seco, seco, seco. Seco pra
burro.

A gente engole em seco. Pelo que passou. Pelo que passo. Pelo que
passarei. A~doença é \versal{INCURÁVEL}. Terei outras crises. Não sei como serão.
Ninguém tem bola de cristal. \versal{ISSO} \versal{AMENDRONTA}. \versal{ISSO} \versal{ASSUSTA}. \versal{NÃO} \versal{QUERO}
\versal{SOFRER} \versal{MAIS}. \versal{ME} \versal{SINTO} \versal{ESGOTADA}. \versal{EXTREMAMENTE} \versal{CANSADA} \versal{DE} \versal{FAZER} \versal{TANTO}
\versal{ESFORÇO} \versal{PARA} \versal{SOBREVIVER}. \versal{PARA} \versal{AGUENTAR} \versal{TANTOS} \versal{EFEITOS} \versal{COLATERAIS} \versal{DOS}
\versal{REMÉDIOS}, \versal{ESPECIALMENTE} O \versal{SONO} E A \versal{TOTAL} \versal{FALTA} \versal{DE} \versal{ENERGIA} E A \versal{GORDURA},
103kg! Afinal, onde está a eficiência do Del Porto, o grande mestre dos
remédios, o frequentador assíduo de congressos internacionais, que vibra
com cada remédio novo que ele tasca em você? \versal{CANSEI} \versal{DE} \versal{SER} \versal{COBAIA}, quem
sente os efeitos colaterais sou eu, o meu corpo, a minha mente e não a
dele.

Estou fodida, estou amarga, estou desesperançada. O~Del Porto falou,
\emph{por telefone}, e eu tinha tido uma consulta \emph{na véspera}, que
só sobrou um remédio para segurar a mania: o terrível Zyprexa. É~um
remédio também usado para abrir o apetite dos anoréxicos, que dá uma
fome e uma larica infernal. A~gente é capaz de sair de casa só para ir
ao supermercado comprar duas latas de leite condensado e tomar de uma
vez. Falo com propriedade, pois já tomei Zyprexa algumas vezes. Foi
quando cheguei no meu maior peso na vida: 107,700kg. Eu tinha muita
vergonha do meu corpo, de ir aos lugares. A~minha sorte foi que achei
uma loja de tamanhos grandes, na Zé Paulino, e as roupas de verão, pelo
menos, eram legais e transadas. No inverno só acho roupas que me servem
em duas lojas do Shopping Iguatemi: a Sinhá e a Erica's. O~problema é
que é tudo muito caro. Mas…. eu tenho que ter o que vestir.
Encontrei também \emph{jeans} que, para falar a verdade, eu adoro, mas
percebi que está em desuso para mulheres de 60 anos, pelo menos no
almoço da Soninha. A~eterna necessidade de me adequar, de ser aceita, a
eterna dificuldade de ser eu com tranquilidade e segurança. Mas eu
queria estar mais feliz com a minha aparência. E~agora essa -- Zyprexa,
claro que é melhor do que uma mania seguida de uma depressão. Em quantos
quilos vou chegar dessa vez? Vou suportar?

Viver, isso dói. Dói muito. Estou cansada, eu estou com medo, eu estou
insegura.

Venho de um período especialmente desgastante. No segundo semestre do
ano passado, 2009, tive uma asma sufocante. Angustiante. Fiquei muito
surpresa de a doença voltar 45 anos depois. Revivi todo o sofrimento da
infância até a adolescência, as bombinhas, as idas ao pronto"-socorro, o
abandono, o dormir praticamente sentada, com vários travesseiros. A~sorte é que achei uma médica muito boa, que zerou o meu pulmão, e o
lucro é que parei de fumar. Quando a asma chegou, eu já estava há vários
meses com uma forte diarreia. Não dava tempo de levantar da cama de
manhã e chegar ao banheiro. Uma vez isso aconteceu no meio de uma
consulta do Del Porto, mas ele não percebeu, acho que só achou estranho
que eu tenha ficado horas no banheiro. As acompanhantes foram
fundamentais e extraordinárias nessa época. A~conclusão do gastro,
depois de examinar os 27 exames que levei, foi: ``Ninguém toma tantos
remédios sem que o intestino seja afetado'', e me pediu uma colonoscopia
depois de dois meses, para o intestino ter um tempo de se refazer da
diarreia que agride muito. Fiquei \versal{ARRASADA}, concluí que a minha diarreia
não tinha cura porque eu precisava continuar a tomar todos aqueles
remédios. O~que fazer? Eu não aguentava mais fazer cocô daquele jeito,
ir na maior insegurança aos lugares e usar um \emph{modess} geriátrico
enorme para ajudar a segurar as pontas da diarreia e da incontinência
urinária.

Sim, porque, ao mesmo tempo, comecei a ter uma incontinência urinária,
às vezes durante o dia e sempre durante a noite. A~Cida, acompanhante,
forrou a cama com um plástico, um lençol por cima, começou a trazer
fraldas geriátricas, e com o tempo passaram a trocar a fralda às 7h00 da
manhã para a cama não encharcar e ter que trocar o lençol todo dia. A~Preta, minha empregada, não aguentava mais lavar tanto xixi e tanto
cocô. Elas, as acompanhantes, foram maravilhosas, só ouvi um pouco de
reclamação. Para elas esse tempo foi duro, foi chato, foi exigente. Eu
me sentia uma bebezona nas mãos delas, sobretudo porque continuava
deprimida, apesar de já estar, nessa altura, no quinto ou sexto
antidepressivo. A depressão começou depois da mania, mais ou menos em
outubro de 2008, e essas doenças ocorrem normalmente juntas, e foi até o
segundo semestre de 2009.

Junto com tudo isso, tive os efeitos colaterais de um remédio chamado
Risperdal, ministrado pelo Del Porto. Fiquei quase paralisada, com muita
dificuldade para andar. Andava só de passinhos e devagar, como uma
velhinha. Não conseguia levantar sozinha nem do sofá nem da cama, só com
a ajuda das acompanhantes. Não conseguia escrever, o banco mandou eu ir
lá preencher um novo cartão de assinaturas, um saco! Detesto ir ao
banco. As minhas mãos tremiam muito. Achei que nunca mais iria escrever,
desenhar, pintar, fazer cerâmica.

\begin{center}\textbf{\asterisc{}}\end{center}


\begin{flushright}\textbf{24/02/2010}\end{flushright}


Há dois dias tive uma gigantesca crise de ódio de tudo e de todos,
sobretudo da doença. Me sentia como um enorme tigre que podia arranhar,
até sangrar e matar a todos, os amigos e a família. Eu estava no auge da
irritação, pensei até em ligar para o Del Porto, mas resolvi dar um
tempo. Esse ódio também me consumia por dentro, queimava e mordia minhas
vísceras. Eu passei perto do suicídio, muito perto. Percebi que esse
tipo de emoção, sentimento ou sei lá o que, é que te dá forças para ir
até o fim. É~uma energia muito forte e a dor e o sofrimento de ter a
doença é enorme. Já são 25 anos de luta diária, de efeitos colaterais
diários, 24 horas sem trégua. Eu estou exausta, eu estou exaurida,
agora, também, dessa incontinência urinária, que já dura mais ou menos
oito meses. Eu estou absolutamente farta de ter acompanhantes, e o Del
Porto não me libera. Há quase dois anos existem quatro mulheres nesta
casa. É~muito. Elas mexeram nas minhas coisas, sumiu um talão de cheque,
ainda bem que estava bloqueado. Enquanto eu estava largada, deprimida,
elas usaram meus cremes. Eu odeio ser invadida. Eu fui super
invadida enquanto elas estiveram aqui, sempre fazendo inalteráveis seu
tricô ou seu crochê.

Viver pelo menos até ter publicado este diário, textos misturados com
desenhos, com fotos, como sugeriu a Bia. Será que vou conseguir? Achei
muito legal a ideia da Bia Bracher de que a publicação fosse assim. Esse
seria novamente um trabalho de ``educadora''. O~intuito é divulgar a
doença para ajudar os seus portadores, ajudar os familiares dos
bipolares e também aqueles que têm a doença e não foram diagnosticados,
ainda não sabem que têm.

\begin{center}\textbf{\asterisc{}}\end{center}


\textbf{}

\begin{flushright}\textbf{02/03/2010}\end{flushright}


Início de uma fase nova: as duas acompanhantes, Cida e Juce, finalmente
irão embora! Virão durante 15 dias, dia sim e dia não. Depois, só duas
vezes por semana. Nem acredito, eu não aguentava mais elas ficarem aqui
18 meses corridos! Eu já estava sufocada e a ponto de perder a Preta,
exausta com a pressão da Cida. Elas são boas pessoas, me ajudaram muito
quando precisei; quando fiquei bebê, regredindo e deprimindo, mas agora
chega, sinto que voltei ao normal, voltei a ser eu, emergi daquela
caverna escura e tenebrosa. Eu estava tão esgotada com a presença delas
que adiantei a consulta com o Del Porto em 22 dias para ele me liberar
das acompanhantes, e deu certo. Ainda bem que ontem eu estava ótima e
daí ele liberou. Eu sabia que ele só iria liberar se me visse ótima.

Eu fui também para perguntar se não seria bom ter uma segunda opinião de
um outro urologista sobre a incontinência urinária, que já dura oito
meses. Ele achou melhor ir a um ginecologista e indicou o Dr. Felipe. Eu
estava cansada de fazer xixi na cama, sou muito moça para isso, preciso
ter paciência de Jó, e a Preta também, por causa dos lençóis e camisolas
molhados. Agora encontrei uma fralda que é uma calcinha, eu mesma posso
me trocar e por isso dá para descartar as acompanhantes. Fiquei muito
chocada quando o urologista falou que essa incontinência urinária é
causada pelo uso do lítio, que tomo há quase 25 anos! O Del Porto tirou
o lítio devagar, ao longo de dois meses. Fiquei três dias sem tomá"-lo,
mas o Del Porto logo reintroduziu, porque tive uma ameaça de mania.
Saco! Saco! Saco! Nada nunca dá certo. Como deu excesso no exame de
sangue, ele diminuiu o Carbolitium para 25mg e aumentou o Depakote para
mais 25mg, que eu tomo depois do almoço.

Fico entupida de remédios e de efeitos colaterais. Só entende isso quem
vive o doença na pele. Emagreço e engordo como uma sanfona. Ah! Muitas
quedas de energia.

Percebi agora que parei simplesmente de tratar a minha pele à noite
porque estou exausta, não tenho forças. Comprei um monte de cremes novos
e nem assim, só dou um trato de manhã: sabonete, tônico, creminho para
os olhos etc. Não é muito. Aprendi com a Soninha, que tem uma disciplina
férrea e uma pele linda. Nenhum bipolar, eu acredito, tem uma disciplina
férrea, a própria doença impede. Só dá para ter ``fases de disciplina
férrea'' quando a gente \versal{ESTÁ} \versal{BEM}\emph{,} \versal{ESTÁ} \versal{ESTÁVEL}. A~gente tem, no
máximo, pequenas fases de disciplina, pelo menos comigo é assim. E~quando eu era jovem, eu era bastante disciplinada, ordeira, tinha força
de vontade. Agora eu preciso estar muito bem para ter essas
qualidades de novo.

\begin{center}\textbf{\asterisc{}}\end{center}


\begin{flushright}\textbf{04/03/2010}\end{flushright}


Hoje vai dar para escrever muito pouco. Pouco. Pouquíssimo. Eu não sei
por que vem assim repetido. Isso é infantil? É senil? Isso começou no
texto sobre ``efeitos colaterais''. Fiquei um tempão com a Soninha no
telefone, ela queria saber o que achei da consulta com o Dr. Caio
Monteiro, ontem, junto com ela e o Zeca. Falei com ela que parecia que
tinha passado um caminhão em cima de mim. Fiquei exausta. Absolutamente
exausta. É~fantástica a dificuldade enorme que mesmo as pessoas mais
inteligentes têm de entender esta doença. Não adianta explicar, não
adianta ler, só entende quem tem. Sobretudo a mania. O~choque, o choque,
o choque. Lá vem a Soninha de novo… O choque, sempre o choque,
eterna fantasia dela. Parece que ela quer que eu passe por tudo que a
mamãe passou. Por que será? Ainda bem que eu já tinha perguntado para o
Del Porto sobre isso e contei para eles. O~Del Porto acha uma judiação
eu ter que passar por 11 anestesias gerais. O~Fernando completou,
dizendo que quem toma choque fica com a memória completamente abalada
por algum tempo. Falou também que o Del Porto é um especialista em
choque, fiquei surpresa, eu não sabia. O~Del Porto também falou de uma
técnica mais moderna que se chama ``não"-sei"-o-que"-transcraniano''. É~um
capacete que colocam na sua cabeça para ativar algumas áreas do cérebro
e responder melhor aos antidepressivos. Eu nem acredito que estou tão
bem sem tomar antidepressivo. Tomara que dure, dure muito! Tenho
acordado cedo, às 7 horas, por causa da troca de fralda, incontinência
urinária por causa do \textbf{lítio} que já dura oito meses. Uma porra!
Ai, que cansaço, que cansaço essa consulta me deu. Um saco, apesar da
boa vontade e generosidade enorme destes dois irmãos, o Zeca e a
Soninha, em me acompanharem.

Uma vez tive uma mania que ninguém percebeu, absolutamente ninguém.
É~inacreditável! Foi quando eu saí do Museu Lasar Segall, em
\emph{janeiro de 1996}. Eu percebi que estava entrando de novo numa
``máquina de moer carne'' e caí fora. Caí fora porque estava
absolutamente frágil devido à morte de mamãe, em dezembro de 1994, e a
de papai, em setembro de 95. Esse intervalo foi muito doloroso porque eu
adorava papai e vê"-lo sofrer era algo quase além de minhas forças.
Fiquei extremamente fragilizada e comecei a sentir a máquina me engolir,
como já tinha me engolido quando eu era chefe do \versal{DAC}, lá no Segall
mesmo. Apesar de eu adorar o projeto que fazia, o ``Conversas
Quartas"-Feiras no Segall'', eu caí fora. Eu não ia segurar toda aquela
mistura de rasteiras, questionamentos, inveja, competição. Não
adiantou, a mania veio por seis meses. A~família não percebeu. Os amigos
não perceberam. Solidão total. Eu vivia na Messiânica perto de casa, era
meu maior apoio, maior que a psicanálise da época. Será que essa
afirmação é mesmo justa? Não sei… Eu estava tão mal que teve
algumas noites em que um pedreiro, que conheci na Messiânica, dormiu lá
em casa porque eu tinha medo. Não houve nada de sexual entre nós, ele
apenas cuidou de mim, velou meu sono. Foi bem difícil essa fase, aquela
sensação de abandono total, de total fragilidade. Estranho que não
gastei rios de dinheiro. Fundo de Garantia. Estranho esse nome. Foi o
Fundo de Garantia que me segurou e o \textbf{S}eguro"-Desemprego, que é
uma mixaria. A~vida é uma grande mixaria de vez em quando. Foi o
Ernesto, meu analista, que me ligou e falou que achava que eu estava em
mania. Obviamente eu estava faltando à análise mais do que nunca, ele
não tinha como me ajudar. Eu mandei uma \emph{corbeille} enorme de
flores lindas para ele. Ele ligou perguntando se era uma despedida. Ele
era um psicanalista. Três dias depois caiu a ficha. Admiti a mania.
Fiquei arrasada, como sempre. Arrasada e dura, não tinha mais dinheiro
para o cigarro, para a gasolina, para a comida. Desespero total. Rogério, santo
Rogério… Fui correndo para ele pedir ajuda. Só a minha família
poderia me socorrer. Quem mais?

Cansada. Estou muito cansada de escrever tudo isso. Reviver tudo isso.
Re"-perceber tudo isso. É~puxado pra caralho. Não sei nem se vale a pena.
De repente começo a duvidar desse desejo, até agora enorme, de publicar
um diário descrevendo meu processo como bipolar. Para ajudar a mim, para
ajudar os outros bipolares, as famílias dos bipolares. Será que vale a
pena? Será que estou no caminho certo? Gasto tanta energia com isso. Eu,
que já tenho normalmente uma energia tão escassa por causa dos remédios,
por causa da doença. Em mim há anos a doença bate assim, minando a
energia que resta, a pouca energia que resta.

\begin{center}\textbf{\asterisc{}}\end{center}


\begin{flushright}\textbf{07/03/2010}\end{flushright}


Acordei às 11h40, tomei café da manhã e já estou morrendo de sono. Fui
deitar às 24h00, dormi bastante. Isso acontece com frequência. Ah, os
remédios! Ah, os remédios! Que paciência a gente tem que ter com os
efeitos colaterais. Como este: o sono. Como a falta de energia, como a
obesidade mórbida. O~lítio causa diurese e, ao mesmo tempo, faz você
reter líquidos. É~demoníaco. Meus pés continuam inchadíssimos, apesar de
o calorão já ter passado. Não posso nem usar os sapatos novos, as
sapatilhas lindas que comprei na hipomania que se revelou quase mania
quando chegou o extrato do cartão de crédito, que é o maior e mais
seguro indicador do quadro.

As acompanhantes foram embora, finalmente! Hoje é domingo, na quinta à
noite a Juce teve um ataque agressivo comigo, foi super malcriada.
Pensei até em pedir para o Rogério vir para cá, mas depois resolvi
ignorar e levar as coisas normalmente. Fiquei morrendo de medo de
ter essa mulher aqui dentro de casa. Quando fui dormir, percebi que
dentro do pires tinha dois comprimidos de Stilnox e não um, que é o
normal. Stilnox é um remédio para dormir. ``Por que ela está querendo me
sedar?'', eu pensei. ``O que ela está querendo fazer comigo?''. O~medo
aumentou. Talvez ela tenha errado na dose, até sem querer. Eu nunca vou
saber. Passei a noite acordada. Não via a hora de ela ir embora e eu
poder trancar a porta.

Na realidade, na segunda"-feira anterior tive consulta com o Del Porto,
quando pedi a ele que me liberasse das acompanhantes. Eu não aguentava
mais a falta de privacidade, a invasão constante, a inveja. E~da parte
da Cida, o autoritarismo, a rigidez absoluta, o super"-controle de cada
gesto meu. Acho que só suportei isso por 18 meses porque estava
realmente muito mal, embora, agora que passou, eu me pegue até
questionando isso. Será que teria sido viável, nesses 18 meses, ficar só
com a Preta e os acompanhantes terapêuticos ``antigos'', aquela moçada
vinculada ao Hospital Dia A Casa? Será que eu não teria conseguido levar
mais sozinha a depressão sem esse inferno de ter acompanhantes 24 horas
por dia? Será que eu não teria me entregue menos à depressão,
caracterizada sobretudo pela total falta de ânimo para fazer qualquer
coisa? Só consegui fazer um esforço gigantesco para fazer análise três
vezes por semana. E~só eu sei que foi um esforço gigantesco. Um saco ter
que levantar da cama, tomar banho e ir. Um saco ficar lá naquela
cadeira, me sentindo totalmente exposta e incomodada com isso. Foi
difícil, foi exigente, mas foi o que me valeu. O~Celso teve uma
paciência enorme comigo. Aquele monte de sessões se sucedendo iguais, ou
quase iguais. Aquela depressão se arrastando, arrastando. ``Eu não
aguento mais esta vidinha'', eu falava para ele, até que um dia ele
disse que a ``vidinha'' não era tão ``vidinha'' assim, afinal, eu fazia
uma boa análise três vezes por semana. Achei que ele tinha ficado
ofendido porque eu o incluí na minha vidinha insossa, insatisfatória,
frustrante, triste, muito triste.

Em abril, graças ao Cymbalta, tive um intervalo hipomaníaco que, apesar
de logo percebido, me custou R\$19.000,00! Era tudo que eu consegui
juntar depois da mania de 2008. Para variar, torrei tudo ou quase tudo
no Shopping Iguatemi. Quando eu me sinto rica, é para lá que eu vou, na
Erica's Boutique. Para falar a verdade, fiz pelo menos boas compras,
tenho usado as roupas até hoje constantemente e vão durar muito porque
são boas. Pelo menos isso.

Percebi dessa vez a mania porque comecei a acordar às 7h30 da manhã sem
o menor esforço e com uma puta energia, como das outras vezes. Esse
sinal, junto com o pensamento acelerado, já é suficiente para detectar a
hipomania. Eu preciso ficar sempre muito atenta.

Percebi a mania pelo olhar de pânico da Preta ao me ver acordada,
tomando café preto na cozinha às 7h30 da manhã. Ela, mais que ninguém,
sabe o que isso significa. A~Cida e a Valdiná, que eram as acompanhantes
da época, não perceberam nada. Mais tarde, a Cida percebeu e tentou me
brecar quando voltei a frequentar a Erica's Boutique. Consegui chegar no
Del Porto a tempo de a mania ser contida.

\emph{Se eu tenho um anjo da guarda, ele se chama Preta}. Ela foi
incansável na loucura de 2008. Ela não entendia nada do que estava
acontecendo, me falou depois que ficava exausta, pois, quando ela
chegava aqui, eu já estava acordada e pedia para ela fazer coisas sem
parar até o horário de ir embora. Normalmente, ela chega às 8h30, e eu
acordo às 11h00. Remexi a casa toda, mudei tudo de lugar, escondi
objetos em cestos normalmente vazios, decorativos. Joguei muita coisa
fora pela janela do banheiro, que é a única que não tem rede. Rogério me
contou que os porteiros disseram para ele que joguei coisas valiosas.
Até hoje não tenho ideia do que foi. Também dei muitos presentes para a
Preta, para uma vizinha, não me lembro mais para quem, mesmo sabendo
que depois eu me arrependo e, alguma vezes, quando percebo, até
peço as coisas de volta. Vivi um verdadeiro inferno na maior alegria e
deslumbramento, esfuziante, contente.

Como é que ninguém mais percebeu? Como é que não fui socorrida antes?
Morar sozinha… O grande furo do Del Porto, o de eu não encontrar
ajuda médica a tempo para me socorrer, uma vez que ele estava de férias.
Ele foi um grande irresponsável comigo e com outros pacientes graves.
Como é que pode um psiquiatra com o gabarito dele não verificar se os
pacientes estão ou não conseguindo entrar em contato com sua equipe? Até
hoje sinto uma mágoa enorme, sinto um ódio maior ainda e continuo
paciente dele… Por quê?

Voltando à ``vaca fria'', na última consulta o Del Porto me liberou das
acompanhantes e, graças a Deus, como era a Cida que estava lá, ele falou
com ela. Falou que na primeira semana elas viriam na segunda, quarta e
sexta"-feira, e na segunda semana, na terça e na quinta. Eu disse para
ele que no primeiro final de semana eu achava bom elas virem no sábado e
no domingo, porque fico muito sozinha nos finais de semana, e é muito
duro, e estava um pouco atemorizada. Na sexta"-feira, a Cida e a Juce
foram embora. No domingo, eu fui almoçar com o Hailton, um acompanhante,
e, ao chegar em casa, às 18h00, dei um cochilo pesado. Tomei dois
chopes. A~bebida ultimamente tem feito um efeito muito maior do que
antes. Certamente deve ser por causa da dosagem dos remédios: 625mg de
Carbolitium e 1250 de Depakote é bastante. Eu sempre bebi, mesmo tomando
remédios. Não consegui renunciar a esse prazer. Tenho, no entanto,
bebido cada vez menos, naturalmente. Há anos não bebo mais sozinha em
casa, como fazia antigamente. Eu adorava tomar uma cerveja, um uísque,
eu adorava pintar bebendo. Teve uma fase em que tinha amnésia alcoólica,
quando ia com minhas amigas beber no Nabuco, um bar na Praça Vilaboim. A~sorte é que eu morava perto. Hoje até adiei um exame de sangue para
poder tomar dois chopinhos no almoço. Aqui, nos finais de semana, o
telefone nunca toca, e, se toca, é alguém me convidando para ir ao
cinema, mas, se estou um pouco deprimida, eu não vou. Eu não consigo.
Acho que me tornei uma pessoa difícil e inacessível para os amigos. Por
isso fico mais isolada. Sem contar que a doença assusta e afasta todo
mundo. É~difícil segurar essa onda. Acho que é desta realidade, da
solidão, que às vezes brota a ideia de suicídio, de que ``estou
esgotada, já sofri muito, não vale mais a pena''. Da solidão além da
doença. Os dois juntos tornam"-se um fardo muito pesado para enfrentar a
vida. A~gente tem que ter muita garra para dar conta. No momento,
escrever este livro, que é ``para os outros'', me motiva muito, pois
como arte"-educadora sempre trabalhei pelo ``crescimento do outro''.
Estou retomando um fio condutor importante da minha vida. Se não
conseguir publicar, vai ser, sem dúvida, uma grande frustração e um
grande vazio porque acredito na \emph{função educativa}\textbf{} deste
livro. Tenho certeza de que existe público para ele.

Escrever, escrever, escrever. Nunca imaginei que fosse ficar assim, tão
tomada pelo desejo de escrever. Escrever está sendo para mim um ato tão
intenso como foi o pintar por tantos anos, o desenhar, o fazer cerâmica.
Estou mergulhada em pleno processo criativo, embora não domine a escrita
como linguagem criativa. Escrevo como leiga absoluta, mas escrever nos
momentos difíceis sempre me ajudou muito. Ajudou a clarear as emoções e
os pensamentos que tomavam conta de mim. Sempre escrevi muito nos momentos
de crise política nas instituições onde trabalhei, nos momentos de crise 
amorosa nas quais me envolvi. Tenho uns seis cadernos de anotações.
Até que não é muito para quem tem 61 anos de vida.

Por que será? Perdi o fio.

Agora não dá para escrever. Hoje é sábado e tem almoço de aniversário da
Marina, filha da Coca.

Pra falar a verdade, estou um pouco sem jeito de ir, embora já tenha
comprado presente e tudo. No fundo, sou tímida. Sempre é um esforço. Mas,
afinal, adoro a Marina e sou amiga da Coca há uns 40 anos, desde o Vera
Cruz.

\begin{center}\textbf{\asterisc{}}\end{center}


\begin{flushright}\textbf{16/03/2010}\end{flushright}


Hoje faz 26 anos que a Tuxa morreu naquela tragédia imensa. Hoje faz
vinte e seis anos que a minha doença começou, a minha tragédia. Dor,
dor, dor e dor. Perda irreparável. A~gente se dava muito bem. Quando a
gente ia para a fazenda, e eu ia muito, ela sempre me mostrava o
trabalho dela e a gente comentava. Ela dizia que eu era a pessoa mais
sensível que ela conhecia, e eu adorava ouvir isso, claro. Ela estava
pintando para valer, já tinha sido aluna do Rubens Matuck, e fez
aquarelas lindas. Quando morreu era aluna do Cabral. Fez paisagens
maravilhosas da fazenda e retratos em aquarela dos empregados. Até hoje
lembro do retrato da Ana, jardineira, e do seu Mariola, marceneiro.

Luto maníaco, reagi ficando doente. Muito uísque, muitos namorados,
muita boate. Licença de 15 dias do trabalho para pintar, pintar, pintar.
Foi a única coisa que consegui fazer. O~Paulo Portella me emprestou um
livro de Goya com a reprodução das gravuras sobre as tragédias. Ele
sacou, era exatamente o que eu estava sentindo, o gosto amargo da
tragédia, a dor imensa da tragédia. Lembro que acordar e dormir era
muito difícil, vinha a sensação de que outra tragédia iminente estava
por acontecer, dava uma angústia enorme.

Romaria a psiquiatras. Começou aí. Fui ao Taboada, antigo terapeuta
junguiano e psiquiatra, mas não deu certo. Eu não tinha nenhuma
experiência com psiquiatras, se o primeiro remédio não desse certo, eu
já caía fora. Teve o segundo, o terceiro, não lembro nem os nomes, não
deram certo. Lembrei do Del Porto, meu professor de Psicopatologia na
faculdade. Resolvi arriscar. Ele foi um excelente professor, muito
culto, levava os últimos artigos de revistas de psiquiatria para a gente
ler. Deu certo. Há 25 anos sou paciente dele, fora os cinco de
``traição'', como ele fala.

Foi o Del Porto quem diagnosticou a minha bipolaridade e me deu lítio
para tomar. Na verdade, fui eu que diagnostiquei a minha bipolaridade,
mas não fui ouvida por ele. Em 1986, tive uma depressão muito forte
devido a um rompimento amoroso. Senti uma tristeza funda, doída e, de
vez em quando, eu tinha muito medo. Crises de medo. Minha casa tinha
sido assaltada no meio dessa depressão, e isso me deixou mais apavorada
ainda. Eu já estava muito frágil e muito instável devido à depressão,
que foi muito longa. Frágil e instável. Frágil e instável. Medrosa.
Solitária. Muito solitária. Naquela época não tinha acompanhante
terapêutico nem família perto. Nem amigos. Solitária. Frágil. Instável.
Mas lutando para dar conta, para sobreviver, para encontrar a luz no
final do túnel, clichê que funciona, que expressa a situação muito bem.

\emph{Imagino} que o Del Porto tenha dado um antidepressivo e eu comecei
a melhorar. Era dezembro quando cheguei lá muito melhor. Em janeiro eu
falei para ele que estava em mania. Ele não concordou, ``Você está
bem'', ele disse, ``Tome esse remédio''. Eu falei a mesma coisa para o
Júlio Noto, meu psicanalista da época, que é também psiquiatra, e ele
interpretou: ``Você não está aguentando o bom''. \versal{SACO}. \versal{SACO}. \versal{SACO}! Você
vai aos melhores especialistas e eles não te enxergam e isso faz com que
você duvide do que você está enxergando em você. Só a catástrofe que
ocorre depois mostra que \emph{você estava certa}.

Foi o que aconteceu. Porque eu só queria dormir, dormir, dormir. Com as
férias chegando, fiquei com medo de ser internada pela família.
Ingenuidade! A família estava muito longe de mim, ninguém tinha noção de
como eu estava. Mas a minha família é assim mesmo. A~gente não se
frequenta. Resolvi reagir e, num esforço heroico, comecei a pintar
camisetas pra vender, montei um ateliê na garagem de casa. Deu super
certo. Levei as camisetas numa loja chamada Arte Assinada, na rua Oscar
Freire, a moça de lá me elogiou muito e disse com todas as letras:
``Você pode ser a nova \versal{MMCL}'', que era na época uma supermarca de
pintura em tecidos. Eu levei o maior susto. Ela até me deu o endereço da
confecção que fazia as camisetas deles. É~raríssimo isso acontecer no
comércio, é contra todas as regras deles. Ela achou que eu deveria
pintar numa camiseta melhor do que a Hering. Eu achei que a melhor coisa
a fazer era passar uns dias em São Sebastião para pensar com calma nesse
sucesso inesperado. Eu não sei se a essa altura eu já tinha mostrado meu
trabalho para o Jean, um supercostureiro da Soninha, que me elogiou
muito e me indicou uma outra loja na Oscar Freire para vender. Eu fui lá
e a dona da loja me deu um monte de amostras de seda para eu fazer um
mostruário pra ela. O~Jean disse que eu devia pintar seda e não
camiseta. Eu já devia estar hipomaníaca quando tudo isso aconteceu. Não
sei.

Fui para \emph{São Sebastião} e lá enlouqueci de vez. A~casa virou um
ateliê e eu pintei milhares de camisetas e alguns tecidos que eu
esticava no corredor para pintar melhor. Um dia quebrou uma garrafa de
cerveja, os cacos machucaram meus pés que ficaram sangrando, mas eu
disse para a caseira: ``Artista é assim mesmo''. No deserto que era
aquela prainha, eu consegui vender várias camisetas para as pessoas que
passavam, vindas de Pitangueiras, a praia vizinha. E~o pior é que eram
pessoas conhecidas e me viram naquele estado! Às vezes eu ia
jantar na cidade. Eu tinha tanta pressa, estava tão impaciente que,
muitas vezes, quando a comida chegava, eu mandava embrulhar para levar
para casa. Eu atraía muito os homens, faz parte do quadro, deve haver
alguma mudança hormonal na mania. É~uma coisa química, ficam como
um urso no mel. Eu dei algumas ``ficadas'', como diz a moçada hoje.
Lembro de ter convidado um rapaz para ir para a casa comigo e lá ficamos,
passeando pelo costão em plena noite, olhando a lua, e depois sucedeu o que
tinha que suceder. A~sexualidade fica mesmo muito exaltada na mania, você absolutamente não tem controle e corre grandes riscos. Naquelas alturas
eu já era uma mulher muito vivida, filha legítima da década de 70, contava
quase 36 anos.

A minha ``mediunidade'' está voltando, tenho sentido isso mais nos
últimos tempos. Hoje foi muito claro, entrei na sala do Celso e me senti
mal imediatamente. O~corpo pesado. Um formigamento na cabeça. Falei para
ele. Aos poucos foi melhorando. Achei que o paciente anterior devia
estar muito mal. Acredito que os ambientes ficam impregnados com as
energias das pessoas que lá estão, lá estiveram. A~energia de uma
igreja, por exemplo, é muito diferente da de um supermercado. A~de um
casamento, a de uma missa de sétimo dia.

Há muitos anos, tia Dagui falou que sou médium, e, com o correr dos
anos, percebi que isso é pesado. Acho que até hoje não consegui
canalizar a meu favor. Ela falou para eu trabalhar com criança e pintar.
Naquela época eu fazia exatamente isso.

Hoje foi terrível. Está sendo terrível até agora. Senti claramente uma
``presença maléfica'' no meu lado esquerdo, essas sensações ocorrem
sempre no lado esquerdo. Dizem que é o lado espiritual. Esse é o lado
mais sensível. Eu já aprendi muito na prática sobre isso. Estava me
sentindo tão mal que não fui ao Dr. Marcos, cirurgião plástico, tirar os
pontos das orelhas. O~trânsito estava infernal, eu me sentindo física e
espiritualmente muito mal, achei que era arriscado demais. Desisti.

Resolvi ir para a Livraria da Vila pegar um livro que encomendei para
dar para a Soninha. Não consegui, porque me deu uma vontade enorme de
fazer xixi e outra vez o trânsito estava parado. Já estava perto de casa
e resolvi vir para cá, arrumar no vaso as rosas brancas que comprei no
caminho. Elas ajudam a dispersar as energias negativas. Aprendi isso. Eu
fico puta quando esse tipo de acontecimento me impede de fazer o que
tenho de fazer. Vou ficando atrasada, muito atrasada. Vou tirar os
pontos amanhã às 12h00, é um horário ruim para mim. Paciência.
Paciência. Paciência. Afinal, não é nada tão grave assim. Agora melhorei,
parece que ganhei a batalha. Estou inteira, não bati o carro, tomei o
maior cuidado. Eu sei muito bem o que é ``estar trabalhada''. É~quando
alguém fez trabalho de ``magia negra'' contra mim, seguindo
provavelmente as orientações do Candomblé, que é uma religião altamente
especializada nesses trâmites. Já vivi isso antes e acho que nunca vou
esquecer. Claro que a primeira pessoa que lembrei foi a Cida, esse bicho
ruim. Hoje assinei as carteiras dela e da Juce. Graças a Deus, estou
livre! Elas devem ter sido comunicadas, daí o ódio. Ficou muito claro
para mim que a Cida nunca acreditou que eu pudesse viver sozinha de novo
nesta casa. Ela falava com todas as letras: ``Imagine você sozinha nesta
casa!''. E~a idiota aqui respondia: ``Há quarenta anos eu vivo sozinha e
faço tudo sozinha''. E~como demorou para eu perceber a verdadeira
natureza da Cida! O Zeca foi o primeiro a sacar, depois a Soninha. Eles
até me chamaram para conversar sobre isso no ano passado. O~Zeca estava
com medo que ela estivesse me explorando economicamente, mas eu garanti
que ela era uma boa pessoa, honesta, e que cuidava bem de mim. Eu estava
ainda muito deprimida, frágil, dependente, sem condições de ter
discernimento. Graças ao Pristiq, saí da depressão e daí, aos poucos,
foi tudo clareando. Estou me sentindo muito bem morando de novo sozinha,
só com a presença da Preta. Fica sempre a sombra de uma mania, de uma
depressão, pelo menos por enquanto. Para falar a verdade, essa sombra é
eterna e faz parte da minha vida. Não é bem uma sombra, é uma realidade,
uma realidade nua e crua. É~um ``estar alerta'', pelo meu próprio bem.
Faz parte da doença, faz parte da saúde, sobretudo da saúde.

\begin{center}\textbf{\asterisc{}}\end{center}


\begin{flushright}\textbf{22/03/2010}\end{flushright}


A aula do Jardim foi absolutamente maravilhosa! Quando a sala estava
vazia e eu entrei para ir ao banheiro, não resisti, dei um abraço nele e
falei: ``Jardim, eu te amo, você é maravilhoso! Você é ouro em pó!''.
Ele deu risada e ficou meio sem jeito, como sempre. Ele já conhece esses
meus arroubos que acontecem de vez em quando. A~gente se conhece há
vários anos, desde quando cursei a \versal{FAAP}, que terminei em 1974, onde fui
aluna dele.

Ele é um conhecedor profundo do desenho e nos dá os fundamentos teóricos
para a prática do desenhar. Hoje ele falou da visão do Apollinaire sobre
o desenho. Nas duas aulas anteriores ele falou de Kenneth Clark, Sérgio
Milliet, Bacon, San Juan de La Cruz, entre outros. O~Jardim tem
\emph{prazer} em dar aula, isso é muito visível, e ele é super didático.
Também, faz isso há mais de 40 anos! Acho que está beirando os
75!

Depois da aula teórica, ele faz uma apreciação dos trabalhos que os
alunos levam. Nesse momento os fundamentos ficam mais claros porque ele
os esclarece para a gente através dos desenhos exibidos. O~que é
fantástico é que ele aceita, acolhe a expressão \emph{individual} de
cada aluno. Cada um tem a sua cara, seu jeito peculiar de ser e de se
expressar. É~raríssimo acontecer isso no meio das artes plásticas. Em
geral, quando a gente vê um trabalho, a gente já identifica qual foi o
professor da pessoa, a escola.

Quando voltei para casa, parei no farol da rua Augusta com a avenida
Brasil e fiquei olhando aquelas árvores iluminadas de roxo no jardim do
lado esquerdo. Olhei um tempão, daí vi uma planta menor, pontuda, também
roxa, e percebi, numa viga branca acima, a sombra roxa e branca da
planta que estava no chão. Percebi que eu estava ``viajando'', parecia
que tinha puxado fumo. Percorri todo o caminho de volta para casa, que
conheço de cor, que já fiz milhões de vezes, percorri ``viajando'',
curtindo deslumbrada. Aliás, o Jardim falou hoje na aula sobre isso,
``deslumbramento''. Eu pensei em passar um \emph{e"-mail} para ele
contando tudo isso, mas agora já está tarde, são 23h30. Amanhã
tenho um monte de coisas para fazer, preciso organizar a agenda.

Ducha de água fria: o Zeca me comunicou que o Fernão e a Soninha
suspenderam os ``acompanhantes terapêuticos''. Fiquei revoltada,
inconformada. Eles não sabem o que é solidão mais bipolaridade, uma
bipolaridade que só piora. Eles não sabem que a doença é \versal{INCURÁVEL}.

É \versal{INCURÁVEL}.

É \versal{INCURÁVEL}.

As crises são \versal{CÍCLICAS}. \versal{OS} \versal{ACOMPANHANTES} \versal{TERAPÊUTICOS} \versal{ME} \versal{AJUDAM} \versal{MUITO}
\versal{NAS} \versal{CRISES}, \versal{AMENIZAM}, \versal{PREVINEM}.

\versal{SUICÍDIO}.

\versal{SUICÍDIO}.

\versal{SUICÍDIO}.

Vou acabar me suicidando de tanta

\versal{SOLIDÃO}.

\versal{SOLIDÃO}.

\versal{SOLIDÃO}.

\versal{MANIA}.

\versal{MANIA}.

\versal{DEPRESSÃO}.

\versal{DEPRESSÃO}.

\versal{DEPRESSÃO}.

\versal{EU} \versal{ESTOU} \versal{CANSADA}, \versal{MUITO} \versal{CANSADA}, \versal{SÃO} 26 \versal{ANOS}…

Quando será que alguém da família vai entender finalmente a doença que
eu tenho há 26 anos? E olha que são todos cultos e inteligentes e vários
deles analisados, terapeutizados, não é por falta de \versal{QI}!

\versal{ELES} \versal{NÃO} \versal{QUEREM} \versal{ME} \versal{VER}.

\versal{ELES} \versal{NÃO} \versal{QUEREM} \versal{ME} \versal{VER}.

\versal{ELES} \versal{NÃO} \versal{AGUENTAM} \versal{ME} \versal{VER}.

\versal{ME} \versal{VER}.

\versal{ME} \versal{VER}.

\versal{ME} \versal{VER}.

\versal{TOMAR} \versal{CONSCIÊNCIA}.

\versal{FUNÇÃO} \versal{EDUCATIVA}.

\versal{FUNÇÃO} \versal{PUNITIVA}.

Eles estão sendo ``educativos'' e ``punitivos''. Eles são assim.

\versal{ELES} \versal{NÃO} \versal{QUEREM} \versal{QUE} \versal{EU} \versal{FIQUE} \versal{BEM}.

\versal{TODA} \versal{FAMÍLIA} \versal{PRECISA} \versal{DE} \versal{UM} \versal{BODE} \versal{EXPIATÓRIO}!

Estou lúcida? Estou sendo mesquinha?

\versal{NA} \versal{MINHA} \versal{FAMÍLIA}, \versal{EU} \versal{SOU} O \versal{BODE} \versal{EXPIATÓRIO}! \versal{NELSON} \versal{RODRIGUES}, \versal{NELSON}
\versal{RODRIGUES}. \versal{GAIARSA}, \versal{GAIARSA}, \versal{GAIARSA}!

Eu estava bem antes do carnaval, as palavras que vinham de dentro de mim
eram:

\versal{RENASCER}!

\versal{ESTOU} \versal{RENASCENDO}!

\versal{FINALMENTE} \versal{SAÍ} \versal{DA} \versal{TUMBA} \versal{ONDE} \versal{FIQUEI} \versal{POR} \versal{DOIS} \versal{ANOS} \versal{CHEIA} \versal{DE} \versal{TERRA} E \versal{DE}
\versal{BICHOS} \versal{ME} \versal{COMENDO}!

Depois que deixaram claro que não era para eu ir para a fazenda do
Fernão e da Soninha porque eu poderia ter um surto \emph{lá}, todo mundo
\emph{sumiu}. Eles só queriam se livrar de mim. O~Zeca. O~Paulinho, a
Soninha, nem sequer telefonaram para mim, já estavam sossegados, longe
do possível surto. Passei o carnaval todo na maior solidão, embora as
acompanhantes estivessem aqui. E~nesse momento foi importante elas
estarem aqui porque eram muitos dias. Só o Rogério me ligou para saber
como eu estava. Ele é de confiança. Se eu tivesse entrado em mania, de
fato, eu estaria internada provavelmente até hoje. Mas eu estava numa
hipomania, feliz, e levei uma porrada tão grande que fiquei deprimida.
Toda a felicidade de ``estar renascendo'' foi para o ralo, foi para o
brejo…

Agora, de novo, cheguei tão feliz e maravilhosa da aula do Jardim ontem!
Estou me reencontrando como artista nessa aula. Não é só a aula que é
importante, é também conhecer novas pessoas interessadas no mesmo
assunto. Embora eu seja tímida, eu adoro aquele cafezinho antes e depois
da aula. Eu adoro as análises que o Jardim faz dos trabalhos dos alunos,
ele sempre vê o lado positivo e negativo, mas reforça o positivo. Acho
que ele, mais do que qualquer um, sabe que ninguém cresce na porrada.
Eu, como arte"-educadora, concordo completamente. Lembrei agora do medo
que a gente tinha do Julio Plaza, que dava Comunicação Visual na \versal{FAAP},
com ele era tudo na porrada. A única coisa que salvava é que ele era um
bonitão.

Estou sentida, estou magoada. Ontem senti muito ódio, muita raiva do
Zeca, da Soninha, do Fernão, do Del Porto com essa história de suspender
os acompanhantes do Hospital Dia A Casa.

É obvio que a Soninha e o Fernão não têm a menor obrigação de me ajudar,
apesar de serem ricos, muito ricos. O~que me consola é que eles já
ajudaram todos os meus irmãos, que são oito, em momentos difíceis que
cada um passou. Eles são muito generosos e discretos nessas ajudas. Eles
são, de fato, pessoas muito especiais. Outro dia, falando nisso, a
Helena comentou que eles ajudam um número enorme de pessoas. Eu não sei
a que pessoas ela se referia.

O Fernão e a Soninha me ajudaram naqueles momentos da vida em que não
consegui me manter. Sempre trabalhei e tive o meu salário, trabalhei por
26 anos numa área que remunera muito mal, a Educação. Trabalhei antes
com estamparia, no Moinho Santista. Trabalhei batendo ponto quatro vezes
por dia e estudando na \versal{FAAP} à noite. Era bem corrido e cansativo. Foi na
época do meu ``casamento'' com o Ivan, e isso foi maravilhoso, me deu
muita força. Ele fazia Desenho Industrial, e eu, licenciatura em Artes
Plásticas na \versal{FAAP}. Eu tinha 25 anos.

Desde menina eu queria ter o meu dinheiro, a minha independência. Com 16
anos, por exemplo, eu dava aulas particulares de Francês, ia de
bicicleta à casa dos alunos que moravam no mesmo bairro, o Alto de
Pinheiros. Dei aulas para a Célia e a Vilma Eid, que encontrei outro dia
na galeria dela, ela é uma superprofissional.

Depois veio a fase dos colares de miçangas para vender na Praça da
República. Minha mãe dava a maior força para esse tipo de iniciativa,
para tudo que favorecesse a independência. Eu amava ir às lojinhas da
Ladeira Porto Geral comprar um monte de miçangas coloridas. Eu ficava
fascinada e muito feliz fazendo esse trabalho. Todo mundo ajudava.
Lembro da mamãe e do Sylvio também fazendo colares. A~Tina e a Bete iam
na Praça da República comigo. Lembro de um dia que levamos um tronco
seco cheio de galhos para pendurarmos os colares. A~feira da praça
estava começando, nos anos 60/70. Não tinha burocracia nenhuma, quem
quisesse chegava com a sua produção e se instalava. O~nível de trabalho
era altíssimo, comparado com o que existe hoje. Os artesãos eram
verdadeiros artistas. Lá eu tinha vários amigos vendendo também o seu
trabalho. Era uma farra! Uma delícia! Quando a feira acabava a gente ia
comer num restaurante e já gastava uma parte do dinheiro.

Quando eu cursava Psicologia nos Sedes Sapientiae, eu fiz um trabalho de
\emph{baby sitter} com os filhos da Lygia Reinach e do Klaus. A~Lygia me
pediu. A~minha família exigiu que a minha irmã, a Tina, fosse junto, e
foi uma delícia trabalhar em dupla, a gente se dá muito bem. Levávamos
as cinco crianças conosco para vender colares na Praça da República! Até
hoje, quando encontro com a Bia, uma delas, que já deve ter seus 40 anos
ou mais, ela fala: ``Gina, aquele foi o único período de minha vida em
que eu não tive uma educação alemã!''. A~gente levava a criançada ao
cinema, às peças infantis, em tudo que fosse do interesse delas. Eu
comprava argila, guache e brincava com elas, eu já era uma
arte"-educadora e não sabia. Outro dia fui assistir a um balé com a Sara,
e daí, na fila de trás, estava a Lygia com todos os filhos. Eu adorei
encontrar a família inteira.

Também enquanto estudava Psicologia no Sedes Sapientiae, eu fui ser
recepcionista no Paço das Artes, que pertencia na época à Secretaria de
Turismo. Fiz o concurso e não passei, fiquei indignada porque meu
francês era perfeito. Eu já estava no quinto ano da Aliança Francesa.
Essa seleção se destinava às ``moças de boa família que falassem francês
ou inglês corretamente''. Eu achava que eu era ``uma moça de boa família
que falava francês muito corretamente''. Movi céus e terra para ganhar o
emprego. Usei o famoso ``quem indica'' e acabei sendo indicada. Quando
tive uma reunião com o secretário de Turismo -- Paulo Pestana, eu acho
-- ele comentou: ``Nossa, a senhora incomodou até o governador!''. Era o
Paulo Egydio, amigo do Fernão. Eu saía da faculdade e ficava no Paço das
Artes das 18h00 às 22h00. Às vezes mamãe ia me buscar, era na avenida
Paulista. Caso contrário, eu voltava de ônibus, naquela época São Paulo
não era uma cidade tão violenta e não tinha perigo, só era bem longe de
casa.

Eu gostava, às vezes, de fazer plantões nos finais de semana em
Congonhas e Viracopos. A~gente orientava os turistas que apareciam. E~uma vez apareceu um francês muito simpático, chamado Jacques, estava com
um amigo. A~gente conversou muito, e eu e minha amiga, que também estava
de plantão, combinamos de levá"-los à noite ao Jogral para ouvir \versal{MPB}.
Claro que isso não fazia parte do nosso trabalho. Deu tudo certo. Foi um
encontro de corpo e alma. Eu não esqueço de vê"-lo tocando flauta para
mim. A~gente passeava pela cidade, pelos lugares mais interessantes. Um
dia a gente foi na Galeria Chelsea, na rua Augusta, tomar um chá, o
acervo da galeria era bom, a gente curtiu bastante. Ele perguntava o
tempo todo se eu iria para a França para ficar com ele. Seria muito
complicado, visto que ele era casado. Ele mandou cartas maravilhosas
amorosas, até que parei de receber… Trinta ou quarenta anos
depois, eu abri um livro em Cotia e tinha uma carta dele. Mostrei para a
mamãe e ela simplesmente disse, com sua seriedade e severidade: ``O que
você queria que eu fizesse? Isso que você fez foi um absurdo!''. Eu
fiquei puta da vida. Por que ela não foi conversar comigo na época e, em
vez disso, sonegou todas as cartas? Será que minha vida teria sido
diferente? Para falar a verdade, hoje eu acho estranho ela ter guardado
uma carta dentro de um livro naquela imensa biblioteca de Cotia, que
contava com 10.000 volumes, e eu ter achado… A única coisa que eu
lembro das cartas do Jacques é ``\emph{Je t'embrasse partout}''. Mamãe
deve ter ficado babando de inveja quando leu minha correspondência
amorosa. Quem é que não fica? Deve também ter ficado chocada, raivosa,
impotente.

Os anos 60 e 70 foram de um maravilhamento e encantamento sem igual. O~homem desceu na Lua, a pílula liberou as mulheres, os \emph{hippies} se
manifestaram. Os The Beatles. Os Rolling Stones. A~Tropicália, Caetano,
Gil, Gal Costa, Bethânia. Antes deles, Roberto Carlos, Erasmo Carlos,
Vanderleia. ``\versal{ATENÇÃO} \versal{PARA} O \versal{REFRÃO}, \versal{TUDO} É \versal{DIVINO}, \versal{MARAVILHOSO}!''
Teve a Janis Joplin, o Jimmy Hendrix , o Joe Cocker. E~teve também o
movimento estudantil no Brasil, o golpe de 64, o terrorismo, a luta
armada. Os guerrilheiros no Araguaia. O~\versal{AI}-5. Teve muita coisa. Coisa
demais. Eu absorvi, eu bebi tudo isso com muita sofreguidão. Eu já tinha
uma ânsia enorme de viver, de liberdade. Uma vontade enorme de conhecer
o mundo além dos muros do Des Oiseaux e da família Sawaya, da moral
cristã e da moral burguesa. Eu, aquela menina tão frágil e asmática da
infância, me tornei uma adolescente rebelde. Eu queria me
aventurar, experimentar. Experimentar tudo o que fosse interessante.
Experimentar. E~experimentei e experimentei, graças a Deus, e
experimentei muito, e saí ilesa. Meu anjo da guarda é forte.

Na adolescência, eu fazia cursos livres de desenho na \versal{FAAP}. Tive
professores ótimos: a Ana Luíza, o Belucci, o Carelli, a Teresa Nazar,
que infelizmente já morreu. Nesse ponto mamãe foi fantástica. Nós
tivemos uma ótima educação. Ela preferia nos dar cultura de ótimo nível
e não aquele vestido de \emph{shantung} ou a calça \versal{LEE} que eu queria
tanto. Não dava absolutamente para dar os dois. Afinal, papai era
professor universitário da \versal{USP} e tinha 10 filhos para criar. Sou muito
grata a ela por isso, Des Oiseaux , Esporte Clube Pinheiros, Aliança
Francesa, Cultura Inglesa, concertos etc.

O Edinizyo? Ou Edinísio? Eu não lembro direito como é que se escreve o
nome dele. Era um baiano magrinho que, de repente, aportou na \versal{FAAP} e
conseguiu um ateliê lá. Ele era completamente criativo e fazia maravilhosamente
bem serigrafia, que na época era uma coisa nova. Não lembro como a gente se conheceu, mas a gente se deu muito bem. Muitas vezes eu matava a aula de desenho e ficava com ele lá no ateliê. Ele era descaradamente \emph{gay}. Fez adereços maravilhosos para o Caetano, a Gal, todo o pessoal da Tropicália. O~Caetano
fazia o \emph{show} semanal -- e eu acho que se chamava \emph{Divino, Maravilhoso} -- e o Edinísio ia regularmente. Esse \emph{show} eu perdi, não sei por que nunca fui. Eu comprei do Edinísio um bracelete lindo de \emph{papier"-mâché}, era
vermelho, verde e amarelo. Acho que ele passou um pouco de breu na
superfície irregular, porque as cores eram muito vivas, e depois fez um
contraste com o preto. Ficou lindo. Uma moça de família como eu, naquela
época, não ficava amiga de \emph{gays} e muito menos de mulatos,
baianos e \emph{gays}.

Mudando de assunto, estou sentindo de novo um ódio agudo nas minhas
costas. De quem será?

Na verdade, estou me sentindo muito sozinha sem as acompanhantes, apesar
de estar muito aliviada. Mas é que tenho ficado muito sozinha,
mesmo… Só agora eu percebi que perdi a análise hoje, que é às
19h00. Eu ainda não me acostumei com esse horário novo. A~análise, com
certeza, teria amenizado a minha solidão, mas a análise não existe para
isso! Eu tomei café da manhã às 11h00, não consegui acordar mais cedo.
Daí escrevi, escrevi, escrevi. Almocei. Escrevi, escrevi,
escrevi… Perdi a hora, saco! Eu me programei \emph{de novo} para
ir ao Pinheiros fazer o tal cartão e não consegui. Até eu estar pronta,
já eram 17h00. Fui na Brenda pegar as roupas no conserto, fui na \versal{MG}
pegar duas calças que ficaram prontas. Fui na Livraria da Vila pegar
dois livros que eu encomendei. Um deles é a história de um bipolar, e o
mais incrível é que eu nem me lembrava disso. Eu anotei o nome do livro na
agenda do ano passado. A~memória já anda péssima. O~outro livro é
justamente sobre a memória, chama"-se \emph{Esculpido na areia}. Fiquei
consolada porque li na contracapa que depois dos 40 a memória começa a
fraquejar. Eu estou com 61 e vira e mexe esqueço tudo, mas tenho visto
muitas pessoas na minha situação. E~eu tomo lítio há 26 anos, por
isso tenho a atenção e a memória prejudicadas há 26 anos.

\begin{center}\textbf{\asterisc{}}\end{center}


\begin{flushright}\textbf{25/03/2010}\end{flushright}


Eu estou me sentindo muito bem, mas tem um fantasma de uma mania
sempre me rondando… Eu achei melhor não parar o Zyprexa 5mg no
final de semana, como o Del Porto mandou. Achei melhor tomar até ontem,
quarta"-feira, por via das dúvidas, e porque hoje tenho uma consulta com
o Del Porto.

O Zeca ligou para ele para ter a opinião dele sobre os Acompanhantes
Terapêuticos do Hospital Dia A Casa. O~filho da puta ao invés de dizer
que ia primeiro conversar comigo, já respondeu que concordava com o
Zeca, que propôs a suspensão. O~que é que eles pensam que eu sou, alguma
oligofrênica? Saco! Só eu sei o chão que me dá ter esses
acompanhamentos nos momentos sofridos, difíceis, paralisantes da
depressão. Na depressão, eu me sinto muito insegura e com muito medo, é
muito difícil ir à analise, ao Del Porto, à farmácia, ao supermercado. É~nesses momentos que eles me ajudam \emph{muito} a manter o meu
cotidiano. Quando não estou bem, eu tenho também dificuldade de ir
sozinha a uma exposição, a um concerto. Posso não estar depressiva a
ponto de ficar dormindo o dia todo, trancada em casa, mas também não
consigo sair sozinha. Nesses momentos eles me ajudam \emph{muito}. Eu
preciso estar \versal{ÓTIMA}, \versal{ÓTIMA}, \versal{ÓTIMA} para ir, por exemplo, sozinha a uma
exposição no Centro Cultural Banco do Brasil, no centro da cidade, e eu
adoro ir lá, as exposições são maravilhosas.

O trabalho dos acompanhantes foi algo que a própria Soninha me
ofereceu, em 2001, quando tive a ``mania dos demônios'' e ela me abrigou
na casa dela. Na verdade, fiquei hospedada numa casa que tinha do lado
da casa dela e estava vazia naquele momento. Tinha uma passagem entre as
duas casas. Eu vi poucas pessoas naquela mania, mas eu sei que nunca
lembro tudo o que acontece numa crise, que é um momento muito intenso e
fragmentado.

O fato é que a Soninha achou que as duas acompanhantes que o Del Porto
mandou eram pessoas muito simples e pareciam mais empregadas do que
acompanhantes. Elas vieram da Granja Julieta. Era a pura verdade. A~Cida
e a Juce fazem parte desse bloco.

A Soninha conversou com o Valentim Gentil, um superpsiquiatra que é
amigo dela. Foi ele quem indicou os Acompanhantes Terapêuticos de A
Casa. A~Tina tinha vindo da Europa para me dar uma força, e nós fomos
juntas para uma conversa com o Dr. Sérgio, diretor do Hospital Dia A
Casa. Eu achava que não precisava mais de acompanhantes, mas ele,
ouvindo em detalhes a minha crise, achou que eu precisava, sim. O~fato é
que finalmente, quando voltei para casa, já passei a contar com o
trabalho dos Acompanhantes Terapêuticos, vinha um de manhã, outro à
tarde e outra à noite. O~fato é que todos eles já têm seus pacientes e
nenhum pode ficar 24 horas comigo. Eu me senti muito bem com eles.
Amenizam o meu sofrimento nas crises. Todos são psicólogos, e converso
muito com eles sobre o que está acontecendo comigo. Eu os chamo, na
verdade, de ``\versal{ANJOS} \emph{\versal{DELIVERY}}''! Tal o bem que me fazem!

Hoje vou conversar com o Del Porto sobre isso, eu estou \versal{PUTA} \versal{DA} \versal{VIDA} \versal{COM}
O \versal{DEL} \versal{PORTO}, \versal{ELE} \versal{SÓ} \versal{FURA} \versal{COMIGO}! \versal{ELE} \versal{SÓ} \versal{FURA} \versal{COMIGO}!

Me dá uma vontade enorme de mudar de psiquiatra e de passar a me
consultar com o Michael Blaich.

Na verdade, eu fui instada pela Soninha e pelo Zeca a pedir uma segunda
opinião com o Dr. Caio Monteiro, colega do Duda. Não gostei. Tive uma
primeira consulta na qual descrevi o meu caso detalhadamente.

Ele pediu uma consulta com um membro da família. Eu pedi para o Rogério
ir, mas ele negou porque acredita no Del Porto e o admira muito. Pedi
para a Soninha e o Zeca irem. Foi um saco, um puta saco. Eu falei que
nunca tinha pensado um mudar de médico, disse que só fui lá porque a
Soninha e o Zeca sugeriram. Defendi o Del Porto com unhas e dentes. O~tal Dr. Fernando me assustou quando disse que como eu estou bem, esse é
o melhor momento para \emph{mexer} nas drogas!

\versal{DROGA}!

\versal{DROGA}!

\versal{DROGA}!

Será que ele não percebe que venho de dois anos de sofrimento intenso e
que não está na hora de mexer em \versal{DROGA} \versal{NENHUMA}? Para os psiquiatras nós
somos cobaias fascinantes para eles experimentarem suas alquimias. Eles
não querem saber do bem"-estar dos pacientes, eles querem é \versal{TESTAR}
\versal{DROGAS}. \versal{DROGAS} e mais \versal{DROGAS}. \versal{TESTAR}. \versal{TESTAR}. \versal{TESTAR}.

``\versal{TÔ} \versal{FORA}'', \versal{EU} \versal{PENSEI}.

\emph{E tô mesmo.}

Daí, por iniciativa própria, fui consultar um psiquiatra que a Helena me
sugeriu no ano passado, o Dr. José Ferreira. Ela ficou sabendo que o
Francisco, que vivia internado, passou a ser paciente dele e não teve
mais nenhuma internação. Isso ficou na minha cabeça e resolvi tirar a
teima por minha conta. Foi ótimo. Foi a melhor coisa que eu poderia ter
feito, eu preciso contar para a Helena.

\begin{center}\textbf{\asterisc{}}\end{center}


\begin{flushright}\textbf{26/03/2010}\end{flushright}


Já estou escrevendo de novo. É~fantástica a necessidade que estou tendo
de escrever nos últimos tempos. Eu me planejei para desenhar e estou
escrevendo completamente autocentrada.

Ontem fui ao Del Porto, e ele me achou \versal{ÓTIMA}. Fiquei feliz. Quantas
vezes saí de lá frágil, me arrastando e sem confiar muito no futuro
remédio, devido a tantos que já tinha experimentado. Ele manteve os
mesmos remédios e a mesma dosagem, fora o Prolopa, remédio para
Parkinson, que ele retirou. Ele disse que eu não tenho Parkinson e
também não tenho mais nenhum vestígio do Risperdal. Graças a Deus!

Estou tomando agora:

1225mg de Carbolitium;

1250mg de Depakote;

1 Stilnox, para dormir;

½ Lexotan 0,3, para dormir;

2 Circadin, para dormir.

Ele queria trocar esses remédios para dormir por 0,25mg de Seroquel.
Como já tomei esse remédio e passei muito mal, eu insisti com ele para
não mudar. Falei que, em vez disso, eu poderia me esforçar para caminhar
todos os dias e assim dormir melhor à noite. Ele topou. Quando estou
bem, eu consigo ser mais ativa nas consultas.

Eu não sei por que estou escrevendo e estou percebendo que estou
cansada, não sei por quê. Esse negócio de baixa de energia o Del Porto
nunca conseguiu resolver. Ele não me fala claramente se tem a ver ou não
com os remédios. Ele sempre insistiu para eu ter uma atividade física e
de preferência pegar o sol da manhã, isso ajuda muito quem tem
bipolaridade. O~Celso e o Rogério sempre bateram nessa tecla também, mas
dessa vez eu ainda não consegui.

Em 2003, eu acho, eu tive um \emph{personal trainer} maravilhoso, o
Pedro Paulo. Foi mais um presente incrível da incrível Soninha, minha
irmã querida. O~mais fantástico é que o Pedro Paulo, depois de me
ensinar alguns alongamentos muito bacanas no primeiro dia, nas outras
aulas só ``conversava'' comigo durante uma hora. Ele é um cara jovem,
alto astral e sabe muito bem como ``motivar'' uma pessoa. Uma vez por
mês, ele fazia uma avaliação e colocava em mim aquele aparelho para
medir a frequência cardíaca. Eu era uma aluna aplicada, mas muito lenta,
no início caminhava \emph{dez} minutos por dia, só tinha energia para
dez minutos. Depois de uma ano, cheguei a caminhar trinta minutos. Eu de
fato tenho pouca energia física, para falar a verdade, acho que foi
assim a vida inteira.

Andar a cavalo, caminhar, nadar: quando eu era jovem, eu fazia tudo isso
junto com os outros na fazenda da Tuxa. Nunca fui modelo de
esportividade como ela, o Zeca, o Tomi, a Bete, o Rogério. Mas
acompanhava bem o grupo. Descia junto com eles nas cachoeiras da fazenda.
Tomava banho de cachoeira no Pinhal, que era a fazenda da Helena.
Se bem que a cachoeira de lá é bem fácil de alcançar. Tem quase uma
escadinha de pedras embaixo. Ai, que delícia! Ai, que saudade!

Hoje vivi uma sincronicidade, eu enviei pelo motoboy um livro de
presente para a Helena e, quando saí da análise, encontrei"-a na padaria
fazendo compras, foi uma delícia. Ela me convidou para ir ao cinema e
ficou de me telefonar. Já são 19h50, ela não ligou, acho que não vai
rolar. Daí vem de novo o problema da energia, eu fiquei com medo de não
dar conta de ir ao cinema, dormir tarde e estar amanhã às 10h15 na casa
da Cecília para ir no Arte na Vila. Para isso tenho de acordar às 8h00!
Porque ainda estou

\versal{MUITO} \versal{LENTA}.

\versal{MUITO} \versal{LENTA}.

\versal{MUITO} \versal{LENTA}.

E \versal{ISSO} É \versal{UM} \versal{SACO}!

\versal{QUE} \versal{VONTADE} \versal{DE} \versal{SER} \versal{NORMAL}\textbf{!}

\versal{MENOS} \versal{DESGASTE}…

\versal{MENOS} \versal{DESGASTE}…

\versal{MENOS} \versal{DESGASTE}…

\begin{center}\textbf{\asterisc{}}\end{center}


\begin{flushright}\textbf{}\end{flushright}
 \begin{flushright}\textbf{28/03/2010}\end{flushright}


\versal{DOMINGO}

A \versal{SOLIDÃO} é uma coisa tão pesada e tão fodida que ela mata

O \versal{DESEJO} \versal{DE} \versal{LER},

O \versal{DESEJO} \versal{DE} \versal{DESENHAR},

O \versal{DESEJO} \versal{DE} \versal{PINTAR},

\versal{DE} \versal{VER} \versal{UM} \versal{LIVRO} \versal{DE} \versal{ARTE},

\versal{DE} \versal{TELEFONAR} \versal{PARA} \versal{UM} \versal{AMIGO}.

A \versal{GENTE} \versal{ACABA} \versal{VENDO} O \versal{FAUSTÃO} \versal{PARA} \versal{NÃO} \versal{SENTIR} \versal{TANTA} \versal{SOLIDÃO}!

A solidão foi tão grande hoje que eu até senti saudade da Cida quando
saí de casa sozinha para ir ao evento Arte na Vila! Estava meio
insegura, foi a primeira vez que saí de casa sozinha desde que as
acompanhantes foram embora, há 23 dias. Deu tudo certo, mas tomar um
lanche sozinha na padaria foi melancólico, absolutamente melancólico. E~chegar de volta em casa foi mais difícil ainda, eu não aguentei, eu com
esse monte de livros de arte, com meu ateliezinho montado, um monte de
tintas Liquetex… Liguei a \versal{TV} no Faustão e peguei uma Coca"-Cola. A~televisão pelo menos tem \emph{voz de gente, você ouve gente e vê
gente}!

Eu fui visitar um ateliê de cerâmica na rua Horácio Lane e gostei muito,
a artista se chama Heloísa Caldas, mas está viajando por um ano, e os
alunos estão trabalhando lá. Conheci o Vicente, um cara muito simpático,
comprei dois vasos minúsculos dele, R\$24,00 os dois!

\begin{flushright}\textbf{30/03/2010}\end{flushright}


É um prazer acordar e tomar um café da manhã simples e caprichado da
Preta. Hoje tinha suco de abacaxi, uma taça verde"-limão, bojuda, cheia
de salada de frutas, queijo branco, pão integral, café. Ela arrumou tudo
igual ao café da novela, e ontem, vendo a novela, percebi de onde ela
tira essas ideias… Eu curto a beleza simples desses objetos que
escolho para o meu dia a dia: a garrafa térmica, bojuda, branca, a louça
``laranja anos 60'', o copo alto com a boca oval, que na verdade seria
um copo de cerveja, mas aqui em casa é de água. O~mais importante em
tudo isso é \versal{TER} \versal{RECUPERADO} A \versal{CAPACIDADE} \versal{DE} \versal{SENTIR} \versal{PRAZER} \versal{NAS} \versal{PEQUENAS}
\versal{COISAS} \versal{DO} \versal{DIA} A \versal{DIA}, \versal{PRAZER}, \versal{PRAZER}, \versal{PRAZER}!

Ontem foi muito difícil ir à aula do Jardim, eu estava morrendo de
vontade de ficar sozinha em casa, desenhando. Estava morrendo de vontade
de ficar ``na minha''. Em geral, quando saio dois dias seguidos, como
fiz sábado e domingo, depois quero ficar quieta. Mas resolvi levar esse
curso a sério e o Jardim é ótimo. Me esforcei e fui.

Foi muito difícil assistir à aula, às 17h00 eu já estava olhando o
relógio, apesar de o assunto ser muito interessante: ``Notas de uma
caderneta de campo, o ambiente mental ou físico que a gente escolhe para
operar nossa poética''. O~Jardim dividiu em seis itens que ele
desenvolveu até às 18h30. Ele falou também sobre \emph{páthos},
liberdade e democracia. E~ele é muito claro: ``Em arte não pode tudo''.
É~realmente fantástico, mas eu fui me sentindo tensa, tensa, exausta,
e mesmo assim fiquei para a leitura dos trabalhos do Samuel, que começou
às 18h30, porque nessa hora a gente aprende muito. O~assunto continuou
sendo o ovo, que ele tratou de várias formas.

À noite aqui em casa, de novo, \versal{SOLIDÃO}, \versal{SOLIDÃO}, \versal{SOLIDÃO}. É~o único
momento em que as acompanhantes fazem falta. Estou fazendo exercício de
\versal{ABERTURA} E \versal{TOLERÂNCIA} e liguei para a Cecília. Ela é extremamente
ansiosa, tenho dificuldade em me relacionar com ela, detesto pessoas
ansiosas. Como tínhamos ido sábado ao Arte da Vila, por sugestão minha,
liguei então para bater um papo. Liguei para a Sara, esqueci que era
noite de aula dela. Liguei para o Tiago para dar os parabéns pelos 33
anos e percebi que ele estava extremamente angustiado, fiquei péssima
depois do telefonema porque adoro ele, estou preocupada com ele.

Essa noite dormi muito mal, amanhã vou para a praia com a Coca e percebi
que estou tensa, continuo tensa também com o assunto ``Zeca, Soninha e
Fernão''. Tenho um monte de coisas para fazer, será que vou dar conta?

\begin{enumerate}
\item
  Separar os remédios para a viagem, contar tudo \versal{SEM} \versal{ERRAR}
  \versal{NENHUM}\textbf{;}
\item
  Telefonemas;
\item
  Vacina gripe \versal{H}1;
\item
  Compra mensal de remédios;
\item
  Supermercado da praia;
\item
  Presente aniversário Preta;
\item
  19h10 análise.
\end{enumerate}
\begin{center}​​\asterisc{}\end{center}
 \begin{flushright}\textbf{}\end{flushright}

\begin{flushright}\textbf{10/04/2010}\end{flushright}


Antes da viagem para Baraqueçaba, peguei dez envelopes e em cinco
escrevi um D bem grande, e nos outros, um N bem grande. Separei os
remédios e guardei lá. Aprendi isso com a Cida e vi que funciona bem,
melhor do que levar um monte de cartelas que ocupam espaço enorme na
mala, e leva um tempão contar os remédios na viagem.

Como ia viajar, eu não parei de tomar o Zyprexa 5,0mg, como o Del Porto
tinha mandado. Na verdade ele falou para diminuir para 2,5mg, é 50\% a
menos, é bem menos. Eu queria ter o máximo de segurança possível de que
não teria um início de crise, pois percebi que, por alguma razão, estava
tensa com a viagem. Além disso, levei uma cartela de 5,0mg de Zyprexa,
que estava sobrando, para o \versal{CASO} \versal{DE} \versal{PINTAR} \versal{UMA} \versal{MANIA}!

\versal{QUE} \versal{MEDO}!

\versal{QUE} \versal{MEDO}!

\versal{QUE} \versal{PAVOR} \versal{DE} \versal{TER} \versal{UMA} \versal{MANIA}!

\versal{PAVOR}!

Estou cansada de saber que \emph{só eu} posso detectar os sintomas e
aprendi a fazer isso bem. Talvez eu seja até um pouco exagerada,
obsessiva, não consigo relaxar.

Na última mania, que foi desastrosa, eu percebi! Pedi ajuda médica e não
tive! O bip do consultório do Del Porto, para falar com os assistentes
dele, simplesmente não funcionou, não tive retorno algum! Não quis
telefonar para o celular do Del Porto porque ele estava viajando, de
férias. Eu não tive logo a ideia de procurar outro psiquiatra, já que o
Del Porto estava de férias. Não insisti, talvez porque a mania no
início dá um puta bem"-estar e eu vinha de dois anos muito sofridos, com
remédios e seus efeitos colaterais, com depressões intermitentes. Eu
estava exausta com esses dois anos de sofrimento e mergulhei no aparente
prazer da mania. Mal sabia eu que teria que pagar tudo isso com um ano e
meio de depressão, largada numa tumba. Agora já faz mais de quarenta
dias que as acompanhantes foram embora. A~viagem da Páscoa foi ótima,
embora eu tenha tido momentos de pensamento suicida ao pensar na minha
dependência econômica. Eu odeio depender. Aos poucos, me envolvi com a
Coca e o Sérgio, e esses pensamentos se afastaram.

Na ida paramos para comer no Rancho da Comadre, que oferece uma comida
maravilhosa feita no fogão de lenha. Comi ``leitão à pururuca'' que há
séculos não comia e adoro. A~Coca e eu conversamos, conversamos e
conversamos. Nós trabalhamos no Vera Cruz na mesma época, mas
estranhamente só ficamos amigas depois de sair de lá. Há muito tempo a
gente não se via. Na realidade, há quase dois anos estou afastada de um
contato mais próximo com os amigos.

A casa que ela e o Sérgio alugaram em Baraqueçaba é pequena, mas
gostosa. Tem uma suíte e dois quartos com um banheiro no meio. É~construída com capricho e lembra a casinha que tínhamos em São
Sebastião, na Praia do Segredo.

Em 1949, papai, biólogo marinho, comprou em São Sebastião um morro
enorme com uma praia, com seus próprios recursos, e ali construiu o
Instituto de Biologia Marinha. Ele contou com a ajuda da \versal{CAPES}, da
\versal{FAPESP} da Rockfeller Foundation. Ele era um grande idealista.
Enquanto o Instituto era construído, nós íamos para São Sebastião nas
férias e ficávamos horas mergulhando naquele mar e catando bichos para o
meu pai. Hoje, quando vejo filmes e fotos de Fernando de Noronha, eu me
lembro da minha infância em São Sebastião, era igual. Eu lembro muito do
nome de uma aula escrito no quadro"-negro do laboratório: ``A fecundação
do ouriço'', para mim isso era um grande mistério.

A certa altura, muito depois de inaugurado o \versal{IBM}, papai comprou um
terreno para nós, e o Sylvio, meu irmão arquiteto, reformou uma casa de
caiçara, que ficou muito boa. Era um pequeno paraíso tropical, podíamos
usar as duas praias, a nossa e a do laboratório. Fazer aquele costão
pulando as pedras era um deslumbramento, tal a diversidade de cores,
brilhos e texturas das pedras. Lá tinha uma pedra que a gente chamava de
``trampolim'', a gente tinha que fazer o costão, que era uma subida, e
no topo se descortinava uma vista maravilhosa do canal e da Ilha Bela.
Era extraordinariamente belo! Daí a gente descia com cuidado, pedra por
pedra, e chegava ao tal ``trampolim'', a gente nadava por ali e, como a
natureza é perfeita, tinha até uma escadinha de pedras, só que com
cracas, para voltar para o trampolim. Muitas vezes a gente voltava de lá
nadando para a praia. A~gente nadava pra caralho. Eu lembro muito bem
de mamãe ir com a ``tropa'' nadando dar a volta à ilhota, era um passeio
comprido, mas era uma delícia. Passei temporadas deliciosas com os
amigos em São Sebastião. Teve uma temporada com a Renata em que a gente
bebia meio litro de uísque por dia depois de nadar, nadar e nadar,
fazer o costão, e daí a gente desenhava, desenhava e desenhava. Foram
milhões de temporadas:

\begin{itemize}
\item
  com o Gaiarsa, que na época era meu namorado, o Pedro Prado, o
  Oswaldo;
\item
  com a Coca;
\item
  com a Helena e o Modesto Carvalhosa;
\item
  com o Dudu e o Vilar, a Peia e o Alex;
\item
  com a Helena Carvalhosa, a Angela Ciampolini e não sei mais quem;
\item
  com o Ivan e outros amigos da faculdade;
\item
  com o Dudu e o Vilar e o Mário Aquiles, meu namorado.
\end{itemize}
Várias pessoas foram várias vezes, e agora não estou lembrando de todas.
Várias vezes também fui com irmãos e sobrinhos e algumas fui
sozinha. Eu tenho as imagens desse lugar de cor na minha cabeça, percebi
isso lá em Baraqueçaba. Eu lembrava, eu me lembrava de tudo. A~cor e a
consistência da areia. O~pôr do sol na praia do laboratório, quando o
sol tingia o mar das baiazinhas de rosa, rosa brilhante misturado com
dourado, balouçante. Monet puro. As pedras do costão, eu lembro delas
muito bem. Eu tenho verdadeira paixão por esse lugar, eu fui a única a
batalhar na herança para que não fosse vendido. Fui voto vencido.
Herança é herança. Gaiarsa que o diga. Gaiarsa e Nelson Rodrigues. Cada
um é cada um. Eu dancei nessa. Eu adoraria poder frequentar esse lugar
encantado hoje em dia, sem dúvida me faria muito bem.

Não é à toa que o que eu mais gostei nesse feriado foram os banhos de
mar em Baraqueçaba, que é uma praia vizinha à praia do \versal{IBM} e o mar é
igual, calminho, calminho, calminho. Teve um dia que foi especialmente
bom, o Sérgio e a Coca voltaram para a barraca, mas eu continuei horas e
horas mergulhando no rasinho ao sabor da correnteza. Eu adoro fazer
isso. Quando sentei na barraca, eu estava completamente relaxada e até
cabeceava de sono, não precisava nem de cerveja, foi um banho como
aqueles de antigamente. Eu me lembro bem de uma cena em São Sebastião: a
Regina, a Rosa e eu, três arte"-educadoras, horas e horas mergulhadas
desse jeito, no rasinho, conversando, conversando… Era delicioso,
era reparador, sinto muita falta disso. Falta de um lugar simples e que
seja meu, na verdade dos meus pais, para ir com meus amigos. Mas…
a vida muda, tudo muda, a gente envelhece, alguns amigos já morreram. Há
vários anos tenho a ``bênção'' de ir durante uma semana, em janeiro,
para a casa do Soninha e do Fernão, em Camburi, que é uma praia linda.
Eles são muito generosos e oferecem ``casa, comida e roupa lavada'', as
pessoas nem acreditam. A~casa é de frente para o mar e o horizonte é
lindo. A~minha única tristeza é que o mar é bravo, tenho cada vez menos
coragem de entrar nele e fico na piscina do condomínio para consolar.
Quem diria…

Mudando de assunto, tenho lembrado muito das acompanhantes. Quando
voltei de viagem, tinha um recado com uma voz muito esquisita da Cida.
Ela tem uma voz muito grossa e falava muito pausadamente como se falasse
com uma débil mental, que seria eu. Ela disse que estava com saudades,
pediu para eu entrar em contato. Deixei recado no celular dela, e até
agora nada. Já faz uma semana, saudades o caralho, ela está assuntando
para saber se eu preciso dela de novo. Tenho lembrado com horror esse
tempo em que tinha acompanhantes vinte e quatro horas, porque é muito
infantilizante em si, e elas caprichavam para me infantilizar ainda
mais:

\begin{itemize}
\item
  Regina, os seus remédios.
\item
  Regina, o seu lanchinho.
\item
  Regina, o adesivo.
\item
  Regina etc, etc, etc.
\end{itemize}
Foi pesado, foi pesado. Esse um ano e meio foi terrível. Mas elas eram
boas pessoas… até certo ponto. Em momentos difíceis, como
quando tive diarreia, incontinência urinária e os efeitos colaterais do
Risperdal, elas foram fundamentais. A~Preta sozinha não ia conseguir
segurar. Na verdade, tenho que ver os dois lados, foi pesado, foi
terrível, mas elas foram necessárias, e eu tenho que agradecer a Deus
por ter um cunhado maravilhoso como o Fernão, que bancou essa despesa
enorme por um tempo tão longo, sem questionar por um segundo.

\begin{center}\asterisc{}\end{center}
 \begin{flushright}\textbf{12/04/2010}\end{flushright}


Eu estou escrevendo desbragadamente para tentar fazer uma ponte entre os
quase dois anos fora do ar e o agora. Desse período, a única lembrança
nítida e importante foi o nascimento da Malu, filha da Carol, minha
sobrinha linda. A~Carol é muito inteligente, esperta, responsável e uma
companhia muito leve. Nós nos damos às mil maravilhas.

Eu senti uma certa tristeza ontem ao ir à feira do Bexiga com a Gilda.
Eu acordei triste hoje, com preguiça de ir à aula do Jardim, com vontade
de ``continuar o fim de semana'' aqui em casa escrevendo, escrevendo,
escrevendo. Será que agora ainda vai pintar um ``luto'' por ter perdido
quase dois anos da minha vida numa crise? Acho que é demais, é
preciosismo, não posso mais perder tempo. Eu tenho muito claro que é a
solidão o que está me incomodando e isso é natural depois de ter ficado
quase dois anos com acompanhantes 24 horas por dia. Eu não esperava por
isso, mas agora acho que não poderia ser diferente.

Eu percebo que escrever sobre o ``período saudável'' da minha vida me dá
um gás para viver a vida de hoje, que é bem diferente. Eu rememoro,
provavelmente ``contaminada'', períodos mais felizes e intensamente
vividos. Eu sinto e percebo que fui muito corajosa ao tentar desbravar
um mundo diferente daquele em que fui criada. Vivi muitos momentos
poéticos. Vivi muito medo. Muitas vezes vivi coisas com ``o cu na mão''.
E~tudo isso foi muito solitário, eu não tive nenhuma companheira de
aventuras. Relembrar tudo isso está sendo passar a limpo uma vida
inteira. Por quê? Pra quê? No momento me parece que é uma questão de
sobrevivência. É~esquisito que eu gaste metodicamente tanto tempo
escrevendo e não desenhando, já que voltei a frequentar o curso do
Jardim e não tenho produzido nada para levar nas aulas. Até comprei o
material que ele mandou, fiz uns desenhos a carvão, mas foram só uns
quatro ou cinco.

\begin{center}\asterisc{}\end{center}
 \begin{flushright}\textbf{13/04/2010}\end{flushright}


O sono, o sono, o sono. Boto o despertador todos os dias para acordar às
10h00 e desligo. Sistematicamente desligo. Sei que a Preta por conta
dela virá me acordar às 12h00. A~Preta é meio ``governanta'' da casa. O~problema com o sono é antigo, na verdade, sempre fui dorminhoca, mesmo
antes da doença. É~uma necessidade biológica, eu acho. A~culpa, a culpa,
a culpa! Afinal, eu tomo \versal{TRÊS} \versal{REMÉDIOS} \versal{PARA} \versal{CONSEGUIR} \versal{DORMIR}! Como é
que não vai ser difícil acordar? Nessas alturas já consegui eliminar
lentamente o Lexotan, depois de tomar meio por uma semana, eliminei
ontem. Hoje, ao organizar os remédios para a noite, botei meio Stilnox,
eu sei que durmo com meio Stilnox, o problema é quando tento passar para
zero, aí não consigo, acho que fiquei viciada. Eu ainda tomo dois
Circadin, que é feito de ``melatonina'', uma substância que a gente tem
no corpo. Esse não é tão nocivo, mas custa uma fortuna… Eu acho
que perco muito tempo dormindo, dormindo. Ultimamente comecei a achar
que dormir é uma espécie de morte. É~um tempo em que você está
absolutamente fora do ar e não produz nada.

No último final de semana dormi muito pesado no sábado de manhã. Eu
dormia, sonhava, acordava e lembrava daquele pequeno sonho. Repeti a
operação várias vezes e várias vezes. Quando eu acordei, finalmente, eu
jurava que eram sete da noite… Olhei no despertador e marcava
15h15. Não acreditei. Olhei no relógio de pulso e marcava 15h15.
Acreditei. Senti um alívio enorme.

Eu não tive filhos, então não tive esse ``despertador natural'' da vida.
O~Gaiarsa escreveu, num livro especial que dedicou às mães, que a
primeira crise de ódio que a mãe tem em relação ao filho se deve ao fato
de ela não poder dormir tudo o que precisa. Mamãe leu esse livro que dei
de presente para ela e concordou com as teses do Gaiarsa sobre o
assunto. Quem diria?!

Enquanto as acompanhantes estiveram aqui, eu passei a dormir às 23h00
religiosamente, para liberar o sofá para elas dormirem. Elas acordavam
às 4h00 da manhã para estar aqui às 7h00. Antes eu dormia às 00h00, à
1h00, 01h30, eu sempre fui boêmia. Quando começou a troca de fralda às
7h00 da manhã, por causa da incontinência urinária, de vez em quando eu
não conseguia dormir mais. Eu rolava na cama até às 9h00 e levantava.
Então eu percebi que seria possível mudar o meu ritmo de sono. Acordando
às 9h00 eu ficava sonada e cansada alguns períodos do dia, mas aguentava
sem dormir de novo.

Será que tem sentido querer mudar tudo isso a essa altura da vida? Aos
61 anos? Afinal, faz 15 anos que eu não trabalho e que posso ter meu
próprio ritmo. Ninguém me controla, a não ser o horário da Preta, ela
tem que sair às 15h00, e por isso eu tenho que almoçar às 14h00. Às
vezes, quando eu me atraso muito, ela deixa o meu prato feito. Conversei
com o Celso, ele acha que o sono deve ser uma necessidade interna e que
seria bom eu deixar o despertador ligado às 10h00. Se eu conseguir
acordar a essa hora, consegui… se não, não… Por que eu me
martirizo tanto com esse assunto até hoje? É que, na verdade, eu sinto
muito perder todo um período do dia. Minha vida anda muito devagar.

E ainda tem aquela coisa: quantas e quantas vezes o Del Porto me falou
que \emph{para um bipolar é importante caminhar e pegar o sol da manhã}?
Ajuda na

\versal{ESTABILIDADE}.

\versal{ESTABILIDADE}.

\versal{ESTABILIDADE}.

A tão almejada estabilidade. Será que um dia vou reverter esse quadro,
dormir às 23h00 e acordar às 9h00 e daí ir caminhar?

Eu sempre fui lenta, mas, devido aos remédios, fiquei cada vez mais
lenta. Hoje levo no mínimo uma hora para tomar café da manhã, por
exemplo. Tudo é

\versal{LENTO},

\versal{LENTO},

\versal{LENTÍSSIMO}.

Isso é um saco! Perco um tempo enorme com as obrigações triviais do
dia a dia. Tenho que me programar bem para sair a tempo para os
compromissos: médicos, análise, dentista etc.

Se alguém me convida na última hora para pegar um cinema, se eu já não
estiver vestida, eu não consigo ir, não dá tempo de eu trocar de roupa
etc.

Esse negócio de dormir muito de manhã também me isola, quando meus
amigos fazem programas de manhã, tipo ir a uma exposição, eles nem me
convidam, porque sabem que eu estou dormindo. A~secretária eletrônica é
que vai atender. Eu queria muito ser igual às outras pessoas nesse
aspecto.

\versal{MUITO}.

\versal{MUITO}. E~\versal{NOS} \versal{OUTROS} \versal{ASPECTOS} \versal{TAMBÉM}…

Nesse momento da minha vida eu absolutamente não estou tendo força de
vontade para nada que seja difícil: acordar mais cedo,

caminhar,

emagrecer.

Escrever não é esforço. Na verdade tem sido uma necessidade compulsiva,
eu nunca imaginei que um dia fosse acontecer comigo. Escrever, escrever
e escrever. As aventuras e as desventuras da vida. O único esforço que
estou concretamente fazendo é ir à aula do Jardim. Mas sempre saio de lá
completamente recompensada. Ele descreve para nós com muita profundidade
o universo do desenho e, de quebra, da pintura. Na última aula, ele
falou da teoria do Zuccaro: ``Euritmia'', o desenho tem que ter
euritmia, eu nunca tinha imginado isso. O~todo e as partes. A~harmonia.
Cada vez ele fala de uma teoria. Eu só tenho escrito. Eu não tenho
desenhado nada. Solidão: \emph{e"-mails}, telefonemas.

\begin{center}\asterisc{}\end{center}
 \begin{flushright}\textbf{16/04/2010}\end{flushright}


Acabei de arrumar os remédios depois do café da manhã. Na caixa pequena
marronzinha, os do dia. Na caixa vermelha em forma de coração, do
tamanho da palma da minha mão, os da noite. Escolhi uma caixa grande de
propósito, ela faz vista no criado"-mudo e assim não esqueço de tomar. Na
hora de dormir é só pegar.

Ontem à noite percebi que oscilei muito o dia todo, pendendo para uma
mania. Conversei muito com a Preta de manhã. Eu estava meio exaltada.
Falamos muito da mania de 2008. Cada vez que a gente fala, ela me revela
um pedaço diferente. Eu não me lembro mais de tudo.

À noite liguei para Renata, eu estava com saudade. Conversamos muito e
eu achei que também fiquei meio exaltada. Eu contei como eu estou agora,
quais os planos, e também relembramos muito, com saudades, as férias em
São Sebastião.

Depois da novela fui abrir minha caixa postal e passar \emph{e"-mails}.
Foi aí que acendeu a luz vermelha: eu não estava com sono, estava com
muita energia. Eu poderia ficar horas no computador. Como? Em geral,
lá pelas 22h30 eu já estou morrendo de sono. Pensei, pensei, pensei e
acho que para garantir, era melhor tomar 5,0mg de Zyprexa e não 2,5 como
o Del Porto havia orientado. Eu tinha seguido a orientação dele, mas a
mania começou a pintar. Esforço, esforço, esforço. A~gente tem que fazer
um esforço desgraçado para driblar a crise. Se eu fosse esperar até hoje
para falar com o Del Porto, talvez eu já estivesse perdida. A~sorte é
que eu conto com um longo aprendizado a meu respeito e a respeito dos
remédios.

Hoje acordei às 9h00 da manhã! Só em mania, mesmo! Mas eu acordei
calma e até levemente deprimida. Eu estou morrendo de preguiça de sair e
fazer o que tenho que fazer. Levar os recibos para a Maria, ir à
Marítima pegar o reembolso, pegar os óculos. Eu tenho que ir. Já era
para ter feito isso ontem e passei o dia todo dentro de casa.

\begin{flushright}\textbf{17/04/2010}\end{flushright}


Hoje acordei lembrando do Edinísio e do Marcelo me levando para comer
comida japonesa na Liberdade nos anos 60. Fiquei maravilhada com as
cores e os sabores dessa culinária. \emph{Sushi}, \emph{sashimi},
\emph{tofu}, \emph{missoshiro} e aqueles rolinhos de alga com arroz
dentro, com um pedacinho de cor no meio, tudo era novo para mim. Não tem
nada mais gostoso que entrar em contato com algo que seja completamente
diferente, sobretudo quando esse algo é bom. Fiquei fascinada pelo
bairro da Liberdade, antes inacessível para mim. Sou fascinada por esse
bairro até hoje. Na última vez em que fui lá com o Sylvio, a Lúcia e a
Tina, num domingo de sol, foi uma delícia, foi no ano passado. O~Edinísio e o Marcelo, também artista plástico, tinham um ateliê na rua
atrás do cemitério da Consolação. Ali era também o lugar onde eu fazia
terapia com o Dr. Espíndola: como não entrei no vestibular para
Psicologia, fui fazer um teste vocacional. Acontecia mais ou menos o
seguinte: eu estava desestruturada, provavelmente tão deprimida que eu
não tinha condição de escolher uma profissão. Eu me dividia entre
Ciências Sociais, Psicologia e Artes Plásticas. Papai insistia na
Psicologia, Ciências Sociais provavelmente para ele não era Ciência. O~importante para mim foi começar a fazer uma terapia, eu sentia que
estava precisando. Para falar a verdade, hoje, olhando de longe, acho
que essa terapia não adiantou nada… não me deu suporte algum.

Eu fui fazer o cursinho para Psicologia no Equipe, recém"-inaugurado,
dissidência do cursinho Universitário. Lá eu conheci o Eric Nepomuceno,
que iria prestar Filosofia. Meses depois do vestibular, nos encontramos
no Jogral, um bar onde tocava o ``Trio Mocotó'', entre outros conjuntos.
A~gente começou a namorar e nos demos muito bem. Era um namoro leve,
gostoso, já estávamos nas respectivas faculdades. Eu admirava muito o
brilho, a inteligência e o talento do Eric. O~pai do Eric era físico da
\versal{USP}. O~Eric era jornalista do \emph{Jornal da Tarde}, um jornal moderno
e de vanguarda que foi lançado nessa época. O ``cabeça'' do jornal era o
Murilinho. O~\versal{JT} era moderno no conteúdo e na diagramação, fez o maior
sucesso, hoje em dia é um jornal comum.

Enquanto namorava o Eric, fiz o trabalho de \emph{baby sitter}, tomando
conta dos filhos da Lygia e do Klaus Reinach. Teve uma noite em que o
Eric levou o Milton Nascimento e a mulher para ficarem com a gente lá! O
Milton estava no início da carreira, mas já era o Milton Nascimento. A~mulher dele ficou com a gente fazendo colar.

A gente frequentava muito o Jogral, do Luiz Carlos Paraná, lugar onde
tocavam excelentes músicos de \versal{MPB}. Íamos também ao Bar Branco, da Maria
Inês, da Carola Whitaker e do Zélio Alves Pinto. Era um bar muito
charmoso, todo branco, que tinha andar térreo, primeiro andar e subsolo.
Um dia a gente estava lá embaixo, todo mundo sentado no chão, e chegou o
Fernando Morais. Ele tinha acabado de percorrer a Transamazônica,
estrada ainda de terra, recém"-inaugurada, ele fez para nós um relato
detalhado e encantador das aventuras e desventuras da viagem. Falou por
horas. Ficamos todos hipnotizados, em silêncio, ouvindo com prazer e
atenção o excelente contador de histórias, que transformou aquele relato
em livro. Se não me engano, ele ganhou o Prêmio Shell com o livro, era
um prêmio importante na época.

Teve uma noite em que fui assistir a um \emph{show} da Sarah Vaugham no
teatro da \versal{PUC} e de lá fui para o Bar Branco sem ter combinado de
encontrar ninguém. Eu arrisquei. Eu fiquei feliz por arriscar.
Provavelmente as pessoas me achavam muito estranha, já que não era
costume uma mulher sair sozinha naquela época. Eu queria me testar. Eu
queria me experimentar. Eu me testei exaustivamente. Tudo o que eu faço,
eu faço exaustivamente. Quando estou bem, faço exaustivamente. Eu canso,
mas vale a pena.

No Colegial eu já tinha uma vida intensa, mas ainda dentro dos
parâmetros socioculturais da época, para uma moça de boa família. A
única brecha fora desses parâmetros era o hábito de frequentar o ateliê
do Ednísio, na \versal{FAAP}. A~intensidade da minha vida era cultural. Eu
frequentava muito as mostras de cinema expressionista da Casa de Goethe:
Fritz Lang, O Gabinete do Dr. Caligari, e~outros filmes, muitos outros.
Eu frequentava também Concertos para a Juventude nos finais de semana,
regidos pelo Diogo Pacheco. Papai e mamãe tinham uma assinatura de
concertos, acho que eram os do Cultura Artística, e a gente ia assistir
durante a semana. A~gente fazia um rodízio, eram dois ingressos.

Outra brecha foi assistir aos ensaios do Quarteto Novo que ocorriam à
tarde na Boate Blow Up. O~Pedrinho, meu namorado na época, trabalhava na
boate e me convidava para assistir! Era o máximo! Que puta som! Hermeto
Paschoal, ah! O Hermeto e mais aqueles maravilhosos músicos conversando
horas e horas através de seus instrumentos… Era uma beleza! Uma
beleza! Ah, se a mamãe soubesse!

No início do Colegial, namorei o Pedro Barreto Prado, que é meu amigo
até hoje. Nos tornamos irmãos. Nossos pais se conheciam da Ação
Católica. Quando a mãe do Pedro estava grávida dele, minha mãe estava de
mim. Foi assim que nos conhecemos…

Eu não me lembro exatamente de como conheci o Pedro. Eu imagino que
tenha sido numa das festas de 15 anos tão comuns naquela época. As
festas eram frequentadas pelas meninas do Des Oiseaux e, à vezes,
algumas do Sion, e pelos meninos do Santa Cruz e, às vezes, alguns do
São Luís. O~Pedro logo se encantou comigo e queria namorar. Eu não
queria. Ele insistia. Eu não queira. Ele insistia e insistia. Insistiu
tanto que eu acabei topando. ``Eu não sou a maravilha que você está
imaginando'', eu dizia. Acho que eu estava insegura. Ele insistiu,
insistiu, insistiu. Acabei topando. Nós vínhamos de famílias parecidas,
de colégios parecidos, tínhamos uma visão de mundo parecida. O~curioso é
que nós dois fomos, no decorrer da vida, nos tornando alternativos.
Naquela época nós adorávamos passear pelo centro da cidade. Às vezes, o
Pedro ia me buscar na saída do colégio, nós tomávamos um lanche e íamos
visitar o \versal{MASP}, que era na rua Sete de Abril. Percorríamos o acervo e
depois ficávamos horas sentados vendo aquele quadro do Van Gogh, o
menino sentado na cadeira. O~Van Gogh era a minha paixão e foi o assunto
do primeiro fascículo dos Gênios da Pintura, que o Pedro me deu de
presente. Outras vezes nós tomávamos chá numa casa chamada Chá Mon, na
Galeria Metrópole, na avenida São Luís. Nós conversávamos e
conversávamos. Nós tínhamos sempre mil assuntos para conversar. Teve uma
época em que fomos colegas na Aliança Francesa, na rua General Jardim.
De vez em quando matávamos aula para fazer nossos programas. Namoramos
por três vezes, entre 15 e 19 anos. Acho eu que nenhum de nós tinha
maturidade para um vínculo mais duradouro, mais estável. Num desses
intervalos, namorei o Otávio, engenheiro da \versal{POLI}, era um cara
inteligente e sensível. Namorar um cara da \versal{POLI} era o máximo! A amizade
entre mim e o Pedro perdurou ao longo dos anos. Ele foi fazer Psicologia
na \versal{PUC}, e eu no Sedes. Quando ele terminou, clinicou um tempo, mas logo
descobriu que queria ser terapeuta corporal. Eu não sei se nessa época
ele já era aluno do Gaiarsa, provavelmente sim. Ele descobriu a Ida
Rolf, foi fazer a formação em \emph{rolfing} nos Estados Unidos e trouxe
esse trabalho para o Brasil. Foi completamente pioneiro e faz esse
trabalho até hoje com o maior tesão. Hoje, o Pedro é um nome
internacional, corre o mundo dando cursos. Apesar disso, continua sendo
um cara muito simples. Só é difícil conseguir encontrar com ele entre
uma viagem e outra. A~gente se vê em geral uma vez por ano, a gente vai
jantar junto e põe a vida em dia. Uma vez é pouco…

Há muitos anos eu me submeti ao \emph{rolfing}. Aquele processo
mexeu muito comigo. Eu lembrei que eu desenhava muito. Eu desenhei uma
mulher sideral cheia de espirais em volta dela. Aquela mulher era eu.
Acho que foi um pouco depois de a Tuxa morrer, foi nos anos 80. Foi um
processo muito intenso. No final da última sessão, o Pedro e eu descemos
a escada rindo muito, a colega dele que estava embaixo achou que nós
``tínhamos puxado um fumo'', não era nada disso, era apenas o
\emph{rolfing} mesmo.

Em 1992 tive a pior depressão da minha vida, e o Pedro me ajudou muito.
Foi muito generoso. Ele foi comigo ao psiquiatra. Tinha uma feira no
caminho, e nós tivemos que largar o carro num estacionamento e correr a
pé. Estávamos em cima da hora. A~consulta era com o Dr. Jorge Figueira.
Eu estava tão deprimida que esqueci que o Del Porto existia. O~Pedro
ouviu pacientemente aquela consulta, deve ter sido chatérrimo para ele.
Teve um dia que ele deitou na minha cama, exausto de tanto trabalhar e
começou a me ajudar a fazer a mala para ir para o Pinhal. Ele dizia:

\begin{itemize}
\item
  Pega quatro camisetas;
\item
  Pega dois \emph{jeans};
\item
  Pega uma echarpe.
\end{itemize}
Ele tinha a maior paciência comigo, a maior disponibilidade.

Uma vez, nos anos 80, o Pedro me convidou para passar um feriado com ele
e um grupo de amigos dele em Mauá. Ele tinha alugado ou emprestado uma
casona que o Godoy tinha lá. Era um grupo grande. No meio do grupo tinha
a Bibi, uma chilena muito interessante, neurolinguista. Ela adorava
tomar \emph{ecstasy}. Um dia, quando estávamos no meio de umas pedras,
ela tirou vários comprimidos de \emph{ecstasy} do bolso e distribuiu.
Foi aquele astral! Aquele astral! O feriado foi muito gostoso. Numas
férias, eu fui passar uns dias com o Gaiarsa em São Sebastião, e o Pedro
foi nos encontrar lá. Nós tivemos uma convivência perfeita. O~Gaiarsa
fez de um barco emborcado, que havia no terreno vizinho, a sua mesa de
trabalho. Ele escrevia direto, horas a fio. Enquanto isso, o Pedro e eu
nadávamos quilômetros. Na hora de comer, ficávamos os três juntos e
conversávamos. O~Pedro conheceu o Gaiarsa muito antes que eu. Os dias
foram lindos, foi uma boa temporada essa. Só sei que foi nos anos 80, em
que ano eu não sei exatamente.

O que eu mais admiro no Pedro hoje em dia é que ele tem o chacra
cardíaco completamente aberto. Quando ele me abraça eu sinto um amor
imenso vindo desse chacra. Eu sinto a energia fisicamente. Um dia eu
perguntei como ele conseguiu isso, mas ele não soube responder. No
encontro seguinte, ele disse que tinha pensado no assunto e achava que
tinha aberto esse chacra trabalhando. Eu acho isso fantástico! Eu não
conheço mais ninguém que emane uma energia amorosa como o Pedro emana. É~muito bonito isso, muito.

Eu admiro muito o pioneirismo de Pedro em relação ao \emph{rolfing}, a
garra que ele teve para fazer o primeiro instituto de \emph{rolfing} no
Brasil. Eu não sei qual é o nome oficial. Eu sei que esse instituto deu
o maior trabalho para ele, muitas burocracias.

Eu admiro sua inteligência, sua intuição, sua dedicação ao trabalho, sua
sensibilidade.

O Pedro é uma pessoa muito preciosa na minha vida. Na última longa
depressão que eu tive, em 2008/2009, ele me ligou muitas vezes e tivemos
conversas compridas ao telefone. Ele foi paciente, muito paciente. Num
dia em que eu estava melhor, acho que em abril, saímos para jantar e
comemorar nossos 60 anos. Ele deu uma festa em janeiro, mas eu não
consegui ir. Estava muito insegura na minha depressão. Foi uma pena!

\begin{center}\textbf{\asterisc{}}\end{center}

\begin{flushright}\textbf{18/04/2010}\end{flushright}


Ontem fui almoçar com o Paulo Portella, no Nozuki, um restaurante
japonês em Higienópolis. Trocamos vários \emph{e"-mails} para marcar o
encontro, visto que o Paulo não usa seus telefones. Um saco! Isso é um
saco. Eu já estava quase desistindo, mas às 12h00 peguei o meu celular e
verifiquei que tinha um recado dele pedindo para ligar para o celular
dele. Finalmente deu certo, conseguimos nos entender. Arre!

Eu me arrumei bem bonita e fui encontrá"-lo. Estreei aquela bata indiana
que comprei na hipomania de fevereiro, perto do carnaval. Botei uma
camiseta por baixo, visto que a bata não está mais fechando!
Também, tomei 5,0mg de Zyprexa por um mês, está explicado. Vou fazer de
novo o regime dos Vigilantes do Peso, é o que mais deu certo comigo até
hoje. Agora estou tomando só 2,5mg de Zyprexa, acho que vai ficar mais
fácil, mais possível emagrecer.

Quando encontrei o Paulo, percebi que eu estava tensa, nervosa. Na
verdade há quase dois anos a gente não se via, a não ser uma vez quando
ele veio aqui no ano passado. Em julho ou agosto do ano passado, o
Claudio Cretti, arte"-educador que trabalha no Serviço Educativo do
Instituto Tomie Otake, me ligou convidando para eu dar meu depoimento
sobre o meu trabalho como arte"-educadora. Eles estavam chamando vários
profissionais da área para gravar um \versal{DVD} que seria encartado numa futura
publicação sobre a História da Arte"-Educação no Brasil.

Eu concordei e marquei um horário. Eu não consegui ir. Me ligaram de
novo e marquei um novo horário. Eu não consegui ir. Eu estava deprimida,
deprimida, deprimida. Eu estava muito insegura. Eu estava com muito
medo. \versal{EU} \versal{NÃO} \versal{CONSEGUIA} \versal{PENSAR}, \versal{EU} \versal{NÃO} \versal{CONSEGUIA} \versal{REVER} \versal{MEU} \versal{TRABALHO} E
\versal{FAZER} \versal{UMA} \versal{SÍNTESE}. O~\versal{TRABALHO} \versal{DE} \versal{UMA} \versal{VIDA}! O \versal{TRABALHO} \versal{MAIS} \versal{IMPORTANTE} \versal{DA}
\versal{MINHA} \versal{VIDA}! A depressão deixa a gente \versal{BURRA} e \versal{LENTA}. Eu já estava
deprimida há uns 10 meses. Experimentando um antidepressivo atrás do
outro e \versal{NADA}.

Foi o Paulo Portella quem me salvou. Ele me ligou para saber o que
estava acontecendo, e eu contei. Ele combinou de vir almoçar comigo
aqui. Eu nem acreditei, a coisa mais difícil é conseguir encontrar o
Paulo ao vivo e em cores. Ele é completamente obcecado pelo trabalho
dele. Ele trabalha, trabalha e trabalha! Sempre foi assim. Mas nos
momentos difíceis ele é como uma fênix que reaparece das cinzas, ele
está sempre presente.

Ele me explicou melhor o projeto de publicação do Instituto Tomie Otake
sobre o Seminário da História da Arte"-Educação no Brasil. O~\versal{DVD} com
depoimentos seria encartado na publicação. Ele insistiu muito para eu
participar, e eu acabei topando. A~gente deu muitas risadas durante o
almoço, relembrando passagens de nossas vidas, sobretudo da vida
profissional. Ele resolveu que o meu depoimento seria filmado aqui em
casa. Eu fiquei tão animada que até vesti o meu vestido colombiano com
suas ``molas'' bem coloridas. As acompanhantes e a Preta falaram que o
Paulo Portella devia vir sempre aqui porque eu fiquei outra
pessoa… Quem não fica outra pessoa com carinho e atenção? Calor
humano, calor humano, acho que foi isso que me fez falta nesses 18 meses
de depressão. Quem é que veio me visitar? A Sara, uma ou duas vezes; a
Bete, duas vezes, uma pelo meu aniversário; a Helena me convidou duas
vezes para ver duas exposições, e, graças a Deus, consegui ir. Isso é
pouco para quem ficou um ano e meio largada na cama. É~pouco, muito
pouco. Será que estou me vitimizando? Às vezes é difícil ter a ``medida
justa'', como o Paulo falava no Gurdjieff.

Eu conheci o Paulo num Curso sobre Teatro na Educação, com a Fanny
Abramovich, que eu montei no Vera Cruz, nos anos 70. Eu tinha muita
dificuldade nessa área, por isso resolvi produzir o curso. A~Fanny topou
na hora, mas duvidou que eu fosse conseguir montar, sua dúvida só me
estimulou. Eu adorava desafios.

Aluguei da ``Tia Iolanda'' as salas de aula necessárias para o curso.
Divulguei bastante, e rapidamente choveram inscrições. Afinal, na época,
a Fanny já era um puta nome em educação. Conheci a Fanny através da Peo,
no Vera Cruz da Frei Caneca, onde justamente ela estava dando um curso.
A~gente sempre se deu muito bem, e ela foi muito generosa comigo,
sobretudo nos meus primeiros anos como arte"-educadora. Vira e mexe eu
ligava para ela com alguma dúvida, e ela me orientava, ela estava sempre
disponível.

O curso durou 13 dias seguidos, foi um absoluto sucesso. Eu não lembro
se eram 30 ou 40 pessoas. A~cada dia uma surpresa, a cada dia a Fanny
vinha com uma proposta super estimulante. Estou lembrando agora dela,
aquela figurinha elétrica, sempre com um cigarro na mão, falando e
falando. Teve um dia em que ela mandou a gente se organizar em grupos,
conforme os elementos dos signos. O~grupo do ar fez um cabeleireiro, o
da terra, um sisudo jogo de xadrez, os outros eu não lembro. Teve outro
dia em que ela pediu para cada um escolher a pessoa com a qual teria
mais dificuldade de trabalhar. Foi uma proposta reveladora, algumas
pessoas se surpreenderam muito ao serem escolhidas. A~Fanny não tinha
medo, ela botava para quebrar. Mas tinha um domínio invisível do grupo.
O~curso foi maravilhoso e inesquecível, deu super certo.

O curso rendeu frutos. Um tempo depois, o Paulo Portella, que fez o
curso, me ligou e me convidou para trabalhar na Pinacoteca, e eu fui.
Estava ganhando um amigo inestimável e um ótimo companheiro de trabalho.
O~Paulo coordenava os cursos que a Pinacoteca oferecia num setor chamado
Laboratório de Desenho. Depois que eu entrei, passou a se chamar
Oficinas de Arte, era para o público infantil e adulto.

Foi muito interessante ir trabalhar no Jardim da Luz, naquele palácio
que é a Pinacoteca, e conhecer uma população diferente daquela do Vera
Cruz. Tudo era novo, tudo era estimulante. Em geral, eu saía no fim das
aulas com a criançada para brincar no Jardim da Luz. Às vezes, aquela
polícia montada assustava um pouco a gente, mas era mais que necessária.
As salas de aula, pra variar, eram no subsolo. É~sempre assim. Mas eram
salas muito amplas, muito boas, com umas mesonas enormes para a
criançada trabalhar. As janelas davam para aquele pátio interno da
Pinacoteca, eram uns janelões enormes, lindos. Trabalhei lá durante oito
anos.

\begin{flushright}\textbf{19/04/2010}\end{flushright}


Ontem acordei às 7h00 da manhã e trabalhei das 7h00 às 9h00. O~meu tempo
de concentração está aumentando. Eu escrevi, escrevi e escrevi. Lá pelas
9h30 eu estava cansada, acho que escrevo com muita intensidade. Comi um
sanduíche de presunto e queijo e fui deitar. Acordei ao meio"-dia e
atendi a um telefonema do Rogério. Foi a única pessoa que ligou…
Eu escrevi de novo e depois abri e enviei \emph{e"-mails}. Estou me
esforçando para resgatar os amigos antigos, mas não é fácil! Estou muito
chateada com isso!

Lá pelas 18h30 tomei um banho e me arrumei para ir ao aniversário do
Rogério. Foi uma reunião pequena na casa do Alexandre, filho dele. Só
estavam o Rogério, filhos, noras, genro e netos, e também o Sylvio, que
disse que percebeu três coisas na nossa educação:

--- que somos muito bem"-educados,

--- que aprendemos a ser generosos

- e que aprendemos a ter garra.

Tudo isso graças ao papai e à mamãe. Eu concordo plenamente com ele.

Será que às vezes a gente ``se acha''?

\begin{center}\asterisc{}​\end{center}

\begin{flushright}\textbf{}\end{flushright}

\begin{flushright}\textbf{20/04/2010}\end{flushright}


Ontem fiz uma linda instalação no móvel de entrada com os caramujos
gigantes que eu tenho e as ``esponjas'' de cerâmica do Antônio. Eu
fiquei muito emocionada com esse trabalho dele, que me lembrou o fundo
do mar da minha infância. Uma das melhores lembranças da minha infância
são os mergulhos em São Sebastião para contemplar o fundo do mar. Eu
ficava fascinada com aquele mundo deslumbrante de cores e texturas,
diferentes peixes nadando, estrelas"-do"-mar vermelhas pousadas sobre
pedras, actínias se mexendo e até as feias holotúrias. Tinha também os
ouriços"-negros, que eu achava lindos e pegava com cuidado para não me
machucar. Tinha também as pedras imensas recobertas de um tecido
esponjoso, às vezes laranja, às vezes verde"-claro, quase branco. Tinha
as algas lindas, balouçantes, e tinha as cracas grudadas nas pedras,
formando desenhos incríveis. Tinha aqueles peixinhos minúsculos,
listradinhos, que passeavam de lá para cá. Tinha também os peixes
maiores, inúmeros em sua diversidade, em cores, tamanhos e formatos. Eu
ficava horas mergulhando, me deleitando com toda essa maravilha. Acho
que o fundo do mar foi uma das imagens visuais mais importantes da minha
infância. Fora do mar tinha a prainha, recoberta de pedras cinzas,
lisas, maravilhosas. A~gente ficava horas lá catando pedras para levar
para São Paulo. Até hoje tenho algumas dessas pedras aqui em casa. Elas
são muito lisas e têm formas muito interessantes. Em geral, elas são
cinza"-chumbo, parecem até um objeto japonês. Eu agradeço ao Dr. Sawaya
ter me apresentado toda essa maravilha! Maravilha que me alimenta até
hoje!

Ultimamente eu tenho lembrando muito da mania de 1998. Foi uma mania
séria. Foi a minha primeira internação. Eu lembro vagamente do início
dela. O~Rogério foi me visitar um dia, e eu adorei a surpresa. Ele
entrou, perguntou se estava tudo bem, e eu contei para ele que tinha
acabado de jogar pela janela uma estatueta. Eu morava no quinto andar. A~estatueta, que tinha mais ou menos um palmo na horizontal por uns 8cm na
vertical, era uma mocinha lânguida recostada num veadinho. Era linda! A
Louise Weiss que tinha me dado de presente. Eu poderia ter machucado
alguém ao jogá"-la. Eu já estava fora de mim. Fora da realidade.
Expliquei para o Rogério que jogar aquela estatueta significava,
provavelmente, eu me libertar do lado passivo do meu feminino. Alguma
coisa por aí, eu me lembro bem. Ele achou ótimo! Notou algumas marcas no
chão. Expliquei que tinha caído cerveja, mas que eu não tinha limpado.
Para mim, aquelas marcas também eram símbolos de alguma coisa. A~mania
de 98, não sei por que, foi toda simbólica. E~olhe que eu fazia
psicanálise na época e não terapia junguiana. Mas eu também fazia
meditação com a Ciça. O~Rogério foi embora e eu me debrucei na janela
para vê"-lo passar lá embaixo. Quando ele passou, recolheu um caquinho da
estatueta e botou no bolso! Ele viajou na minha viagem. Depois de
acabada a mania, eu fiquei sabendo que foi a Soninha que pediu para o
Rogério ir me ver. Ela soube por alguma pessoa que eu não estava muito
bem e achou melhor conferir. Às vezes a família tenta ajudar, tenta
brecar uma crise no seu início, mas as circunstâncias não favoreciam.

Nessa época, eu e o Rogério recebemos do Sylvio a casa de papai e mamãe,
em Cotia, que nos coube como herança. O~Sylvio tinha morado lá desde
setembro de 1995, quando papai morreu. Houve uma reunião entre nós três
em Cotia, quando combinamos tudo o que tinha que ser combinado. Chamamos
o João, caseiro, e o Sylvio me apresentou a ele e falou que eu passava a
ser a responsável pela casa e pelo jardim.

Acho que era muita areia para o meu caminhãozinho. As lembranças dessa
fase são confusas. O~que mais lembro é que fiquei muito feliz em ter a
casa e, para inaugurar, organizei um churrasco com poucos amigos.
Estavam a Aninha e o Juan, a Francesa, a Germana, o Rogério e a Rita,
sua namorada. O~Juan, argentino, fez um churrasco delicioso. Eu estava
tão feliz que até dancei com o Rogério naquela salona vazia. Depois
conversamos horas e horas no terraço. Rogério, dando aula, pra
variar… Lembro que ele falou muito sobre Jung. Depois disso a
Francesa até emprestou um livro para ele. Nesse dia eu ainda estava
bem, mas depois…

Eu só lembro que fiquei muito ligada no jardim de Cotia. Eu passava
naquelas chacrinhas perto da casa da Soninha e comprava um monte de
mudas. A~Renata, minha amiga que mora perto da casa da mamãe e do papai,
me arrumou um jardineiro. Ela também me arrumou um marceneiro, e eu
encomendei uma cama de casal bem rústica. Eu passei por uma loja de
móveis na Bela Cintra e comprei um sofá novo para o apartamento da rua
Pernambuco, era um antigo sonho. O~sofá eu mandei para Cotia. O~motorista que trouxe o sofá novo levou o velho para Cotia. Eu estava
super eficiente! Reparei que tinha uma loja de móveis de vime do
lado direito da Raposo Tavares. Fui lá e comprei uma mesa com cadeiras
para a salinha de almoço e umas duas cadeiras confortáveis para a sala.
Encomendei móveis para o lado da casa que ficou com o Rogério, queria
fazer uma surpresa para ele.

Eu andava na maior vula na estrada para Cotia. Acho que só não tive um
acidente porque, sem falsa modéstia, eu dirijo muito bem. Teve um dia em
que fiquei fascinada e hipnotizada por um motoqueiro que andava na minha
frente. Para mim, ele parecia um anjo moderno, caído do céu, e eu fui
indo atrás dele. Por sorte não aconteceu nada.

Teve um dia em que eu fui até o Shopping Morumbi. Vi, no canteiro
central de uma avenida, um pessoal vendendo móveis rústicos. Parei o
carro e fui conversar com eles. Falamos sobre religião --- provavelmente
eram evangélicos -- a conversa acabou com um deles me abençoando e eu o
abençoando. Eu me arrisquei muito, eu me arrisquei muito.

A \versal{MANIA} É \versal{UM} \versal{RISCO} \versal{TOTAL}, \versal{MUITAS} \versal{VEZES} A \versal{MANIA} É \versal{UM} \versal{RISCO} \versal{DE} \versal{VIDA}, \versal{DE}
\versal{MORTE}, \versal{DE} \versal{VIDA}.

O João, caseiro, nunca me obedecia. Um dia fiquei irritada, botei ele no
carro e dei mil voltas com ele, não sei como, por estradinhas paralelas
e perpendiculares à Raposo Tavares, cheguei na avenida Marginal. Voltei
para Cotia e larguei ele numa estação de ônibus. Ele ligou para a Peo,
ex"-mulher do Sylvio, que mora na casa ao lado da casa dos meus pais; ele
pediu para ela ir buscá"-lo porque não tinha dinheiro para a condução. A~Peo veio me perguntar por que eu fiz aquilo. Não lembro o que eu
respondi.

O João nunca abria o portão de arame farpado que dava acesso à estrada
que o Sylvio fez para a gente chegar lá embaixo, num terreno que estava
à venda. Eu me enchi e um dia acelerei o carro, fui em frente e abri o
portão na marra. Quando cheguei lá embaixo, perto da mata, vi um monte
de figurinhas pretas se erguendo como se fossem sacis. Uma alucinação
visual, com certeza, penso hoje. Mas hoje é hoje. Aquele momento era
aquele momento, e então eu tirei um monte de Coca"-Cola e saquinhos de
salgadinhos que tinha no carro e joguei no terreno. Era tudo simbólico,
era um ritual de posse. Eu me lembro que eu comprava pacotinhos de
salgadinhos em função da cor, o vermelho significava isso, o amarelo,
aquilo, e assim por diante. O~meu carro ficou atolado. Eu estava
exausta, mas tive que subir o morro a pé. Entrei dentro de casa e dormi.
No meu quarto eu tinha criado um altar para o Rudi, guru da meditação.
Em cima de uma mesinha coloquei um tecido com uma faixa laranja e outra
cor de rosa forte -- eu achava que essas eram as cores do Rudi. O~tecido
era muito bonito. Eu adorava. Em cima coloquei um porta"-retrato com a
fotografia do Rudi e um porta"-incenso. Provavelmente tinha uma vela. A~minha relação com a meditação era bastante intensa naquele momento. Azar
meu…

No dia seguinte, eu acordei e o Sylvio estava lá no terraço. Não sei se
foi bem assim, esse pedaço está confuso para mim. Enfim, teve um momento
em que encontrei o Sylvio no terraço. Ele disse que já tinham tirado o
meu carro lá de baixo e que, quando entrou com o dele, também atolou.
Ele foi embora logo, mas voltou depois. Eu lembro de uma cena aterradora
com ele. Nós estávamos no meu carro, estacionado em frente à casa da
Peo. A~casa dela era vizinha da dos meus pais, mas cada uma tinha a sua
entrada, sua estradinha. As duas casas eram unidas por um grande gramado
em declives, e, quando a gente estava em uma delas, dava para enxergar a
outra. O~Sylvio queria descer do carro e entrar na casa da Peo. Eu não
queria de jeito nenhum que ele fizesse isso. Ele teimava, teimava e
teimava em ir. Eu não queira que ele fosse porque achava que a Peo
implicava com todas as namoradas dele. Eu achava que ela não queria que
o Sylvio casasse de novo. Mais uma vez era simbólico. Para segurar o
Sylvio, eu mordi a mão dele até sangrar. A~Peo veio ver o que estava
acontecendo. Quando chegou perto do carro, eu xinguei, berrei, falei um
monte de impropérios. Falei, é claro, que eu achava que ela não deixava
o Sylvio casar de novo. O~que é que eu tinha de me meter na vida do
Sylvio, naquelas alturas um marmanjo com seus 55 anos ou mais? Neura,
neura, só pode ser neura. A~Peo já tinha percebido antes que eu não
estava bem e tinha ligado para o Rogério perguntando se eu continuava a
tomar lítio. Ele me contou isso depois e considerou uma invasão a
atitude dela, mas, na verdade, ela só queria ajudar. Depois de eu voltar
da clínica, ela me contou que me viu muitas vezes dançando, fazendo
``movimentos rituais'' e ficou preocupada.

Durante a mania, fui ao condomínio vizinho algumas vezes, onde fazia
``manobras rituais'' na entrada da garagem de várias casas. As manobras
eram sempre iguais, repetidas, e faziam um desenho assim: Y\. Eu não
tenho a menor ideia do que significava isso, mas eu tinha que fazer, era
uma necessidade interna. Parece até que alguma entidade me obrigava a
fazer isso.

Eu não sei quantos dias fiquei em mania. Talvez 15, talvez 20. Num
domingo, eu estava muito feliz lá em Cotia, quando vi chegarem o Fernão
e a Soninha de óculos escuros e com cara de enterro. Achei que tinha
acontecido alguma coisa grave com alguém da família. Um acidente ou
coisa assim. Não desconfiei que o problema era comigo.

O Duda veio conversar comigo com aquela calma que lhe é peculiar. Ele é
médico. Caminhamos pelo jardim. Ele falava que eu não estava bem, e eu
respondia que estava ótima, que tinha feito isso, mais aquilo etc. Ele
teve a maior paciência. Aos poucos fui me acalmando, fui me assentando.

Chegamos ao terraço onde a Soninha e o Fernão nos esperavam. Eles me
falaram que eu precisava de ajuda psiquiátrica. Eu aceitei e disse que
meu psiquiatra estava viajando. Esqueci completamente do assistente
dele, que eu já tinha conhecido. Lembrei que tinha o número do bip do
meu psiquiatra anterior, o Dr. Jorge Figueira. Colaborei. Graças a Deus,
eu colaborei.

O Dr. Jorge Figueira foi maravilhoso e nos atendeu em pleno domingo,
final de tarde. Lembro que fui de Cotia para São Paulo no jipe do
Sylvio. Ainda tive uma fantasia de fuga para um hotel, mas logo desisti.

Na consulta, eu partia os cigarros em dois e ia amontoando sobre a mesa.
Isso pesou muito no diagnóstico. Eu não me lembro bem como foi. Se o Dr.
Jorge Figueira conversou primeiro comigo sozinha e depois comigo, com a
Soninha e com o Sylvio. O~fato é que ele decidiu pela internação. Ele disse
que eu não precisaria ser internada se tivesse a presença constante da
família e possivelmente um acompanhante terapêutico. A~Soninha falou que
não teria esse tempo e perguntou para o Sylvio se ele teria. Agora eu
estou lembrando, eu estava junto com eles. É~óbvio que me senti mais
rejeitada do que nunca. Ser internada porque meus irmãos não tinham tempo
para cuidar de mim! Claro que a questão não era tempo, era medo de
entrar em contato com a minha loucura. Hoje eu acho que isso tem de ser
respeitado, limite pessoal é limite pessoal. Mas, naquele momento, eu
sofri. Eu sempre tive horror a ser internada, eu me senti muito
rejeitada, uma leprosa.

A angústia e a ansiedade dos meus irmãos eram enormes naquele momento, e
eu absorvia tudo. Eu sou mesmo uma esponja. O~Rogério tinha chegado
atrasado. Talvez, se ele tivesse feito uma dobradinha com o Sylvio, como
em 87, a internação teria sido evitada. A~Soninha sempre foi a favor da
internação, em 87 ela já quis me internar, mas o Sylvio e o Rogério não
deixaram e assumiram a responsabilidade por mim. Eu me recuperei bem e
tranquilamente em casa, com a ajuda da Maria, uma empregada maravilhosa
que o Sylvio me arranjou.

O caminho para a clínica foi um suplício, aquela angústia dos irmãos
pesou muito. E~a minha também. Já que era para ser internada, queria
logo acabar com aquilo. Com aquele inferno.

Quando chegamos à clínica, teve uma reunião com o diretor. Quando
acabou, ele disse para eu entrar. Soninha perguntou se eu queria que ela
me acompanhasse, e eu respondi:

--- Em certos lugares a gente entra sozinha. 

E~dei tchau.

Entrei e vi aquela sala cheia de gente vendo televisão. Um rapaz me
chamou a atenção, ele parecia muito calmo, equilibrado, não combinava
com aquele contexto. Eu fui até um balcão me servir de suco de uva e
comer umas bolachas. A~enfermeira veio com um copinho de remédio e não
me deixou comer. Eu joguei o suco de uva nela e dei um berro muito alto.
O~diretor da clínica veio ver o que estava acontecendo e mandou ela
deixar eu comer antes de tomar o remédio. Abuso de poder de pequena
autoridade, diria Gaiarsa. De certa forma, ela foi humilhada na frente
de todos os pacientes. Era alta, magra, negra, uma pessoa muito forte.
Fiquei completamente paranoica em relação a ela, achava que ela não ia
me deixar sair da clínica, que ia fazer uma macumba contra mim.

A clínica era melhor do que eu imaginava, fora as grades ostensivas,
muito ostensivas. Eles ofereciam várias atividades, tinha trabalho
corporal, tinha terapia em grupo, tinha uma oficina de arte, lá fiz um
pássaro grande de argila. O~pássaro é um assunto recorrente no meu
trabalho. Fiquei amiga do Roberto, o rapaz que me chamou a atenção no
primeiro dia. Demos uma namorada. As enfermeiras não deixavam a gente
ficar de mãos dadas.

Na clínica tinha um enfermeiro chamado Jamilson. Um dia, ele chegou
perto de mim e disse:

--- O seu problema é do espiritual. Você não devia estar se tratando
aqui; você deveria estar se tratando em outro lugar. Não diga a ninguém
que eu falei isso, nós somos proibidos de ter esse tipo de contato com
os pacientes. Eu posso ser demitido por isso.

Eu entendi o que ele quis dizer, eu estava ``trabalhada'', toda aquela
loucura pela qual passei foi causada por um ``trabalho'' de macumba ou
de candomblé.

Eu fui internada num domingo e, no sábado seguinte, tinha a tarde livre.
Pedi para o Rogério me levar ao Templo Cabocla Guaciara, o templo da
Dagui. Ao chegarmos lá parecia que todo mundo já sabia que íamos chegar.
Um rapaz falou: ``Você é a Regina? Estaciona ali'', disse para o
Rogério.

A gira era uma gira de preto velho comandada pelo preto velho do Saulo.
Entrei na fila para receber o passe. Quando chegou a minha vez, o preto
velho mandou eu me ajoelhar e pôs as duas mãos sobre a minha cabeça. Meu
corpo estremeceu todo. Estou livre do ``trabalho'', eu pensei. Depois
ele mandou eu segurar no bastão dele. Ele falou que era para eu levar
umas bananas para oferecer para meus inimigos na clínica.

Rogério e eu voltamos para São Paulo e paramos numa banca de frutas da
avenida Sumaré, onde compramos as bananas. Aproveitei e tomei uma água
de coco que ofereci à mãe Severina, o espírito de uma baiana velhinha
que fica perto de mim e me protege. Ofereci as bananas na clínica e a
enfermeira negra comeu. Fiquei mais sossegada. O~Jamilson, quando me
viu, assentiu com a cabeça, ele percebeu que eu tinha me tratado
espiritualmente, que tinha resolvido o problema.

Quando o Ernesto e o Milan ligaram para lá, eu fiquei super feliz. Eu
contei para eles que de vez em quando eu cuspia os comprimidos, e eles
falaram para eu não cuspir, claro! O Dr. Jorge Figueira foi me visitar
umas duas vezes, eu tentava me controlar e parecer bem. Finalmente tive
alta. No dia em que saí, o Santos, motorista da Soninha, não podia ir me
buscar. Eu saí sozinha da clínica e corri até a esquina para pegar um
táxi, eu estava morrendo de medo que alguém da clínica viesse me puxar e
levar de volta para lá. Quando entrei no táxi foi um alívio enorme.

Eu não sabia, mas continuava em mania. Fui liberada ainda em mania!
Comprei milhões de coisas para a casa de Cotia, louças, talheres,
panelas. Eu tinha tanta louça que até dei um aparelho completo para
uma amiga minha. Eu continuava sonhando em usufruir um pouco da natureza
naquela casa, em convidar os amigos para um churrasco ou para um simples
almoço de domingo.

Gastei muito dinheiro, gastei R\$30.000,00. Passada a mania, eu
perguntava para o Rogério como ele tinha deixado eu gastar a entrada de
um apartamento, e ele respondia:

--- Ninguém te segurava naquela época, ninguém te segurava.

Com a medicação, a mania passou. Lembro que eu tomava Stelazine, um
remédio que deixa a gente andando de passinho. Veio a depressão, meu
ciclo é sempre assim. Eu falei para o Ernesto, meu analista, que me
sentia como um vaso estilhaçado no chão, apenas cacos e mais cacos
espalhados. Começamos pacientemente a juntar os cacos. O~Ernesto é ótimo
nisso. Ele é chinês, tem uma paciência enorme.

Acho que foi em 99 que, graças a uma enorme insistência do Ernesto,
comecei a ter aulas de pintura com o Sérgio Fingermann. Eu me sentia uma
E.T\ naquele aquário de madames. Só tinha uma moça alternativa, mas ela
não era muito dada. As madames trabalhavam muito, e seus trabalhos
cresciam. Eu me sentia um E.T\, porque só pintava paisagem, todo mundo
fazia pintura abstrata. Um dia uma delas me disse: ``Estou tão feliz por
estar aprendendo pintura contemporânea!'' Eu pensei: ``Que absurdo!''.

Eu quase morria para estar lá às 9h00 da manhã. Eu dormia mais cedo e
provavelmente acordava às 7h00, já que demoro para me arrumar, tomar
café etc.

Muitas vezes eu chegava atrasada e de vez em quando eu faltava. O~Sérgio
começou a pregar a minha lona para mim. Quando eu chegava já estava
pregando o meu trabalho. Achei muita delicadeza da parte dele e fui
conversar com ele. Ele disse que eu não era a única aluna que tomava
remédio, a Benê também. Fiquei chateada por ele saber que eu tomava
remédio, a Sara Carone, amiga em comum, deve ter contado. De qualquer
forma, ele foi gentil, percebeu a dificuldade que eu tinha em fixar
aquela lona e resolveu ajudar.

Eu não sei ao certo quanto tempo frequentei o ateliê dele, um ano
talvez, um ano e meio. Eu gostava de pintar com cores fortes, sempre
pintei com cores fortes. Ele insistia para que eu usasse uma paleta
pastel. Eu não sei, os trabalhos ficaram bonitos, mas eu, insegura, não
conseguia me opor a ele. Não conseguia me afirmar.

Não conseguia usar as cores do jeito que eu gostava. Eu não estava bem
nessa época. Estava deprimida. Não conseguia pintar em casa. Sentia um
grande vazio, um vazio enorme dentro de mim e também uma solidão enorme,
uma tristeza. É~impressionante o vazio da depressão. Ele é
característico da depressão, faz parte da depressão, vem junto com ela,
vem lá do fundo e toma conta da sua vida.

O Sérgio não tinha o menor saco de ver os trabalhos antigos das alunas,
aqueles que fizeram antes de entrar lá. Eu, como educadora, acho isso o
fim da picada. Um dia até levei uns quatro ou cinco trabalhos, mas era
tão pouca a receptividade que acabei não mostrando. Apesar de tudo, ele
era um ótimo professor, acompanhando com muita paciência e atenção o
processo individual de cada aluno. No início da aula ele mostrava e
comentava o trabalho de algum pintor importante. Naquela época eu tinha
facilidade em imitar o que ele tinha mostrado.

Muitas vezes, porém, eu parava o meu trabalho, olhava os trabalhos de
minhas colegas e achava tudo muito esquisito… Teve um dia em que,
quase sem perceber, comecei a fazer pintura abstrata, eu só conseguia
fazer na aula. Fiz por um tempo e resolvi parar de frequentar as aulas.
Até hoje não sei bem por quê. Em muitos momentos eu me arrependi dessa
decisão, acho que principalmente porque fiquei cinco anos sem pintar
nada. Outro dia o Celso me disse que naquela época eu pensei muito
antes de sair do ateliê do Sérgio. Pesei os prós e os contras. Fiquei
muito aliviada quando ele disse isso.

Passei então a botar minha energia na cerâmica e frequentei o ateliê da
Sara Carone, uma grande ceramista, fiz até uma exposição individual na
Galeria da Aliança Francesa, no Jardim América, em 2003. A~exposição se
chamava ``Pássaros, paisagens e pedras''. Foi a Helena que me propiciou
o encontro com o Pierre Clement, diretor da Aliança Francesa. A~essa
altura, já fazia sete anos que eu produzia minhas pequenas esculturas em
cerâmica. Quando a montagem da exposição ficou pronta, não resisti e
pedi para o Jardim vir olhar. Ele gostou muito do trabalho. Não esqueço
dele pegando um pássaro na mão, olhando para mim e dizendo: ``Isso é
arte, não é artesanato''. No \emph{vernissage} foi muita gente, numa
bonita manhã de sol. Eu fiquei espantada ao ver as pessoas tão
interessadas em comprar os meus trabalhos. Fiz uma boa venda. Depois da
exposição continuei por algum tempo a frequentar o ateliê da Sara. Ela é
uma professora muito bacana porque dá liberdade total aos alunos e os
orienta a partir de sua expressão pessoal, ela não impõe um modelo. A~Sara também é uma grande artista, já expôs quatro vezes no Japão.

Quando você tem um professor, você tem que ceder, a não ser que ele seja
o Evandro Jardim, que acolhe e aceita cada aluno na sua individualidade.
Meu sonho é ter um professor de pintura igual ao Jardim. Mas não existe.
Ele só dá aula de gravura. Agora sinto que a pintura está latente em mim
e vou pintar sem professor, mesmo porque não tenho mais dinheiro para
pagar um curso. E, antes de mais nada, tenho que conseguir fazer a lição
de casa do curso do Jardim. Tenho que aproveitar melhor o curso.
Desenhar, desenhar, desenhar.

Hoje passei a tarde com o Gaiarsa. Anteontem tive um impulso e liguei
para ele. Quando eu falei quem era, ele disse que eu tinha sido um tempo
de alegria na vida dele, que nós éramos crianças.

Antes de ir, eu estava meio apreensiva, acho que havia uns quatro anos
que não me encontrava com ele. Me atrasei um pouco porque fiz questão de
levar um vaso lindo de flores. Ele adora flores.

Aos seus 90 anos, ele estava inteiraço, fiquei impressionada logo ao
vê"-lo. Pedi um copo de água e fui até a cozinha com ele. Eu me lembrava
muito bem do apartamento. Na sala, ele fez um altar em dois níveis para
uma deusa, que é a mulher. No nível mais alto, colocou uma escultura
muito bonita de uma mulher em pé e, atrás, uma escultura de uma criança.
No nível mais baixo tem uma escultura de uma mulher meio primitiva,
fazendo uma reverência, e tem um barco pequeno, simples, com vários
talismãs. Acima de tudo isso, indo de uma parede a outra, tem um fio de
onde pende um aviãozinho. Esse altar se localizava num ângulo entre duas
paredes. Como o Gaiarsa atendeu um telefonema meio comprido, eu pude
observar bem. O~Gaiarsa sempre teve uma criança muito viva dentro dele,
em parte por isso é que nos demos tão bem. Hoje mesmo ele falou que cada vez
mais ele é um moleque. Ele é fantástico, ele é mesmo fantástico.

Nós passamos a tarde juntos, foi muito bom, foi alimentador, foi
apaziguador. Relembramos algumas coisas do nosso passado, foi gostoso.
Eu falei também de alguns dos meus problemas atuais, e aí virou terapia.
Teve uma hora que eu o chamei de Del Porto… Ele disse que hoje em
dia tem horror de gente, disse que só se dá com umas sete ou oito
pessoas porque as outras são todas previsíveis, fazem o mesmo discurso
sempre. É~verdade.

Nos anos 80 fui fazer um curso de Arte"-terapia com a Carmen Levy. Ela
resolveu apresentar os alunos para profissionais de diferentes linhas:
psicanálise, \emph{Gestalt}, Jung, Reich. Foi aí que eu conheci o José
Ângelo Gaiarsa. Ele nos falou de Reich e da terapia corporal. Eu fiquei
impressionadíssima com ele, com seu entusiasmo, com sua vibração pela
terapia corporal, pelo que ele fazia. Ele parecia uma taça de champanhe
borbulhante! Resolvi fazer o curso dele só para estar perto daquela
energia toda, daquele cara que era entusiasmo puro. Só trabalhar no Vera
Cruz estava muito chato.

É claro que, conforme fui fazendo o curso, fui me apaixonando por este
ser antissistema, iconoclasta e anárquico que é o Gaiarsa. E~também fui
aprendendo propriocepção, respiração, leitura corporal. O~que eu mais
gostei de aprender foi de fazer leitura corporal. Esse instrumento
passou a me ajudar muito no meu trabalho como professora de Arte no Vera
Cruz e na Pinacoteca. Naquela época, um novo mundo se descortinou para
mim através da leitura corporal. Passei a conhecer muito melhor as
pessoas, fiquei até um pouco fanática. Depois o fanatismo passou, mas
até hoje, em situações difíceis, eu uso a leitura corporal
instintivamente.

Acho que já fazia o curso há um ano quando um dia fui pedir umas
apostilas para o Gaiarsa. Ele disse que não tinha trazido, mas que
morava perto, falou que eu poderia acompanhá"-lo até a casa dele para
pegar. Ele deu a deixa e eu topei. Fiquei dois anos buscando as
apostilas… Foi maravilhoso. Foi o melhor namoro da minha vida!

A nossa vida era simples. O~Gaiarsa era apaixonado pelo trabalho dele e
eu pelo meu. Isso dava equilíbrio à relação. Nós dois trabalhávamos com
gente, conversávamos muito sobre isso. Passávamos muitos finais de
semana no apartamento dele, era alimentador. A~questão comida era com
ele mesmo, já que sou uma nulidade. Ele fazia saladas e sopas
deliciosas. Lembro de um dia em que fomos à feira do Pacaembu, era um
sábado. O~Gaiarsa brincava com os feirantes e fez o maior sucesso com
eles, que queriam ficar com o boné do Gaiarsa. Quando voltamos, fizemos
uma sopona natureba, daquela em que se colocam os caules e as folhas dos
vegetais.

Às vezes íamos a um restaurante que ele gostava, acho que era na alameda
Casa Branca, onde serviam uns grelhados deliciosos. Teve um dia que eu
quis ir a um \emph{show} da Rita Lee, no Ginásio do Ibirapuera. Convidei
o Gaiarsa e, para meu pasmo, ele topou. Teve algumas vezes que fui a
festas, que naquela época eram muitas, todo sábado tinha festa. O~Gaiarsa ficava me esperando, deitado no sofá. Agora percebo que isso não
era necessário, afinal, eu tinha a minha casa, podia voltar para lá. Nós
não tínhamos a chave da casa um do outro. Não sei quem exigiu essa
combinação, mas deu muito certo. Era uma garantia de privacidade. Eu sei
que eu prezo muito isso. Talvez tenha partido de mim.

O Gaiarsa dizia que nunca tinha visto alguém gostar tanto de ser
agradada como eu. Também com aquela qualidade de agrado! O
toque do Gaiarsa era especial, era macio, era gostoso, era muito
masculino, mas também era feminino. Eu não sei explicar direito, é
difícil falar verbalmente a respeito de sensações.

Nossa relação durou dois anos. No final do segundo ano, apareceu a Rose.
Eu já sabia, é claro, que o Gaiarsa era polígamo, ele apregoava isso
publicamente. Ele não me disse nada. Acho que foi muita areia para o meu
caminhãozinho. Resolvi ir passar um mês na Bahia, em Salvador. Eu estava
muito ligada no Gaiarsa e vira e mexe ligava para ele a cobrar, de algum
orelhão. Pensei, pensei, pensei e achei que o melhor era dar um fim
naquela relação. Tinha a poligamia do Gaiarsa personificada na Rose,
naquele momento, e tinha a diferença de idade, 28 anos, comecei a sentir
vontade de ter um companheiro da minha geração. Quando voltei para São
Paulo o Gaiarsa estava muito interessado em mim. Ele insistiu para
ficarmos juntos. Ele dizia que não tinha em São Paulo nenhum homem mais
interessado em casar comigo do que ele. Mas eu fiquei firme. Não topei
mais.

Se eu soubesse o desastre que seria o meu próximo relacionamento
amoroso, com o Sílvio, com certeza eu teria ficado com o Gaiarsa. Mas a
gente nunca sabe o que está arriscando, e não dá para prever o futuro.
Mas, com certeza, eu não suportaria a poligamia do Gaiarsa. Eu iria me
machucar muito. E~essa bandeira, há muitos anos, ele defendia com ardor
e com verdade.

De qualquer maneira, esses dois anos de relacionamento com o Gaiarsa
foram muito ricos e proveitosos para mim. Ficou uma boa lembrança. Claro
que eu devia ter queixas e críticas em relação a ele, mas com o tempo
elas desapareceram, ficou só o bom. Acho que é um bom sinal. Um ótimo
sinal.

\begin{center}\asterisc{}​\end{center}

\begin{flushright}\textbf{02/05/2010}\end{flushright}


Em agosto de 1973 comecei a fazer um estágio na Oficina de Arte do Nível
I do Vera Cruz, o Verinha, como era chamado. Até então eu contava com
meu estágio na Escola de Arte de São Paulo, da Ana Mae, e com o curso de
arte para crianças que dei com a Cristina, na \versal{ACM} da Vila Mariana, no
final dos anos 60. Nós fizemos um pequeno folheto para o curso, que era
escrito em ``vermelho mercúrio"-cromo'' sobre branco. O~texto terminava
assim: ``Nós queremos que as crianças sejam felizes''. Pensamento com o
qual concordo até hoje. Eu contava com a experiência de aluna da Escola
de Arte Brasil: lá, a liberdade era considerada como algo fundamental no
processo do artista. O~catálogo deles, Nasser, Fajardo, Rezende,
Baravelli, era maravilhoso, colocava isso claramente.

Ao começar a observar as crianças na oficina, fiquei fascinada. Elas
vinham organizadas em turmas de cinco, seis, sete anos, cada uma na sua
vez. Cada criança escolhia o material preferido e se punha a trabalhar.
Os materiais oferecidos eram argila, madeira, serrotes, martelos e
pregos, os papéis, as tintas e os pincéis, o material gráfico, o lápis
de cor e as canetinhas, as sucatas variadas.

As crianças trabalhavam na maior espontaneidade, com muito prazer e
alegria. Algumas vinham já com um projeto pronto:

--- Hoje vou fazer um avião!

--- Vou fazer uma tartaruga!

--- Vou fazer um robô!

Outras se mobilizavam para o trabalho a partir do material disponível,
do colorido, do que as atraía. Elas trabalhavam com independência e
envolvimento, com concentração. Às vezes o professor interferia para
dar alguma ajuda que se tornava necessária, outras vezes, para ensinar a
técnica de algum material.

A Peo, minha cunhada na época, era a diretora pedagógica da escola e
orientava a oficina. Com ela aprendi a importância da liberdade no fazer
criativo. A~importância de acolher a manifestação individual de cada
criança. A~importância de respeitar o ritmo próprio de cada criança. Com
ela aprendi que o fazer na oficina é um grande ``brincar'', que a
criança aprende e se desenvolve.

Fiz disso os meus princípios de trabalho, que apliquei por dez anos como
professora de Arte no \versal{NÍVEL} \versal{II} do Vera Cruz, por oito anos na Pinacoteca
do Estado, por um ano e meio no Clube da Turma no Parque Ecológico do
Tietê, e por sete anos no Segall.

Muitas vezes refleti com a Peo sobre o meu trabalho. Nós ficamos amigas,
e eu aprendi muito com ela. A~Peo, em 1973, já fazia a síntese do Piaget
e do Jung na cabeça dela. Ela é muito inteligente, muito antenada e
muito sensível.

\begin{center}\asterisc{}​\end{center}


\begin{flushright}\textbf{26/05/2010}\end{flushright}


Ontem o Megumi e a mulher dele, a Naoko, vieram trazer minha árvore que
ele restaurou. Chegaram lá pelas 14h15 e foram embora às 16h30. O~Megumi
logo mostrou a árvore, que ficou perfeita, não sei como ele conseguiu
essa mágica, o tronco da árvore antes estava partido em dois.

O Megumi pediu para ver o meu trabalho de cerâmica e eu mostrei. Ele
disse que tem imanência material e que a pintura também tem
espiritualidade e verdade. Ele gostou de ambos, cerâmica e pintura.
Disse que estou pronta como artista e não preciso mais de professores.
Comentou que tenho humildade e doçura. Gostou do meu carinho com a
Preta.

O Megumi ficou completamente à vontade e falou muito. Ele falou que o
meu trabalho tem verdade. Acho que foi disso que eu mais gostei. A~verdade no trabalho de arte, para mim, é essencial. A~verdade na vida é
essencial. Eu sempre busquei a ``verdade''.

Nunca imaginei ter um encontro tão bom com o Megumi. A~visita dele foi
como um bálsamo nesse meu coração ferido pela depressão. Eu acabei
contando que estou escrevendo e que sou bipolar. Percebi que o Megumi
ficou fascinado pela minha loucura. Ele queria ver algum trabalho feito
na mania ou na depressão, mas eu só achei aquela pintura grande que está
na sala. Ele disse que eu devia ``transfigurar'' a doença através da
arte. Hoje foi bem difícil arranjar coragem para escrever, estou
deprimida há 24 dias.

\begin{center}\asterisc{}​\end{center}


\begin{flushright}\textbf{01/06/2010}\end{flushright}


Seis meses após ficar bipolar, em 1987, fui trabalhar no Clube da Turma,
um projeto da Secretaria do Menor, que atendia 600 crianças carentes da
Zona Leste. Eu e minha equipe trabalhávamos à tarde, atendendo 300
crianças. Eu tinha acabado de começar a tomar lítio. Engordei 10 quilos
em um mês, fiquei inchada e sentia muito sono durante o dia. Várias
vezes dormi na minha sala, lá no projeto. Recostava a cabeça na mesa e
dormia. Até hoje não sei se alguém me viu dormindo.

Coordenei a equipe de cultura: música, artes plásticas, teatro, dança.
Tive a sorte de ter uma equipe maravilhosa. Trabalhei lá por um ano e
meio, foi muito gratificante. Lá apliquei todos os ensinamentos que
aprendi no Vera Cruz, uma escola de elite.

\begin{center}\asterisc{}​\end{center}


\begin{flushright}\textbf{10/06/2010}\end{flushright}


Entre 1989 e 1996 eu trabalhei no Museu Lasar Segall. Entrei no museu
para ser chefe do Departamento de Atividades Criativas, \versal{DAC}. O~\versal{DAC} tinha
cinco divisões: Artes Plásticas, Música, Cinema, Fotografia e Criação
Literária. Os professores dessa equipe estavam no Segall há muito tempo.
Para mim foi uma experiência terrível porque eles me boicotaram
continuamente durante quatro anos, até me derrubarem. Conheci então o
que é a sordidez. Eu tive a pior depressão da minha vida, emagreci 11
quilos em um mês.

O Maurício Segall, então, sugeriu que eu produzisse um projeto novo no
museu, que seria composto por dois debates por mês, acerca de assuntos
atinentes à arte e à cultura. A~minha função passou a se chamar
Assessora da Direção para Eventos Especiais e Publicações. O~projeto se
chamava ``Quartas"-Feiras Conversas no Segall''. O~Maurício, por
distração, me deu liberdade total. Por distração porque os projetos de
todos os departamentos tinham que ser aprovados pelo Colegiado,
instância máxima do museu. No começo foi muito difícil tocar esse
projeto devido à depressão, mas, conforme ela foi passando, eu fui me
envolvendo e gostando cada vez mais dessa atividade. O~projeto tinha um
conselho que se reunia no início de cada semestre e era formado pelo
Maurício Segall, Marcelo Araújo (chefe da Museologia), Rodrigo Naves
(crítico de arte), Roberto Schwarz (crítico literário) e Pedro Puntoni.

Fiz ao todo 52 debates, exemplifico aqui alguns deles e seus
debatedores:

\begin{enumerate}
\item
  ``Arquitetura da destruição'': Renato Mezan (psicanalista) e Paulo
  Mendes da Rocha (arquiteto);
\item
  ``O mito de Don Juan'', Olgária Mattos (filósofa) e Luiz Tenório
  Oliveira Lima (psicanalista);
\item
  ``Sobre a exposição \emph{O~Brasil dos viajantes}'' (\versal{MASP}), Ana
  Belluzzo (curadora da exposição e crítica de arte), Rodrigo Naves
  (crítico de arte) e Ivo Mesquita (crítico de arte);
\item
  ``Sobre a amizade'', José Arthur Giannotti (filósofo) e Luiz Tenório
  de Oliveira Lima (psicanalista);
\item
  ``O pensamento do projeto \emph{Arte"-Cidade}'', Nelson Brissac Peixoto
  (autor do projeto e crítico de arte), Alberto Tassinari (crítico de
  arte) e Lorenzo Mammi (crítico de arte e músico).
\end{enumerate}
Eu não sei como eu tinha coragem de convidar intelectuais tão brilhantes
para os debates. Acho que foi porque eu estava respaldada por duas
instituições muito sérias no meio da cultura, o Museu Lasar Segall e a
Pinacoteca do Estado. Quando se trata de fazer um trabalho de qualidade,
eu tenho mesmo a maior cara de pau.

No início de 1996, quando voltei de férias, senti que ia de novo entrar
numa máquina de moer carne, como tinha entrado quando no \versal{DAC}. Eu estava
frágil devido ao luto recente pela morte de meu pai e de minha mãe.
Senti que não teria forças para enfrentar esse combate. Pedi, então,
demissão do museu. Num certo sentido foi uma pena porque eu já tinha
imaginado muitos outros debates para produzir.

\begin{center}\asterisc{}\end{center}


\begin{flushright}\textbf{13/06/2010}\end{flushright}


Em 1997, li no Estadão de domingo que a Pinacoteca do Estado iria
realizar vários debates sobre ``Paixão e Loucura'', a propósito da
exposição de esculturas de Camille Claudel. Por acaso, nesse mesmo dia,
fui a uma exposição de escultura no \versal{MUBE} e lá encontrei Emanoel Araújo,
então diretor da Pinacoteca. Eu já o conhecia, pois o tinha convidado
para participar de um debate no Segall. Falei para ele que eu adoraria
produzir os debates sobre ``Paixão e Loucura'' e ele topou. Todas as
terças"-feiras de outubro e novembro haveria um debate. Exemplifico aqui
alguns debatedores e seus temas:

\begin{itemize}
\item
  Jorge Forbes (psicanalista lacaniano e psiquiatra): ``Criação e
  reconhecimento'', \\Paulo Mendes da Rocha (arquiteto): ``… a
  esfera dos sentidos…''\\Mediadora: Maria Lúcia Montes
  (antropóloga), José Celso Martinez Corrêa (diretor e artista de
  teatro): ``Paixão palpável''
\item
  José Miguel Wisnik (professor de literatura e músico): ``Paixão
  fanática''\\Mediador: Olívio Tavares de Araújo (crítico de arte)
\item
  Roberto Piva (poeta e xamã): ``Quando a insanidade é uma bênção:
  poesia e xamanismo''\\Antônio Bivar (escritor): ``Paixão e loucura,
  letras e artes''\\Mediador: Diógenes Moura (jornalista)
\end{itemize}

Uma das vantagens de ter trabalhado com Emanoel Araújo foi a liberdade
total que ele me deu. Além disso, tinha a assessoria de imprensa que
divulgava os debates, tornando"-os acessíveis a muita gente. Já no
Segall, porém, eu não contava com uma assessoria de imprensa, e os
debates eram divulgados só para a mala"-direta do museu, o que tornava o
público mais reduzido.

Na mesma época em que produzi os debates, o Emanoel me pediu para
coordenar a monitoria da exposição ``Escultura brasileira: perfil de uma
identidade'', no Banco Safra, na avenida Paulista e no \versal{BID} -- Banco
Interamericano de Desenvolvimento, Washington -- \versal{DC} -- \versal{USA}. Emanoel
Araújo foi o curador, e Sérgio Pizoli foi o curador"-assistente. Levei as
monitoras para conhecer o trabalho de três artistas nos seus ateliês:
José Rezende, Liuba e Vlavianos. Forneci textos sobre os artistas, e a
partir deles fizemos encontros para discutir a percepção que cada uma de
nós íamos tendo sobre a exposição.

\begin{flushright}\textbf{25/09/2010}\end{flushright}


Estou com muito ódio e muita raiva de ter essa doença. Eu não suporto
mais. Eu não tenho mais uma gota de energia para enfrentar as crises.
Estou absolutamente esgotada, estou muito cansada. Exausta. E~muito
solitária. Muito.

Eu estou apavorada com o meu futuro: depressões, manias, internações. Eu
tenho pavor desse futuro. A~doença é incurável, há muito tempo sei
disso, mas parece que só agora a ficha caiu completamente. Eu sempre me
iludi depois de cada crise, achando que tinha sido a última. Agora vejo
que não é assim. Agora a ficha caiu completamente, e estou cheia de
horror, cheia de medo do futuro sofrimento, do atual sofrimento. E~não
aguento mais sofrer, não aguento mais essa dor no fundo do meu peito.
Estou arrasada com a minha circunstância.

É claro que penso muito no suicídio. Seria a libertação, seria a paz tão
almejada, seria a saída lógica, já que me encontro num beco sem saída.
Mas sei que por enquanto não consigo. Talvez eu seja muito católica para
me matar. Eu sei que é pecado mortal e que suicida não tem missa de sétimo
dia. Eu quero ter missa de sétimo dia. Provavelmente não me suicido por
pura covardia. E~fico mergulhada nessa angústia e nesse medo que não
resolvem nada, só me paralisam, me dilaceram.

Em agosto eu faltei a todas as sessões de análise e às aulas do Jardim.
Fique literalmente dormindo o dia todo, comendo e vendo televisão. Nunca
tinha acontecido isso antes. Eu sempre me esforcei muito para não perder
a análise durante as crises, mesmo que eu tivesse que ir de táxi. Eu não
tenho mais a mesma energia para enfrentar as crises. É~evidente. Estou
cansada, esgotada. O~Celso foi maravilhoso e ligou várias vezes para me
apoiar. Ele é mesmo um cara muito bacana. Faço análise com ele há 11
anos, é a minha análise mais longa. A~vida inteira, desde os 18 anos, eu
faço terapia.

O que vai ser de mim? O que vai acontecer comigo? E agora?

\begin{flushright}\textbf{10/10/2010}\end{flushright}


Eu só percebi agora que há dois anos estou continuamente em crise e
provavelmente por isso estou tão esgotada.

Tudo começou com aquela gigantesca mania de 2008. Eu fiquei duas vezes
internada porque o Del Porto me liberou em mania.

Depois fiquei um ano e meio deprimida, experimentando vários
antidepressivos, que nunca funcionaram. Em abril o Del Porto me deu
Cymbalta, e em 15 dias eu estava em mania. Fiquei um ano e meio com duas
acompanhantes. Foi tudo muito desgastante! Foi tudo um saco!

Futuro. Futuro. Qual será o meu futuro agora? Estou muito assustada e
com muito medo e sem coragem de encarar o que a vida me reserva. Porque
eu já sei o que ela me reserva: crises e mais crises. Sofrimentos e mais
sofrimentos.

Chega! Eu não aguento mais!

Tô fora!

\begin{center}\asterisc{}​\end{center}

\begin{flushright}\textbf{São Paulo, 26/10/2010}\end{flushright}


\subsection{\emph{Na Granja Julieta}}

Eles conseguiram e conseguiram. Fui internada hoje às 16h00. \emph{Eu
iria sozinha, de táxi, como combinei com o Del Porto.} Sozinha não, com
o Rafa, acompanhante que estava comigo, uma pessoa chata, chatésima,
absolutamente irritante. É~foda!

Eu já estava arrumando as malas quando Paulinho ligou e se ofereceu para
me trazer para cá junto com o Rogério. Ele combinou de chegar em casa às
15h00 e chegou às 13h00! Haja ansiedade! Eles ficaram em pânico nessa
situação, super ansiosos, e é terrível porque essa ansiedade familiar me
contamina muito, é um horror, é muito pesado, esse clima que me
enlouquece. Quando eles chegaram, eu já estava com a mala quase pronta,
mas, no meio de tanta ansiedade, deixei de trazer coisas importantes
para mim, como uma malha leve, um pente, coisas que trazem um certo
conforto. Você já está fodida por ser internada, é importante ter um
certo conforto nas coisas básicas, materiais, ajuda bem.

Foi o Del Porto quem tomou essa decisão, \emph{a pedido do Fernão}, e
ligou para mim para dizer que eu seria internada porque seria melhor
para mim nesses dias de final de ano. \emph{Parece que todas as famílias
resistem a ter por perto a presença do parente bipolar, seja no meio do
ano, no Carnaval, no Natal ou no}\textbf{} Réveillon. A~não ser quando
você está bem depressivo, bem fraquinho, triste, melancólico,
definhando. Aí você não é uma ameaça aterrorizante para eles como você é
na mania. Afinal, a única ameaça aterrorizante é o suicídio, que talvez,
apesar de ser algo terrível, fosse um grande alívio para minha família ,
isto é, para alguns dos irmãos e alguns dos amigos. A~Regina, a Rê, a
Gina teria se transformado em pó e não iria incomodar mais com suas
manias e com suas deprês…

No ano passado fiquei muito perto do suicídio durante alguns períodos.
Tinha vezes que eu \emph{sentia} uma vontade enorme, \emph{era um
impulso muito forte de me suicidar}. Outras vezes eu tinha \emph{o
pensamento obsessivo}\textbf{} do suicídio. Eu planejava tudo: como
seria, onde seria, em que hora seria. Eu contei isso para Soninha em
duas ocasiões ou mais, quando ela me telefonou para saber como eu
estava. Nem por isso ela veio me visitar, ou me convidou para almoçar na
casa dela, só nós duas, e não nos almoços das segundas"-feiras, com tanta
gente. Me senti uma total \emph{leprosa}, alguém que ninguém quer chegar
perto porque tem muito medo de se contaminar com aquele mal terrível, a
``lepra!'', ou melhor a bipolaridade, hoje denominada ``Distúrbio
Bipolar do Humor'', nome que disfarça \emph{a gravidade e a tragicidade
da doença.}

Tomei meu café da manhã no quarto, como é hábito aqui. Dois pães
franceses, um potinho de manteiga, um de requeijão, um de geleia de
goiaba e um bule com leite. Eu adoro porque vem um bule grande de café
preto e dá para tomar café à vontade. Depois eu fiquei escrevendo e,
quando cansei, fui passear no jardim, que é a melhor coisa que essa
clínica tem. Eu adoro \emph{contemplar} as carpas nadando, revoluteando
naquele laguinho artificial. Como tem um banco pertinho, dá para sentar
e curtir, viajar. Depois percorri os caminhos de lajotas e fui
revisitando cada canto do jardim do qual me lembrava bem. Acho que esse
jardim foi meu melhor apoio na mania de 2008, e também o Raymond Juneck,
um psicanalista lacaniano fantástico que atende na clínica.

\begin{flushright}\textbf{27/12/2010}\end{flushright}


A Rosélia, minha acompanhante, continua me irritando com seu jeito de
``polícia'', essa clínica treina suas acompanhantes para serem
verdadeiras ``policiais''. Ela gruda em mim e vai me seguindo muito de
perto como uma sombra. Eu tenho que dar uns ``chega pra lá'' nela e dou
mesmo, mas isso enche, é exaustivo, é cansativo, me irrita muito e me
desgasta. Essa é a pior parte do tratamento na clínica. Um saco! Um
verdadeiro saco! Eu poderia falar com a Isaura, que é a ``chefa'' aqui,
para mudar de acompanhante, mas não tive coragem. Resolvi aguentar mais
um pouco. Quero ser política e não causar transtornos para conseguir ter
alta o mais cedo possível. Já aprendi muito sobre isso nas internações
passadas. É~uma luta! É uma guerra!

Como tudo conta ponto aqui, eu resolvi ir até à sala da Terapia
Ocupacional só para dar uma boa impressão. Eu quero ganhar todos os
pontos que puder ganhar. Eu estou a menos de um dia e já estou
desesperada para voltar para a minha casa. No começo, eu fiquei até
aliviada de vir para cá, pois diminuiu a enorme pressão familiar que eu
estava sentindo, via telepatia e ao vivo, principalmente do Rogério que
já estava me chamando de ``descompensada'' e ``excitada'' nas nossas
conversas telefônicas.

Eu vi em um mural de um consultório médico esta frase anônima: \emph{``A
minha paciência opera verdadeiros milagres na minha vida''}. Achei a
frase muito sábia, perfeita, e tenho lembrado dela em muitos momentos da
minha vida. Esse pensamento me dá forças para aguentar situações
difíceis, para suportar as adversidades da vida e as da bipolaridade,
especificamente.

Voltando à \emph{generosidade enorme} do Rogério: durante um ano e meio,
depois da mania de 2008, ele me \versal{TELEFONAVA} \versal{TODAS} \versal{AS} \versal{NOITES} para saber se
eu estava bem. Ele fazia uma visita \emph{de irmão e de médico}.
Perguntava tudo sobre a depressão, a diarreia ou a prisão de ventre,
sobre os efeitos colaterais paralisantes do Risperdal, enfim, sobre
tudo que eu estava passando. Ele é o ser humano mais afetivo, generoso e
amoroso que eu conheço. É ``iluminado''.

Quando eu estava internada, ele veio várias vezes me visitar. Trazia
aquelas esfirrinhas do Arábia, que eu adoro, e também aquelas
bananadinhas de Ubatuba, que eu também adoro. Ele trouxe também um
livrinho com reproduções das pinturas do Van Gogh. Aquelas pinturas tão
queridas foram também um bálsamo nas minhas feridas. Era um livrinho em
que cada página era um cartão postal, e havia muitas reproduções de
vasos de flor, que eu amo. Eu me animei toda e dediquei alguns cartões
para as acompanhantes da clínica, mas … não tive coragem de dar,
eu nunca tinha visto tantos vasos de flor pintados pelo Van Gogh e
fiquei com o livrinho para mim. Deixo"-o na mesinha ao lado do sofá de
casa e, vira e mexe, dou uma curtida nele. O~Van Gogh foi a minha
primeira paixão pela pintura, durante a adolescência.

O Rogério é o único irmão que não tem medo da minha loucura. Ele sempre
me apoiou e me deu força em todas as crises, generosamente. Ele sempre
esteve \emph{muito presente, e isso é claro que faz a maior diferença
para mim.} Ele me dá chão e me ajuda a pensar, como o Celso, meu
analista, como o Del Porto, cada um do seu jeito, na sua linguagem, na
sua loucura, na sua especialidade. O~Rogério fica muito ligado nos
remédios que eu tomo.

Agora já são 22h00 e, não adianta negar, estou tristíssima porque
ninguém ligou, nem o Rogério, nem a Soninha, nem o Paulinho, que são os
irmãos que sabem que estou aqui. Eu não sei mais quem está sabendo. O~Rogério disse que a notícia da crise ficaria só entre eles três…
No momento, só tenho o direito de dar duas ligações diárias! Portanto,
se as pessoas não me ligam de fora, eu fico quase completamente isolada.
Eu escolhi ligar para a Preta para saber se estava tudo bem e ligar para
a Sara, com quem bati um bom papo. Eu adoro a Sara, me entendo super bem
com ela. Pedi a ela para ligar para a Soninha e pedir para a Soninha
mandar para mim 12 bananas, 12 caixinhas de água de coco, pão
pinheirense e 12 latinhas de guaraná, que eu acho que não vieram. A~falta de memória causada pelo \versal{ECT} é muito maior do que eu imaginava. De
noite eu não lembro o que fiz de manhã. Mas o Del Porto disse que isso
dura dois meses. Tomara. Eu acabei de fazer quatro sessões de
eletroconvulsoterapia que deram errado. Em vez de me estabilizar,
desencadearam uma mania.

Eu tenho muita consciência e clareza de que estar internada no Parque
Julieta, tido como a melhor clínica de São Paulo, é um grande
privilégio, que tenho graças à generosidade da Soninha e do Fernão. Eu
me lembro muito bem do estágio que fiz na clínica pública da Vila
Mariana, quando estudei Psicopatologia, por sinal com o Del Porto, que
lecionava lá nessa época, na faculdade de Psicologia São Marcos. Era um
horror, horror total. Eles faziam de tudo para despersonalizar as
presas, ou melhor, \emph{as loucas.} Era um hospital só para mulheres.
Todas elas tinham os cabelos cortados, bem curtinhos, e usavam aventais
xadrezinhos de branco e azul, ou de branco e vermelho, \emph{conforme o
grau de loucura e de periculosidade}! Ali ninguém era ninguém. Elas
viviam sua solidão absoluta num amontoado de solidões absolutas.

Foi aí que ficou claro que eu não tinha a menor vocação para ser
psicóloga, eu tinha mesmo vocação para ser artista. Eu ``viajava'' nos
delírios e alucinações das pacientes. Eu achava muito interessante os
conteúdos ali expressos. Havia uma paciente, por exemplo, que era
fascinada e apaixonada pelo ``cabo Bruno'', um personagem famoso da
crônica policial da época. Ela me contava mil histórias de encontros que
ela havia tido com o cabo Bruno, inclusive um, numa churrascaria, que
havia sido fantástico. Eu simplesmente não cumpria a minha função de
terapeuta, isto é, trazê"-la para o chão, para a realidade, para perceber
duramente que tudo aquilo era puro delírio, pura fantasia sem a menor
condição de se realizar… Afinal, o cabo Bruno estava preso naquele
momento.

Esse estágio no hospital psiquiátrico público da Vila Mariana me
impressionou muito, me marcou muito. Eu lembro muito bem de tudo que
observei ali, as imagens estão intactas na minha cabeça. Foi uma espécie
de ``circo dos horrores''. Mal sabia eu que alguns anos mais tarde eu me
tornaria também uma ``doente mental'', uma ``psicomaníaca depressiva'',
ironias da vida…

Aquela situação das internas, naquele depósito de loucas que
apresentavam todo tipo de loucura, era algo terrificante, uma das cenas
mais escabrosas que já vi na minha vida. Elas adoravam quando a gente
levava de presente espelhinhos e batons. Era a maior festa! Elas
tentavam provavelmente resgatar seu feminino, já soterrado pela loucura
e pelo que é um hospital psiquiátrico público, puro abandono. E~fazer
uma anamnese, então. Para mim era um horror. A ``louca'' ficava
sentada num banquinho, e nós ficávamos em volta dela, perguntando,
perguntando e perguntando. \emph{Invasão total}, sem a menor cerimônia.
O~fascínio dos \emph{``sãos''} pelos ``\emph{loucos}'' se explicitava
ali. Pela minha própria história percebo que todo mundo tem um certo
fascínio pela loucura… Os olhos das pessoas brilham quando a
gente conta uma crise, uma mania arrasadora, como a de 2008, por
exemplo, ou qualquer outra crise. As pessoas demonstram a maior
curiosidade pelo assunto. Eu ficava muito aflita com essa situação da
anamnese (não sei direito como se escreve até hoje). Eu levei
vários materiais para as presas poderem se expressar: argila, papéis de
tamanhos variados, tinta, todas as cores possíveis de tinta, pincéis e
pincéis hidrográficos. Elas amaram e trabalharam muito. Essa oficina de
arte improvisada foi a minha maneira de ajudá"-las, de me solidarizar.
Teve um dia em que o Paulo Portella, meu amigo, foi comigo ao estágio, e
uma das pacientes fez uma pintura de um homem barbudo igual ao Paulo.
Quando ficou pronta, logo exclamou: é o Tiradentes, é o Tiradentes!
Sonho de liberdade.

\begin{center}\asterisc{}​\end{center}


\begin{flushright}\textbf{31/12/2010}\end{flushright}


São 19h00 e já estou bonita e arrumada esperando a visita do Sylvio, do
Rogério, da Bete e do Paulinho aqui no Parque Julieta, onde estou
internada há cinco ou mais dias. Como passou rápido! Botei aquele
vestidão que eu adoro, bem florido com o fundo rosa e flores em tons de
azuis, marrons, amarelos e brancos. Me perfumei com aquela amostrinha do
perfume Paloma Picasso. Me sinto de novo meio \emph{hippie} usando esse
vestidão. Estou com as guias que fiz na T.O., que são bem bonitas, bem
vistosas. Uma delas traz Iemanjá, Preto Velho, \emph{Cabocla Guaciara} e
\emph{Rosa}; a outra, Ogum, Cabocla Guaciara e Preto Velho. Azuis,
verdes e marrons lindos. Nas contas.

Eu fui até a T.O\, com o maior preconceito e daí, de repente, adorei fazer
colares. Fiz um monte, uns oito ou nove, eu acho. Mostrei para o
Rogério, que gostou muito. Até mandei um de presente para a Nanda, um
lindo, super sóbrio, em tons de bege, que combina com a Nanda. Estou
pensando em fazer mais colares para vender quando sair daqui, ou seja,
logo, logo. O~Del Porto acha que em princípio devo ficar aqui até 6 ou 7
de janeiro. Eu prefiro passar meu aniversário (4 de janeiro) na clínica,
para ficar possivelmente mais estável por mais tempo em vez de arriscar
uma possível instabilidade.

A Bete foi muito generosa e hoje mandou para mim uma cesta incrível de
Natal do Santa Luzia. Tem tudo que a gente gosta: queijo provolone,
queijo muçarela, panetone, tâmaras, passas. Vai ser gostoso dividir
essas guloseimas com os irmãos hoje à noite.

\begin{center}\asterisc{}​\end{center}


\begin{flushright}\textbf{01/01/2011}\end{flushright}


Apesar de todo o medo que eu tinha de me encontrar com o Paulinho, o
Rogério, o Sylvio e a Bete aqui na clínica para comemorar a ``passagem
de ano'', deu tudo certo. Eu pretendia passar o meu \emph{réveillon}
aqui na clínica quietinha, me ``fazendo de morta''. Achei que seria mais
interessante para mim no sentido de não mexer com as forças da dinâmica
familiar, que nem sempre são positivas, como bem falaram o Gaiarsa e o
Nelson Rodrigues, meus mestres absolutos, e, antes deles, Freud, Melaine
Klein, Bion, Winnicott, entre outros, é claro.

No fim deu tudo certo e foi muito gostoso reencontrar os irmãos. A 
última vez que tínhamos nos visto foi no dia 12/12/2010, no jantar de
Natal que a Soninha oferece anualmente para as famílias Barros e Sawaya.
Eu fui para o jantar muito insegura e com medo, afinal havia
praticamente dois anos que eu não via a família, devido à longa
depressão que tivera. Quando cheguei ao jantar o Fernão me elogiou
muito e falou que eu estava muito bonita. Eu caprichei mesmo na roupa e
na maquiagem para me sentir mais segura. Vesti um vestidão comprido com
uma estampa geométrica em tons de preto, ocre e branco. Eu não pude
ficar até o fim do jantar porque estava com uma lombalgia muito forte,
sentindo muita dor nas costas e nas pernas. Além disso, eu tinha que
chegar em casa até às 23h00 porque a Débora, acompanhante, estaria me
esperando. Na véspera quem me acompanhou foi o Ricardo, dia 20. Eu senti
``microssinais'' de crise de mania e falei com o Del Porto, que aumentou
a dose de Zyprexa para 7,5mg e achou ótimo eu ter solicitado os
acompanhantes para o período da noite, o horário em que me sinto mais
frágil. Ele foi subindo o Zyprexa devagar, até os 20mg, que é a dose que
estou tomando há uns três ou quatro dias. Não me lembro bem quando ele
aumentou.

A Bete telefonou para a clínica e ficou aos prantos no telefone.
Conversamos sobre a eletroconvulsoterapia, e é claro que ela estava puta
da vida porque foi a última a saber, como acontece sempre.

Pedi para ela as coisas que estavam me fazendo falta: água tônica,
queijo, pão, bananas. A~Bete disse então que ia mandar fazer uma
cesta no Santa Luzia, que era para ser surpresa. Mas ela não aguentou e
contou. Ela caprichou e mandou uma cesta muito generosa, com tudo que eu
gosto.

\begin{center}\asterisc{}​\end{center}


\begin{flushright}\textbf{02/11/2011}\end{flushright}


Esta temporada está animada! Foi o que comentei com a Áurea, enfermeira,
quando ela veio trazer os remédios. Perguntei se a moça que está
surtando violentamente há três dias melhorou, e ela disse que não. Não
melhorou porque não aceita tomar remédios, só injeção. Eu não consigo
entender porque as injeções não fazem o mesmo efeito que os remédios e
não acabam com o surto. Dou graças a Deus por aceitar bem os remédios há
26 anos! Claro que tive minhas fases de não tomar remédio. E~aí me fodi,
me fodi mesmo. Mas é assim que a gente aprende a importância dos
remédios, na prática. É~se ferrando. Se ferrando muito!

Uma das minhas metas deste ano é voltar aos Vigilantes do Peso para
eliminar os quilos a mais, mas enquanto eu tomar Zyprexa vai ser
impossível, esse remédio abre tanto o apetite que é usado para resolver
problema de \emph{pacientes com anorexia grave}!\textbf{} Dá para
imaginar o tamanho da fome que ele dá, fome sobretudo de açúcar. Quando
eu tomo esse remédio, eu costumo comprar dois potes de Häagen Dazs,
Dulce de Leche, e devoro de uma vez ao chegar em casa. É~uma fome de
açúcar animal! Bruta! Incontrolável. É~uma droga, droguíssima!

Estou na Clínica Parque Julieta, lugar que outrora foi uma fazenda. Os
espaços são bastante amplos e tem um jardim maravilhoso, os jardineiros
daqui são verdadeiros artistas. Eles têm uma coleção de orquídeas
belíssima. Eu sempre me lembro muito da mamãe quando passeio nesse jardim.
Eu peguei o último quarto com banheiro dessa temporada. Quando liguei
para a clínica para reservar um quarto para mim, este, o 21, era o
último quarto disponível com banheiro. Fui salva pelo gongo! Quarto sem
banheiro é um horror… Estou muito mal"-acostumada…

Meu quarto é pequeno, mas é gostoso, parece com aquele quarto que no
Pinhal a gente chamava de Van Gogh: tem uma cama e um armário encostados
na parede à direita da porta de entrada. Entre o armário e a cama, tem a
porta que dá para o banheiro, que é bastante bom. À~direita da cama tem
um criado"-mudo com uma gavetinha, uma prateleira para sapatos mais
embaixo, e um abajurzinho. À~esquerda da porta de entrada tem uma
\emph{bergère} com estampa florida, que é muito funda, então fica meio
difícil levantar, mas é confortável para ler. Em seguida tem um
frigobar, que está lotado com o que sobrou do \emph{Réveillon} e,
depois, tem esta mesinha com uma cadeira onde estou escrevendo. O~quarto
tem um janelão enorme, de mais ou menos 1,20m x 3m, que dá para um
jardim lindo, com árvores muito antigas, muito desenvolvidas, uma
beleza! Eu tenho o maior prazer em abrir a janela e ficar contemplando
aquele monte de verdes se entrecruzando, iluminados por pontos amarelos,
e também as flores da trepadeira ``primavera''. A única pena é que a
janela tem tela, o que atrapalha um pouco a visão.

Quando cheguei aqui, achei a clínica menos decadente que da última vez.
A~dona Helena morreu, e quem assumiu o lugar dela foi a férrea Isaura,
que é muito competente, mas é um ``cão policial feroz''. Nas duas
internações anteriores eu a detestava, mas agora tenho me relacionado
melhor com ela. Pelo visto, foi ela que acabou com o clima decadente que
a clínica tinha antes, clima de abandono total…

Essa clínica foi outrora uma ``casa grande'' de fazenda, é por isso que
os aposentos são muito amplos e o parquê dos pisos muito bonitos. Ela
acabou de ser pintada, pelo visto. A~construção é cor de telha queimada
e os detalhes marrom"-escuro. Quando a gente entra, depois de percorrer
um longo corredor no meio do jardim, à esquerda da porta, no \emph{hall}
de entrada, tem uma imensa árvore de Natal, mais ou menos de três
metros, super colorida e enfeitada. Está linda e é o maior astral. Na
porta da maioria dos quartos, há uma guirlanda de Natal bem bonita
enfeitando. No \emph{hall} seguinte ao da entrada, à esquerda, colocaram
um enfeite lindo, todo vermelho e dourado e branco, que é um ``anjão'',
muito bonito, sobre uma mesinha. A~Isaura de fato conseguiu criar um
clima alegre de Natal aqui na clínica. Isso foi uma verdadeira proeza,
eu imagino… Ela me disse que todas as guirlandas foram feitas na
T.O\,
A Áurea, enfermeira da clínica, me disse que no momento há 18 pessoas
internadas, algumas com acompanhantes 24 horas, outras com acompanhantes
12 horas. Graças a Deus, o Del Porto me liberou pelo menos da
acompanhante diurna, eu não aguentava mais… Que saco! Ter
acompanhante do nível que são essas da clínica é um saco! O Del Porto
viajou, está de férias, então logo, logo vou pedir para a Dra. Mara
tirar a acompanhante noturna. Mas quero pensar bem antes de fazer isso.

Uma parte dos internos se constituiu de uma moçada viciada em drogas.
Eles são relativamente jovens, cerca de 30 anos, três moças e três
rapazes. Alguns são casados e têm filhinhos. Uma judiação… No dia
31, os filhos do Bruno vieram visitá"-lo, são três crianças lindas,
moreninhas como ele, trouxeram a maior alegria para a clínica, foi
comovente.

Às vezes a gente fica fumando no terraço depois do almoço e do jantar, e
dá boas risadas. Teve uma noite que o assunto foi ``drogas'', cada um
contou sua experiência, suas histórias… Eu, apesar de 30 anos
mais velha que essa moçada, até fiquei por dentro da conversa porque
fumei maconha numa época da minha vida e cheirei pó umas duas vezes.
Depois de cheirar, no dia seguinte fiquei completamente deprê e percebi
que o pó para mim é um perigo, porque você deprime e você quer cheirar
mais para ficar eufórico de novo, graças à droga, e escapar da deprê. Eu
fui muito sábia, visto que nessa época eu não era ainda bipolar, pelo
menos não estava diagnosticada. Eu contei para eles também a minha
experiência com a ``loló'', em Olinda, no carnaval. A ``loló'' é uma
espécie de ``lança"-perfume'' caseiro que eles vendem lá. Não é à toa que
eu descia e subia as ladeiras de Olinda na maior vula no carnaval.
Enfim, nessa noite a conversa com a moçada foi gostosa, me entrosei com
o grupo.

Porém, desde ontem estou meio cheia das pessoas. Percebi um leve ``ar de
rejeição'' por mim quando entrei na T.O\, e eles pararam de conversar,
deviam estar ``falando de mim'', foi a reação paranoica imediata. Detesto
isso e lembrei do que o Celso sempre me fala: ``Um grupo é por si só
enlouquecedor''. E~de fato é. Pude comprovar isso várias vezes na minha
vida. Sou gato escaldado. Por isso ontem e hoje me isolei e escrevi
bastante, ontem não fui jantar, só almocei os queijos e as frutas que
tenho aqui no quarto. A~Bete exagerou na compra, ontem ainda dei frutas
e bolachas para uma acompanhante e só agora percebi que ainda tem duas
tranças enormes de muçarela na geladeira, graças a Deus, porque eu adoro
muçarela.

\begin{center}\asterisc{}​\end{center}

\begin{flushright}\textbf{03/01/2011}\end{flushright}


Hoje resolvi parar para jantar com a turma e parei um pouco de
escrevinhar. O~jantar estava bom: arroz, hambúrguer, esfirrinha de
queijo e de entrada uma saladinha ótima, bem variada. Tinha também
abobrinha refogada e manjar branco de sobremesa. Eu acho a comida daqui
muito boa, mas vários pacientes reclamam bastante. O~refeitório fica
numa espécie de ``jardim de inverno'' em forma de L e tem umas 12
mesinhas com toalhas de um brocado branco sobre branco, bem discreto, e
por cima um losango azul"-anil de algodão. As toalhas brancas têm o
monograma da clínica bordado em azul"-marinho. ``É tudo muito fino'',
como diria minha mãe. De vez em quando o monograma aparece pequenininho
nas peças de louça. ``É tudo muito burguês'', diria o Rogério.

Eu sempre sento numa mesinha que tem uma vista linda para o jardim, já
que meia parede do refeitório é de vidro. Eu adoro ver aquele monte
lindo de bananeiras, acho fantástico as formas das folhas das
bananeiras, sempre adorei contemplar bananeiras. Às vezes ninguém vem
comer comigo, e então eu faço a refeição sozinha. Eu prefiro isso a
almoçar com aquela mulher psicótica, esquizofrênica, a Sônia, que está
internada aqui há 10 anos! O filho a internou aqui há 10 anos! Ela ficou
\versal{MORANDO} na clínica! Ela é uma mulher muito bonita, loira, de olhos
azuis, alta, esguia e deve ter por volta de 50 ou 54 anos. Mas está com
os dentes completamente podres, o que me dá uma aflição terrível. Ela se
veste muito bem, um pouco inadequada, diga"-se de passagem, usando
microssaias bem micros, o tempo todo. Na última internação resolvi ser
``caridosa'' e almocei com ela. Foi um desastre, ela me sugou até a
alma. Desisti de ser caridosa. Agora tenho sido ``caridosa'' só comigo.
Evito os muito loucos e os muito chatos, chatérrimos, aqui na clínica.
Chega de ser Madre Teresa de Calcutá. Já paguei meu carma.

Esse refeitório, com certeza, foi anexado à antiga construção da sede da
fazenda. O~pé direito é bem mais baixo. Hoje eu estava reparando na
beleza do salão principal. Ele tem uma porta gigantesca com um lindo
trabalho de marchetaria no rendilhado de ferro que se sobrepõe às portas
de vidro. A~gente, quando senta no fundo do salão, vê o jardim através
desse rendilhado, e é muito bonito.

Do lado direito e do lado esquerdo do portal tem um óculo muito bonito,
com formas arredondadas, difíceis de descrever. Eu hoje reparei que no
piso do salão também tem um óculo com o mesmo formato, feito com os
tacos na madeira. Com certeza deve simbolizar alguma coisa importante. A~escada que leva para o primeiro andar tem dois lances, e, no fim do
primeiro lance, há um vitral maravilhoso, enorme e muito colorido, com
rosas azuis, vermelhas e amarelas, mas agora não estou lembrando das
figuras ali representadas, a não ser do tal símbolo que também está lá
presente, desenhado no vidro. Hoje eu reparei que o teto da imensa sala
de estar é super trabalhado com gesso, formando figuras muito bonitas
com conchas e flores barrocas. A~sala é gigantesca e tem uns cinco
recantos, e em cada um tem um sofazão ou de couro ou de tecido -- e
precisando de uma boa lavada… Num desses recantos tem uma \versal{TV} bem
grande, deve ter 40 a 42 polegadas e tem também, além do sofá, duas
poltronas. Um outro canto tem uma \versal{TV} menor e dois sofás. Tem também dois
cantinhos para jogar xadrez ou baralho. Cada mesinha tem um pano verde
com os naipes do baralho estampados nas suas cores.

\begin{center}\asterisc{}​\end{center}


\begin{flushright}\textbf{04/01/2011}\end{flushright}


É inacreditável, mas hoje acordei de súbito às 6h30 e não consegui mais
dormir. Não sei o que aconteceu. Resolvi, então, continuar a escrever.
Gosto do silêncio que reina neste horário na clínica. Ouço apenas os
piados dos passarinhos e o vozerio longínquo das cozinheiras na cozinha.

Na verdade, desde que parei de tomar dois comprimidos de Carbolitium \versal{CR}
450mg, um à noite e um de manhã, meu sono se normalizou, passei a dormir
espontaneamente oito horas por noite, às vezes, nove horas. Depois de
passar dois anos dormindo das 23h00 às 11h00, isto é, 12 horas de sono,
na depressão, o que vivo agora é quase uma ressurreição! Uma maravilha!
\emph{E pensar que passei anos me culpando porque dormia demais…}
e era tudo efeito colateral de remédio! Ô sofrimento!… \emph{Não
era para eu ter culpa nenhuma.} Mas o mundo da culpa é foda! Afinal tive
educação ``católica''…

Depois de tomar o café da manhã, fui fumar um cigarro no terraço e me
diverti com a moçada. O~Bruno e o Fernando criaram o \versal{CPJ}, tipo \versal{CCC},
Comando Vermelho e outros, ou seja, Comando Parque Julieta! ``Lá dentro
eles mandam, eles falaram; da porta para fora, no terraço e nos jardins,
mandamos nós''. Eles são muito divertidos, inteligentes, cheios de
energia! Demos boas risadas. Todo mundo está muito impressionado com a
moça que está num surto violento há dias. Ela berra, grita, fala para os
meninos que eles a estupraram, que botaram veneno na água dela, e eles
ficam irritados, sobretudo porque não conseguem dormir, visto que ela
acorda berrando às 4h30 da manhã. Teve um dia que ela desceu a escadaria
de fio dental e foi o maior escândalo. Eu acho estranho as pessoas
ficarem com raiva, afinal ela está muito doente e merece compaixão e~muita pena também.

Aqui tem umas figuras incríveis, muito simpáticas. A~Gigi é uma delas.
Ela tem 37 anos, é uma moça alta, corpulenta e prática, e se fantasia de
pomba"-gira. A~cada dia ela usa uma roupa super produzida, vermelha,
lilás ou preto são as suas cores… Os adereços, em geral dourados,
é tudo legítimo, segundo ela. Ela é filha única e parece ser bastante
rica. O~pai é desembargador. Ela usa sandálias com salto 10\, cm, do
Fernando Pires, que segundo ela são muito confortáveis. Na verdade, as
sandálias são muito bonitas. Eu adoraria conseguir andar assim nas
alturas sem levar um tombaço. A~Gigi é toda \emph{sexy}, completamente
\emph{sexy} com seus saltos altos, seus batons, sua maquiagem pesada,
que dura 24 horas por dia. Além disso, ela usa diversas perucas loiras,
ruivas, com um cabelo longuíssimo, ora dourado, ora vermelho, ora
preto. Ela tem o corpo \emph{inteirinho tatuado}, não sobra espaço
para mais nada. As tatuagens são esotéricas: tem do Santo Graal; num dos
ombros tem uma diabinha que parece a ``tiazinha''; numa das mãos, um
naipe do baralho em cada dedo (na outra mão não lembro o que tem). Ela é
uma pessoa muito simpática e inteligente, eu gosto de conversar com ela
e de apreciar sua produção, diferente a cada dia. Cada dia é uma
surpresa. Mas eu nunca vi alguém agredir tanto seu corpo através do uso
desenfreado de tatuagens e mais tatuagens. Ela criou ``uma segunda pele
esotérica para ela''. Talvez para se proteger…

Ontem foi meu aniversário e foi uma delícia. O~primeiro telefonema que
recebi foi da Preta, super afetiva e carinhosa. Depois ligaram a
Soninha, que enviou de presente um par de brincos lindos, ovalados,
compridos, em ouro e madrepérola, e ligaram também a \emph{Sara}, o
\emph{Daniel}, a \emph{Fátima}, a \emph{Bete} etc. Minha memória recente
está mesmo muito prejudicada, não lembro mais de quem ligou.

O aniversário foi uma delícia! À tarde eu estava fazendo colar e, de
repente, o grupo cantou parabéns para mim com a maior animação. Eu, que
sou tímida, fiquei toda sem jeito, mas adorei. A~Martina, orientadora da
T.O., ofereceu um panetone à guisa de bolo, e eu ofereci o primeiro
pedaço para o Paulo, que ficou roxo de timidez.

De manhã, no dia 04/01, quando eu entrei na T.O., estava tocando, a
altos brados, \emph{\versal{SATISFACTION}}, com os Rolling Stones. Nem acreditei,
eu adoro essa música dos anos 70. Dos meus anos 70. Mais tarde teve a
Janis Joplin cantando \emph{Cry me a baby!.} Eu morri de emoção. Eu
estava me sentindo muito bem e muito feliz fazendo colares e ouvindo um
\emph{rock} pauleira. Ofereci o primeiro pedaço de ``panetone bolo''
para o Paulo, porque senti que era a pessoa mais próxima de mim naquele
grupo. Ele é filho adotivo de um casal que conheço há séculos, porque
são muito amigos de juventude da Soninha e do Fernão. O~Paulo é um
cara educadíssimo, doce, que está enfrentando uma crise de alcoolismo.
Ele é médico, gastro e cirurgião. Trabalha no Sírio"-Libanês. Uma
judiação. Ele tem 48 anos. Desde que eu cheguei aqui, ele está fazendo
com o maior capricho um avião, daqueles de aeromodelismo, o que demanda
uma concentração enorme e uma motricidade fina \emph{idem}. Ele consegue
trabalhar muito bem no meio da zoeira que é aquela sala da T.O., onde
sempre toca um \emph{rock} altíssimo, ensurdecedor.

Eu estou com preguiça de descrever todos os pacientes da clínica e acho
que nem conseguiria. Afinal, não sou uma escritora… Em todo caso
tem o Fernando, também se recuperando das drogas, que diz com a maior
naturalidade que, quando ele se droga, conversa com as plantas, com as
árvores, com as flores e que acha esse delírio muito positivo se
comparado com o da moça que fica berrando e xingando todo mundo às 4h30
da manhã. E~tem a Luciana, que me contou, logo no primeiro dia em que a
conheci, que tentou se suicidar tomando duas caixas de Seroquel…
Ela disse que depois teve que fazer uma lavagem estomacal que foi
terrível, parece que foi meio malfeita e ela ficou toda machucada, foi
maltratada. Ela tem um jeito calmo e doce e seria a última pessoa que
você iria imaginar que tentou o suicídio. Aqui, as pessoas acabam dando
esse depoimento com a maior naturalidade, e eu acho isso muito bom,
visto que com as pessoas ``tidas como normais'' você nunca pode falar
desses assuntos, todo mundo fica logo angustiado, meio em pânico e sem
saber o que fazer.

Quantas marcas de caquinho eu tenho no meu corpo e na minha mente! Raku.
Raku é aquela técnica de cerâmica que deixa o trabalho todo craquelado.
Pelo menos o resultado é lindo! A vida marca, a vida marca muito. Todos
somos craquelados. Todos temos marcas. E~isso é lindo, é isso que faz a
gente ser a gente, e não o outro e~vice"-versa. Uns mais marcados,
outros menos, cada um na sua idiossincrasia. Palavra bonita…

\begin{center}\asterisc{}​\end{center}

\begin{flushright}\textbf{14/02/2010}\end{flushright}


Voltando à ``vaca fria''…

Eu tenho que tomar muito cuidado, muito, muito cuidado agora. Ontem o
Rogério já usou comigo, na conversa telefônica, o termo
``\versal{DESCOMPENSADA}''. ``Não precisa ficar descompensada!''. Tudo porque eu
me irritei, e me irritei mesmo, com aquele questionário chatérrimo e
infindável que ele faz sempre sobre a novela. Ontem eu estava cansada.
Parece sempre que estou fazendo um exame vestibular! \versal{SACO}! \versal{SAQUÍSSIMO}!
Eu sempre tenho o maior saco e ``guento'' firme, mas ontem não consegui
me controlar. A~luzinha acendeu, \emph{eu tenho que me esforçar muito}
agora para \emph{discriminar} o que é \emph{``irritação maníaca'' e o
que é ``irritação natural''}.

Me observar, me observar e me observar. Como sempre. Mas às vezes a
gente ``viaja na maionese'', e eclode uma crise… Eu sei mais que
ninguém que \emph{eu corro esse risco}! Não vai eclodir nenhuma crise!
Eu não vou deixar. Eu estou fazendo o máximo, ``o supermáximo para
isso''! Eu estou dando tudo de mim, mas tem um monte de coisas que podem
me ajudar e que eu ainda não consegui começar:

\begin{enumerate}
\item
  \versal{REGIME}. Estou pesando 97,500kg, Zyprexa, Zyprexa, Zyprexa…
  Saco! Como é que dá para emagrecer tomando esse remédio? Mas agora
  talvez dê, porque estou tomando só 2,5mg diariamente. Já cheguei a
  tomar 12,5mg e então cheguei nos 107,500kg! Não podia ser diferente.
  Foi um terror, um verdadeiro terror. Fiquei deformada, enorme,
  inchada, uma verdadeira ``balofa'', horrível!
\item
  \versal{CAMINHAR}. \versal{APROVEITAR} O \versal{SOL} \versal{DA} \versal{MANHÃ}. O~Del porto já me falou umas 500
  vezes sobre os benefícios que isso traz aos bipolares, mas…
  preguiça, preguiça, preguiça e preguiça. Agora vou encarar essa
  preguiça, estou disposta, apesar do calorão que tem feito. Quando tive
  \emph{personal trainer}, por duas vezes na vida, foi bem fácil. Vou
  procurar um que caiba no meu orçamento, na Revista da Vila devo
  encontrar anúncios.
\item
  \versal{NADAR}. E~eu adoro nadar! E sou sócia do Pinheiros! Amo aquele piscinão
  da minha infância, mas nunca vou… Burra…
  Preguiçosa… Regina, chega de ser tão preguiçosa e chega de se
  atacar, se atacar, se atacar, como bem diz o Celso.
\item
  \versal{UMBANDA}, \emph{\versal{JOHREI}}, \versal{MEDITAÇÃO}: não, não e não. Acabam me
  desestruturando. Aprendi a duras penas, agora \versal{CHEGA}! Só uma vezinha
  raro em raro. Quero muito ver a Dagui e o Saulo, tô morrendo de
  saudades!
\end{enumerate}

A Filó, minha gata, deu cria há muito tempo e teve quatro gatinhos. Até
hoje eu me arrependo muito e sinto culpa porque não deixei um dos
gatinhos ficarem em casa, fazendo companhia para ela. Ela tem uma vida
muito solitária. Quando chega alguém em casa, ela vem correndo para a
sala ao ouvir a campainha. Ela adora gente. Gato é um bicho de rua, mas
a Filó é uma gatinha doce, meiga e feminina, de apartamento.

Quando eu chego, ela logo tenta sair pela porta e eu tenho que afastá"-la
rápido.

A Filó é a minha mais fiel acompanhante terapêutica. Ela me faz
companhia 24 horas por dia e ainda dorme comigo, entre as minhas pernas,
para onde vai logo que eu deito de bruços para dormir.

Antes disso ela me espera sentadinha na frente da porta do banheiro,
enquanto eu escovo os dentes. Quando eu chego e sento no sofá, ela logo
deita do meu lado para eu coçar a barriga dela. Ela adora receber
agrados. Quem não gosta?

Eu rezo para morrer antes dela porque eu acho que a morte dela seria
insuportável para mim. Eu amo muito essa grande companheira. Essa fiel
companheira. Eu não sei como iria sobreviver sem ela me pedindo atum
\emph{light} para comer o dia todo.

\begin{center}\asterisc{}​\end{center}

\begin{flushright}\textbf{14/01/2011}\end{flushright}


Saí da clínica dia 6, depois de uma consulta com a Dra. Mara, que me deu
alta. Não foi fácil convencê"-la de que eu estava bem e que, naquelas
alturas, a clínica, com o excesso de loucuras que estava abrigando,
estava me fazendo mais mal do que bem. Ficar ali estava insuportável. Eu
já estava bem, mas a loucura à minha volta, sobretudo os gritos daquela
mulher que não saía do surto, me contaminavam. Os gritos da mamãe. Os
gritos descontrolados da mamãe! Os sustos enormes com os gritos da
mamãe. Aquele horror. Aquele medo. Voltava tudo. Era insuportável ouvir
os gritos daquela mulher. Eu precisava, pela minha saúde, sair da
clínica naquele momento. Tive uma conversa boa com o Paulo naquelas
alturas. Ele estava sentindo a mesma coisa e também queria sair.

O Rogério veio me buscar. Eu estava tensa porque ele chegou às 12h30 e o
Zeca falou que ele chegaria às 12h00. É~claro que eu estava prontíssima
e muito ansiosa desde às 11h00. Na hora de sair da clínica sempre me dá
um medão de não sair, de não conseguir, de alguém me segurar lá. Alguém,
seja da terra ou do céu, do inferno ou do paraíso.



\begin{center}\asterisc{}​\end{center}

\begin{flushright}\textbf{}\end{flushright}

\begin{flushright}\textbf{17/01/2011}\end{flushright}


Fiquei arrasada com a notícia que o Hailton me deu ontem. Na última
consulta com o Del Porto, ele chamou o Hailton e lhe disse, em
particular, que não estava conseguindo acertar a dose do Zyprexa comigo.
Se o Del Porto não acertar a dose para me estabilizar, quem vai acertar?

De fato eu estou perto do meu normal, mas não voltei ao meu normal.
Ainda fico um pouco \emph{impaciente} com as pessoas e sou um pouco
\emph{ríspida}. Ou muito.

Eu fico num verdadeiro drama em relação ao meu trabalho espiritual na
Messiânica e na Umbanda. Eu não pretendia voltar a essas religiões. Eu
queria conseguir colocar o espiritual todo na minha arte. Colocar a
minha \emph{mediunidade} na minha pintura e na minha cerâmica. Mas não
estou conseguindo porque gosto de trabalhar à noite e a presença
obrigatória dos acompanhantes tiram a minha privacidade.

Estou com acompanhante desde 20 de dezembro. A~iniciativa de chamá"-los
foi minha, quando percebi que estava muito fragilizada, especialmente à
noite. O~Del Porto achou boa a iniciativa e, por ordem dele, estou com
acompanhantes até agora. Mas neste momento estou detestando ter
acompanhantes, estou super exausta nesse sentido porque ter um
acompanhante é sempre algo muito invasivo. É~um estranho na sua casa.
Meu apartamento é pequeno, eu não tenho um quarto para os acompanhantes,
eles dormem no sofá da sala e, em geral, ficam comigo na sala, vendo
televisão. Ontem eu segui a sugestão da Soninha e falei para a Josi que
eu queria ficar sozinha na sala e pedi para ela ficar lendo revista na
cozinha. Depois eu fui para o computador e fiquei um tempão lendo e
respondendo os \emph{e"-mails}. Assim foi melhor.

A Sara sugeriu que eu proponha para o Del Porto que ele divida o meu
caso com o Carlos Moreira, já que ele não está conseguindo acertar a
dose do remédio. Eu achei uma boa ideia, terei consulta daqui a dois
dias e vou falar com ele.

Em termos espirituais, já fiz de tudo que poderia fazer. Voltei à
Messiânica, e o ministro reconsagrou meu \emph{orikari} que havia caído
no chão. Recebi \emph{jhorey} três dias seguidos e fiz cultos para os
antepassados: para mamãe, vovó Úrsula, tio Chico e Tuxa. Mandei também
rezar missa para eles na capelinha de Nossa Senhora Aparecida, da Vila
Beatriz. Hoje, como senti a presença muito forte da vovó, mandei rezar
uma missa para ela e uma para mim.

Estou fazendo tudo que está ao meu alcance para debelar essa crise.
Tudo. Tudo. Tudo. Tudo! Eu me ocupo com esse trabalho espiritual e acaba
sobrando pouco tempo para eu me ocupar com meus assuntos pessoais.

Hoje, a Preta, que se demitiu num dos ataques agressivos, quando gritei
muito com ela, na mania, veio conversar comigo. Eu intuí que o Zeca já
deveria ter falado com ela por conta própria e verifiquei que estava
certa, porque hoje ele me telefonou e contou que já tinha falado com
ela. Nessa vontade de me ajudar, ele me tratou como uma débil mental.
Alguém que não tem competência de cuidar dos seus problemas sozinho.
Ainda bem que tenho condições de perceber que a vontade dele é de
ajudar, mas é horrível ser tratada como débil mental, como oligofrênica,
como uma criança que não sabe decidir.

Essa doença é horrível. É~muito cruel comigo e com a minha família. Traz
muita angústia para todo mundo. Muito sofrimento. Mas para falar a
verdade, estou me sentindo melhor. As dores na perna causadas pela
lombalgia praticamente cessaram. Acho que foi graças ao \emph{jhorey}.
Eu não tenho a menor dúvida quanto à sua eficácia. Acho também que estou
mais calma, menos impaciente.

Estou mais tranquila e conseguindo organizar a minha casa. Hoje veio o
empalhador para empalhar as três cadeiras da cozinha que estavam
completamente estragadas. Eu \emph{consegui} comprar lâmpadas, e o
Antônio trocou as da cozinha, do banheiro e uma da sala, que um dia caiu
no sofá, misteriosamente, sem essa nem aquela. O~sistema elétrico da
casa ficou bem arruinado com os últimos acontecimentos. É~tudo muito
misterioso… Como eu queria entender todo esse mistério… Já
fui duas vezes até o Campo Limpo e o Templo Cabocla Guaciara estava
fechado. Foi o maior desaponto.

Hoje senti a presença delicada e iluminada da mamãe, no lindo vaso de
rosas que coloquei na sala. Senti perto de mim a \emph{presença boa da
minha mãe}. Daquela mãe que fez aquele maravilhoso jardim em Cotia.
Agora quando a gente vai lá, na casa do Sylvio, a gente sente logo essa
\emph{presença boa}, que tem um \emph{encantamento}. Presença que o
Sylvio chama de ``\emph{radiosa}''.

\begin{center}\asterisc{}​\end{center}

\begin{flushright}\textbf{}\end{flushright}

\begin{flushright}\textbf{23/01/2011}\end{flushright}


Ontem fui ao Templo Cabocla Guaciara pela terceira vez. Nas duas
primeiras, estava fechado. Na hora de ir, no caminho, caiu um grande
temporal, o Ricardo (\versal{AT}) e eu paramos por 10 minutos num posto e depois
continuamos. O~Templo é muito longe, é lá no Campo Limpo. Eu não sei por
que dessa vez eu estava com medo de ir até lá.

Quando cheguei, estava tensa, não estava à vontade e fascinada como
antigamente. O~Templo agora está muito mais organizado. A~gente preenche
uma ficha na entrada e já sabe que médium vai atender a gente. Eu,
graças a Deus, fui atendida pelo Saulo. Eu achava que ele nem
incorporava mais.

O Caboclo dele falou que eu ``não estava trabalhada''. A~médium ao lado
dele fez um ``descarrego'' em mim, passando em todo o meu lado direito
um maço de folhas verdes, que tinha alecrim, acho que também tinha
arruda. O~Caboclo me deu uma vela azul e uma rosa para acender em casa,
uma espiga de milho para deixar secar na cozinha e, quando secar, eu
devo debulhar perto de uma árvore. Esse milho é para garantir que eu
tenha sempre comida em casa. Ele deu também uma flor de girassol, sem
caule, para eu pôr num vasinho. Eu estava muito insegura e a Corina, que
eu conhecia dos velhos tempos, me acompanhou o tempo todo.

Em seguida, eu fui falar com a Dagui, que estava lá fora e parecia uma
rainha naquela cadeira enorme. Parecia, não: ``\emph{ela é uma rainha}''.
Eu contei para ela que há um mês ou 20 dias, eu não sei, eu tenho
sentido a \emph{presença dela e do Saulo}, me orientando para desmanchar
um trabalho. Contei que eu fiz tudo que eles me ensinaram. Roupa cor de
rosa, bolsa cor de rosa. Vasos de rosas cor de rosa. Vasos de rosas
brancas. Guaraná e balas para São Cosme e Damião. Água de coco na
xicrinha para a Rosa. Enfim, vários rituais. Eu nunca tomei tanto
guaraná. A~Dagui foi muito honesta e falou que isso era tudo criação da
minha cabeça. Ela falou que eu deveria ir lá sete sábados em seguida
para me tratar. Eu desanimei. Foi a minha primeira reação. O~desânimo. O~Caboclo também mandou eu tomar três banhos de água de coco misturada com
água mineral e bastante açúcar. Eu desanimei também. Achei que seria
insuportável ficar toda melada nesse calor gigantesco que tem feito, e
depois do banho de ervas a gente não pode se enxaguar, tem que deixar o
banho secar no corpo. O~bom foi que eu resolvi a minha dúvida com ela.
Acho que eu nunca tinha falado com ela de forma tão direta. Eu, pra
falar a verdade, fiquei arrasada porque compreendi que todos aqueles
rituais que eu estava fazendo eram delírios e alucinações minhas, era
pura loucura. Eu percebi que a doença continua muito presente, embora eu
esteja me sentindo razoavelmente bem.

Na volta, o Ricardo e eu paramos para tomar um suco no Senzala e
conversamos bastante. Foi muito bom. Ele disse que não se envolveu no
ritual, que não acreditava e que ia jogar no lixo o milho e a flor que
ele ganhou. Problema dele.

Hoje, sem dúvida, acredito muito mais na Messiânica do que na Umbanda.
Por quê? Eu não sei, fé é uma coisa inexplicável. Mas é claro que a
Messiânica é uma religião mais fácil de seguir. É~perto de casa, eu
posso ir a qualquer hora e ficar o tempo que eu quiser, ministrando e
recebendo \emph{jhorey}. Eu lembrei de um dia em que, saindo com a Rosa
do Templo da Dagui, ela me perguntou se o contato com esses caminhos não
abriam mais esse canal que eu já tenho aberto por causa da doença. Eu
acho que sim. Eu nunca esqueci essa colocação dela, e eu acho que é
sensata.

Ontem, depois de ficar um tempo arrasada, fiquei muito leve, muito feliz
por não ter mais a obrigação de continuar com aqueles rituais
espirituais de ``limpeza'' da casa, tão trabalhosos.

Eu acho que, se tenho necessidade de seguir um caminho espiritual, tenho
que me esforçar para conseguir isso. O~ministro falou para eu fixar
\emph{um dia} na semana para ir à Messiânica e me dedicar a ministrar
\emph{jhorey}. Eu sempre sou muito exagerada e começo a ir todos os dias
e daí me arrebento por causa do exagero e piro… A minha
psique não aguenta. Eu tenho que ir devagar, bem devagar, muito devagar.
Eu não posso querer suprir a carência afetiva que tenho através disso.
Através de frequentar muito a Messiânica, também porque lá encontro
outras pessoas que em geral são muito amáveis, muito generosas. Eu já
fiz muito isso antigamente, tenho essa lucidez agora, só agora. Tenho há
muito tempo a lucidez que a solidão é a minha maior vilã. É~o que me
deixa frágil e vulnerável. A~carência afetiva também. Pra falar a
verdade, acho que sou igual às outras pessoas. Quem não fica frágil e
vulnerável quando sente solidão e carência afetiva? O problema é que eu
também tenho a bipolaridade, que me fragiliza mais ainda, fragiliza
muito.

Eu agradeço a Deus todo apoio que tenho, especialmente da Soninha e do
Fernão. Sem eles, nessas alturas, eu não teria acesso a um ótimo
psiquiatra, a um ótimo psicanalista, aos remédios caríssimos, a
internações em boas clínicas. Eles são muito generosos e amorosos
comigo. \emph{Eles me dão chão}. Eu nunca conheci na vida duas pessoas
tão generosas e maravilhosas. E~eu sei que não é só comigo, é com várias
pessoas. Com o correr dos anos a minha aposentadoria achatou, e eu não
teria mais condições de manter o meu tratamento, então pedi socorro à
Soninha e ao Fernão, que me deram suporte imediatamente.

Eles me ajudaram muito em um dos momentos mais difíceis da minha
vida. Quando entrei em mania, em 1998, eles foram com o Duda até Cotia
para me resgatar. Eles me acompanharam na consulta com o Dr. Jorge
Figueira e, depois, na internação na Clínica Conviver, junto com o
Sylvio e o Rogério. Foi a minha primeira internação. Quando tive outra
mania, em 2000, eles me acolheram na casa deles. Foi ótimo porque assim
não precisei ficar internada numa clínica, que é uma coisa muito dura de
se viver. Eu tinha, então, duas acompanhantes que vieram da Clínica
Parque Julieta. Foi o Del Porto quem indicou. A~Soninha não gostou do
nível das acompanhantes, e eles me deram o incrível presente de ter
acompanhantes terapêuticos vinculados ao Hospital Dia A Casa, que tenho
até hoje e que têm um nível muito superior. Eles também estiveram
presentes na minha internação no final de 2011. Lembro que fiquei
surpresa ao encontrar o Fernão e a Soninha na clínica, quando cheguei lá
com o Paulinho e o Rogério. Fora tudo isso, eles me convidaram muitas e
muitas vezes para passar o final de semana ou os feriados na fazenda com
eles, que é um sonho de lugar, um verdadeiro paraíso. É~uma delícia
ficar lá, sobretudo quando a sobrinhada vai junto.

Eu adoro os filhos deles, talvez porque tenha ficado tomando conta deles
quando eram pequenos, em 1970, quando a Soninha e o Fernão viajaram por
dois meses. Ficou um elo muito forte nos ligando. Eu adoro a Soninha e o
Fernão e sinto que eles gostam muito de mim. Existe um afeto muito forte
entre nós. Esse afeto me dá chão.

Eu agradeço a Deus, também, a generosidade do Rogério, que depois que eu
saí da clínica, em 2008, me ligou \emph{diariamente durante um ano e
meio} para me acompanhar. Até hoje ele liga umas duas vezes por semana
para saber de mim, e a gente bate um bom papo. Ele sempre foi
extremamente dedicado durante as minhas internações, sempre me visitando
várias vezes e levando sempre coisas gostosas para eu comer.

Eu agradeço a dedicação do meu irmão Zeca, que me ajuda muito deixando a
questão financeira ao encargo da Maria, sua secretária. É~para ela que
eu entrego os recibos dos médicos e as notas dos medicamentos para ser
ressarcida. Depois de uma mania, o Zeca sempre coloca a secretária à
minha disposição para ela me ajudar a me reorganizar financeiramente.

Eu agradeço a ajuda de todos os outros irmãos e dos amigos que torcem
por mim, que de vez em quando conseguem \emph{me tirar de casa} para ir
ao cinema ou a uma exposição.

\begin{center}\asterisc{}​\end{center}


\begin{flushright}\textbf{09/02/2001}\end{flushright}


Eu fui com o Rodrigo (\versal{AT}) ver a exposição do Baselitz e fiquei
encantada. Eu me afinei completamente com o trabalho dele, cheio de
gestos, de emoção violenta. Os quadros são enormes, talvez 2,5m por 5m,
eu não sei. São coloridíssimos, em cores fortes, mas tem uma série em
branco e preto que é maravilhosa. Eu andava atrás do Baselitz, até
comprei um livrão dele na Livraria Cultura, mas não estou achando aqui
em casa. Estou com medo de ter esquecido na própria livraria. Os \versal{ECT}'s
prejudicaram muito a minha memória, mais do que eu imaginava. Esqueço
muito o que eu faço.

Sinto uma tristeza enorme dentro de mim. Uma daquelas profundas, porque
a ``eletroconvulsoterapia não funcionou''. O~tiro saiu pela culatra. Em
vez de estabilizar, depois do \versal{ECT}, eu tive uma mania que resultou numa
internação. Fiquei internada no \emph{Réveillon} e no dia do meu
aniversário! Eu tenho um barril de choro para chorar e não consigo
chorar. O~choro fica entalado. Só ontem caiu a ficha, quando o Del Porto
falou que não me faria mais \versal{ECT}'s porque causam mania. Ele falou que eu
tenho razão em estar cética e que eu preciso ter ``determinação'' na
luta contra a doença. Ele mudou a dosagem do Zyprexa, abaixou, e deu
Wellbutrin \versal{SR} também. Eu estou me sentindo desamparada. O~desaponto
porque o \versal{ECT} não funcionou é enorme, gigantesco.

Eu estou tentando resgatar a minha garra e o meu espírito de luta.
Ontem, pelo menos, fui à análise, de táxi, mas fui. Não consegui ir ao
Rodrigo Naves nem à Sara. O~Rodrigo, às 9h30 da manhã, é impossível para
mim. Já cancelei e entrei na fila de espera da turma da noite. Hoje
acordei melhor, \emph{consegui} tomar banho, almoçar e ir ao banco fazer
os pagamentos. Na saída, dei um belo raspão no carro, que é novinho, uma
judiação. Os remédios… Eu estou disposta a lutar o máximo
possível para sair dessa deprê. Eu não quero mais ficar em ``internação
domiciliar'', como já fiquei três anos da minha vida. Três anos
completos e perdidos. Outra coisa boa é que pedi para o Del Porto
suspender as enfermeiras da noite e ele concordou. Eu não suportava mais
a presença delas aqui. É~muito invasiva e a gente se sente cada vez mais
doente com a presença delas. É~um saco! Garra, garra, garra. Eu preciso
ter garra. Resgatar a Regina guerreira que enfrenta com coragem os
problemas, as dificuldades, as inseguranças, os medos. Eu tô insegura,
eu tô frágil, eu tô com medo, mas eu tô encarando a real. Estou lutando
para não ter acomodação física nem psíquica para não submergir…
Acho que é o único caminho para sair dessa.

\begin{center}\asterisc{}​\end{center}


\begin{flushright}\textbf{22/05/2011}\end{flushright}


Agora faz um pouco mais de um mês que eu estou bem. Finalmente a
estabilidade chegou, graças à sabedoria do Del Porto, e ficou.

Há 15 dias parei de tomar Pondera, um antidepressivo, e na semana
seguinte o Del Porto suspendeu o Wellbrutrin. Fiquei só com o Depakote,
1000mg diários, com o Zyprexa, 5mg diários, ``estabilizadores'' de
humor, e com o Dormonid, 20mg diários, para dormir.

Eu engordei 8kg por causa do Zyprexa, no início do ano, quando eu estava
em mania; na clínica, eu cheguei a tomar 30mg. É~uma dose de elefante, é
a maior dosagem permitida! Porém, quando o Del Porto foi
``gradativamente'' reduzindo esse remédio, eu comecei a naturalmente
perder peso, perdi 5kg. Mas, depois, ele voltou a aumentar de 2,5mg para
5mg, porque eu tive uma leve ameaça de mania e daí eu engordei de novo.
Esse remédio dá uma larica infernal por açúcar. Quase todas as noites eu
levanto e vou para a cozinha comer doce. Quando acordo no dia seguinte,
encontro no meu criado"-mudo um pratinho, um garfo, embalagens de bala
Toffee etc. Finalmente eu me convenci de que nunca mais vou voltar a
pesar 80 kg como antigamente, estou com 99,7kg. Fiz uma faxina e dei
todas as minhas roupas antigas mais bonitas para a Vera, que adorou.
Fiquei só com o que me serve hoje ou vai servir, quando eu conseguir
chegar aos 90kg. Eu ainda tenho esperança de conseguir isso! Em breve!

Tenho aproveitado esse ``estado de saúde'' para reorganizar a minha
casa, que ficou três anos abandonada. Nesses três anos eu fui acumulando
e acumulando coisas sem jogar nada fora. Agora aproveitei uma ``poupança
involuntária'' que fiz no Banco do Brasil, fiquei sem gastar na
depressão, e acumulei um dinheiro. Já comecei a arrumar tudo aqui em
casa. Troquei de geladeira, de televisão e vou comprar um computador
novo. O~meu tá bem velhinho, já tem uns 10 anos. Finalmente vou colocar
um \emph{Skype} com câmera para poder conversar bastante com a Tina.
Esse é um plano bem antigo que eu nunca consegui botar em prática, agora
vai dar. Comprei também uma torradeira nova, uma sanduicheira e uma
grelha enorme, redonda, maravilhosa. Outro dia fiz uma faxina com a
Preta nas minhas alfaias. Dei três lençóis usados para ela e três para o
Antônio. Descobri que tenho um monte de toalhinhas de bandeja lindas e
pus em uso, assim como paninhos bordados para pôr em cestinhas com pão.
Encontrei várias tapeçarias lindas, que trouxe do Peru, e aproveitei
para pedir para o Antônio pregar uma no quarto do computador e outra na
sala, ao lado da gravura do Jardim, uma árvore maravilhosa.

Estou adorando fazer tudo isso, minha casa vai ficar linda e em ordem de
novo. Aproveitei para mandar lavar o tapete da sala, reformar o sofá,
tingir de laranja os tapetes do meu quarto. A~Preta e eu descobrimos uma
colcha linda de filé, que eu trouxe de Natal, e nós a colocamos na cama,
ficou demais. Na cabeceira da cama, acima, eu já tinha botado umas três
toalhinhas de filé, uma espécie de tapeçaria que eu comprei há séculos
numa viagem a Alagoas, quando subi o Rio São Francisco, faz uns 30 anos.

Teve um dia em que faxinei com a Preta a mesa do quartinho para deixá"-la
livre para o Sylvio desenhar. Fiquei boba com a quantidade de pincéis
que ajuntei ao longo da vida, devo ter quase uns cem! Joguei muita coisa
fora nesse dia. Ainda falta pôr ordem em duas prateleiras e no resto do
quartinho. Tem até coisas que eu comprei na mania gigantesca de 2008 e
que eu nem revi! Vou ter um monte de surpresas…

Ontem teve o casamento do Carlinhos, que foi, para variar, uma festa
linda na casa da Soninha e do Fernão. Tinha umas 400 pessoas, uma moçada
linda que levou os seus filhinhos pequenos. Eles ficaram, é claro, perto
da casinha onde havia duas mesas postas para eles almoçarem em um
bufezinho. O~casamento foi lindo, muito alegre, alto astral. Eu estava
tão bem, que até dancei \emph{rock}, não resisti! Fui com um casquinho
de \emph{tweed} chiquésimo, que comprei na Erica's, uma calça preta e
botinhas. Botei uma flor vermelha no casaco, que é branco e preto.
Graças a Deus, estou recuperando o meu charme e a minha feminilidade, já
não era sem tempo, falta só emagrecer. Mas nesses dias gelados que tem
feito, é impossível fazer regime. Eu não consigo. Não consigo mesmo.

\begin{center}\asterisc{}​\end{center}


\begin{flushright}\textbf{12/06/2011 -- 18h30}\end{flushright}


A Preta se demitiu, e eu começo uma nova vida com a Cida amanhã. Espero
não repetir com a Cida os erros que cometi com a Preta. Espero não
confundir de novo ``empregada'' com ``amiga''. \emph{Dar limites
claros}, esse é o segredo para qualquer relação ser boa e dar certo. Não
foi fácil aceitar a demissão da Preta, depois da enorme dedicação que
ela teve por mim, nestes últimos três anos. Ela é um verdadeiro ``cão
fiel'', tem uma fidelidade canina e sempre gostou muito de mim, e eu
dela, ela foi uma verdadeira mãe para mim. Nos últimos dois meses,
porém, quando fiquei bem e retomei o papel da dona de casa, ela ficou
muito irritada e enciumada. Começou uma verdadeira luta de poder, ``taco
a taco''. Eu tive crises de irritação com ela por causa disso. Crises
pontuais que duravam alguns segundos apenas. Mas eu não admitia
interferência e ia ocupando na marra o meu lugar na casa e tinha um
grande prazer nisso! Um dia ela insinuou que ganharia muito mais como
faxineira. Perguntei se a questão era salário, ela disse que sim e pediu
R\$1.600,00 reais. Eu conversei com o Zeca e ele me aconselhou a ficar
com ela, porque existe uma falta enorme de mão de obra na praça. Até
uruguaias estão vindo trabalhar como domésticas no Brasil. Eu então
aumentei o salário para R\$1.600,00, o que é um absurdo, minhas amigas
pagam entre R\$750,00 e R\$850,00. Juntando o salário mais a condução dá
mais de R\$2.000,00 reais, é um valor mais caro que a minha análise! Se
bem que o Celso sempre me deu um puta desconto. Eu já tinha até
encontrado uma pessoa quando na semana seguinte ela pediu demissão de
novo e não quis de jeito nenhum falar por quê. O~clima na casa, o mau
humor recíproco ficou insustentável, até que um dia ela disse que queria
ir embora porque eu fico muito irritada com ela, e também que eu quero
uma empregada que saiba ler etc. Dessa vez eu não pedi para ela ficar,
eu não implorei, que era o que eu faria se estivesse deprimida, e nem
propus um aumento maior de salário, o que seria absurdo.

Finalmente marquei com quatro candidatas para entrevistá"-las no sábado.
Na realidade, só vieram duas, e eu escolhi a Cida porque ela tem uma
referência de seis anos que eu pude checar na hora. Ela também trabalhou
um mês na casa da Mazinha como faxineira, e essa é uma ótima referência.
Amanhã ela começa, espero que dê tudo certo! Estou com muita vontade de
reassumir as responsabilidades que são minhas nesta casa, em vez de
ficar delegando, delegando e delegando… Vou até fazer um roteiro
do trabalho semanal para ela, como sempre fiz nos tempos em que eu tinha
saúde e a empregada sabia ler. Nem acredito que vou ter uma empregada
alfabetizada! Que maravilha! Poder selecionar uma receita e dizer:
``Experimente fazer''.

É muito doloroso conviver no dia a dia com uma pessoa que não aguenta a
sua saúde, o seu bem"-estar e fica torcendo pela sua doença. Torcendo
para que você fique deprimida, frágil e dependendo cada vez mais dela.
Assim ela volta a ter poder quase total sobre você, e é disso que ela
gosta. É~assim que ela se sente importante e se afirma. Ela se acha
melhor, muito melhor que você, pois afinal ela é saudável e você apenas
uma doente dependente, frágil, muito frágil, que precisa de ajuda para
tudo. Eu acho que vai ser muito bom conviver com a Cida, que nunca me
viu deprimida e nem em mania. Espero que por um bom tempo ela não tenha
que conhecer a ``Regina bipolar''. Estou bem e como o Del Porto diz:
``\versal{FORA} \versal{DAS} \versal{CRISES}, \versal{VIDA} \versal{NORMAL}''. Esse pensamento foi um dos melhores
que ele me disse nesses 23 ou 24 anos em que sou paciente dele. O~outro
é: ``Não acredite na leitura depressiva da sua vida''. Esse pensamento
me ajuda muito nas depressões.

Na verdade, estou bem, mas há dois dias liguei para o Del Porto para
falar das minhas pequenas crises de irritação, e ele dobrou o Zyprexa de
2,5mg para 5mg. Estou bem porque estou de olho vivo, prestando muita
atenção no meu processo, no ``como me sinto e me percebo'' no dia a dia.
Não tem outro jeito, só eu posso detectar uma ameaça de crise. Ainda bem
que, depois de 28 anos de doença, estou bem treinada para perceber os
sinais, às vezes muito sutis, de um início de crise. Sendo medicada a
tempo, dá para segurar a crise, e, para falar a verdade, há anos eu me
``automedico'' e ligo para o Del Porto só para ele conferir. Na maioria
das vezes eu acerto.

Na sexta"-feira passada, por exemplo, saí do Celso e parei no Bistrô St.
Honoré para tomar um lanche. No começo eu ainda estava bem, mas aos
\emph{poucos} foi me dando uma tristeza profunda, profundíssima, um
verdadeiro e inexplicável luto. Eu pretendia ir ao \emph{shopping} para
comprar dois \emph{jeans}, mas desisti, não tive ânimo, resolvi voltar
para casa. Eu preciso muito dos \emph{jeans} e estou adiando essa compra
há séculos. Mas, naquelas alturas, eu não via a hora de chegar no meu
cantinho e me sentir abrigada, protegida. Foi difícil porque levei horas
para achar um táxi, estou a pé há uma semana, mandei consertar o carro.
Ficar a pé em São Paulo, hoje em dia, é um verdadeiro inferno, existem
pouquíssimos táxis disponíveis. Não sei por quê.

No sábado fui ficando em casa até que arranjei coragem para ir ao
\emph{shopping}. Estava meio desanimada, e aí a Mazinha me ligou
convidando para ir à casa dela ver umas coisas que a Gaía trouxe da
Índia. Como adoro coisas indianas, resolvi ir. Para variar comprei
muito:

\begin{itemize}
\item
  uma \emph{pashmina} de \emph{cashmere} bege xadrezinha de marrom;
\item
  um pano enorme de seda vermelha com um bordado vermelho e colorido nas
  extremidades que é usado nos casamentos;
\item
  uma almofada rosa"-choque linda;
\item
  uma veste de algodão rosa"-claro, bem levinha.
\end{itemize}

Estavam lá a Fátima Golan, a Fu, a Marina, o Zeca. Eu fui logo escolher
o que queria e conversei bastante com a Gaía. Na verdade, tudo era meio
caro, mas, como eram produtos especiais, valia a pena. A~Gaía esqueceu
de incluir na minha conta uma \emph{pashmina} azul"-claro, que eu adorei,
e eu vou tentar falar com ela para poder comprar.

Eu bebi umas três doses de uísque com muita pressa. Na verdade, quando
tenho sentido essas crises de irritação misturadas com angústia, tenho
tomado umas doses de uísque. O~Del Porto falou para não fazer isso de
jeito nenhum, porque pode precipitar uma crise, mas… Desde que me
tornei uma bipolar nunca parei de beber, só que fui diminuindo bastante.
Atualmente, na prátcia, só bebo socialmente.

Hoje eu tinha programado ir ao Museu do Negro com o Ricardo e depois
assistir a um \emph{show} de \emph{jazz} que haveria ao ar livre, no
Ibirapuera. Acordei completamente desanimada, sem energia, tô de pijama
até agora. O~Ricardo veio até aqui e batemos um bom papo. Suspeitei o
advento de uma depressão e, por precaução, já tomei os antidepressivos:
Wellbutrin e Pondera. Preciso prestar muita atenção para perceber se
eles não estarão desencadeando uma mania. Eu não tenho sossego mesmo,
mas sei, com clareza, que com a idade as crises são cada vez mais
frequentes. Olho vivo, Regina! A vida tá muito boa para ter uma crise
agora! Chega de crises!!! Chega de sofrer, sofrer e sofrer!!! \emph{Vida
nova!}

\begin{center}\asterisc{}\end{center}

\begin{flushright}\textbf{}\end{flushright}

\begin{flushright}\textbf{16/06/2011}\end{flushright}


Hoje é um dia histórico na minha vida: formalizei a demissão da Preta.
Ela veio aqui e assinou o pedido de demissão e o recebimento das verbas
rescisórias. Dei R\$500,00 de gratificação pelos bons serviços
prestados, e aí ela disse que ia chorar… mas não chorou, não!

\begin{center}\asterisc{}\end{center}

\begin{flushright}\textbf{29/06/2011}\end{flushright}

\epigraph{``Todo ser em movimento é perigoso. Todo ser que se transforma,
incomoda.''}{(Paulo Leminski)} 

É exatamente isso o que estou vivendo. Este é o terceiro mês em que
estou bem, estável, e isso está incomodando muita gente. As pessoas
querem que eu continue ocupando o ``lugar de doente'' de qualquer jeito.

Estou organizando uma temporada para Camburi e percebo com clareza que
muitas pessoas estão me estranhando. Em parte com razão, porque estou
mandona e autoritária, um pouco impaciente e, às vezes, irritada. Em
parte é natural que eles estranhem porque fiquei em crise de mania ou
depressão nos últimos três anos corridos sem que eles convivessem
comigo. Em janeiro fiz um esforço gigantesco para ir para Camburi,
porque eu estava deprimida. O~Celso e o Del Porto recomendaram que eu
fosse. Eu já tinha convidado 10 pessoas, logo depois do \versal{ECT}, em
dezembro, quando achei que estava ótima, mas depois deu pra perceber que
eu já estava em mania. O último \versal{ECT} foi em 2 de dezembro, e a mania
começou exatamente aí, deu pra verificar isso quando chegou o extrato do
meu cartão de crédito. Fiquei internada de 26 de dezembro a 6 de
janeiro, e essa internação foi um verdadeiro inferno. Quando eu saí da
clínica, porém, fiquei bastante deprimida.

Lá em Camburi, quando eu acordava, fumava todos os dias uns seis ou sete
cigarros até de ter coragem de pôr o maiô, descer e encontrar os amigos.
O~Antônio foi maravilhoso, ele tomava café da manhã comigo e depois me
convidava para ir à piscina. O~resto do pessoal, nessas alturas, já
estava no aperitivo. De qualquer modo, o calor e o carinho das pessoas
me fizeram muito bem, mas em muitos momentos eu me senti um fantoche, um
verdadeiro fantoche que estava lá só para assegurar aquela temporada
para os amigos. Não foi à toa que, quando cheguei em São Paulo, tive uma
crise suicida. O~Rodrigo era o acompanhante que estava comigo, e ele foi
completamente incompetente. Eu chorei muito por causa dos fortes
impulsos suicidas, e ele só falava abobrinhas. No final da tarde, depois
de muita angústia, eu lembrei de ligar para o Del Porto no celular,
porque era sábado. Ele sacou no ato o que estava acontecendo e disse:
``Regina, você vai voltar a ter estabilidade'', e me mandou tomar um
Pristiq (antidepressivo) naquele momento, outro na segunda"-feira, e ir
ao consultório dele na segunda"-feira. Foi o que eu fiz.

Na consulta de segunda"-feira, eu fiquei arrasada quando o Del Porto
disse que não aplicaria mais o \versal{ECT} em mim, porque em mim ele desencadeia
mania. Eu me senti completamente ``a pé''. Pareceu"-me que a medicina, a
psiquiatria, não tinha nada mais a me oferecer. O~Del Porto mesmo falou
para o acompanhante que eu tinha razão em estar cética, mas disse para
mim que eu precisava ``ter determinação''. Eu continuei deprimida e
desesperada, me sentindo absolutamente ``sem saída''. \emph{A única
saída que eu enxergava era o suicídio}. Mas eu não tinha coragem.

A Tina estava aqui nessa época e me ajudou muito, foi muito carinhosa e
me fez muita companhia. Eu me sentia super bem quando estava com ela.
Ela veio ao Brasil para me ver, porque eu tinha sido internada, e para
ver a Soninha, que começava a ter um probleminha de memória.

Eu fiz um esforço gigantesco para reagir àquela depressão. Por
insistência da Sara, pedi uma segunda opinião para o Carlos Moreira. Ele
foi um estúpido. A~Bia, minha sobrinha, e o Zeca me acompanharam à
consulta. Primeiro, conversei durante uma hora com o Carlos Moreira.
Levei o meu texto: ``Efeitos colaterais: a outra face da moeda'' para
ele ler. Ele ficou interessadíssimo e foi anotando todos os nomes dos
remédios que eu citava no texto. Depois eu contei para ele em detalhes
as crises desses últimos terríveis e sofridos três anos. Ele então
chamou a Bia e o Zeca para o consultório. Nesse momento ele passou a me
ignorar, como se eu fosse uma débil mental e ficou se dirigindo só à Bia
e ao Zeca. Ele nem me olhava, afinal, ali eu era a ``doente mental'', a
``louca'', e ele só conseguia ver isso em mim. Eu fiquei puta, mas não
me manifestei. Tive muita paciência.

Eu fiquei surpresa com a participação do Zeca na consulta. Percebi que
ele tinha gravado o nome e a dosagem de vários remédios que eu já tinha
tomado. Em 2010, no início de 2010, ele tinha me acompanhando junto com
a Soninha a uma consulta com o Fernando Asbarh, amigo do Duda, para quem
eu fui pedir uma segunda opinião por sugestão da Soninha, que ficou
muito preocupada comigo no \emph{Réveillon} que passei na fazenda. Fui
sozinha à primeira consulta, expliquei o meu caso, e o Fernando Asbarh
pediu que eu fosse a uma segunda consulta com um membro da família. Ele
também me tratou como a ``doente mental'', a ``louca''. Achei ele um
porre e deixei bem claro que eu nunca tinha pensado em mudar de
psiquiatra, defendi o Del Porto com unhas e dentes. Ele foi muito
antiético e forçou a barra insistentemente, de modo vulgar, para eu sair
do Del Porto e me tratar com ele. Ele disse para a Soninha e o Zeca que,
como eu estava bem naquele momento, aquele era o melhor momento para
experimentar uma combinação de drogas que ele tinha em mente. Fiquei
horrorizada, eu tinha demorado tanto tempo para ficar bem, e ele queria
mexer nas drogas logo nessa circunstância? Que idiota!!!

Para a maioria dos psiquiatras, nós, bipolares, doentes mentais, somos
ratos de laboratório, interessantíssimos para eles testarem seus
remédios, as combinações de remédios que eles inventam, é claro. E~eles
ficam super excitados com essa possibilidade… Até hoje eu não
entendi por que ele fazia tanta questão de me tratar. Para competir com
o Del Porto ou por que eu sou de um meio social diferenciado, o que ele
percebeu quando a Soninha me acompanhou à consulta? Nunca vou ter essa
resposta e, pra falar a verdade, isso nem me interessa tanto. Continuei
no Del Porto, é claro. E~o que acho mais fantástico é que o Del Porto
nunca fez com que eu me sentisse uma ``doente mental, louca'', nas suas
consultas. Ele sempre me considerou uma pessoa inteligente, culta e que
tem o azar de ter essa doença. Nas consultas, quando estou razoavelmente
bem, nós discutimos política, literatura etc. O~Del Porto é muito culto
e tem uma memória de elefante. Eu acho isso importante porque as pessoas
cultas têm mais parâmetros para pensar, para raciocinar, para refletir,
para tomar decisões. Voltando ao Carlos Moreira: detestei ele. No
primeiro momento, ele até me pegou, mas depois achei absurdas suas
sugestões de tratamento. Ele falou muito animado que tinha cinco
sugestões para me dar, ou melhor, para dar para o Del Porto. Duas foram
eliminadas na hora porque eram de remédios que eu já havia tomado.
Elesou não surtiram efeito, ou desencadearam muitos efeitos colaterais.
Como eu lembrava dos nomes desses remédios, deu para eliminar essas
sugestões no ato. A~terceira era sugestão já me havia sido dada por
outro remédio, que o Del Porto, ao verificar o meu histórico, constatou
que não dava para eu tomar devido aos efeitos colaterais terríveis que
já tinha me causado. A~quarta sugestão foi que eu tomasse sempre uma
superdosagem de hormônios da tireoide, porque eles fazem os
antidepressivos surtirem mais efeito. Uma violência. A~quinta sugestão
do Carlos Moreira foi que o Del Porto me ``entupisse'' de Zyprexa. Ele
usou esse termo mesmo: ``entupisse'' Depois eu deveria me submeter a
sessões de eletroconvulsoterapia para sair da depressão. Ele achou que
essa solução seria interessante porque acreditava que eu tinha reagido
muito bem ao \versal{ECT}. Na verdade, reagi tão bem que entrei em mania antes
mesmo de receber a última aplicação de \versal{ECT}. Perguntei se ele considerava
que o meu caso era grave, e ele respondeu que era, mas que eu não devia
perder as esperanças… Achei essas duas últimas sugestões
absolutamente ``brutais'', mas disse que, no caso de ter que optar entre
as duas, eu preferiria a ``superdosagem'' de hormônio da tireoide, que
teria um efeito colateral ótimo, o emagrecimento. Estou com 106kg devido
ao efeito colateral do Zyprexa, que é dar apetite gigantesco, sobretudo
de açúcar.

A Tina, minha irmã que mora em Londres, ficou uma fera com essa solução
de hiperdosagem de hormônio. Ela me telefonou e me deu a maior dura,
disse que papai, que foi médico por seis anos, e depois biólogo até os
92 anos, era completamente contra essa estória de mexer com os
hormônios. Ele achava que era muito perigoso porque, ``ao mexer com os
hormônios de uma pessoa, você está mexendo com tudo''.

Eu já estava exausta de todo esse percurso: mania em dezembro,
internação, depressão e, quando fui ao Del Porto, falei pra ele
claramente \versal{QUE} \versal{NÃO} \versal{QUERIA} \versal{QUE} \versal{NINGUÉM} \versal{MAIS} \versal{MEXESSE} \versal{COM} O \versal{MEU} \versal{CORPO} E \versal{COM}
A \versal{MINHA} \versal{CABEÇA}. Fui muito firme e direta ao falar isso, e ele ficou
surpreso e \versal{EMBASBACADO}. Como eu estava relativamente bem ao ter essa
consulta, apesar da depressão, \versal{PUDE} \versal{TER} \versal{UMA} \versal{POSTURA} \versal{ATIVA}, \versal{CLARA} E
\versal{AFIRMATIVA}. O~Del Porto, que já estava para me prescrever dosagem de
hormônio tireoidiano, acatou a minha decisão e resolveu continuar com o
tratamento tradicional, que é uma combinação de antidepressivos: \versal{PONDERA}
E \versal{WELLBUTRIN}, mais 2,5mg de Zyprexa e 1000mg de \versal{DEPAKOTE} e 20mg de
\versal{DORMONID}. Esses três últimos remédios eu já vinha tomando há tempos.
Ainda bem que ele tem jogo de cintura e percebeu como eu estava me
sentindo. A~combinação de antidepressivos funcionou, e acho que em 20
dias ou um mês eu saí daquela depressão prolongada e passei a me sentir
muito bem, como antigamente. Esse bem"-estar já dura com certeza três
meses: abril, maio e junho. Tive, no entanto, que \versal{SEGURAR} \versal{DUAS} \versal{PEQUENAS}
\versal{AMEAÇAS} \versal{DE} \versal{MANIA} \versal{NESSE} \versal{PERÍODO}. Como percebi A \versal{TEMPO}, liguei para o Del
Porto, ele me medicou, e eu fiquei bem e estou bem até agora. Percebo,
no entanto, que continuo um pouco \versal{IMPACIENTE} e às vezes \versal{IRRITADA},
aspectos que são sinais de mania. É~por isso que parei de tomar o
\versal{WELLBUTRIN} por minha conta mesmo. Hoje de manhã não tomei o \versal{PONDERA}
porque o Del Porto já havia suspendido há tempos.

Um dia, eu deixei um recado \versal{DESESPERADO} no celular dele porque não
estava aguentando \versal{OS} \versal{EFEITOS} \versal{COLATERAIS} \versal{DA} \versal{RETIRADA} \versal{DO} \versal{ZYPREXA}. Ele
sacou que era isso que estava acontecendo e me mandou retirar o Zyprexa
mais devagar, tomando um dia sim, um dia não, e aí eu melhorei. Os
efeitos colaterais eram enjoos terríveis, tonturas, falta total de
energia, entre outros. Cheguei a ir ao otorrinolaringologista por minha
conta mesmo para ver se eu estava com labirintite, mas o resultado do
exame do ouvido foi negativo. Gastei uma grana e fui ao ``pai de
santo'', o Marcelo, que eu adoro, para ver se todo aquele mal"-estar era
devido a algum ``egum'' (encosto) ou a algum trabalho que alguém tivesse
feito para mim. Ele jogou os búzios e falou que estava tudo bem na área
do espiritual, só me mandou tomar uns ``banhos de ervas'' para melhorar.
Os banhos, que a Preta fazia para mim, me ajudaram muito a me sentir
melhor. Lembro de algumas ervas desses banhos:

\begin{itemize}
\item
  anis"-estrelado,
\item
  canela,
\item
  erva"-doce… o resto não lembro mais.
\end{itemize}

Agora que estou bem, tenho que me haver com as pessoas que estão ao meu
redor. A~amiga que está verdadeiramente feliz, porque saí da depressão,
é a Lia. Ela se interessa muito pela doença e pergunta muito sobre o
assunto. Eu adoro esse interesse dela. Todo mundo precisa de um \versal{BODE}
\versal{EXPIATÓRIO}. No almoço de segunda"-feira, na casa da Soninha, percebo que
algumas pessoas também estão me estranhando muito, sobretudo a Adelaide,
que é amiga da Verita e da Gilda. Ela tem me olhado com olhos
arregalados, como se eu fosse um E.T\, que baixou naquele almoço, naquele
instante. Talvez eu esteja muito paranoica, e essa reação das pessoas
talvez possa ser até um cuidado comigo mas… o que diria o
Gaiarsa… e o Paulo Leminski… Eu não vou esquecer nunca
aquele vídeo magistral do Gaiarsa sobre esse assunto, a família! Aquelas
imagens continuam até hoje muito claras na minha cabeça!

Como eu sei que a minha doença é incurável, eu estou tentando aproveitar
ao máximo essa fase boa. Tenho pintado trabalhos enormes de 2m X 1m (na
mesa), e quando o Jardim veio aqui me dar uma supervisão falou que eu
evoluí muito. Ele acha que agora sou de fato uma pintora profissional.
Fiquei super feliz! Exultei!!! Ele disse que agora preciso de uma
galeria para expor.

Tenho desenhado muito também e aprendi a usar a câmera fotográfica que
comprei em dezembro e é ótima. Vou levar para Camburi. Retomei as aulas
com a Carmen e aprendi até a mandar mensagem de texto pelo celular.
Aprendi a fotografar \emph{sozinha} pelo celular. Nas próximas aulas
quero aprender a passar \emph{e"-mails}. Instalei o \emph{Skype} no
computador e na semana passada, com a ajuda da Carmen, falei um tempão
com a Tina. Nós nos vimos porque nossos computadores também têm câmeras.

Tenho clareza de que nem todo mundo vai ficar feliz porque estou bem. Já
aprendi o suficiente com a vida. Por isso serei comedida e vou moderar o
meu entusiasmo, que é devido à \versal{ESTABILIDADE} que alcancei… Tomara
que ela dure bastante, vou fazer todo o possível para mantê"-la. Vou
continuar muito \emph{atenta} ao meu processo.

\begin{center}\asterisc{}\end{center}


\begin{flushright}\textbf{18/11/2011}\end{flushright}


Ledo engano, essa fase de estabilidade, que eu achava que iria durar um
ano pelo menos, acabou em três meses. Bem que o Del Porto falou que com
a idade as crises seriam mais frequentes, mas eu não imaginava que
fossem tão frequentes. Frequentes demais, frequentíssimas!

Eu me senti pega de calça curta e fiquei revoltada. Não consegui ainda
fazer as coisas sozinha: dirigir, ir ao banco, ir até a análise. Quando
me vi muito insegura ao guiar na Faria Lima, com aquele monte de
motoqueiros azucrinando, percebi que os meus reflexos não estavam bons e
parei de dirigir. Passei a andar de táxi, eu sempre faço isto. Eu adoro
andar de táxi porque dá para observar bem a paisagem, reparar nas lojas
e nos restaurantes da Vila Madalena, onde moro. Pena que o táxi seja tão
caro. Houve dias em que pedi a ajuda de acompanhantes para fazer as
coisas. Eu queria continuar a fazer tudo sozinha, mas não conseguia mais
e ficava muito chateada e frustrada com isso. Os acompanhantes são uma
bênção, sem eles a depressão fica cada vez mais árida e mais difícil de
suportar. Eu vivi isso bem durante os sete meses de depressão do ano
passado, foi terrível, foi demais! Foi muito sofrido. Muito. Foi um
verdadeiro filme de horror… horror…

Durante os primeiros 20 anos, eu vivi a doença muito sozinha e foi
duríssimo. Duríssimo mesmo. Como a família não se frequenta, acho que
houve crises de depressão que eles nem ficaram sabendo. Nas manias, eles
sempre estavam presentes. As manias são mais preocupantes. Depois da
mania de setembro de 2001, a Soninha me proporcionou ter os
acompanhantes que ela descobriu através de um contato com o Valentim
Gentil, um excelente psiquiatra. Eles me acompanham há 12 anos já!

No início dessa depressão, eu regredi bastante, comecei a me sentir como
um bebê que queria que alguém trocasse as fraldas, o cueiro e desse
mamadeira. Queria um enorme colo de mãe. Sentia muita vontade de ter um
colo para me aninhar, me proteger, receber carinho e calor. E~eu tenho
62 anos! Mas depressão é assim.

Eu já tinha tido essa experiência em outras depressões. Às vezes, no
começo, é assim mesmo. Como não havia colo nenhum para me abrigar,
resolvi deixar a minha cama bem gostosa e comprei quatro jogos floridos
de lençóis que eu estava precisando mesmo, dois têm \emph{laise} na
fronha e na vira. Eu me sinto uma princesa quando durmo neles. Aquela
loja fantástica, O Mundo do Enxoval, fez uma liquidação e eu aproveitei.
Realizei um sonho antigo, comprei uma toalha branca para pôr na minha
mesa, que fica oval quando eu coloco mais uma tábua. A~toalha é linda!
Pela primeira vez na vida comprei guardanapos de tecido branco para
poder receber o Fernão e a Soninha. Há muito tempo eu queria recebê"-los,
há muito tempo eu queria ter guardanapos de tecido.

Nos primeiros quatro finais de semana da depressão, eu fiquei trancada
em casa. Não tinha a menor força para sair. No primeiro, estava sozinha
e me sentindo tão fraca que cheguei na cozinha para tomar os remédios e
o café e, simplesmente, não consegui. Meu corpo parecia feito de
algodão, voltei para a cama e dormi mais um tempo. Quando acordei,
liguei para o Del Porto, porque fiquei preocupada, eu não me lembrava de
ter me visto assim antes. Ele disse que essa falta de energia era
própria da depressão e que aos poucos, com a ação dos remédios, ia
melhorar. Eu sosseguei. No domingo estava um pouco melhor. Achei que
esse cansaço todo vinha do esforço enorme que tive que fazer durante a
semana para encarar a vida. Fiquei muito tensa e me dizia o tempo todo:
``Eu vou conseguir'', ``Eu vou conseguir''. E~ia mesmo, a grande custo
consegui fazer o que eu tinha que fazer.

Depois passei quatro finais de semana trancada em casa, e o Ricardo
vinha me ver aos sábados. O~Ricardo é um acompanhante terapêutico muito
especial. Ele é muito inteligente, pensa rápido e é muito observador.
Além disso, é afetivo e carinhoso. Teve um domingo em que ele veio me
ver no final do dia e trouxe uma bandejinha de doces da Doceria Leo, que
fica aqui perto, na Praça Panamericana. Eu nem acreditei! Fiquei muito
feliz. Eu adoro doce, e nós comemos rapidamente todos aqueles docinhos:
brigadeiro, quindim etc.

A defesa de dissertação de mestrado dele foi ``A Arte no Metrô de São
Paulo''. Nós temos uma grande afinidade neste assunto, arte. Um dia
passamos a tarde olhando um livro do Rotko que eu tenho e comparando o
trabalho do Rotko com o do Paulo Pasta. O~Ricardo também dá aula na \versal{UNIP}
e já é psicanalista, tem uma pequena clínica particular, que aos poucos
está crescendo. Eu não sei como ele consegue fazer tudo isso com aquele
bom humor de sempre. Além de tudo, é meio piadista. A~tese de doutorado
dele vai ser sobre ``Atendimento Terapêutico'', ele é apaixonado por
esse assunto. O~Del Porto ficou muito bem impressionado com o Ricardo e
indicou"-o para outro paciente que está com depressão. No começo, o Del
Porto dizia para mim que eu não precisava de acompanhantes terapêuticos
formados na \versal{USP}, mas aos poucos ele começou a gostar da moçada. O~Ricardo fez questão de assistir até às quatro sessões de
eletroconvulsoterapia que eu fiz.

No quarto final de semana fiquei muito feliz porque finalmente
``consegui'' sair de casa e fui com o Ricardo à exposição da Louise
Bourgeois. Eu queria muito ir desde que recebi o convite, mas mergulhada
naquela depressão não dava, não conseguia sair de casa. No começo, eu me
senti um pouco um E.T\, porque o Instituto Tomie Otake estava lotado de
gente e eu tinha ficado muito tempo sozinha em casa. Logo, porém, fiquei
completamente impactada com a obra da Louise Bourgeois e me envolvi
completamente com a exposição. Achei as instalações fortíssimas, gostei
muito também daqueles totens que estavam enfileirados numa linha
horizontal numa sala dos fundos. Tudo me atraiu muito! Que mulher forte
e corajosa para desnudar assim seu mundo interno! Despudorada, a Louise
Bourgeois, e completamente honesta, fiel a si mesma! Me fez muito bem
ter visto a exposição, foi um grande alimento!

No final de semana seguinte eu fui com o Ricardo passear naquela praça
que tem em frente à casa do Duda, meu sobrinho, e que eu adoro. É~uma
praça oval, relativamente grande, que tem árvores antigas e frondosas
misturadas com árvores mais jovens. Tem um gramado bem cuidado, um
pequeno parquinho para crianças e um caminho para quem quer caminhar. Eu
só consegui andar metade do caminho, antes eu conseguia, de cara, dar
uma volta completa na praça. Eu estou completamente descondicionada,
fora de forma, e isso me prejudica muito, meus músculos estão todos
atrofiados. Afinal, eu passei três anos indo do sofá para a cama e da
cama para o sofá. Logo me sentei num dos bancos espalhados pela praça e
me diverti olhando aqueles cachorros lindos brincando sob os olhos dos
donos. Gostei também de contemplar a praça, suas árvores, seu horizonte.
Ficar na praça me faz muito bem, eu sinto muita falta de natureza, de
horizonte e, hoje em dia, não tenho muito para onde ir. Antigamente
podia ir para São Sebastião, para o Pinhal, para a Fazenda Santa Rita,
onde morava a Tuxa. Era uma fazenda incrível, que tinha cavalos,
piscina, cachoeiras e paisagens belíssimas.

Nessa depressão aconteceu de várias vezes eu ficar sentada horas no
sofá, só fumando e pensando, sem fazer nada. Isso acontecia muito quando
eu voltava da análise. Eu ficava de uma hora a duas horas só pensando.
Eu lembrei que um dia mamãe me viu assim em Cotia, chegou perto de mim e
disse:

-- É assim mesmo, minha filha, eu ficava horas olhando pela janela. As
pessoas chegavam perto de mim e falavam:

-- Você não quer ouvir música?

-- Você não quer ir ao cinema?

-- Você não quer ler um livro?

-- Mas eu só queria ficar olhando pela janela.

Quando eu contei isso para o Sylvio, ele achou bonito, disse que eu
devia me dar esse tempo quando pudesse. Eu me dei esse tempo várias
vezes e acho que me fez muito bem. Como moro sozinha, ninguém ficava
super ansioso com o meu silêncio, a minha postura. Nesses momentos,
morar sozinha é uma bênção, a gente pode viver esses momentos com
tranquilidade. O~meu apartamento tem uma vista bonita, e eu curtia muito
quando o sol se punha e o céu ficava todo tingido de laranja, rosas e
azuis. Quando sento no sofá, graças a Deus, a minha vista é o céu, por
enquanto, pelo menos. A~especulação imobiliária é foda. Ontem mesmo eu
descobri um prédio novo sendo construído aqui perto. Espero que ele não
prejudique a minha vista. A~Vila Madalena, esse bairro que no seu início
era de pequenos chacareiros, está, na verdade, virando um paliteiro.

Aos poucos, os antidepressivos foram agindo bem e passei a solicitar a
ajuda do Hailton e do Ricardo só para as coisas mais difíceis, como ir
comprar algodão cru na 25 de Março. Isso eu não consigo mesmo fazer
sozinha. Aos poucos voltei a dirigir e recuperei a minha liberdade e a
minha independência. Voltei a frequentar o almoço semanal das primas e
das amigas, na casa da Soninha.

Essa depressão foi ``sopa'' perto das muitas outras que já encarei
nesses quase 30 anos de bipolaridade. Ela foi curta, durou três meses e
meio. Quando começou, eu fiquei em pânico porque sabia que poderia durar
um ano, dois anos, três anos… Conforme o meu organismo reagisse
aos antidepressivos. Quando uma crise começa, a gente nunca sabe quando
vai acabar, isso é absolutamente imprevisível e gera uma enorme
angústia. O~estranho é que, nessa depressão, não senti aquela tristeza
imensa, que vem do fundo do coração. Cada depressão é diferente da
outra. Sempre. Graças a Deus.

\begin{center}\asterisc{}\end{center}


\textbf{}

\begin{flushright}\textbf{28/11/2011}\end{flushright}


No final de 1973, parei de trabalhar no Moinho Santista, onde eu criava
estampas para toalhas e lençóis, no ateliê. O~Paulinho, meu irmão, me
arranjou esse emprego. Eu fui até lá com a pasta de desenhos e a Gesine,
chefe do ateliê, me escolheu.

No começo, ela mandava eu ir ao Largo São Francisco comprar flores e
desenhar as flores, fazer desenho de observação. Ela queria ``limpar'' o
meu desenho. Ela também me ensinou a desenhar um círculo perfeito a mão
e uma forma oval. Se ela não me ensinasse, eu jamais teria condição de
fazer isso.

No ateliê trabalhavam também o Marlon e o Alberto, que tinham uma enorme
capacidade de produção. Cada um tinha uma mesona de fórmica branca e, ao
lado, um carrinho com quatro ou cinco gavetas cheias de guache Talens,
um fantástico guache alemão. Eu desbundei com aquele carrinho, foi o que
mais me seduziu no ateliê. Ao meu lado trabalhava a dona Ane, só meio
período. Tinha também uma moça que agora eu esqueci o nome.

O ritmo era puxado, eu tinha que bater ponto às 8h00, ao meio"-dia, a uma
hora e às 17h00hs. Eu nunca tinha feito isso. A~gente almoçava num
restaurante imenso do Moinho Santista. Às 17h00, eu pegava o ônibus na
Praça do Patriarca e ia para casa. A~Santista era na rua Bela Vista. Eu
passava sempre pela rua Álvares Penteado e ficava babando com aquele
prédio do Banco do Brasil que se tornou depois o Centro Cultural Banco
do Brasil.

Por sorte, eu morava naquele predinho alternativo da rua Pernambuco, que
era super perto da \versal{FAAP}. Tomava um lanche rápido e às 19h00 ia a pé para
a \versal{FAAP}, em geral carregando um monte de tralhas. As aulas, acho que
acabavam às 22h00. Muitas vezes a gente saía da faculdade e ia para
algum \emph{vernissage}, que naquela época acontecia mais tarde do que
hoje.

Com o passar do tempo, foi ficando claro para mim que eu não aguentava
aquele ritmo de trabalho e estudo, embora eu adorasse o ateliê e criar
estampas. A~dona Ane sempre dizia para mim que eu tinha que optar entre
o trabalho e a faculdade. Pensei bastante e optei pela faculdade, que
não queria largar, como tinha largado a Psicologia, no terceiro ano do
Sedes Sapientiae. Com pena, saí do Moinho Santista e fui fazer estágio
na Escola Vera Cruz para tentar dar aulas, que era um trabalho de meio
período.

Eu adorava crianças e me dava muito bem com elas. Eu ``tinha jeito'',
como se diz. Levava na bagagem um estágio que fiz na Escola de Arte São
Paulo, da Ana Mae Barbosa, e as oficinas que tinha dado na A.C.M\, da
Vila Mariana durante um tempo. Depois de sair do Santista, fiz estágio
no Nível I do Vera Cruz, onde a Peo era a responsável pelo trabalho da
oficina e também diretora desse nível. Estagiei também com os
adolescentes do Nível \versal{III}. Adorei estagiar, fiquei fascinada pelo
trabalho. No final do estágio, a diretora pedagógica do Nível \versal{II} me
convidou para dar aula para as segundas séries, e é claro que eu topei.
Nas férias fiz aquela viagem linda que foi subir o Rio São Francisco com
um grupo de amigos. Eu não me lembro bem se foi nesse mesmo ano que me
pediram também para dar aulas para as terceiras séries. Acho que foi.

\begin{quote}
\section{\versal{OFICINA} \versal{DE} \versal{ARTE}: Escola Vera Cruz e Pinacoteca do Estado}

Vemos a oficina de Artes como um processo vivo e integrado entre o
\versal{PROFESSOR}, a \versal{CRIANÇA}, o \versal{ESPAÇO}, os \versal{MATERIAIS} \versal{EXPRESSIVOS} e as \versal{SERVENTES}
que ajudam a manter o espaço cuidado e atraente.

\begin{itemize}
\item
  A oficina é o \versal{PROFESSOR}:
  \begin{itemize}
  \item
    atento,
  \item
    amoroso,
  \item
    observador.
  \end{itemize}
\item
  Na percepção:
  \begin{itemize}
  \item
    de cada criança, de cada individualidade, de como melhor ajudar a
    cada um na sua dificuldade;
  \item
    de como incentivar cada um nos seus processos pessoais de
    descoberta, invenção, experimentação.
  \end{itemize}
\item
  É o professor:
  \begin{itemize}
  \item
    vibrante, ao acompanhar o desenvolvimento de processos grupais e
    individuais em toda sua riqueza;
  \item
    satisfeito ao ver a criança crescer no domínio e elaboração formal
    dos diversos materiais, o que significa crescer na elaboração e
    domínio de si mesma;
  \item
    aprendendo com a espontaneidade, alegria e liberdade, que são as
    crianças;
  \item
    se desenvolvendo e se descobrindo continuamente, enquanto ser humano
    e artista, para poder conviver com as crianças em liberdade, sem
    diretividade e imposição nas atitudes e nos padrões estéticos;
  \item
    exercendo o trabalho cotidiano e às vezes difícil, mas fundamental,
    com os limites de uso do espaço e do material que dão segurança à
    criança na sua soltura;
  \item
    satisfeito ao ver uma criança superar suas dificuldades, se
    desenvolver com os materiais, soltar o corpo, se expressar;
  \item
    preocupado, às vezes, ao conviver com uma criança que passa longo
    tempo insatisfeita, sem conseguir se expressar;
  \item
    dando toques, atuando, percebendo a melhor forma de ajudar a criança
    que não se expressa (Às vezes, aparentemente nada está acontecendo,
    até que, de repente, um dia, a criança materializa um trabalho forte
    e elaborado, resultado do espaço que teve para viver o aparente
    ``vazio''. Vazio de produção externa, mas pleno de ``produção e
    elaboração interna''. ``E assim arrancar de dentro da noite a barra
    clara do dia'', como diz o Gismonti. E~aí a gente fica feliz de ter
    conseguido dar esse espaço e acreditado nela.)
  \end{itemize}
\item
  A oficina é:
  \begin{itemize}
  \item
    o professor impaciente, gente que às vezes exagera no grito e na
    atitude;
  \item
    o professor atento, que ajuda a criança a qualificar e diferenciar
    suas vontades, que lhe dá a referência da linguagem específica de
    cada material, quando necessário; que introduz novos materiais e
    técnicas conforme o momento da criança e, assim, amplia seu
    conhecimento.
  \end{itemize}
\item
  A oficina é o espaço onde a \versal{CRIANÇA} acontece no seu movimento
  espontâneo:
  \begin{itemize}
  \item
    ora pra dentro,
  \item
    ora pra fora,
  \item
    ora alegre,
  \item
    ora triste;
  \item
    às vezes sozinha, no seu cantinho, fazendo o seu barro, a sua
    pintura;
  \item
    às vezes trabalhando em grupo: uma ideia aqui, outra acolá. E
    aparece muitas vezes o dono do grupo: ``Aqui é azul!'': reage o mais
    forte, se encolhe o mais frágil, e o professor atua.
  \item
    Mistério, emoção, sensação, intuição.
  \item
    Conhecimento físico, lógico, matemático, analógico e social: tudo
    acontecendo junto e se relacionando, corpo, cabeça, coração
    trabalhando juntos. Análise e síntese.
  \end{itemize}
\item
  Dificuldades:
  \begin{itemize}
  \item
    às vezes é não conseguindo se relacionar com os materiais, cara
    triste e quieta, corpo encolhido, só olhando;
  \item
    ou correndo excitada pelo espaço, perturbando a tranquilidade de
    quem está concentrado, às vezes, até destruindo o trabalho de outras
    crianças.
  \end{itemize}
\item
  Facilidades:
  \begin{itemize}
  \item
    olho brilhando, satisfação na descoberta e invenção, seja
  \item
    tomando banho de esguicho;
  \item
    tecendo a teia de barbante;
  \item
    serrotando e pregando com afinco a madeira;
  \item
    alisando, acariciando, moldando e desmoldando o barro;
  \item
    se concentrando no lidar com o fogo do pirógrafo e da vela;
  \item
    fazendo o fogo e o pão;
  \item
    simbolizando, representando, imaginando com o arco"-íris das tintas,
    dos lápis de cor, das canetinhas;
  \item
    reinventando e ressignificando as sucatas variadas;
  \item
    qualificando e diferenciando, pouco a pouco, os diversos materiais,
    cada um correspondendo à emoção, à necessidade de cada dia, de cada
    momento.
  \end{itemize}
\end{itemize}
A oficina é dor e prazer, satisfação e insatisfação, alegria,
divertimento e trabalho. Ela é o \versal{ESPAÇO} amplo, bem pensado e organizado
que facilita a organização e independência da criança, que propicia sua
leitura pessoal do espaço, na medida em que ela dele se apropria como
quer, conforme seu estado de espírito naquele dia.

\begin{itemize}
\item
  A oficina é o \versal{INSTRUMENTAL} necessário para a elaboração com os
  materiais:
  \begin{itemize}
  \item
    martelos, pregos e serrotes;
  \item
    tesouras, pincéis;
  \item
    barbante, durex, cola, materiais de ligação;
  \item
    pirógrafos, lápis, canetinhas, lápis de cor;
  \item
    bacias para as misturas, entre outros mais.
  \end{itemize}
\end{itemize}
É o trabalho cotidiano e paciente das \versal{SERVENTES} que ajudam a manter tudo
cuidado e em dia, junto com os professores e as próprias crianças, que
colaboram na limpeza no final da oficina.

\begin{itemize}
\item
  A oficina acolhe e abriga a intimidade da criança:
  \begin{itemize}
  \item
    o sorriso e a lágrima;
  \item
    o claro e o escuro;
  \item
    o limpar e o sujar;
  \item
    a alegria e a tristeza;
  \item
    o destruir e o construir.
  \end{itemize}
\item
  Ela é o ``fazer'':
  \begin{itemize}
  \item
    descoberta e invenção;
  \item
    experimentação;
  \item
    satisfação e decepção;
  \item
    tentativa;
  \item
    ensaio e erro;
  \item
    vislumbre;
  \item
    contração e descontração;
  \item
    o individual e o coletivo;
  \item
    a magia, a alquimia da mistura das materiais;
  \item
    o experienciar sensível do mundo;
  \item
    o estabelecer de infinitas relações.
  \end{itemize}
\end{itemize}
``\versal{FAZER} E, \versal{EM} \versal{FAZENDO}, \versal{FAZER}"-\versal{SE}'', como dizia o Sartre.

\end{quote}
\begin{quote}
\begin{itemize}
\item
  A oficina é:
  \begin{itemize}
  \item
    dramatizar;
  \item
    imaginar;
  \item
    se fantasiar, se travestir, se enfeitar;
  \item
    brincar de ser herói ou estrela, monstro ou bicho;
  \item
    é interagir, saber ouvir, saber falar, saber se mostrar, se
    expressar corporal e verbalmente;
  \item
    é respeitar a si mesmo e ao outro, adquirindo a dimensão do coletivo
    e do social.
  \end{itemize}
\end{itemize}
A Oficina de Arte é importante na escola porque, estando liberada de um
padrão de expectativas, a criança pode descobrir de modo mais livre o
seu padrão pessoal. A~criança percebe, pouco a pouco, como é o seu tempo
interno, na medida em que é ela quem decide sempre com que material vai
trabalhar e quanto tempo vai permanecer nesse ou naquele projeto. Na
medida, também, em que ela desenvolve diversas organizações motoras do
seu corpo, exigidas pelas características diferentes de cada material,
de cada projeto de trabalho.

Acreditamos que esse fazer, onde a criança seleciona, qualifica e
diferencia cada material, leva"-a a qualificar e diferenciar a si mesma,
fora da oficina, nas diversas situações que encontra na vida.

Cada vez fica mais claro para nós como é fundamental propiciar, para a
criança, esse espaço onde ela pode se expressar em todas as suas
dimensões e, assim, se desenvolver globalmente como ser humano:

\begin{itemize}
\item
  captante,
\item
  expressante,
\item
  flexível,
\item
  independente,
\item
  sensível,
\item
  individualizado,
\item
  consciente de si mesmo e do mundo.
\end{itemize}
A liberdade da arte é vital. Usualmente consideramos a expressão
artística como uma atividade de luxo e de passatempo, quando não é
vista, como no caso dos artistas famosos, como uma graça divina. No
entanto, nenhuma atividade melhor para equilibrar o viver por demais
pré"-determinado de hoje em dia. O~cotidiano padronizado é uma espécie de
morte, na medida em que pode ser vivido quase sem consciência. Nesse
contexto, a atividade livre é deveras vivificante.

Esse trabalho é desenvolvido na Escola Experimental Vera Cruz, nível \versal{II},
com crianças de 8, 9 e 10 anos. E~também no ``Laboratório de Desenho'',
da Pinacoteca do Estado, com crianças de 7 a 11 anos. Na Pinacoteca,
também desenvolvo cursos para arte"-educadores e professores de primeiro
grau, junto com Paulo Portella Filho. Nesses cursos procuramos passar
essa filosofia de trabalho de arte e educação acima descrita. No Vera
Cruz, as turmas são de 17 alunos, e o tempo da aula é de uma hora e meia
por semana. Desenvolve"-se o trabalho plástico e o jogo dramático. Nesse
``fazer'', as crianças escolhem livremente o material e manifestam
espontaneamente todas as fases da História da Arte. Acontece desde a
pintura das cavernas, a \emph{body"-art}, o impressionismo, a arte
gestual, o grafite, todas as inesgotáveis possibilidades da expressão
humana. Em cada aula são 17 processos diferentes acontecendo ao mesmo
tempo. Cada professor acompanha sua turma durante três anos.

Tem sido um trabalho difícil, exigente para o professor, mas muito
gratificante, de uma riqueza enorme.

\medskip{} 

\begin{flushright}\emph{Maria Regina Barros Sawaya}\end{flushright}

\begin{flushright}\emph{Professora, Coordenadora, Assessora da Área de Arte da Escola Vera
Cruz, nível \versal{II}}\end{flushright}

\begin{flushright}\emph{Professora de crianças e adultos na Pinacoteca do Estado}\end{flushright}


\end{quote}
\begin{center}\asterisc{}\end{center}

\begin{flushright}\textbf{04/01/2012}\end{flushright}


Trabalhei 10 anos no Vera Cruz. Nos últimos três anos, além de
professora, fui coordenadora e assessora.

No dia 16 de março de 1984, perdi a Tuxa, minha irmã muito querida, num
trágico acidente de carro na estrada entre a fazenda dela, a Santa Rita,
e a cidade de São Carlos. Ela estava indo encomendar os doces e salgados
para a festa de aniversário que daria na semana seguinte. Quando alguém
tecia algum comentário sobre a Tuxa, em geral era assim:-- Que linda que
ela é!

-- Ela é um sol!

-- Que energia inesgotável que ela tem!

-- Que alegria de viver!

E por aí afora…

Foi nessas circunstâncias que me tornei bipolar. Eu não suportei a dor
da perda, o luto. Eu vivi o luto do lado inverso. Em mania. Eu nunca
bebi tanto, namorei tanto, pintei tanto. Essa foi a minha primeira
mania. Mal sabia eu o que o futuro como bipolar me reservava… Mal
sabia eu que o meu calvário apenas tinha começado… Agora já faz
quase 30 anos. Uma vida!

Na comunidade em que eu morava, eu punha o som muito alto para dançar,
dançar e dançar. O~Gaiarsa me disse que isso é uma forma de luto, de
ritual de luto em sociedades mais primitivas. Acho que eu sou meio
primitiva… Nessas horas eu preciso dançar. É~um jeito de
exorcizar.

É claro que os outros três moradores da casa ficavam putos comigo por
causa do barulho e quiseram me expulsar de lá. Mas eu ganhei essa briga
de foice e consegui expulsar os três, falei com advogado e tudo. Pedi a
ajuda do Paulinho, meu irmão, e ele não achou nada melhor para me dizer
do que:

-- Que falta de \emph{fair play}, Gina. Olha se pode!Esse período da
minha vida é meio confuso para mim. Acho que morei sozinha até o Fábio
Prado, irmão do Pedro Prado, vir morar comigo em 87.

Morar em comunidade foi a pior decisão que tomei na vida, não tem nada a
ver comigo que sou individualista e preciso de um canto só meu. Preciso
de um pouco de solidão todos os dias. Eu entrei nessa para poder
trabalhar menos, viver com menos e ter tempo para terminar a faculdade
de Psicologia, que acabei terminando em 1987. Meu sonho era ser
arte"-terapeuta, usando tudo o que já tinha aprendido durante 10 anos
como arte"-educadora. Não consegui porque, quando estava terminando a
faculdade, eu tive uma depressão enorme no final de 1986, por causa de
um fim de relação amorosa, seguida de uma mania devastadora em janeiro
de 1987, que foi seguida de mais seis meses de depressão… Foi
duro… Foi foda… Eu comecei a achar que, como eu tinha
ultrapassado a barreira do consciente, eu era ``completamente louca'' e
não poderia de forma alguma clinicar, embora tivesse me esforçado muito
durante anos para ter o tal diploma de ``Psicologia''. Acho que o meu
analista da época não me ajudou muito. Ele insistia para eu entrar na
Sociedade Brasileira de Psicanálise, que não tem nada a ver comigo. Além
disso, eu jamais teria dinheiro para pagar uma ``análise didática''. Eu
estava confusa. Muito confusa nessa época. Ter a tal mania me assustou
muito, e eu estava ainda me havendo com isso. Eu até ``esqueci'' que o
meu objetivo sempre foi ser uma terapeuta de base junguiana, apesar de
estar me tratando com psicanálise e trabalhar com as crianças através de
brincadeiras com a terra, a água, o ar e o fogo. Coisa que eu já tinha
feito muito no Vera Cruz. E~eu tinha o espaço perfeito para isso naquela
casa onde eu mesma já tinha montado e mantido uma escola de arte para
crianças, por dois anos, junto com a Germana Monte"-Mór e a Ruth, minhas
amigas. Além de tudo, a casa tinha um bom quintal e, por isso, daria
muito bem para brincar com água. Tinha também um muro grande que já
estava pintado pelas crianças anteriores. Mas… infelizmente não
consegui realizar nada disso.

Em 1984, eu passei o ano todo meio maníaca. Eu critiquei tudo que tinha
para criticar no Vera Cruz. Eu, absolutamente, deixei de ser política.
Botei o dedo em todas as feridas da instituição e incomodei muito.
Incomodei muita, muita gente. No fundo foi uma forma de manifestar a
minha dor e o meu desespero com a morte da Tuxa. Parece que tudo o mais
perdeu totalmente a importância, o sentido, eu perdi o sentido da vida.
Eu, nesse ano, fui também uma pessoa completamente sem limites. Isso
incomoda muito em qualquer instituição. É~claro que, no final do ano, a
Lucília, diretora do Nível \versal{II}, me deu uma prensa:

-- Gina, ou você se mantém adequada aos limites do Vera Cruz, ou você
sai.

Eu adorava a Lucília, sempre a achei uma pessoa fantástica, muito
inteligente e muito sensível.

-- Eu saio, eu respondi.Por causa de uma mania você pode perder um
emprego de 10 anos, mesmo que durante 10 anos você tenha trabalhado
direito, tenha sido muito eficiente… A doença é muito destrutiva,
completamente destrutiva. Tanto na mania como na depressão. É~foda… Em geral nenhuma instituição aguenta. Ela expulsa você.

\begin{center}\asterisc{}\textbf{}\end{center}

\begin{flushright}\textbf{07/01/2012}\end{flushright}


Eu gostava de sair pela porta grande, da grande sala de jantar, onde
tinha aquela mesona e aquela cristaleira enorme cheia de enfeites
bonitos e atraentes. Eu descia os degraus da escada e contornava o
``jardim francês'', com aquele canteiro redondo e lindo no meio. Grande.
Ele era meio grande. Antes dele, à direita, tinha o banco marrom de
ferro com mesinha redonda e três cadeirinhas. Eu saia pelo portãozinho e
dava na alameda principal, que tinha todo um corredor de
manacás"-da"-serra às suas margens. Em novembro, dezembro, aquilo virava
uma festa de brancos e roxos. Tinha também um riachinho que percorria a
alameda do lado esquerdo, que fazia aquele delicioso barulho de água
corrente. Depois tinha aquela raiz estranha que saía da terra, quase na
vertical. Era a minha ``raiz''. Toda vez que eu passava naquele pomar
majestoso, eu ia cumprimentá"-la. Mais à frente, à esquerda, tinha duas
imensas moitas de bambus juntas. A~primeira era de bambus mais finos,
amarelos com listrinhas verdes. A~segunda era de bambus grandes,
grossos, verdes. Dentro era oco e, quando a gente entrava, parecia que
estava numa gigantesca catedral gótica, natural. Era uma beleza! Uma
beleza mesmo. Depois, eu virava e subia um caminho ladeado, à direita,
por mangueiras centenárias e, à esquerda, por um murão ``rosa"-caipira''.
Logo, à direita, tinha a avenida das jabuticabeiras ancestrais. Eu
parava bem no começo e ficava admirando, admirando um tempão. Depois, eu
caminhava por ela até um trecho onde havia dois bancos, um de cada lado
da alameda. Um de frente para o outro. Eu deitava sempre no da esquerda
e ficava horas olhando o céu através das copas das jabuticabeiras. Era
lindo! Lindíssimo! Grandioso! Depois, ia até o final da alameda, virava
à esquerda e ia subindo pelo caminho de terra. No final desse caminho, à
direita, tinha um jatobá enorme, maravilhoso, frondoso, antigo, muito
antigo. Mais acima tinha um plano cimentado, retangular, com dois
bancos, um em frente ao outro. Tinha também um arco para pôr trepadeira
entre os bancos. Na descida, do lado direito, tinha a formosa
``escadinha de água'' que o Conde do Pinhal mandou fazer. Parece que ele
copiou de uma estação medicinal da Europa. Ao longo dos degraus, que
eram baixos e compridos, descia água corrente até lá embaixo. Parece que
era bom para artrite, artrose, esse tipo de doença. Muitas vezes a gente
ficava sentado nos bancos lá em cima, conversando por horas. Depois, a
gente descia pela escadinha ou pela terra e voltava para casa. Eu tive a
sorte de poder me encontrar com esse pomar maravilhoso e fantástico
durante 30 anos.

\begin{center}\asterisc{}\end{center}

\begin{flushright}\textbf{10/01/2012}\end{flushright}


Ainda bem que o Celso voltou. Estava difícil ficar sem análise. Na
segunda semana de férias dele, quando cheguei da Baleia, onde passei o
\emph{Réveillon} com Soninha, Fernão e filhos, eu estava muito triste e
angustiada, não sabia por quê. Como eu fiquei morrendo de medo de estar
entrando numa daquelas depressões profundas, sem saber ao menos quando
ia terminar, achei melhor ligar logo para o Del Porto, na terça"-feira.
Só conseguimos falar na quinta"-feira porque na quarta"-feira foi meu
aniversário e um monte de gente me ligou. E~eu fiquei muito contente. O~Del Porto aumentou o Pondera, um dos dois antidepressivos que eu já
estava tomando. Passei, então, a tomar quatro comprimidos na
quinta"-feira mesmo. Alguns dias depois, eu liguei de novo para o Del
Porto porque estava precisando tomar dois Dormonid para dormir, em vez
de um, que era a dose habitual, e por isso eu passava o dia seguinte
chapada, acabava dormindo praticamente o dia todo. Falei para ele,
também, que na véspera tinha conseguido dormir, mas acordei às 3h00,
elétrica, com muita energia. Ele comentou comigo:

-- Será que você está dando uma ``virada'', Regina?

Eu disse que achava que não porque tinha passado os dias anteriores
dormindo muito. Só assim consegui ficar sem a análise naquela
semana… Evasão total… Fuga total… Até agora eu não
sei o que estava me amedrontando tanto, qual era o bicho"-papão.

O fato é que, dois dias depois, passei a noite toda em claro. Só
consegui dormir das 9h00 às 10h00 da manhã e acordei cheia de energia
para levar a Filó, minha gatinha, ao veterinário. Aí eu me convenci de
que estava em mania e liguei para o Del Porto. Ele falou para eu começar
a tomar 2,5mg de Zyprexa naquela noite mesmo, tomar só um Pondera na
manhã seguinte e eliminar completamente o Wellbutrin 300mg, que era o
outro antidepressivo que eu também estava tomando. Ele deu uma mudada
rápida e radical na minha medicação. Sábio Del Porto! Até agora deu para
segurar a mania, acho que dessa vez escapei a tempo. Ô inferno de
doença!!! Saco!!! Estou cansada!!! Mas estou satisfeita por ter
conseguido controlar a mania. Dessa vez nem gastei dinheiro
excessivamente, pra falar a verdade, até agora me controlei bem.

Eu retomei a análise na quarta"-feira passada, já tive duas sessões. Eu
estava tão ansiosa por causa da mania e de outros assuntos que tinha
para tratar com o Celso que acabei ligando para ele na terça"-feira à
noite, na casa dele, não consegui esperar a sessão da quarta"-feira, às
15h30! Ele foi muito generoso e, na prática, fez uma sessão por telefone
mesmo, me acalmando muito, muito mesmo.

O Celso é alto, magro e, segundo um dos meus acompanhantes, ``bonitão''.
Eu também acho. Ele tem barba, bigode e também um sorriso lindo. Seus
dentes são branquinhos e muito perfeitos. Mas pra falar a verdade, ele
sorri pouco, ele é mais pra sério. Sério e ``pão, pão, queijo, queijo''.
Ele é muito direto e isso é muito bom para mim que gosto de me
``evadir'', me evadir, me evadir…

Ele também é muito inteligente, culto e sensível. Na mesinha, ao lado da
cadeira dele, tem sempre vários livros: Mitologia, Filosofia, Escola de
Frankfurt, Mircea Eliade. Só livros complicadíssimos. E~isso me faz ter
mais confiança nele, porque mostra que ele é curioso. Está sempre
estudando, se informando.

A análise me ajuda muito a aceitar a bipolaridade, porque, pra falar a
verdade, até hoje, quase 30 anos depois, às vezes ainda é difícil. Às
vezes eu ainda me revolto muito por ter a doença. No ano passado, quando
uma depressão séria me pegou, depois de uma fase longa de estabilidade,
eu fiquei puta da vida! Eu me senti apanhada de ``calça curta'', mas o
Celso, com toda a paciência do mundo, me convenceu de que era melhor
viver do que me suicidar. A~análise também me ajuda muito a tomar esse
monte de remédios que tomo todo dia e a aguentar os terríveis efeitos
colaterais. Outro dia eu contei. Tomo 12 comprimidos todo dia. Nem todos
são remédios psiquiátricos. Nessas alturas, eu tenho que tomar também
remédio para pressão, hormônio, cálcio etc. Comoqualquer pessoa de minha
idade tem que tomar tudo isso, sinto um certo consolo… Nada como
a desgraça dos outros… para consolar a gente. Que horror! Que
pensamento mais mesquinho!

O Celso também é muito generoso. No ano passado, quando eu estava muito
deprimida e não tinha acompanhantes, eu pedi o telefone da casa dele na
véspera de um feriado longo de quatro dias. Ele me deu o telefone na
hora. Graças a Deus, aguentei o feriado sozinha e não precisei ligar
para ele.

Além de ser muito culto, para minha felicidade, ele também gosta muito
de pintura. Na sua sala de espera, de um lado, tem uma reprodução linda
do Van Gogh e, do outro, uma do Picasso, cheia de azuis, vermelhos e
brancos. Tem também um aparelho de som que está sempre tocando música,
em geral clássica, baixinho. Muitas vezes é Mozart. Por causa disso, eu
retomei o contato com Mozart e comprei uns \versal{CD}'s para eu ouvir em casa.
Curti e curto muito.

Eu fiquei no divã no início da análise, mas… depois passei um
tempão na cadeira. Devia ser medo, claro. Mas o Celso nem reclamou e nem
criticou. No final do ano passado, eu voltei a deitar no divã e achei
ótimo. Estou mais corajosa agora.

Teve um dia que, quando a sessão começou, o Celso me deu um cartãozinho
de uma loja chamada Home Machê, porque eu tinha achado bonitos os vasos
que ele comprou lá e colocou na prateleira da sala de espera. Eu fiquei
surpresa com esse gesto. Achei que ele era um cara livre da ética rígida
da Sociedade Brasileira de Psicanálise, o que é ótimo!

No começo da análise, de vez em quando, eu levava uns presentes para o
Celso. Teve um dia que levei uma caixinha vermelha, em formato de
coração, cheia de pequenos chocolates da Lindt, que eu comprei na
padaria. Dei uma igual para Soninha e fiquei com outra para mim. É~nela
que ponho os remédios da noite que tomo antes de dormir, de manhã já a
coloco no criado"-mudo. Apesar disso, às vezes, eu me esqueço de tomar os
remédios, como aconteceu no sábado passado. Eu só lembrei disso às 18h00
e, aí, não dava mais! Eu tomei só os da noite e lembrei que o Del Porto
um dia me disse que pular só uma dose não era tão grave assim… No
domingo, eu tomei tudo direito, de manhã e à noite! Viva! Haja saco!

A sala do Celso é bem ampla e iluminada, eu gosto disso. Do lado
esquerdo, tem o divã, encostado à parede, e uma \emph{bergère} em cada
extremidade. Ao lado da do Celso, tem a mesinha baixa com os livros.
Como sempre levo um copo de água pra sessão, o Celso passou a colocar
uma mesinha pequena ao lado da \emph{bergère} ou do divã. Do lado
direito da sala, o Celso fez um pequeno escritório, com armário, mesa,
computador, telefone. A~sala tem também uma porta de vidro enorme que dá
para um terracinho redondo. É~bem bonito. O~lugar onde o Celso senta tem
uma janela que fica atrás da poltrona, mas, mesmo assim, é meio escuro.
No meio da tarde, ele já tem que acender a luminária ao lado dele.

O Celso é um cara muito calmo, tranquilo e ponderado. Ele não faz drama
com nada. Eu acho isso ótimo, principalmente para mim que sou turca e,
por isso, muito intensa e exagerada nas emoções. Ele também não faz
longas interpretações freudianas, kleinianas ou winnicottianas, ou,
talvez, eu não perceba que ele faça. Graças a Deus! Ele é muito direto,
é pão, pão, queijo, queijo, mesmo, e eu admiro muito essa grande
competência que ele tem.

Um dia encontrei o Fábio Moreira Leite, que também foi paciente do
Celso, e ele disse que nunca tinha conhecido um analista tão objetivo na
vida. E~o Fábio, assim como eu, tinha muita prática no assunto. Eu
fiquei muito triste quando o Fábio morreu. Ele era muito inteligente,
brilhante e completamente empolgado e envolvido com seu trabalho de
artista plástico. Eu nunca vou esquecer aquele seu trabalho fantástico
com carimbos.

O Jardim também gostava muito dele e até o convidou para ser seu
interlocutor, num dos cursos de gravura em metal que deu no \versal{SESC}. Um
dia, há muito tempo, eu fui no ateliê do Jardim comprar umas gravuras
para dar de presente. Acabou que passamos a tarde conversando. Foi uma
delícia! O Jardim me deixou ver todo o trabalho dele. Lá tinha milhões
de gavetinhas cheias de gravuras lindas. Quando conversamos sobre o
Fábio, deu para perceber que o Jardim tinha uma enorme admiração por
ele. Eu até contei para o Jardim que tinha namorado o Fábio durante um
tempo. Pra falar a verdade, sempre que eu encontrava o Jardim, acabava
contando muitos episódios da minha vida para ele. Eu não sei nem por
quê. Ele é um ótimo ouvinte. Super atento e super sensível.

\begin{center}\asterisc{}\end{center}


\begin{flushright}\textbf{12/01/2012}\end{flushright}


A Filó é a gatinha vira"-lata que a Marinês me deu 12 anos atrás. Quando
chegou em casa, a primeira coisa que fiz foi pegá"-la pelo cangote e
passar um \emph{spray} antipulgas, como a Marinês tinha recomendado. Foi
um dos melhores presentes que eu ganhei na vida!

Logo, logo, ela começou a dormir no pé da minha cama e eu adorei. Ela
era pequenininha, magrinha e muito delicada. Quando estava no meu colo,
e eu fumava, ela seguia a fumaça com o olhar, virando a cabeça. Depois,
teve uma fase em que ela encarava a \versal{TV} quando estava ligada. O~que será
que ela via?

Um dia, ela estava brincando na mureta da área de serviço e parece que
se enganchou num dos vasos, caindo lá embaixo. A~sorte foi que um dos
meus vizinhos, que gosta muito de animais, viu e levou"-a, junto com a
Preta, para o Hospital Veterinário Rebouças. Ela tinha tido uma
hemorragia no pulmão e tinha quebrado uma patinha. Eu ia visitá"-la todas
as tardes, e os funcionários do Hospital me disseram que ela só comia
quando eu estava lá.

A Filó é uma ótima companhia. É~uma tigrinha com listas cinzas, brancas
e pretas, é rajada, vira"-lata total. Os olhos dela são lindos, e a gente
se comunica pelo olhar, ``olho no olho''. Eu a ensinei a piscar, de vez
em quando dá certo: ela pisca pra mim, e eu pra ela. A~gente
``conversa''. Eu acho isso o máximo! Vai ver que já estou ficando
gagá… Quando eu chego já a encontro miando na porta e tenho que
tomar cuidado para ela não sair. Outro dia, ela conseguiu sair e foi até
a porta do vizinho da frente, ainda bem que consegui pegá"-la logo. Se
não… teria que subir um monte de escadas.

Às vezes, ela se empoleira no sofá, atrás de mim, e de repente eu vejo o
vulto dela. Como todo gato, ela é completamente silenciosa. Outras
vezes, se escarrapacha ao meu lado no sofá, o que é um pedido sutil para
que eu coce sua barriga. Ela adora tomar banho de sol nas janelas. É~incrível como os gatos adoram o sol. Ela vai perseguindo o sol pela casa
até ele chegar no sofá.

Uma vez eu fui para um sítio com os amigos e, chegando lá, percebi que
tinha deixado a Filó trancada em um armário. Fiquei desesperada. Uma das
pessoas do grupo, porém, era veterinário e falou que a Filó ficaria bem
até três dias, se o armário fosse bem ventilado. Ele era, então tive
coragem de voltar no terceiro dia, bem cedo. Abrir a porta do armário e
vê"-la sair foi um alívio enorme. Outras duas vezes, sem querer, claro,
deixei"-a entre a veneziana de madeira e a rede que coloquei no
apartamento, justamente para ela não cair mais. Assim que entrei no
apartamento, ouvi os miados desesperados dela e fui soltá"-la.

Eu achei que a Filó tinha direito a dar uma cria pelo menos. Uma vez,
deixei"-a numa clínica para cruzar antes de eu fazer uma viagem para
Goiânia. Todos os dias eu ligava para a clínica para saber como ela
estava. Meses depois, ela pariu quatro gatinhos, no pé da minha cama,
claro. Uma noite acordei com os miados dos gatinhos que ela carregava na
boca pelo apartamento inteiro. Era lindo de ver! Lindo!

No final do ano passado eu fiquei muito triste porque um ultrassom
revelou que a Filó está com um tumor maligno no intestino… Eu
fiquei arrasada. A~gente, o Rogério, eu e o Hailton (\versal{AT}), já foi três
vezes levá"-la ao Hospital Rebouças. Ela foi medicada duas vezes e reage
bem. Agora come uma latinha de ração cremosa por dia, além da ração
normal. Ela adora a ração cremosa e fica muitas vezes miando perto da
geladeira, pedindo desesperadamente para a gente dar. A~gente dá três
colheres de sopa por dia, como a veterinária nos orientou, às 8h00, às
14h00 e às 19h30.

A sorte é que a Cida, minha empregada nova, adora a Filó e cuida bem
dela, que se esfrega muito nos pés dela fazendo a maior festa..

Enquanto eu tomo o meu café da manhã, a Filó fica sentadinha no chão, do
lado esquerdo, ergue a cabeça e espera até eu dar para ela um pouco de
requeijão, na ponta do meu dedo. Ela lambe rapidamente e logo pede mais
e mais. Ela lambe, lambe e lambe. E~adora!

\begin{center}\asterisc{}\end{center}

\begin{flushright}\textbf{26/01/2012}\end{flushright}


Foi nos anos 80 que conheci a Dagui e o Saulo. Mulher e marido.
Maravilhosos. Mãe de santo e Pai de santo. Umbandistas. Corajosos.
Generosos. Muito generosos. Formaram muitos ``médicos'', como a Dagui
brincava, ao longo de vários anos. Trabalho árduo, muito bonito e
profundo, de autoconhecimento. Muitos e muitos anos de dedicação total.
Fé. Amor. Muito amor.

Foi a Rê Stella que me apresentou os dois, em meados de 1987. Eu tinha
tido uma mania gigantesca, naquela vez em que fiquei em São Sebastião.
Pintando, pintando e pintando. Tecidos lindos esticados no chão do
corredor. E~eu pintava… Um, com cor bordô, eu pintei inspirada no
Pollock, o outro, verde"-esmeralda, era mais árabe. Fiz um leve grafismo
dourado nele. Discreto. Discretíssimo. Bem bonito.

Pomba"-gira, Pomba"-gira, Pomba"-gira. A~Dagui falou que naquele tempo todo
eu tinha estado com um monte de pombas"-gira em cima da minha cabeça. Me
azucrinando. Me amaldiçoando. Me atacando. Me machucando. Elas faziam de
tudo pra eu ficar louca. E~eu fiquei. Bipolar. Louca. Louca,
louquíssima. Louca total. Sem cura. Até o último dia da minha vida.
Crueldade. Uma enorme crueldade. Quem será que fez isso comigo? Foi a
Pi, ex"-mulher do Sílvio, que era o meu namorado da época, com certeza
que sim. Aquela filha da puta que, apesar de trabalhar com crianças com
paralisia cerebral, é uma puta. Uma putona… É
inacreditável…

A Dagui falou que a macumbeira ali era ela. Então era ela que ia
desmanchar esse trabalho tenebroso. Ela só pediu pra eu ficar uma semana
sem tomar café e sem usar as cores preta e vermelha. Eu deveria também
acender uma vela para meu anjo da guarda. Eu esqueci e, quando percebi,
já tinha ido trabalhar a semana toda com aquele casaco velho de veludo
cotelê preto, finíssimo, que eu adorava… Mas os santos não me
castigaram.

Também lá no Clube da Turma fazia bastante frio. Ela mandou, também, eu
receber \emph{jhorey} de ``baciada''. Eu não sabia nada acerca do
\emph{jhorey,} nem da existência da Igreja Messiânica. Ela pediu para o
Saulo fazer um ``culto de antepassados'' para a Tuxa, na Messiânica. Eu
nunca tinha ouvido falar desse trabalho maravilhoso de cura espiritual,
através da imposição de mãos. Eu fui. O \emph{jhorey} me fez muito bem.
Levei a Helena na Messiânica para receber \emph{jhorey}, e ela gostou
muito. A~gente saía da difusão se sentindo ótima, com muita disposição
para enfrentar o dia.

Eu fiquei adepta do \emph{jhorey} e da Messiânica. Uns cinco anos depois
me tornei messiânica. Sou até hoje. Recebi meu \emph{orikari} numa
cerimônia solene e muito bonita. Precisei fazer duas vezes o curso! O
ministro disse que era porque eu nem sabia andar direito, como ele
observou bem… Eram os remédios. Malditos remédios. Benditos
remédios. Em todo caso, extremamente necessários. Uma tentativa heroica
de ter um pouco de saúde mental. Um pouco de paz. Um pouco de paz de
espírito. A~qualquer preço. Custe o que custar. Doa o que doer. Doa em
quem doer. É~foda. Fodíssima!

Depois de cinco anos frequentando com muita dedicação a Umbanda, eu
tinha melhorado pouco. Muito pouco, considerando o esforço enorme para
ir até lá, na lonjura onde é o Campo Limpo. Ir todo sábado quase, não
era mole. Apesar do fascínio pelos atabaques e pela ``gira'', eu me
desencantei. Apesar disso, levei um monte de gente para fazer consulta
lá. A~Jussara, a Marlene, a Patrícia, a Helena. Entre outros tantos.
Muitos outros tantos. Ficavam todos maravilhados e encantados, como eu,
fascinados mesmo. Os rituais eram, na verdade, muito bonitos. Aquele
monte de mulheres com saias brancas rodadas. Os homens todos vestidos de
branco. Os atabaques batendo, tocados por adolescentes bonitos e
saudáveis, para inaugurar a gira. Tum, tum, tum… e os médiuns
começavam a incorporar seus santos de cabeça. Era lindo! Lindo mesmo!!!
Eu adorava ir lá!

Banhos de ervas. Arruda. Alecrim. Guiné. Mágicos banhos de ervas. Da
Dagui. Da ``mãe Dagui''. Da ``Cabocla Guaciara'', santa de cabeça da
Dagui. Da Rosa, aquela baiana velhinha, desbocada, que também era santa
de cabeça da Dagui. Pra falar a verdade, eu nunca entendi muito tudo
isso, mas ficava absolutamente fascinada. Encantada. Eu adorava preparar
meus banhos de ervas. Eu saía do Templo Mãe Guaciara e ia direto naquela
chacrinha, em uma esquina da avenida Sumaré, onde uma velhinha tinha um
terreno enorme com todas as ervas de que eu precisava. Lá tinha também
uma árvore linda, que ficava carregada com bolas e mais bolas de flores
cor de rosa. Eu babava com essa árvore. Era uma maravilha!

Eu fazia tudo como a Dagui me ensinou. Pegava uma sopeira branca,
bonita, e punha as ervas lá. Fervia a água e jogava sobre as ervas, até
a metade da sopeira. Completava com água fria para ficar um banho
morninho. Tomava um banho normal. Fechava a torneira, jogava o banho de
ervas sobre a minha cabeça e, depois, abria o chuveiro para as folhas
que caíam no chão saírem, elas não podiam ficar lá. Eu tinha que catar
as folhinhas também. Depois não me enxugava, depois do banho de ervas a
gente não pode se enxugar. Tem que deixar o banho secar no corpo. Para
afastar os maus espíritos. Para afastar o mau"-olhado. Para afastar a
inveja. Inveja que seca e maltrata a gente. Isso todo mundo sabe. Todo
mundo já sentiu. Todo mundo já sofreu. Um dia na vida todo mundo já
sofreu, e muito. Um dia na vida todo mundo já viveu. Inveja. Pecado
mortal e destruidor. Que mata. Que quer matar. Destruir. Arrebentar.
Foder. Foder mesmo! Droga! Droga! Droguíssima!!! Por que será que alguns
invejam tanto alguns outros? Mesmo que eles sejam muito fodidos e tenham
uma doença incurável, como eu. Não é choro nem vela. Isso é a pura
realidade. A~verdade total dos fatos.

Já faz muitos anos que a Dagui se dedica integralmente à ``Casa do
Zezinho'', que ela fundou com seu grupo. Eu imagino que foi com seu
grupo de médiuns. Eu não sei ao certo. Eu só sei que esse era o maior
``sonho'' da Dagui. Ela falava bastante nisso para a gente. Lá eles
atendem 1400 crianças carentes, todos os dias, em tempo integral. Casa,
comida, estudo. É~assim.

Eu até estranho toda essa paixão que tenho pela Umbanda e pelo
Candomblé. É~algo só meu que, absolutamente, não tem a ver com a minha
educação. Com a minha família ou com aquele meu colégio cheio de urubus
negros. As freiras. As chatas das freiras. Eu só gostava de Madre Maria
do Redentor, aquela inglesa divertida que me deu aula de desenho no
ginásio. Rosáceas e mais rosáceas. Ela mandava a gente fazer rosáceas o
tempo todo. Rosáceas coloridas. Bem bonitas. Eu adorava fazer rosáceas
com meu compasso e lápis de cor. Eu pintava as ditas cujas bem de
levinho com a minha caixa de lápis de cor. Eu também achava lindas
aquelas faixas coloridas que a gente usava no colégio. Elas tinham mais
ou menos três centímetros de largura, e a gente tinha que enfiar nos
passantes. Primeiro na cintura, daí a faixa ia até as costas, a gente
cruzava e puxava pelos ombros e, depois, por baixo dos braços, até dar
um nó leve e um laço atrás do qual pendiam duas pontas. Era uma operação
lenta e delicada que a gente repetia todos os dias. No primário, a gente
usava uma faixa rosa, depois, uma faixa rosa"-claro com risquinhos
brancos, em seguida, uma bordô e, por fim, uma verde. Todo dia.

Todo dia.

Todo dia. Sempre. Sempríssimo.

``Todo dia ela faz tudo sempre igual, me sacode às seis horas da
manhã''.

Eu não sei por que grande parte das pessoas chamam a Umbanda e o
Candomblé de ``religiões primitivas''. De primitivas elas não têm nada.
São muito complexas, sofisticadas, têm uma simbologia muito rica.
Riquíssima. Coloridíssima. Isso sim, elas têm mesmo. Como que alguma
coisa que veio da África, berço da nossa ancestralidade, pode ser
primitiva?… E a arte Africana, é primitiva? E onde ficam aquelas
máscaras fantásticas que inspiraram até Picasso? Me diga, me diga aí!
Não digo que fico brava. Bravíssima. Guias, guias e guias. Cada uma
delas simboliza um santo, uma santa. Azul"-anil vivo, Ogum ou São Jorge.
Azul"-claro ou branco brilhante, Iemanjá ou Nossa Senhora da Aparecida.
Amarelo"-vivo, Oxum ou … pra falar a verdade, eu não sei mais a
santa equivalente a Oxum na Igreja Católica. Preto e vermelho é Exu, o
demo. O~que era tão bonito que queria ser Deus. Virou demônio, o que
abre as giras, o que abre a vida. Os caminhos da vida. Preto, preto, ou
vermelho, vermelho. Branco é Oxalá. Sexta"-feira é dia de Oxalá e, em
homenagem a ele, a gente deveria se vestir de branco. Pra falar a
verdade, eu nunca faço isso. Eu esqueço. Agora vou começar a fazer.
Tenho uma calça branca de \emph{shantung} nova, que é linda, e algumas
blusinhas. Falta mesmo é uma calça branca de lãzinha. Acho que vou
mandar fazer. Posso também usar um abrigo que é uma coisa bem mais
simples, e mais barata, e muito mais confortável. E~as comidas de santo,
então? Eu nunca tive a sorte de experimentar, mas, quem teve, diz que
são comidas divinas. Requintadíssimas. Delicadíssimas. Levíssimas. Eu
adoraria experimentar. Pelo menos uma vez na vida… Vatapá,
mungunzá, acarajé de santo, deve ser tudo ótimo! Já pensou? Só imaginar
os alguidares com comidas de santo já me dá água na boca. \versal{ALGUIDAR}.
Palavra linda. Como tem ``al'' no início, deve ser de origem árabe, como
a gente bem aprendeu na escola. Eu acho os árabes fantásticos.

Eu sou suspeita para dizer porque meu avô era libanês, libanês mesmo.
Ele foi mascate em Minas Gerais. Ele ia de casa em casa, no seu
burrinho, vendendo rendas, panos, botões e não sei o que mais. Depois
ele teve uma lojinha em Carmo do Rio Claro, que é a cidade mineira onde
eles fazem aqueles doces de mamão e abóbora lindos, cheios de desenhos
recortados de frutas e flores. Depois, ele veio para São Paulo e teve
uma loja na 25 de março. Claro. Eu tenho o maior orgulho da minha origem
libanesa, libanesíssima! Nada como ter ``\emph{sangue novo}''. Eu sempre
gostei e me orgulhei. Sangue novo de imigrante. Garra de imigrante.
Esperteza, no bom sentido, de imigrante. \versal{SAWAYA}. Esse é meu sobrenome de
libanesa ou pós"-libanesa.

Eu acho graça porque às vezes quando digo meu sobrenome para alguém no
telefone, às vezes as pessoas perguntam se é de origem japonesa.
Imaginem, eu japonesa… Nem pensar… Aquela coisa toda
certinha e reguladinha dos ``japas'' não tem nada a ver comigo. Já a
arte japonesa, isso sim! Eu daria a vida para ter uma gravura do
Hiroshigue. Daquelas bem bonitas. Coloridas. Levemente coloridas.
Delicadíssimas. Salve, mestre Hiroshigue! O mestre dos mestres da
gravura japonesa. Ozu, Ozu, Ozu. A~gente viu um filme do Ozu outro dia
aqui na fazenda. Ele é o \versal{MESTRE} \versal{DO} \versal{ENQUADRAMENTO}. Não tem outro. Não tem
igual. Não tem mesmo. E~os tecidos dos quimonos, então?…
Maravilha. Total maravilha! Maravilha total. Flores e folhas
delicadíssimas. Desenhos absolutamente sutis. Eu vi uma exposição
fantástica de kimonos japoneses na Pinacoteca do Estado. Até comprei o
livro com as reproduções, de tão maravilhosos que eram. E~valeu a pena
comprar. Livro é o tipo da coisa que sempre vale a pena comprar. Sou uma
digna filha do Dr. Sawaya, mesmo… Ainda mais \versal{LIVROS} \versal{DE} \versal{ARTE}.
Graças a Deus, tenho uma bela coleção para ficar olhando, curtindo e
\versal{APRENDENDO}.

E a comida japonesa, então? \emph{Missoshiro}, \emph{sushi},
\emph{sashimi}. Para falar a verdade, eu conheço só o mais básico, mas
eu adoro comer essa comida sofisticada e linda que leva tanto arroz e
tanto peixe cru. E~os \emph{temakis}, então? São Paulo agora está
repleta de ``temakerias'', eu não entendo por que eu ainda não
experimentei. Não entrei em nenhuma. Na rua Tabapuã, bem perto da minha
análise, tem uma, mas eu sempre prefiro ir no Fifties, comer um bom
x"-búrguer salada e um \emph{milk"-shake} de chocolate ou morango. O~Zyprexa dá mesmo uma fome enorme, gigantesca. Não tem \emph{milk"-shake}
melhor que o de lá. Não tem mesmo. E~aquelas batatinhas fritas
sequinhas, então? Nunca vi tão boas em outro lugar. É~um manjar dos
deuses. Elas são perfeitas. Super perfeitas. Delicadas, cortadas
fininho, e vêm naquela caixinha linda de papelão vermelho. Mostarda e
\emph{ketchup} para acompanhar. Pratinho branco pequeno. Fifties.
Fifties. Fifties. De vez em quando, eu gosto de ir lá. Equilibrar os
pensamentos, as intuições e os sentimentos depois da análise maravilhosa
com o Celso Vieira de Camargo.

\begin{center}\asterisc{}\end{center}


\begin{flushright}\textbf{28/01/2012}\end{flushright}


Papai foi da geração de brasileiros fundadora da \versal{USP}, dos professores
que substituíram os estrangeiros que vieram para fundá"-la. Filho de
imigrante libanês, batalhou pra chuchu. Vestiu a camisa até o último de
seus dias. Suou a camisa. Suou. Suou. Suou. Nos deu um exemplo magnífico
do que é lutar por seus ideais. Eu agradeço a Deus por ter tido esse pai
maravilhoso, grandioso, brilhante. Não é exagero, não. Ele era tudo isso
mesmo, tudo ao mesmo tempo. Brilhante, maravilhoso, grandioso,
inteligente, inteligentíssimo!!!

Eu tive a sorte de ser a ``queridinha`` do papai. Não por mérito meu,
mas simplesmente porque ele me escolheu. A~gente se gostava muito. A~gente se entendia muito. A~gente se olhava muito. Olho no olho. A~melhor
recordação da minha infância era aquele olhar embevecido, doce, do papai
me fitando. Aquele olhar árabe. Ele salvou a minha infância asmática. Eu
vivia no pronto"-socorro e dormia com um montão de travesseiros. Por isso
fui uma criança adulta. Eu sempre me senti completamente rejeitada pela
mamãe. E~fui, fui mesmo. Afinal, eu tinha cara de \versal{TURCA}. Talvez ela me
rejeitasse porque eu era a filha mais libanesa de todas. Tudo que meus
olhos ``cor de mel'' tinham de bonitos, o meu nariz grande tinha de
feio. Nariz árabe, nariz libanês. Eu não me orgulhava nada dele. Aos 19
anos, fiz uma plástica fantástica com o Dr. Martinho. Foi muito
bem"-sucedida, porque ele tirou só o carocinho feio do meu nariz. Não
teria cabimento cortar mais, afinal eu tenho rosto grande. Uma bocona
grande, bonita e bem desenhada, como todos os Barros, Barbosa Barros,
todos Camargo Andrade Barbosa Barros têm.

A família da minha mãe é quatrocentonésima. De Campinas. Tem povo mais
quatrocentão que o de Campinas? Vovô Barros foi um grande médico lá.
Ginecologista. Até hoje tem um busto dele em frente à maternidade.
Depois, eles vieram para São Paulo e passaram a morar naquela casa
maravilhosa do Ramos de Azevedo. Parece que foi meio por acaso que foram
parar lá, no bairro do Paraíso. Paraíso, nome lindo para um bairro. Eu
não me lembro bem. Eu só me lembro daqueles vitrôs coloridos,
fantásticos, que tinha na sala de jantar. Eles ficavam na vertical,
formando um suave semicírculo. Um depois do outro, um depois do outro.
Uma verdadeira alegria! A sala, do lado esquerdo, onde tinha os vitrais,
era meio arredondada… A luz se filtrava colorida. Azuis, amarelos
e vermelhos. Lá estava também aquela fantástica e imensa mesa oval de
madeira pesada, onde deveria caber umas 15 pessoas. Vovô e vovó tiveram
13 filhos! Eu lembro muito bem de toda a sala. Os vitrais, a mesa com as
suas cadeiras acompanhadas de uma soberba cristaleira. Todas as peças
tinham detalhes em bronze, magníficos, lindos. Lembranças de infância. A~gente tem as boas e também tem as más. A~casa ficava na Rua Alfredo
Ellis. Até hoje a gente se refere a ela como a ``Alfredo Ellis''.

-- Você lembra daquele porão da Alfredo Ellis onde a gente brincava?

-- Você lembra que o piso da copa era de ladrilhos brancos, minúsculos,
sextavados?

-- Você lembra da Sofia, aquela cozinheira fantástica da Alfredo Ellis?
Ela fazia quitutes deliciosos.

-- Você lembra daquele macaco que tinha lá? No jardim! E das
jabuticabeiras?…

-- E daquela sacada de onde descia uma escadinha com a grade toda
rendilhada que ia dar no porão?

Alfredo Ellis, boas e mágicas lembranças da minha infância. Eu adorava
ir lá.

Eu não conheci vovô Zuza. O~meu parto foi o último que ele fez. Eu o
conheci só um pouquinho. Senti sua amorosidade. Quarenta dias depois,
ele morreu, teve um infarto lá na fazenda Atibaia, que era do tio Tá e
da tia Cecília. Tios maravilhosos. Encantadores. Eu bebi leite de luto!
De muito luto. De muita tristeza. Ele era adorado por vovó e por todos
os filhos. Devia ser uma pessoa muito carismática. Eu sempre ouvi dizer
que ele era também muito charmoso e elegante, com seus belos ternos e
bengalas com castão de marfim, de ouro, talvez. Vovó Úrsula era mais
severa. Desde que tia Lúcia morreu, ela passou a usar luto fechado até
morrer. Luto, preto, preto, preto. Até morrer. Eu só conheci vovó de
luto. Sempre de luto. Lutíssimo. Doído e sofrido.

Tia Lúcia, pelo que todos contam, era uma mulher lindíssima. Tocava
piano. Esqueci de dizer que na Alfredo Ellis tinha também uma sala de
música, se não me engano, com um belo piano de cauda. Eu não me lembro
muito bem. A~memória trai a gente. Às vezes a gente escreve a memória do
desejo, não memória da verdade. E~por acaso existe verdade? No quê? Me
diga. Me diga. Me diga. Onde? Como? Quando?… ``De perto ninguém é
normal'', como bem diz o Caetano. Só sei que aquela sala era linda. Até
hoje, fechando os olhos, enxergo tudo. Ou quase tudo. Tinha também a
sala de estar com aqueles móveis durinhos, bem formais, tapete, tapete,
tapetão e duas cristaleiras lindas, cheias de objetos bonitos,
atraentes. Objeto do desejo. Desejo. Desejo. Desejo.

Quando eu era criança, eu acho que desejei muito poder, pelo menos,
mexer naqueles objetos. Nem pensar em abrir, em mexer, em pegar. \versal{NÃO}
\versal{PODIA}, \versal{MESMO}! \versal{NÃO}, \versal{MESMO}! Claro que eu ficava fascinada por tudo aquilo
e queria mexer em tudo. Brincar um pouco. Brincar com cuidado…
``Eu prometo que vou ter cuidado, vó''. Até hoje sou apaixonada por
objetos bonitos. Quase sem perceber, fiz uma pequena coleção de garrafas
e vasos de vidros coloridos que ficam na prateleira branca da minha
cozinha. Tem um pouco de tudo. Garrafas de cerveja que eu acho que têm
um desenho interessante, a da ``Norteña'', por exemplo. Vasos coloridos.
Turquesa. Eu amo a cor turquesa. É~a de que eu mais gosto. Lá tem um
vidro lindo turquesa"-escuro com tampa e também um solitário, daqueles
que é para pôr só uma florzinha. Verdes, vermelhos, amarelos. Lá tem
vaso de tudo quanto é cor. Alguns eu comprei na feira do Bexiga, outros,
na praça Benedito Calixto e, outros, numa loja incrível que chama L.S\, e
fica na rua João Cachoeira. O~problema é que tudo que tem na L.S\, é
muito caro, então eu vou pouco. Como a minha psicanálise é no Itaim, eu
passo três vezes por semana pela rua João Cachoeira. Olho sempre a
vitrine da L.S\, e fico babando, mas não paro e não entro. Ainda bem que
não dá nem tempo de fazer isso. A~loja é tão fantástica porque é da
Lúcia Steiner. Ela é filha do Benjamin Steiner, que foi um dos mais
refinados e cultos antiquários de São Paulo. Eu nunca conheci ninguém
com um olhar tão apurado para objetos como Lúcia e o marido. Eles
trabalham muito com vidro, que eu adoro. Numa ``crise consumitiva'', eu
comprei lá três pesos para papel, redondos, de vidro, lindos. Um é meio
malhado de branco e preto fosco. Outro é transparente e tem uma espiral
preta desenhada em toda a volta, por dentro. O~terceiro, agora, eu não
estou lembrando, só sei que é bonito, muito bonito. Quando eu estava
reformando o apartamento, também comprei lá três garrafões
vermelho"-vivo, que eu botei no chão, na cozinha, e ficou muito legal.

Às vezes, eu consumo bastante, mais do que deveria, nas manias,
então… mas devido à bipolaridade, às vezes, eu fico trancada
muito tempo em casa e, em ficando, curto muito contemplar meus objetos.
É~verdade, não é desculpa esfarrapada, não. Olhar objetos bonitos me faz
bem. Muito bem. Ver \versal{TV} também. Adoro as novelas da Globo. Talvez me
consolem e ajudem a aguentar os trancos e os barrancos, eu não tenho
dúvida. Possuir um pouco de beleza faz bem. Posse. Posse. Posse. Maldita
posse. Pecado mortal? Pecado mortal? Acho que é problema meu. Educação
católica. Culpa. Maldito mundo da culpa… Eu conheço pessoas que
têm mil vezes o que eu tenho de objetos belos e não estão nem aí…
Compram, compram e compram. No Brasil, ou no exterior, onde estiverem.
Itália, França, Japão, Londres. Marrocos, quem sabe. Na Sotheby's, quem
sabe. Garimpam, garimpam e garimpam. No Marché aux Puces, em Paris.
Portobello Road, em Londres. Eu achei a Portobello Road o máximo!
Comprei um casaco tipo militar, azul"-marinho, inesquecível. Onde
estiverem, as pessoas, compram, compram e compram, e fazem coleções
fantásticas. Maravilhosas. Sem dúvida elas têm mais saúde mental do que
eu por não terem tanta culpa e mandarem bala. Eu não consigo fazer isso
e nem posso. Não tenho bala na agulha. Eu vou comprando devagar e com
muito prazer o que eu vou podendo comprar. E~morro de prazer.

No meu ateliê, que fica dentro do meu apartamento, do lado esquerdo da
mesa de pintar tem umas quatro ou cinco prateleiras, repletas de ``arte
popular'' que eu fui ajuntando ao longo da minha vida. Tem quatro
casinhas de barro fantásticas, com frontão e tudo, que eu trouxe da
Paraíba há uns 40 anos, quando passei um julho lá, junto com o Ivan. Tem
um boi preto que eu trouxe de Cusco. Lá tem um lugar próprio para beber
``chicha'', quando terminam as colheitas e o povo comemora. É~uma bebida
de milho fermentada que eu vi vendendo no mercado, mas não tive coragem
de beber. Depois me arrependi de não ter ao menos experimentado. Eu
também trouxe do Peru duas réplicas de esculturas da Fundação Larco
Herrera, que achei numa loja do aeroporto quando parti. Uma é um gato,
com o lugar próprio para beber a chicha, a outra é um vasinho. Eu quase
pirei quando visitei essa fundação, nunca imaginei que houvesse um
acervo tão fantástico de cerâmica peruana. São milhões de peças
belíssimas. Como na época eu fazia cerâmica, o entusiasmo foi enorme.

Se eu for descrever todos os objetos que estão nas minhas prateleiras,
eu não vou acabar nunca esse livro. Então, chega! Chega, Regina. Tome
tento! Eu só vou contar que lá tem um monte de cerâmicas lindas e
minúsculas que eu também trouxe do Peru. Tem dois caminhõezinhos
carregados de frutas que cabem na metade da palma da minha mão. Tem um
circo cheio de pessoinhas na arquibancada redonda que é inimaginável.
Tem bumba meu boi. Tem vários. Um deles foi a Sara Carone quem me deu
uma vez, quando veio do Nordeste. Que mais que tem? Não tô lembrando.
Agora, chega. Chega, mesmo. Tem também um fogãozinho amarelo de lata com
o qual todas as pessoas se encantam. Ele está bem lá em cima. Árvores.
Eu fui fazendo uma bela coleçãozinha de árvores, quase sem perceber.
Árvores delicadamente recortadas a \emph{laser}, na madeira fininha,
agora eu não estou lembrando o nome da artista. Até o Jardim no dia em
que veio aqui em casa pediu pra eu mostrar para ele o nome dela, a gente
tem que destacar a arvorezinha do apoio para ver o nome. Pra falar a
verdade, achei a tal arvorezinha bem semelhante às das gravuras do
Jardim. Ele também é apaixonado por árvores, só que as dele dão de 10 a
zero nas dela, claro. Tem também um conjunto de quatro palmeirinhas de
metal escuro que todo mundo adora. É~uma árvore linda, cheia de galhos
com passarinhos, que um dia eu tive a enorme coragem de comprar na
Benedixt. Valeu a pena. Ela é linda, e eu curto muito até hoje; tem
também duas árvores lindas de cerâmica do Megumi, fantásticas, são as
mais bonitas.

Pra meu desespero, há pouco tempo abriu na Vila Beatriz, que é muito
perto de casa, uma loja especializada em \emph{design}, que se chama
Casa Canela. Os objetos de lá são uma maravilha. Eu já cometi algumas
ousadias lá. A última foi a compra de um vaso de murano transparente que
está no limiar do cafona. Lembra uma ânfora grega, segundo a Helena. A~Sara, quando viu, disse que jamais compraria um vaso como aquele. Ela é
um barato. Livre, leve e solta. É~por isso que adoro ela. Paguei o vaso
em três ou quatro vezes, e é fato que ele ficou maravilhoso na minha
salinha. Eu o coloquei naquele móvel abaixo da janela. Eu gosto tanto
dele que voltei a comprar flores frescas semanalmente, lírios brancos
bem compridos, antúrios vermelhos, rosas e outras flores. A~cada semana
uma escolha. A~cada semana, uma curtição diferente. Acho que só por isso
já valeu o gasto. Tem coisa melhor do que fazer um vaso novo, com flores
diferentes a cada semana? Faz bem para a alma. Faz bem para o coração.
Faz bem para tudo que a gente é. E~também para o que a gente não é.
Será?

\begin{center}\asterisc{}\end{center}

\begin{flushright}\textbf{Fazenda -- 29/01/2012}\end{flushright}


Retomando… Papai, Paulo Sawaya, foi da geração fundadora da \versal{USP}.
Foi do primeiro grupo de brasileiros que substituiu os estrangeiros que
criaram a \versal{USP}. Como o Lévi"-Strauss, por exemplo, entre muitos outros.
Muitos eram judeus que fugiram da Europa em guerra.

Papai era muito apaixonado pelo que fazia e, quando ele queria, queria.
E~queria logo. A~jato. Rápido. Ele era um homem de decisões rápidas e
virava um trator. Eu, pessoalmente, acho que esse foi mais um ótimo
exemplo que deixou para a gente. Nós todos somos meio tratores, nós, os
10 filhos que ele teve, que agora são nove. Quando a gente quer, a gente
quer e manda bala, a gente age. Trabalha pra chuchu. Veste a camisa. Se
entrega. Sobretudo no trabalho. Pelo menos eu. Com os meus namorados,
que foram muitos, eu nunca fui assim. Talvez devesse ter sido. Talvez
tivesse sido melhor. Mas eu não tenho o que reclamar dos meus namorados.
Gostei muito de todos eles, e eles me fizeram muito bem. Bem mesmo.
Paixão é doença, mas, às vezes, é bom. Bonzíssimo. Alimenta.
Desalimenta. Desalimentante, mesmo.

Papai foi um dos fundadores da Sociedade Brasileira para o Progresso da
Ciência, que nasceu lá em casa, junto com outros dois cientistas, um era
o Dr. José Reis e o outro eu não estou lembrando o nome agora. Isso
ocorreu lá pelos anos 50. Eu lembro uma vez que a gente viajou para Belo
Horizonte com papai porque a \versal{SBPC} foi lá. Foi o máximo! Eu adorei! Foi
quando conheci Sabará, Tiradentes, Mariana. O~barroco mineiro. Ouro.
Ouro. Ouro fosco. Imagens de santos belíssimas. O~mistério das igrejas.
Rezar. Eu gosto de rezar. Quando pequena, eu saía do Madre Alix, ia para
casa de vovó e ficava rezando o terço com ela. Mistério. Ela me ensinava
os mistérios do terço. Dolorosos. Gozosos. E~os outros? Misteriosos. Ela
tinha muita fé. Era super católica e morava em frente à Igreja São José,
no Jardim Europa. Acho, imagino, que ela ia todos os dias à missa.

A Bete ficava lá embaixo vendo desenhos animados na \versal{TV}, porque lá em
casa não tinha \versal{TV}, papai achava que \versal{TV} era uma droga e que a gente tinha
mesmo era que ler, ler muito, sempre, sempre. Sempre, sempre, sempre.
Ele lia livros de todos os tipos. A~Bete, embaixo, e eu, lá em cima, com
a vovó Úrsula, aprendendo a rezar o terço. Eu tinha dó daqueles pés
inchadíssimos que ela tinha. Isso é uma das recordações de dor da minha
infância. Eu achava aquilo um horror porque devia doer muito. Eles
ficavam muito apertados naqueles sapatos pretos. Pretos. Pretos. Pretos.
Que dó! E eu nunca a vi reclamar sobre isso. Ela, como as pessoas da sua
geração, não reclamava nunca. Nunquíssima… Ela bordava vestidos
lindos. Bordados muito sutis e delicados que a gente adorava. Eu nunca
me esqueci daqueles bordados, daqueles vestidos da minha infância que eu
amava. Vovó também fazia camisinhas de cambraia para bebês, que tinham
os mesmos bordados delicados em azul"-claro, rosa"-claro, amarelinho. Seu
bordado era um primor. Acho que naquelas alturas eu já era bem vaidosa.
Como continuei sendo. Até agora. Mentira! Até agora, não.

Pra falar a verdade, eu ando bem desleixada. Verdade ou mentira? Quando
posso, compro roupas de boa qualidade porque acho que elas disfarçam
melhor a minha obesidade. Nada como uma boa roupa, com um bom corte e
tecido bom, para disfarçar os pneuzinhos, ou pneuzões… Merda!
Zyprexa, Zyprexa, Zyprexa… Um porre! Eu adoro roupa boa. Do
Shopping Iguatemi. Lá tem a Erica's e a Sinhá, com \emph{alguns} modelos
de roupa que cabem em mim. Se estou com bala na agulha, compro lá, se
não estou, vou até a Julmitex, na rua José Paulino, que tem alguns
modelos bacanas bem em conta. O~que eu mais gosto, na José Paulino, é
daquele monte de camelôs. Echarpes, colares, \emph{bijoux}. Eles vendem
um pouco de cada coisa. Um pouco de tudo, até abridores de lata,
descascadores de legumes, trim, tesourinhas e tesourões. Uma beleza! Eu
não resisto! Eu adoro uma feira. Qualquer feira. Não adianta, eu sou uma
turca mesmo.

De todos os sintomas colaterais, o que me deixa mais triste é mesmo a
obesidade. Tenho dificuldade para levantar do sofá, do carro, da
cadeira, de qualquer lugar… Não é mole carregar 20 quilos a mais,
que equivalem a quatro sacos grandes de açúcar. Açúcar União. Açúcar da
minha infância. Perdi muito em agilidade e em flexibilidade. Colesterol
demais também faz mal ao coração. Tudo que é demais faz mal para alguma
coisa. Já aprendi e aprendi bem, na marra.

Quando jovem, eu era naturalmente mais para a magra. Tinha um corpo
bonito, sabia disso, e gostava disso. Seios grandes, cintura fina e
quadril largo. O~típico corpo da mulher brasileira, não é por nada. Eu
era mesmo. Claro que eu gostava disso. Qual é a mulher que não gosta de
saber que é gostosa e desejada? Qual é a mulher que gosta de ser um
elefante, como sou agora? Ser uma mulher feminina, atraente e gostosa.
Atrair os olhares masculinos é muito gostoso. A~gente se sente mulher,
mais mulher. Aos 63 anos, eu sinto muita falta disso. No Brasil, pelo
menos, nenhum homem olha para uma mulher da minha idade. Quase nenhum. É~raro acontecer. É~a praxe. A~praxe. A~cultura secular brasileira.
Mulheres de 60… fora! Foríssima. Todas as minhas amigas têm a
mesma queixa. Claro que isso é triste. Tristíssimo. Então, a gente,
quase sem perceber, vai ficando meio desleixada. A~gente nem liga mais
tanto para as roupas e as \emph{bijoux} que a gente usa. A~gente começa
a preferir mais o conforto à beleza. A~gente acaba preferindo usar um
abrigo confortável e uma crocs. O~que tem suas vantagens. Abrigo e crocs
são duas coisas bonitas, baratas e confortáveis. Muito confortáveis. E~andar de havaianas, então? Tem coisa mais deliciosa?

Agora comecei a me lembrar do tempo em que eu usava biquíni. Mamãe, que
era muito severa e pudica, não me deixou usar. Eu comprei um biquíni
meio grande, azul"-marinho com um arremate branco, e pedi para a Soninha
dizer para a mamãe que era presente de Natal para mim. A~Soninha, minha
eterna e querida aliada, topou a empreitada. Foi assim que pude começar
a usar biquíni, como todas as moças da minha época faziam. Quase todas.
Algumas mães eram ainda mais pudicas que a minha mãe e não deixavam de
jeito nenhum que as filhas usassem. O~meu primeiro biquíni era de
helanca, uma helanca meio grossa, pra falar a verdade. Mais tarde vieram
os de pano, era só achar um bom modelo, uns paninhos de florzinhas e
depois mandar a costureira copiar. Tenho até hoje uma foto que papai
tirou de mim na piscina da Fazenda Santa Rita. Eu estava numa posição de
lado, de biquíni, Édipo, Édipo. Vai e volta. De um lado para outro. De
outro lado para o um. É~um tráfico de influência, influência de tráfico.
Tráfico. Tráfico. Tráfico.

Traficar afeto é bom. Muito bom. Eu adorava papai, mesmo, e ele era, de
fato, um cara adorado por muita gente. Em primeiro lugar por mamãe, fã
total e absoluta dele. Depois, pela maioria dos filhos. Especialmente
eu. Depois, por aquele monte de alunos da \versal{USP} e de outros lugares. Vira
e mexe tinha algum estudante nordestino morando com a gente até achar um
lugar decente para morar. Nossa casa era modesta, mas muito aberta e
movimentada. Sempre, ou quase sempre, tinha algum hóspede de fora
morando conosco, e isso sem dúvida era muito bom para abrir a nossa
cabeça para o mundo. Sair do ``nosso mundinho'', pequenininho,
apertadinho, quando ele era assim ou estava assim. Pequeno e arejado.
Desapertado. Desarejado. Desarejadíssimo.

Papai também fez parte, durante séculos, da Sociedade São Vicente de
Paulo, criada por Frederico Ozanan, na França. É~claro que ele foi
presidente dessa sociedade por longos anos. Ele era, de fato, um cara
ambicioso, muito ambicioso, simpaticamente ambicioso. Isso é óbvio. Mais
que óbvio. Papai e mamãe tinham ``os pobres'' deles e, às vezes, iam
levar cobertores para eles. Eu só acho estranho que nunca levavam a
gente junto com eles nessa boa ação.

Papai também participava do Conselho da Cultura Inglesa. Entrou como
aluno e foi parar no conselho, claro! Como sempre. Ambição. Ambição.
Ambição. Saudável ambição. Ambição pelo bem dos outros. Nem sempre, é
claro. Afinal, ele era um ser humano falível. Falível como qualquer
outro ser humano. Ter ambição, sobretudo no trabalho, foi outro bom
exemplo que papai nos deixou. Nesse sentido, pelo menos, somos todos
ambiciosos, bastante ambiciosos. O~Sylvio, por exemplo, jamais teria
sido diretor da \versal{FAU}/\versal{USP} se não tivesse um mínimo de ambição. E~ele teve
a grande sorte de nascer, ou ter se tornado, político. Um bom político
para defender as causas em que ele acredita. Coitado, na gestão dele,
pegou o abacaxi da reforma do telhado da \versal{FAU}. Coisa que nenhum diretor
antes dele conseguiu resolver. É~muito estranho, pelo menos para mim,
que um prédio criado pelo Artigas tenha esse tipo de problema, que
parece insolúvel até hoje. Outro abacaxi que ele pegou em sua gestão foi
uma greve de alunos, que não aconteceu só da \versal{FAU}. De nós todos, só ele
seguiu a carreira universitária. Em 64, era diretor do \versal{DCE} e quase teve
que se exilar porque assinou o manifesto dos sargentos. Ele era diretor
do Diretório Central dos Estudantes, e o Serra, da \versal{UNE}. Ficaram
próximos, mais ou menos, por causa da política estudantil.

\begin{center}\asterisc{}\end{center}

\begin{flushright}\textbf{31/01/2012}\end{flushright}


Eu escrevi este livro para ajudar outros bipolares e, especialmente,
suas famílias, porque percebi que a minha fica em pânico quando entro em
crise, sobretudo em mania. Até hoje, 28 anos depois de eu ter ficado
bipolar. Espero que eu tenha conseguido ajudar bastante todo esse povo
sofredor. Sofridíssimo. Porque esta doença é mesmo um eterno sofrimento.
Um sofrimento enorme que vem de repente e vai de repente, sem avisar.
Por isso é uma doença muito cruel, crudelíssima. Sofrida. Sofridíssima.

O maravilhoso foi que, ao escrever, eu fui exorcizando os demônios da
minha bipolaridade. Eu não tinha a menor expectativa de que isso
acontecesse, pudesse vir a acontecer. Os demônios da bipolaridade são
fortíssimos. A~gente tem que ter muita força para enfrentá"-los. Tem que
ter até vigor físico. Por isso, também, me fez muito bem escrever este
livro. Agradeço a Deus!

\versal{XÔ}, \versal{SATANÁS}!!!

\versal{XÔ}, \versal{LÚCIFER}!!! \versal{FODAM}"-\versal{SE}!!!

Não me encham mais, nunca mais, porque eu já cansei de frequentar
religiosamente a Umbanda e o Candomblé para expulsar vocês da minha
vida… E pra falar a verdade, não consegui. Quando eu sinto algo
que não é do domínio do meu racional e nem da minha intuição, apelo para
o jogo dos búzios. Até hoje. Eu respeito muito o Candomblé e a Umbanda.
Há pouco tempo, inclusive, eu fiz um \versal{EBÓ} e um \versal{BURI}, tudo na mesma noite,
rituais de Candomblé. Mais uma vez entreguei minha cabeça para Iemanjá
cuidar dela e eu sofrer menos. O~mínimo possível. Por coincidência, meus
santos de cabeça são Iemanjá e Ogum. Nossa Senhora da Aparecida e São
Jorge. Tenho uma pequena imagem de cada um no meu quarto.

No início deste ano, aconteceu algo inacreditável. Pelo menos para mim.
Depois de eu sofrer durante 28 anos com a bipolaridade, quatro irmãos
meus pediram, espontaneamente, uma reunião com o Del Porto para se
informarem sobre o meu caso. Foi em fevereiro de 2011, quando a Tina
estava passando um tempo aqui. O~Zeca, o Sylvio, a Tina e a Soninha
foram até o consultório do Del Porto. Ele disse que não faria mais
eletroconvulsoterapia comigo porque gera mania. Disse também que eu
estava muito sozinha e precisava muito de apoio afetivo da família para
enfrentar a doença. Segundo o Sylvio, que é meio exagerado, o Del Porto
teria até sugerido um rodízio de irmãos. Eu acho que dificilmente o Del
Porto falaria isso. Ele é muito discreto e equilibrado.

O fato foi que o Sylvio se sensibilizou e foi por conta dele procurar
outro psiquiatra, que é o Dr. Paulo Toledo, que tem também um grande
conhecimento de Jung e de Astrologia e valoriza muito a Arte"-terapia
como instrumento de trabalho. Ele disse ao Sylvio que é muito importante
o paciente formalizar o seu inconsciente através da expressão artística.
O~Sylvio, então, me ``intimou'' a passar as tardes das quintas"-feiras
com ele. Depois do almoço ele ficaria desenhando e eu pintando. Eu
topei, aliás, não tinha como não topar. Era uma intimação, mesmo, e pra
mim, até hoje, ``irmão mais velho é irmão mais velho''… Depois de
um certo tempo, a gente começou só a drinkar uísque, beliscar castanha
de caju e conversar.

A arte"-terapeuta que o Dr. Paulo recomendou era muito simpática, mas era
uma mocinha. Pra falar a verdade, sem falsa modéstia, eu dava de 10 a
zero nela. Afinal de contas, já faz algum tempo que eu sou uma artista
plástica profissional. Uma pintora. Uma boa pintora. Eu não queria
perder tempo com aulas que não acrescentavam nada para mim e as
abandonei. Combinei com o Sylvio que, de vez em quando, eu levaria meu
trabalho para o próprio Dr. Paulo fazer uma análise simbólica, porque
que ele é ``bamba'' nisso. Já levei uma vez, e foi muito bacana. Adorei
esticar os panos pintados no chão e ver o todo. Agora que o Patrick Paul
passou a morar no Brasil, também pedi que ele às vezes fizesse análises
simbólicas do meu trabalho. Ele é um médico francês que desenvolveu um
conhecimento esotérico excepcional, a partir dos dogmas da Igreja
Católica. É~o único guru católico que conheço. Já me consultei umas três
ou quatro vezes com ele. Foi o Rogério que me apresentou.

\begin{center}\asterisc{}\end{center}

\begin{flushright}\textbf{05/02/2012}\end{flushright}


Espero ter deixado bem claro neste livro a importância de aprender a
discriminar os primeiros sinais das crises para poder debelá"-las logo no
início. A~gente consegue. Treinando bastante, a gente consegue. Espero
que também esteja claro que é importantíssimo ter um ótimo psiquiatra,
no qual você confie muito. Procure, procure e procure. Você vai achar. É~preciso respeitar religiosamente as consultas que ele marcar. Sejam
semanais, quinzenais, mensais, o que for. Tomar religiosamente os
remédios que o médico receitar, \emph{exatamente no horário}\textbf{} em
que ele mandar. Treine"-se para aguentar os efeitos colaterais. Você tem
que aguentar, é melhor do que ficar pirada e ser internada, o que é um
verdadeiro inferno. Se você conseguir detectar o comecinho de uma mania
ou depressão, lembre"-se de que, se você não cuidar a tempo, ela pode
resultar numa internação. Corra para o médico. Ele pode conseguir cortar
a crise.

Caminhadas ao sol da manhã ajudam a manter a \versal{ESTABILIDADE}, mas eu nunca
consegui fazer isso. Existe pesquisa científica sobre o assunto. Não se
esqueça disso. Se esforce. Encare uma dieta suportável para ficar menos
obesa e melhorar a autoestima. Se expresse, pinte, desenhe, borde ou
faça o que quer, desde que esteja se expressando prazerosamente.
Cerâmica, cerâmica, cerâmica. O~contato com o barro, símbolo da ``mãe
terra'', é muito benéfico. Tente, mesmo que seja difícil, construir um
grupo de amigos e parentes para apoiar você na doença. Ajuda muito. Faça
uma rede de amigos ou parentes que de vez em quando telefone pra você e
pergunte se você está bem, se está tomando os remédios e aguentando os
efeitos colaterais. Leia. Se informe. Leia muito sobre a bipolaridade.
Quanto mais informação, maior a possibilidade de tecer parâmetros e
elaborar estratégias para enfrentar a doença. Truques, pequenos truques,
tipo deixar de manhã, no criado"-mudo, os remédios da noite para você não
esquecer de tomar. Com o tempo a gente vai descobrindo milhões de
truques que ajudam muito. Muitíssimo.

Existem na praça alguns livros em português sobre ``bipolaridade'':

\emph{Uma mente inquieta} -- Kay Redfield Jamisom -- Martins Fontes --
1996.

Dos outros só sei os títulos:

\emph{Tristeza Maligna} (depressão)

\emph{O~Diabo do meio"-dia} (depressão)

\emph{Eu não sou uma, sou duas} (bipolaridade)

\emph{À espera do sol} (bipolaridade)

Acho que seria muito bacana se muitos bipolares resolvessem escrever
suas experiências com a doença e publicar. A~gente poderia formar \versal{UMA}
\versal{IMENSA} \versal{REDE} de ajuda. Um ajuda outro, que ajuda outro, que ajuda outro.
Pode ser que alguém conheça um calmante que não interfira com os
remédios. Algum remédio para tomar antes de dormir que nos livre dos
stilnox, dos dormonid, dos lexotam da vida e seus terríveis efeitos
colaterais. Ou um chá para diminuir o apetite e se livrar de alguns
quilos. Quem sabe? Fica aqui a ideia. Tomara que dê certo, eu adoraria
conhecer as aventuras e desventuras de outros bipolares.

Para mim foi muito bom escrever este livro. Se escrever me fez bem,
espero que também faça bem para você. Espero que seja um alimento, uma
força, um incentivo para cuidar da doença. Afinal a pessoa que mais tem
amor pela gente é a gente mesmo. Somos todos narcisos. Não dá pra
disfarçar. Faz parte.

\begin{center}\asterisc{}\end{center}


\begin{flushright}\textbf{\versal{MEMÓRIA} -- 06/02/2012}\end{flushright}


4004--3535. É~fantástico. Após 11 anos consegui decorar o telefone do
banco. Onze anos! Apenas 11 anos! Memória derretida. Lítio, Zyprexa,
bipolaridade. É~foda. Foda, mesmo. Derrete tudo. Derrete a gente.
\versal{LIQUEFAZ}. \versal{LIQUIDIFICADEIRA}. \versal{BATEDEIRA}. \emph{\versal{ORANGE} \versal{JUICER}}. \versal{MERDA}.
\emph{\versal{JUICER}}. Alô, alô, Realengo, aquele abraço!

\begin{center}\asterisc{}\end{center}

\begin{flushright}\textbf{\versal{ATRASO} -- 06/02/2012}\end{flushright}


É tudo assim! É sempre assim… Eterno atraso,
atrasadíssimo… Um bipolar vive no eterno atraso. Finasterida.
Receita de novembro de 2011. Encomenda de fevereiro 2012. Cabelo caindo.
O~meu cabelo ainda vai ficar caindo por \emph{falta do remédio}. Só
falta ser além de bipolar, ficar calva, aí, eu me suicido…

Ser bipolar é um porre mesmo. Atrapalha toda a vida. Em todas as suas
singularidades.

\begin{center}\asterisc{}\end{center}

\begin{flushright}\textbf{\versal{DORES} -- 06/02/2012}\end{flushright}


Terreiro, terreiro, terreiro. Eu preciso de um terreiro para exorcizar.
Para curar dores. Dores quase insuportáveis nas pernas. Eu preciso de um
\versal{EBÓ} e de um \versal{BURI}, pipoca, pétalas de rosas brancas e alecrim, arruda e
guiné. Preciso de tudo isso. Além da bipolaridade, agora dores no corpo.
Nas pernas, na cervical.

\begin{center}\asterisc{}\end{center}

\begin{flushright}\textbf{06/02/2012}\end{flushright}


A minha saúde agora está ótima! Coragem! Viva!!! Estou feliz, produtiva,
apesar das dores terríveis nas pernas… Eu mesma tô achando que é
melhor passar de 5mg de Zyprexa para 7,5mg para evitar uma mania, como
aprendi com o sábio Del Porto.

Não adianta, eu esqueço tudo mesmo. Hoje cedo, a Sara falou que o
entrevistado de hoje do Roda Viva é um cara muito bacana. Eu não consigo
lembrar de jeito nenhum. Ótimo, vai ser surpresa total. Agora lembrei, é
o Laerte, aquele humorista fantástico. O~Laerte que ultimamente tem se
vestido de mulher. Adoro o Roda Viva. Tem alguns entrevistados
fantásticos. Teve um dia que, quando eu liguei para ver, o entrevistado
era o Marcos Jank, filho da Tuxa, meu sobrinho! Ele é o bambambã do
etanol. É~presidente da instituição especializada em etanol mais
importante do Brasil. E é uma simpatia. Morro de orgulho dele. Foi quem
viu o acidente mortal da Tuxa, só percebeu que a vítima era a Tuxa,
quando foi socorrê"-la lá embaixo no barranco. Tinha apenas 21 anos! Na
madrugada do velório, ele escreveu um texto \emph{lindo} sobre a Tuxa.
Nós fizemos uma homenagem para ela. Colocamos num papel muito bonito o
retrato da Tuxa, numa paisagem linda, com céu azul de um lado e o texto
do Marcos do outro, numa tinta discreta, suave, um azul"-acinzentado.
Enviamos para toda a família e para os amigos da Tuxa. Foi assim que a
gente conseguiu ao menos homenageá"-la. Mas… o que a gente queria,
mesmo, era ter a presença dela entre nós até hoje. Aquela alegria,
aquela energia, aquela força. Ah, como seria bom! Que falta eterna eu
sinto dessa irmã querida, muito querida.

\begin{center}\asterisc{}\end{center}

\begin{flushright}\textbf{08/02/2012}\end{flushright}


Eu vou passar o Carnaval em Recife de qualquer jeito. Mesmo que eu vá
sozinha. Já convidei a Vera, há três dias, mas ela não responde nunca.
Ontem convidei a Lia, e ela ficou de me ligar hoje de manhã, mas até
agora, 16h30, nada.

Fui visitar a exposição da Mira Schendel:

Têmpera sobre juta, aglomerado, óleo sobre juta.

Só podia ser judia! Tanto talento e competência!!!

Têmpera sobre tela, aglomerado.

Têmpera e madeira, têmpera acrílica, gesso e madeira.

Técnica mista sobre juta, madeira, tela.

\begin{center}\asterisc{}\end{center}

\begin{flushright}\textbf{}\end{flushright}

\begin{flushright}\textbf{14/02/2012}\end{flushright}


Amanhã vou para Recife! Graças a Deus, Recife. Não vejo a hora de chegar
lá. Caboclinhos, maracatus, frevos, tudo ao mesmo tempo. É~o carnaval
mais lindo do Brasil. Galo da Madrugada… Beleza!!!

Estou almoçando no restaurante da loja 62º. A~comida é muito boa, mas é
cara.Tudo, ou quase tudo que é bom, é caro.

O dia hoje foi cheio. Elas, da loja 62º, não quiseram ficar com os meus
colares em consignação. Eu não sei por que, os colares que estavam lá,
dessa vez, eram horríveis. Às vezes são lindos. Já comprei muitos lá.
Depois comprei os patês da Filó. Antes já tinha ido cortar o cabelo,
hidratar, pintar as unhas da mão.

Daí, fui para o Shopping Villa Lobos mandar transferir o celular novo
para pós"-pago. Não consegui. Fiquei uma hora esperando e daí consumi,
comprei:

\begin{itemize}
\item
  Vestido lindo para levar para Sofia, em abril;
\item
  Três jogos de lençol maravilhosos na Casa Almeida. Tenho fixação por
  lençóis. Comprei também duas almofadinhas lindas e muito macias,
  bordadas em estilo indiano. Meu sofá está super confortável e alegre
  agora.
\end{itemize}
Na 62º, eu perdi a cabeça e comprei várias coisas. Fico maluca nessa
loja. Comprei lá uma bandeja linda, cor de rosa, para dar para a Tina.
Eu preciso parar de consumir desvairadamente, senão não vou ter grana
para ir para Londres e Paris, em abril.

Às 17h30, eu estava no Del Porto, que achou muito legal eu ir para
Recife. Achou que eu podia ir sozinha, visto que a Sara e a Lia não
toparam. É~muito difícil mesmo arranjar companhia. E~olha que ofereci a
viagem de graça para a Sara porque já tinha pago tudo para duas pessoas
e ela é muito dura. Mesmo assim… Nada! Fiquei meio puta, mas
encarei a viagem sozinha. Eu queria muito ir, aproveitar que estou bem.

Passei a noite toda em claro, ora arrumando a mala, ora escrevendo
bilhetes para a Lena. Expliquei tudo que ela teria que fazer pra mim. O~principal era cuidar da Filó, minha gatinha que está doente. No último
dia, ela ficou com o cocô mole e eu fiquei preocupada. Mas ela não
chegou a ter diarreia, que é um perigo para ela neste momento.

Às 6h00 tomei um bom banho e me aprontei. Vim bem bonita. Vestido azulão
bem leve e guias brancas e azuis de Iemanjá. No caminho para o aeroporto
eu dormi, apaguei e só me lembro do Rodrigo me dizer para não ficar
caindo em cima dele.

Voo terrível. Tudo apertado. Humilhação. Cinto de segurança com
extensor!!! Que horror. Barriga enorme. Zyprexa, Zyprexa, Zyprexa. O~preço é altíssimo, mas… estou estável. Isso que importa. Roncos,
todo mundo roncava, menos eu. A~minha bolsa enorme no meu pé…
Incomodava. Bolsa verdona e dourada. De perua. Comprei na liquidação da
Arezzo. Adorei! Agora estou curtindo ficar um pouco perua, nos detalhes.

Finalmente cheguei. Quase chorei quando vi duas personagens do maracatu
e um pessoal dançando frevo, logo no saguão das bagagens. Peguei de cara
um carregador. Eu estava exausta. Na verdade curto muito mordomias. Lado
quatrocentão? Pode ser. Afinal as escravas abanavam os senhores com
leques de palmeira… Não era? Eu adoraria viver isso. Adoraria
mesmo. Meeesmo.

Tinha que fazer hora para chegar na hora de iniciar minha diária na
entrada, no Hotel Aconchego, que é um aconchego. Passeio por Olinda,
glorioso. O~Sobral é o motorista. Gente boa, muito boa. Galinha de
cabidela. Restaurante bem simples. Ele conhece tudo. Fui rezar na Igreja
Nossa Senhora das Neves. Uma boniteza. Exaustão. Exaustão. O~guia
insistia muito para eu continuar o passeio. Fiquei irritada. Gritei com
ele. Pedi desculpas. Dei R\$ 10,00 e fui embora. Antes, aquela vista
maravilhosa do alto do morro em Olinda, onde a gente vê o mar naquela
paisagem descortinada. Sem cortinas. Descortinadíssima. Tudo uma beleza.
Uma boniteza. Eu amo Olinda e Recife. Amo muito. Ingressos para o Galo
da Madrugada, para mim e o Sobral, ele topou ir comigo como
``segurança''! É no sábado. Nem acredito. Ficaremos numa arquibancada
tomando café da manhã e apreciando. Vai ser sábado. Depois de amanhã. É~bem cedo. Às 8h00. Preciso descansar bastante para conseguir ir. Depois,
Banco do Brasil, esqueci de pegar grana para viagem, em São
Paulo… Olha só… Freud explica. Acho que eu queria tudo de
graça.

Eu fiz tudo isso com grande esforço, mas valeu a pena. As pernas
continuam doendo muito. O~Duda disse que é por causa da depressão, da
bipolaridade. Maldita doença que judia de mim. No hotel, dormi das 18h00
às 21h00. Direto. Daí, primeira providência: arrumar os remédios no
criado"-mudo. Eles vieram numa bolsa separada, que trouxe comigo. Se eu
perdesse a mala, não perderia os remédios. \versal{PRUDÊNCIA}, \versal{PRUDÊNCIA},
\versal{PRUDÊNCIA}. A~bipolaridade requer muita:

\begin{itemize}
\item
  \versal{ORGANIZAÇÃO},
\item
  \versal{PRUDÊNCIA},
\item
  \versal{PÉ} \versal{NO} \versal{CHÃO},
\item
  \versal{RECONHECER} \versal{OS} \versal{BENEFÍCIOS} \versal{DOS} \versal{REMÉDIOS}. Mal com eles, pior sem eles.
\end{itemize}
Tomar, tomar e tomar. Deglutir, absorver. Deixá"-los tomar conta do
corpo, do cérebro. Para você viver razoavelmente bem. E~talvez com o
cérebro todo esburacado…

O hotel é minúsculo. Mais parece uma pousada. Tem uma pequena piscina no
centro. Em volta ficam os quartos. Como um claustro moderno. Numa beira
da piscina, tem quatro espreguiçadeiras, no lado oposto, um bar e
mesinhas redondas com quatro cadeiras cada uma. Estou escrevendo numa
delas. Estou me sentindo muito bem e em paz. Maravilha!!! Nem acredito!
Graças a Deus e a meu imenso esforço, porque eu não desisto mesmo!

\begin{flushright}\textbf{}\end{flushright}

\begin{flushright}\textbf{16/02/2012}\end{flushright}


Ir à farmácia, comprar:

\begin{itemize}
\item
  talco,
\item
  desodorante,
\item
  Novalgina,
\item
  Voltarem Retard, 100 mg,
\item
  bronzeador 30 Nivea,
\item
  escova de cabelo,
\item
  \versal{CIGARRO},
\item
  \versal{RED} \versal{BULL}.
\end{itemize}
Esforço, esforço, esforço. As pernas doem muito. O~Duda disse que é por
causa da depressão. Foi a primeira vez que ele fez esse diagnóstico. Eu
achei que estava com ``fibromialgia''. Ele disse que não. Ele estava
meio seco nessa consulta. Depois dela, um dia no \emph{shopping}, tomei
um Voltarem, porque não aguentava a dor. Nunca senti tanta dor nas
pernas, nos joelhos, nos braços e até nas pontas dos dedos dos pés e das
mãos. Ela se espalhou. Continuei com o Voltarem. Adorei. Eu nunca tinha
tomado antes. Só ouvia falar que é um remédio fortíssimo. Arrisquei.
Acertei. O~Celso, porém, não achou bacana a minha iniciativa e disse
para eu falar com o Duda. Falei. Ele mandou substituir por Novalgina. O
único problema é que o Voltarem não é bom para quem tem cólon irritável,
como eu. Tive um pouco de diarreia. Mesmo assim tô com vontade de
arriscar. Afinal, estou em Recife e quero aproveitar o máximo possível.
A~Novalgina não funciona tão bem como o Voltarem. Amanhã vou comprar.
Comprei.

Mercado de São José, uma pequena lindeza tropical. Lembra o antigo
Mercado Modelo de Salvador. Como era nos velhos tempos. Frutas,
verduras, peixes, carnes e artesanato, guias de umbanda e
candomblé… Adorei. De cara, ainda na rua, comprei de um ambulante
dois guarda"-chuvinhas minúsculos. Presente para o Rogério e o Sylvio.
Símbolo do frevo. Lá dentro fiquei maluca. Máscaras
\emph{fantásticas}\textbf{} para tampar os olhos. Influência veneziana
no Carnaval? Comprei para dar para a Sara, a Helena, o Rogério e o
Sylvio. Comprei um trabalho fantástico, é uma máscara oval colada numa
tela cujo fundo é preto. Deve ter uns 30cm X 40cm. É~um trabalho
fortíssimo. A~máscara é tridimensional e toda recoberta com lantejoulas
brancas, douradas, pretas e azuis. Tem um pouco de vermelho, são as
cores da bandeira de Recife. Lantejoula, lantejoula, lantejoula.
Lantejoulas coloridas. Coloridíssimas. É~o Recife. Por isso que amo esta
terra. Amo muito, mesmo! Amo demais! Toalhinhas vindas provavelmente do
Ceará, de renda renascença. Outras bordadas a mão, outras a máquina.
Mais presentes: para as secretárias do Duda, para a secretária do Del
Porto. Esqueci, ou melhor, não tive tempo de comprar no Natal, eu estava
deprê no Natal. Aprendi com a Helena e a Margarida a fazer um ``armário
de presentes''. Outro dia fui numa loja bacana, lá na Vila, que chama Ô
de Casa. Comprei um monte de presentes que eram uma lindeza, que são
para levar nos almoços de segunda"-feira da Soninha. Se bem que agora vou
deixar de ir para ir aos Vigilantes, como recomendou o Duda.

Eu não aguento mais esse peso. Essa obesidade. Ir ao banheiro no avião
foi um tormento. Sempre é. Pra andar. Pra levantar. Pra tudo. De manhã,
quando venho tomar café da manhã aqui no hotel, morro de vergonha da
obesidade e da dificuldade para andar, que nesse horário é enorme.
Depois de tomar duas Novalginas, melhora. Depois de um tempo em pé,
melhora.

Voltei do Mercado de São José, dei um mergulho na piscina para refrescar
e dormi um pouco. Tenho sentido necessidade de fazer essas pausas.
Então, faço mesmo. É~a terceira vez que passo o Carnaval aqui, não
preciso ficar fissurada em visitar isso, visitar aquilo, visitar tudo.

À noite, depois de um bom banho, fui comer uma moqueca no Bargaço, um
restaurante chique, caldinho de feijão e uísque de entrada, no
aperitivo. Eu estava com vontade de tomar um bom uiscão. Depois a
moqueca, divina: peixe, pirão, farofa amarelinha de dendê, arroz
branquinho. Trouxe três quentinhas, amanhã já tenho com o que encher a
barriga. Casquinha de siri também. Duas. Fome enorme. Durante o dia, só
besteiras. Cocada preta e cocada branca, divinas. Café expresso bem
tirado. Adorei! Adorei! Adoreisíssimo!!! Comer é bom pra segurar a onda
e em restaurante bom é melhor ainda.

\begin{center}\asterisc{}\end{center}

\begin{flushright}\textbf{18/02/21012}\end{flushright}


\versal{IR}:

8h -- Galo da Madrugada, voltar para o hotel, Polo das Fantasias, Praça
do Arsenal;

16h -- Maracatu Mirim de Baque Virado, Nação Erê;

16h30 -- Desfile de bonecos gigantes;

17h -- Caboclinhos 7 flechas mirim, rua da Guia, Rampa do Arsenal, rua
Bom Jesus, avenida Rio Branco;

19h -- Maracatu Baque Virado Leão Formoso de Olinda;

19h30 -- Maracatu de Baque Virado Leão da Campina.

\versal{IR}:

Instituto Ricardo Brennand, Oficina do Francisco Brennand, Recife
antigo;

Domingo -- meia"-noite -- Tambores Silenciosos, Passeio de catamarã,
Porto de Galinhas, restaurante Biruta, Bairro do Pina, à noite.

Comprar algodão cru, embalar colagem de máscara e despachar.

\begin{flushright}\textbf{}\end{flushright}

\begin{flushright}\textbf{17/02/2012 --- sexta"-feira}\end{flushright}


\versal{IR}\textbf{:}

A partir das 16h00, rua das Moedas.

Concentração do cortejo, \versal{NANÁ} \versal{VASCONCELOS}!

Batuqueiros e cortes das 10 nações de Maracatu de Baque Virado: Aurora
Africana / Encanto do Pina / Leão da Campina / Oxum Mirim / Raízes de
Pai Adão / Estrela Brilhante / Porto Rico / Cambinda Estrela / Gato
Preto / Encanto da Alegria / \versal{MARACATU} B.S\,\versal{CRUZEIRO} \versal{DO} \versal{FORTE}.

\versal{AFOXÉ} \versal{ARA} \versal{ODÉ}.

Voltar para o hotel.

19h30 -- 500 batuqueiros de maracatu, sob a regência de \versal{NANÁ}
\versal{VASCONCELLOS}.

\begin{flushright}\textbf{18/02/2012}\end{flushright}


Voltei, Recife! Foi a saudade que me trouxe pelo laço!… Pelo
braço!… Alceu Valença

… quero sentir a embriaguez do frevo…

Pierrôs,

Colombinas,

Marinheiros,

Marinheiras,

Presos,

Enfermeiras,

Presas,

Duquesas,

Diabinhas,

Diabinhos…

Tem de tudo.

Tudo isso é o Galo da Madrugada!!! Acordei às 7h00 da manhã e cheguei lá
às 10h00 com o Sobral. Andei três quilômetros para chegar na sede do
Galo!!! Andava, a perna doía. Parava. Sentava um pouco quando doía.
Andava e parava. Sobral com a maior paciência, com medo que eu perdesse
a saída do Galo. Mas deu tudo certo. Chegamos ao camarote, que na
verdade era uma laje coberta, a tempo.

Deslumbramento! Deslumbramento! Deslumbramento! É o único termo que
define o carnaval de Recife.

… Ei, pessoal!

Ei, moçada!

Carnaval começa no Galo da Madrugada…

Eu tenho verdadeira paixão por este carnaval. Comprei mais um monte de
adereços de pena. Vou pendurar tudo lá em casa, na sala, no meu quarto,
onde couber… As fotografias que eu tirei descrevem o Galo da
Madrugada muito melhor que as palavras. Acabei de ver as fotos, ficaram
bem bonitas. Tudo é luz, tudo é cor, tudo é brilho! Muita luz, muita cor
e muito brilho! Um imenso descortinar de luz, cor e brilho. Tudo é a
mais pura alegria. Pessoas felizes. Pessoas cantando muito alegres:

moçada,

velhinhos,

criançada.

Ninguém resiste ao som incrível dos frevos e dos maracatus, das
marchinhas.

Ontem à noite fui à cidade antiga apreciar os maracatus com a mulher do
Sobral, como acompanhante terapêutica… Tô muito descolada,
arranjei até \versal{AT} aqui em Recife… Ela se chama Marlise.

… Ei, pessoal!

Ei, moçada!

Carnaval começa no Galo da Madrugada…

… A manhã já vem surgindo e clareia a cidade…
cristais…

… O Galo também é de briga, as esporas afiadas…

Ontem à noite teve uma hora que vi uma moça dando um tapa. Não resisti e
pedi para ela para eu dar um também. Ela olhou espantada para mim e
depois para o amigo dela. Ele assentiu com a cabeça, e eu dei ``um tapa
na pantera''. Daí, ele abriu uma caixinha cheia de baseados
perfeitinhos! Eu agradeci e disse que não podia fumar. Merda! \versal{MERDA}!
Merda de doença! A maconha atua nos mesmos canais que os remédios, então
é perigoso. Merda de doença que me rouba todos os prazeres. Já não chega
a obesidade. … Não posso nem queimar um fumo. No Galo da
Madrugada ia ser o máximo dar uma bola. \versal{IA}. \versal{IA}. \versal{IA}.

O Rogério ligou. Ele tá super preocupado porque estou sozinha aqui.
Ligou para saber se eu tinha chegado do Galo da Madrugada. Conversamos
bastante. Ele recomendou que eu não vá visitar em Olinda aquele grupo de
mineiros que estava aqui e agora foi para lá. Alugaram uma casa lá, tem
três cozinheiras e 3500 latinhas de cerveja! Eu já tinha pensado e
concluído que são mesmo é um bando de machões mineiros que vêm passar o
carnaval em Olinda para beber, beber, beber e comer todas as gatinhas
possíveis. O~Rogério acha que poderia até pintar alguma violência sexual
pela competição que existe entre os homens. Eu já tinha pensado nessa
hipótese também.

\versal{TÔ} \versal{FORA}. Amanhã eu vou para Nazaré da Mata, às 12h00, para ver o
Encontro dos Maracatus. Tô louca para ir. Sempre fui louca pelos
maracatus.

\begin{center}\asterisc{}\end{center}


\begin{flushright}\textbf{20/02/2012}\end{flushright}


A cerveja do hotel é boa, está sempre bem geladinha. Apesar de sempre
tomar Skol, aqui tenho preferido a Heineken porque é a que está mais
gelada de todas. E é gostosa. Nem forte nem fraca.

As pernas agora não doem, porque estou sentada. Mas, quando eu ando,
elas doem muito. Parece que a parte posterior das minhas coxas e da
barriga da perna estão sendo ``\versal{ALFINETADAS} O \versal{TEMPO} \versal{TODO}''. Dói muito,
muitíssimo. Só uma doida como eu para vir passar o carnaval em Recife
neste miserável estado de saúde. Mas quando perguntei ao Duda se eu
deveria ou não fazer a viagem, ele liberou. Resumo da ópera: muito
Voltarem e muita Novalgina. Para ``segurar a onda''.

Mas tá valendo a pena aguentar tanta dor. A~viagem está sendo
fantástica, absolutamente fantástica. Ficar em Nazaré da Mata,
contemplando e fotografando os maracatus foi maravilhoso. Fiquei tão
comovida com a ``beleza'' e a ``verdade'' dessa expressão de cultura
popular que quase chorei. Quase mesmo. Tive que me segurar. Quando vi,
estava no meio do maracatu, fotografando, não sabia nem como sair. Festa
de luz, de brilhos, de cores. Festa de luz. Imensa luz espiritual,
transbordante. O~cravo preso na boca, o símbolo do ``sagrado''. Aqueles
velhinhos bem velhinhos, segurando com devoção a enorme e sagrada
roupagem coloridíssima, reluzentíssima. Eles transpiravam dignidade,
integridade e \versal{FÉ}. \versal{MUITA} \versal{FÉ}. Coragem. Vida, com V maiúsculo,
\emph{segurando} com coragem e ardor aquela roupagem linda e
pesadíssima.

De turista ali tinha só eu e mais dois ou três, graças a Deus. Um rapaz
da rádio local veio me entrevistar. Adorei! Adoro dar entrevistas. Falei
que foi o Antônio Carlos Nóbrega que me aconselhou vir até Nazaré da
Mata ver os maracatus. Só que isso foi há séculos, há uns 20 anos,
quando vim para Recife na primeira vez. Mas eu nunca esqueci e sempre
sonhei com esse dia, que aconteceu agora. Eu nunca imaginei que a doença
fosse dar esta brecha para eu viver de novo um carnaval maravilhoso em
Recife. Graças a Deus, manifestado no Sobral e na Marlise, meus anjos da
guarda recifenses, eu estou aproveitando muito esta viagem.

Lá em Nazaré vi uma coisa muito antiga acontecer. Pessoas estavam
sentadas na calçada, em frente às suas casas, para esperar o maracatu
passar. Armaram mesinhas cobertas com um toldinho e nelas botaram um
monte de copos cheios de água para os passantes sedentos. Claro que,
sedenta como sou, por causa dos remédios, fui logo avançando na água. É~esse Nordeste generoso e delicado que eu amo. Quem é que não gosta? Eu
adorei ficar uma tarde inteira numa cidade pequena, no interior de
Pernambuco, apreciando o maracatu com seus guizos e os caboclinhos.
Passavam também dois caboclinhos antes dos maracatus. Penas, penas e
mais penas. Arcos e flechas, são os caboclinhos. Deu para fotografar
tudo.

Ontem teve um encontro de maracatu, desde às 6h00 da manhã até às 20h00,
lá em Nazaré da Mata. Eu não quis ir, não sei por que, achei que vê"-los
no dia anterior e também à noite, no centro da cidade, no Marco Zero,
tinha sido emoção demais. Eu não sei. Eu não sei. Eu não sei bem. Eu só
tinha a clareza de que não queria ir. Embora, raciocinando, achasse que
era a melhor coisa a fazer. Afinal, vim de São Paulo sonhando com os
maracatus, seus reis e rainhas, suas cores e luzes. Seus guizos
religiosos. Suas cortes. Mas eu estava necessitando de horizontes
abertos, de sal, sol e mar. Talvez para digerir todas as emoções do
carnaval até então.

Dia 15 --- terça"-feira -- desembarque, Recife

Dia 16 --- quarta"-feira -- Olinda

Dia 17 --- quinta"-feira -- Marco Zero, Mercado São José

Dia 19 -- sábado -- Galo da Madrugada

Dia 20 -- domingo --- Nazaré da Mata

Dia 21 --- segunda"-feira -- Porto de Galinhas

… Voltei, Recife, foi a saudade que me trouxe pelo braço…

… Tomar umas e outras e cair no passo…

\begin{center}\asterisc{}\end{center}


\begin{flushright}\textbf{21/02/12}\end{flushright}


Acabaram de passar pelo bar do hotel dois ``presidiários'' que
provavelmente vão para Olinda, apesar de estar chovendo canivetes.
Chove, estia, chove, estia. Estia, chove…

Ontem tive muita sorte, pois fez sol e fui conhecer Porto de Galinhas,
que eu adorei, e Calhetas, que tem uma vista \emph{deslumbrante}. No
meio do trajeto, passamos pelo Cabo de Santo Agostinho. Estamos bem no
alto do morro. Descortinou"-se uma vista fantástica do penhasco. Marzão
verde, marzão azul. Uma lista colada na outra formando a imensa
paisagem. Imensidão de céu. Imensidão de alma! Toda esta beleza preenche
muito a alma, o coração, a cabeça. Essa viagem de ontem me fez um bem
danado.

Tomei dois banhos de mar. No primeiro, fiquei com um pouco de medo. As
ondas, aparentemente calmas, na verdade eram fortes e me derrubavam.
Fiquei um pouco tensa, mas amei finalmente tomar um banho de mar, já que
no \emph{Réveillon}, na Baleia, estava tão deprê que não tomei nenhum.
Banho de mar para mim é a \versal{REDENÇÃO}. \versal{REDENÇÃOTÍSSIMA}.

Depois do primeiro banho, pegamos a estrada de novo, o Sobral e eu, em
direção a Porto de Galinhas. Paramos para almoçar em um
\emph{self"-service} muito bom. Comprei um vidrão de compota de jaca e
dei para o Sobral. Falei que era pelo tanto que ele tinha cuidado de mim
no Galo da Madrugada. Claro que no final da temporada vou dar uma boa
gratificação para ele. Sem ele, eu não estaria aproveitando a metade do
que estou aproveitando desta viagem.

Porto de Galinhas. Coqueiral. Imenso coqueiral, mar, areias. A~cidadezinha é grande. Paramos na Praça 6. Tomei um banho de mar bem
demorado, daqueles que os dedos da mão começam a ficar murchos. Mar
calmo, sem ondas, só um pouco de correnteza. Saudades enormes daquela
prainha de São Sebastião que era nossa. Só nossa. Para mim é claro que
quando eu saio procurando praia, é a ressonância da Praia do Sonho, ou
do Cabelo Gordo de Dentro, que está batendo em mim.

Banho de mar. Cerveja geladinha. Uma senhora muito bonita e simpática
perguntou o meu nome, de onde eu era. Perguntou se o Sobral era meu
companheiro!… Eu expliquei a presença do Sobral ali, ela
entendeu. Mas ela falou com todas as letras que Porto de Galinhas é a
praia da elite de Recife. Ela não me ofendeu, foi muito simpática. Pra
falar a verdade, acho que continuo até hoje tendo ``cara de elite''.
Cara de??? Hoje não brigo mais com isso, como na adolescência. Hoje
usufruo. Uso a cara de elite, a educação de elite a meu favor. Em
algumas circunstâncias, é um instrumento poderoso. Demorei para me
certificar de que ``ter boas maneiras'' e ser ``bem"-educada'' ajuda
muito na vida. Viva, Dona Sônia Sawaya! A mulher mais ``lady'' que eu já
conheci. Quando estava bem.

Voltando à ``vaca fria''. Porto de Galinhas é uma praia imensa.
Coqueiral imenso, muitas pousadas e um hotel gigante numa das
extremidades. O~centro estava bem cheio e movimentado. Lojinha de
artesanato, de roupas, pequenos restaurantes, a maior muvuca!

Vi de relance algumas roupas com pano africano e quase parei para olhar.
Resisti. Bravamente resisti. Saco!!! Sinal que não estou em mania. Estou
conseguindo me controlar, apesar de ter verdadeira paixão por panos
africanos, pelas estampas maravilhosas que eles têm.

Depois fomos pra Calhetas apreciar a vista para o mar, do alto do
penhasco. O~caminho de terra é péssimo. Muita gente deixa o carro no
meio do caminho e vai a pé. Lá eu comprei doce de caju, de pasta de
caju, de goiaba, em potinhos pequenos. Comprei também uma sacola linda
de palha por R\$35,00 e também um conjuntinho cafona de três borboletas
para pendurar na parede. O~que me atraiu foi o roxo das asas. Imaginei
esse roxo na parede amarela da minha casa. Talvez fique bonito…
Talvez. Se não ficar, dou de presente para alguém.

Em Porto de Galinhas comprei uma caravela maravilhosa, com as velas
todas enfunadas, por R\$50,00! O vendedor queria R\$70,00, mas eu
negociei. Fiquei até com vergonha. O~trabalho é lindíssimo. De Calhetas
voltamos para Recife. Resolvi que de hoje em diante, quando eu tiver
grana, venho passar os feriados prolongados aqui em Recife. Tendo grana
é muito mais fácil do que eu imaginava, agora que reaprendi a viajar,
claro. E~que conto com a presença do Sobral aqui.

Já são 14h45, e eu ainda estou no hotel porque acordei com uma diarreia
pastosa. Achei melhor tomar logo dois envelopes de Questran e dar um
tempo para ver como o intestino se comporta. Não atrapalhou mais. Acho
que agora dá pra sair. Vou tomar mais Imosec, só tenho uma caixa. O~intestino estava tão bom desde o final de ano que eu achei que não
precisava de mais do que isso. Prevenir é melhor que remediar…
Saco! Nunca achei que ia ter tanto problema com intestino na vida.

\begin{center}\asterisc{}\end{center}

\begin{flushright}\textbf{23/02/2012}\end{flushright}


Hoje o dia foi uma delícia. Como tinha planejado na véspera, acordei,
tomei café da manhã e fui pra piscina.

O café do hotel é muito bom, dá uma boa forrada, dá até pra não almoçar.
Toda manhã como bastante melancia e depois uma tapioca deliciosa
recheada com coco. Tomo também um suco de cajá, de graviola ou de outras
frutas nativas, que eu adoro. Hoje, como eu estava faminta, comi também
dois mínis pães franceses com presunto e queijo. Como o café da garrafa
térmica é horrível, eu peço logo um expresso.

Eu tenho a maior vontade de tomar café da manhã de camisola, eu adoro
ficar a manhã toda de camisola. Mas… hotel é hotel. As
formalidades sempre me enchem o saco…

Eu já tomei café de maiô e de saída de praia, porque o dia estava bonito
e eu queria curtir um sol com calma. Eu estava um pouco envergonhada de
ficar de maiô na frente daquela mineirada atacada, mas encarei.

Tinha uma moça que já estava tomando sol, e a gente começou a conversar.
Ela é paulista também, mora no Campo Limpo e se chama Polyana! Ela disse
que é mesmo a Polyana encarnada, que faz eternamente o ``jogo do
contente'', como a do livro… Odeio gente que é poliana, como a
minha mãe que, fora das crises, era absurdamente poliana. Talvez por
defesa. Deve ser por isso que eu desenvolvi um espírito crítico muito
aguçado e até exagerado.

Mas a moça era muito simpática. Ficamos conversando. Ela conhece a Casa
do Zezinho! Mora no Campo Limpo, e a Casa do Zezinho é lá. Chegou uma
amiga dela. Paulistana, bem escurinha, e também legal. Trabalha com
eventos em um banco. Depois chegou o namorado dela, alemão branquíssimo.
Depois uma senhora mineira. Mora em Ouro Preto. Que sorte a dela!

Eu, a Polyana e a amiga estávamos falando sobre o eterno assunto
feminino: empregadas. A~mineira logo se enturmou no papo. Mil
observações. Mil palpites, mil conclusões.

\begin{itemize}
\item
  Toda patroa já foi roubada por alguma empregada. Fiquei consolada. Não
  é só comigo que acontece, não sou tão paranoica assim… Até
  agora não me conformo porque a Cida, minha última empregada, me roubou
  logo o colar indiano com pedrinhas de âmbar que foi da Tuxa. Merda!
  Paguei todas as verbas rescisórias e mandei passear. A~nova empregada
  se chama Lena. Tô apostando nela.
\item
  Se a empregada é boa, invista nela. Trate muito bem. Todos os direitos
  e condução, é óbvio. Dê limites claros. Empregada não é amiga, é
  empregada. E~todas querem ficar sendo ``amigas''. Foi assim que
  ``estraguei'' várias empregadas, a Preta inclusive. Como sou bipolar,
  a minha relação com as empregadas é mais difícil do que para uma
  pessoa normal. É~muito mais difícil… Dificílimo. Na depressão
  montam em mim. Na mania, têm medo de mim. Na estabilidade, ficam
  surpresas e disputam o poder comigo. Não suportam me ver bem, dando
  limites claros. É~foda! Esse capítulo ``empregada'' na minha vida é
  foda! Como diz a Miriam: mal com elas, pior sem elas.
\end{itemize}
Depois de tanto papo, tanto sol, tanto bronzeador, me vesti para sair.
Liguei várias vezes para o Sobral. Não consegui contatar. Deixei três
recados no celular. Nada… Fiquei chateada, mas resolvi pegar
outro táxi. Tinha que ir na Geraldo Araújo Tecidos, deu problema na
compra ontem. Quando ouvi o recado no celular, fiquei gelada. Na
verdade, o erro foi do Joel, vendedor, que me vendeu alguns tecidos por
um valor maior que o verdadeiro. Como fiquei com crédito, é claro que
preferi trocá"-lo por mais tecidos maravilhosos… Credo! Já comprei
ontem quase 200 metros de algodão cru! O algodão de Recife é um dos
melhores do Brasil, se não for o melhor, fiquei super feliz com a
compra. Comprei também uns 50 metros de percal branco, para quando eu
quiser ter um fundo branco na pintura. É~claro que, encantada com os
padrões, as estampas, comprei também tecido floral e três cortes de
tecido xadrez, bem fininho, um mais lindo que o outro. Minha origem
árabe estava cantando dentro de mim! Eu adoraria ter conhecido meu avô
Jorge Sawaya, que foi mascate. Adoraria mesmo! Mas a vida não quis
assim. Quis de outro jeito. É~de outro jeito que eu vivo.

Convidei o Sobral para almoçar comigo no Leite. Eu sabia que era um bom
restaurante, o Felipe, amigo do Sylvio, que indicou. Mas eu não tinha
ideia de que era um restaurante finíssimo, requintadíssimo,
antiquíssimo. Eu não tinha a menor ideia de que ele era tudo isso. Acho
que foi demais para o Sobral, tanto é que hoje ele sumiu completamente.
Eu quis homenagear e, talvez, na melhor das intenções, tenha humilhado.
Sei lá… A conta foi caríssima, é óbvio. O~Sobral fez questão de
conferir… Saco! A água mineral servida lá é Leite e também tem um
vinho português feito especialmente para o restaurante. O~rótulo é
Leite, escrito em letras elegantes e discretas.

Duas, ou melhor, várias saias gigantes plissadas, beges, pendem do teto.
Luminárias grandiosas, origamis lindos! Tudo é lindo, tudo é \emph{art
déco}, anos 20, 30, o desenho das cadeiras inclusive, do bar, das mesas.
O~serviço é impecável. Os guardanapos dobrados sobre os pratos imensos
parecem uns microtransatlânticos. Claro que eu, ``branquinha'', entrar
lá com um senhor negro dos seus 60 anos, grisalho, causou o maior
\emph{frisson}. Mas, como todo mundo lá é ``muito fino'' e muito
educado, não houve nenhuma manifestação. Graças a Deus!

Depois, o Sobral disse que ficou morrendo de vergonha enquanto comia lá.
Ele já é bastante grande, poderia ter dito que não iria e pronto. Mas
não resistiu a um bom arroz de polvo com coco ralado. Envergonhado ou
não, o primeiro comentário do Sobral sobre o restaurante foi:

--- É muito fino!

E é mesmo, um restaurante de altos executivos e banqueiros, pelo que
ouvi. E~eles querem sempre o melhor. Sabem o que é melhor. Sabem muito
bem o que fazer com o nosso dinheiro. O~Leite é também um restaurante
frequentado por políticos importantes. Eles também sabem muito bem o que
fazer com o nosso dinheiro…

Depois, Casa da Cultura. Pena. Que pena!… As pernas doíam muito.
Só fui a três boxes. Comprei uma toalha de mesa de \emph{richelieu},
maravilhosa, duas, para bandeja de café em renascença, e uns sachês
bonitos para dar de presente de renda renascença, em forma de coração.
Uma graça. Mas as pernas doíam, doíam e doíam. Então voltei correndo
para o hotel. Já eram também 17h00. Fiquei aqui, vi a novela das 18h00 e
jantei \emph{carpaccio} de carne. Hoje vou sair sozinha mesmo. Não tem
por onde. Mas já estou meio grogue com estas duas caipirinhas de…
esqueci o nome da fruta… cajá.

\begin{flushright}\textbf{24/02/2012}\end{flushright}


Praias onde fui:

\begin{itemize}
\item
  Candeias, antes do pedágio;
\item
  Praia do Paiva, onde tem o pedágio;
\item
  Itapuama: primeiro banho de mar, ondas fortes;
\item
  Entrada de Praia de Pedra do Xaréu, bom para pescar;
\item
  Porto de Galinhas: praia mansa, segundo banho de mar, arrecifes.
\end{itemize}
Na volta:

\begin{itemize}
\item
  Calhetas, vista linda do alto do morro;
\item
  Gaibu, mansa, arrecifes;
\end{itemize}
São João, dias 23 e 24 de junho, Caruaru, cheio, ou Garanhuns, mais
vazia. Clima mais frio.

Praias:

\begin{itemize}
\item
  Itapuama, Calhetas e Gaibú pertencem ao Cabo de Santo Agostinho;
\item
  Porto de Galinhas é localizada no município de Ipojuca;
\item
  Tamandaré só tem pousadas;
\item
  Praia dos Carneiros, banho de argila;
\item
  Maragogi/Alagoas.
\end{itemize}

\begin{center}\asterisc{}\end{center}

\begin{flushright}\textbf{25/02/2012}\end{flushright}


Estou no avião, voltando para São Paulo, triste de deixar Recife. De
deixar o mar. De deixar a magia dessa cidade. Outra noite, quando jantei
no restaurante Biruta, que é pé na areia e tem um segundo andar com uma
vista maravilhosa para a praia, pensei em morar uns tempos em Recife. Me
faria muito bem. O~mar. O~mar. O~mar. Amar, amar, amar. Ritmo mais
tranquilo. No geral. Calma. Menos violência. Andar de carro com os
vidros abertos. Me enturmar com algum grupo de artistas malucos de lá.
Deve ter. Com certeza tem. Será? É necessário meditar sobre o assunto.
Libertação. Morar em Recife me parece uma libertação. Sobretudo da
família. Sempre a família. Não sei por que… Gaiarsa me socorre!
Me ajuda a entender.

Ontem, dia puxado. Embalar as compras. Sobral maravilhoso. Fez
praticamente tudo sozinho. Não aguentei. As pernas e a coluna doíam.
Muito. Dor, dor e dor. Me maltratando. Preciso enfrentar. Aquaterapia.
Caminhar. Aula de dança. Fisioterapia. É~muita coisa. E~quando vou
pintar? Dentista. Análise. Aula do Jardim. Consulta com Del Porto. É~muita coisa, preciso me organizar bem.

São Paulo. Passaporte, a primeira tarefa. Continuar a tomar sol no
Pinheiros, a segunda tarefa. Voltar para a análise. Tô com saudades. Tô
sentindo falta…

Acertar a vida econômica. Pra variar gastei muito mais do que podia.
Paguei nove quilos de excesso de peso. Mandei sete volumes por
transporte terrestre para São Paulo. Tomara que chegue mesmo. Ficar
atenta aos \emph{e"-mails}. Lena: prestar atenção aos recados e anotar.
Estou cansada. Cansadíssima. Louca para chegar em casa e ver a Filó.
Pegar no colo. Olhar para os olhos dela. Graças a Deus tive a boa ideia
de pedir para o Ricardo ir me buscar no aeroporto. Por que não pedi para
o Sylvio ou o Rogério? Afinal eles são irmãos… Sei lá. Não sei
não. Eu adoraria se um deles tivesse tido a ideia de ir me buscar.
Mas…

\begin{center}\asterisc{}\end{center}


\begin{flushright}\textbf{27/05/2012}\end{flushright}


Cheguei de volta a São Paulo dia 25 de fevereiro. O~Ricardo (\versal{AT}) foi me
buscar no aeroporto como eu tinha pedido. Ele é um santo, super
disponível e generoso no seu trabalho. Fomos comer uma \emph{pizza} na
Oficina de Pizzas perto daqui de casa, na rua Purpurina. Eu estava
ótima, eu pelo menos achava que estava, contei toda a viagem a Recife
para ele.

Dez dias depois, porém, eu deprimi. Cancelei a excursão com a \versal{CVC} para a
Europa. Do jeito que eu estava não dava para ir. Eu não tinha ideia,
como sempre, de quanto tempo aquela deprê ia durar. Um mês, quatro
meses, um ano? Esse é um dos infernos da doença. Um grande inferno. Um
corte na vida, sem data para começar e para acabar. A~gente fica inerte.
Sem energia. Fica tudo parado. Tendo psiquiatra, tomando remédios
ultramodernos, mesmo assim, a gente fica à mercê da doença. Esse demônio
é mais forte que nós. Por mais que a gente tente domá"-lo, a gente não
consegue. Não consegue mesmo. É~um inferno. A~doença é demoníaca. É~cruel. É~terrível. É~insalubre.

Em seguida chegaram os extratos dos meus cartões de crédito e aí eu
deprimi de uma vez total, totalmente. Pelo tanto que eu gastei,
R\$50.000,00, percebi que tinha tido uma hipomania ou uma mania mesmo.
Gastei uma fortuna! Como que eu não percebi que estava em mania,
acordando muito cedo em Recife e gastando, gastando, gastando,
comprando, comprando e comprando. Pelas datas dava para ver que a mania
começou alguns dias antes de eu ir para Recife, aqui em São Paulo, e
terminou uns dias depois de eu voltar. R\$50.000,00, uma fábula! Mesmo
tendo comprado em Recife mais uma mala e um sacola grande para trazer o
que eu tinha comprado, eu não desconfiei de nada! Voltei com nove quilos
de excesso de peso! Nove quilos de toalhinhas de renda, toalhonas
artesanais lindas, roupinha para o bebê da Carol, uma infinidade de
coisas… Fora os adereços de carnaval que eu comprei com o maior
entusiasmo: máscaras, saiotes de pernas, as sombrinhas típicas do frevo,
tudo multifacetado de cores. Eu fico fascinada por cor. Eu amo as cores.
Eu pirei! Eu fiquei desmiolada com tanta cor! Cor, cores e cor.
Azuis"-anil, vermelhos, amarelos. Tudo lindo! Tudo indescritível!

Como pagar, eis a questão. Fiquei arrasada porque não percebi quando a
mania chegou. Culpa, culpa, culpa. Fiquei morrendo de culpa, de
angústia, de tristeza, apavorada. Como pagar? Angústia, angústia,
angústia. Como eu não percebi a mania chegar, eu que me achava uma
craque em evitar manias? Na verdade, ao longo da doença, consegui evitar
muitas, muitas mesmo, percebendo a mania chegar e ligando para o Del
Porto na hora certa. O~mais difícil, sem dúvida, é perceber os primeiros
sintomas. A~gente tem que estar alerta, muito alerta. A~gente tem que
estar atenta. Para mim, começar a acordar cedo e estar às 7h00 tomando
café na cozinha, \emph{é o primeiro sintoma}. Começar a me encantar com
tudo e a gastar muito dinheiro, é o segundo. Existem outros sintomas:
pensamento acelerado, energia em excesso, autoritarismo com todo mundo,
muita irritação etc. Parece que a gente tem um demônio dentro do corpo.
Sem dúvida, a gente tem um demônio dentro do corpo. Eu ouvi dizer, ou
li, ou fantasiei, que os bipolares na Inquisição eram tidos como bruxos
e por isso eram queimados. Será? Será que pareço uma bruxa para algumas
pessoas? Já não chega ser artista plástica e ter uma gata? De certa
forma é ser uma espécie de bruxa. É~alguém que transforma isso naquilo.
Alquimia. Transformação. Bruxa. Ouro. Mistério. ``Os alquimistas estão
chegando! Em lindos discos voadores de cristal forrados de veludo
rosa''. Tenho ouvido muito esse \versal{CD} do Jorge Benjor. É~lindo!

Foi o Zeca que me orientou para eu conseguir resolver as dívidas. Foi um
processo angustiante e desgastante, com vários sustos. Ele colocou à
minha disposição a Maria, secretária dele, que é muito eficiente. Ela
sempre me ajuda nas situações difíceis como essa. O~Ricardo também me
ajudou muito. Ele fez três tabelas com as somas das parcelas dos cartões
de cada mês. Vou ficar ``lisa'' até setembro. Saco! A gente estava tão
aflito para se situar que até fomos ao Santander no domingo à noite para
pegar um extrato do mês! É o fim mesmo! Quantas vezes ainda terei que
passar por esse inferno, eu pensava. A~doença é incurável. Terror.
Terror. Isso é um terror. O único futuro previsível e certo na minha
vida é ter crise e mais crises e, por causa da idade, cada vez mais
frequentes. É~desalentador. Muito desalentador. É~um futuro povoado de
horror. De loucura. Do horror da loucura. Uma barra! Se a gente não
esquece, a gente não vive. Não vive mesmo. Tem que esquecer e tem que
lembrar. É~foda!

O Del Porto me deu de novo a dupla Wellbutrin e Pondera para a depressão
passar e eu me equilibrar. Eu melhorei e fiquei bem mais animada graças
a esses antidepressivos. Sábio Del Porto, na última consulta, na semana
passada, ele pediu para eu diminuir gradativamente o Pondera e também o
Zyprexa, um estabilizador, que eu ainda estava tomando. Graças a Deus
ele tirou, assim eu vou poder emagrecer um pouco, pelo menos. Agora
estou só com 1000mg de Depakote, estabilizador, Wellbutrin 300mg e
Dormonid 15mg. É~até pouco perto de outros tempos. Essa depressão
ocorreu em fevereiro, março e abril. Até que foi rápida, agora eu estou
bem.

Eu estou aproveitando ao máximo o meu bem"-estar. Voltei a desenhar num
cadernão e a fazer aquarelas. Copiei várias vezes uma fotografia linda
que eu tirei há muitos anos, em São Sebastião: pedras, a vista da Ilha
Bela, céu, mar, sol. Tudo que eu amo. É~uma vista que eu contemplava
muito quando me sentava no gramadinho em frente a casa, tomando uma
cervejinha. A~ilha ia mudando de cor: roxo, verde, azul, cinza, preto.
Era um espetáculo magnífico, matissiano sem dúvida. Foi um privilégio
enorme ter acesso a essa paisagem durante tantos anos. Graças ao papai,
que foi quem descobriu e fez lá o Laboratório de Biologia Marinha para a
\versal{USP}. Ele realizou com muito esforço e trabalho o sonho dele. Ele
trabalhou tanto, por paixão. E~por isso eu sempre o admirei muito! Eu
acho o máximo quem tem uma paixão na vida e consegue realizá"-la. Eu vivi
intensamente a minha paixão pela arte e educação. Foi muito realizador.
Agora vivo a paixão pela pintura, que na verdade é muito antiga. Acho
que eu nasci com ela.

Nesta última quarta"-feira, à noite, tive mais uma sessão ``contas'' com
o Ricardo. De novo somar, dividir e dividir mais ainda. De novo
telefonar para os bancos para saber o saldo e os últimos lançamentos.
Conferir com o talão de cheques. Um saco! Eu estava exausta por causa de
gripe forte que tinha pegado há uma semana. Deixei praticamente tudo nas
mãos do Ricardo, que é muito organizado e tem a maior paciência. Eu não
aguento mais fazer contas e ter angústia por causa de dinheiro. Eu já
tive muito isso na vida. Esta fase pós"-mania é bem difícil de viver. Na
mania de 2008, a mais grave que eu tive, foi muito mais fácil resolver
porque eu tinha dinheiro aplicado. Eu mesma resolvi tudo sozinha, paguei
tudo quando saí da clínica. Tendo dinheiro é fácil, é só pagar. Poupo na
depressão e torro na mania, um inferno. Fico sempre sem nada. Eu não
quero mais viver isso.

Eu não me conformo por ter torrado R\$50.000,00! Eu torrei duas
possíveis viagens confortáveis à Europa. A~viagem tão sonhada há tanto
tempo! Há uns vinte anos, passar uma semana com a Tina em Londres e ir
com ela fazer uma viagem, passando por Paris e indo depois para a
Itália, Roma, Veneza, Firenze, a Galeria Uffizi, contemplar a grandiosa
arte da Renascença. Em Londres, Rothko, Turner, Constable, entre outros.
Ver ao vivo a sala do Rothko na Tate Galery, eu tenho paixão pelo
Rothko. Por que será que ele se suicidou? Eu sinto muita falta de ver os
originais ao vivo. É~muito diferente ver a reprodução de uma pintura ao
invés de ver a pintura original. Eu ainda vou conseguir fazer essa
viagem. Vou conseguir juntar dinheiro de novo. Eu preciso me informar.
Eu preciso continuar a me formar. O~artista está sempre se formando.
Todo mundo e cada um estão sempre se formando. O~ser humano está sempre
se formando. Uns dias mais. Outros dias menos. Mas todos os dias um
pouco, pelo menos.

Eu tenho lembrado muito de uma passagem daquele livro, \emph{Uma Mente
Inquieta}. Quando a autora teve uma mania grave, ao ficar diante de
pilhas de contas, o irmão dela entrou na sala com duas taças de
champanhe, ajudou"-a fazer as contas e também emprestou o dinheiro para
ela poder pagar. Fiquei morrendo de inveja desse episódio da vida dela.
Até recentemente, eu me virei sempre sozinha, e foi muito duro para mim.
Por três vezes não consegui cobrir os cheques da mania e por isso fiquei
cinco anos sem cheque e sem cartão. No total, quinze anos. Um inferno!
Uma dureza! Por conta dos gastos exorbitantes das últimas manias, agora
tenho uma dívida com o banco que só irá vencer em 2021. Isso, se eu não
tiver outras manias avassaladoras. Acho que nessas horas \emph{o apoio
da família é fundamental}. Quem gasta muito dinheiro na mania, gasta
porque está doente, não por falta de responsabilidade. Gastar muito,
compulsivamente, eu não sei por que, é um dos ``sintomas'' da doença. As
pessoas precisam tentar entender, por mais absurdo que pareça. Entender
e ajudar. Levar a pessoa para um bom psiquiatra que saberá medicar.
Ajudar a pagar as dívidas, por mais malucas que sejam. A~pessoa não teve
culpa. \emph{Ela é doente. Ela tem uma doença grave. Às vezes,
gravíssima!}

Perdas fazem parte da doença. Perdas. Perdas e perdas. Todo mundo sofre
perdas na vida, mas os bipolares perdem mais. Mais vezes. Uma vez eu
estava numa puta depressão, e a Fanny Abramovich me convidou para
escrever dois livros para a Editora Ática. Um seria sobre a minha
filosofia de trabalho como arte"-educadora, outro, sobre o uso da sucata
nas oficinas de arte e educação. Apesar de afundada na depressão, topei.
Logo em seguida veio uma mania devastadora. Conclusão: não consegui
escrever os livros. Perdi uma grande chance na vida. Só consegui pensar
no título do livro sobre a minha filosofia de trabalho: ``Navegar é
preciso, viver não é preciso''. Eu adoro esse pensamento do Fernando
Pessoa. Teve uma outra vez em que o Luís Camargo me convidou para
escrever um capítulo para o seu livro sobre Arte e Educação. Eu estava
deprimidíssima. Recusei na hora, apesar da insistência dele. Eu já
sabia, já tinha aprendido que não iria conseguir mesmo, estando naquele
estado lamentável. Conclusão: perdi, por causa das depressões, duas
ótimas chances de escrever sobre o meu trabalho. Até hoje não publiquei
nada sobre ele, mas também não corri atrás. Apesar dos anos que
passaram, a memória e a emoção sobre esse trabalho continuam muito vivas
dentro de mim. Foi um trabalho feito com muita paixão. Quem sabe um dia
eu ainda escrevo. Pelo menos tenho montanhas de ``slides'' (que coisa
antiga!), através dos quais registrei o trabalho.

\begin{center}\asterisc{}\end{center}

\begin{flushright}\textbf{18/08/2012}\end{flushright}


Agora só penso na minha exposição de pintura cujo \emph{vernissage} será
no dia 20 de setembro, na galeria Garcia. Fiquei um tempão sem escrever
por que deprimi de novo. Ô inferno considerável! As crises não são só
mais frequentes com a idade, como diz o Del Porto, elas passam a ser
frequentíssimas. Só neste ano tive:

\begin{itemize}
\item
  uma depressão em janeiro, resquício da de dezembro de 2011, o
  \emph{Réveillon} foi terrível;
\item
  uma mania em fevereiro, em São Paulo e Recife;
\item
  uma nova depressão em março, abril e meados de maio;
\item
  um período de estabilidade do fim de maio até o fim de junho;
\item
  uma nova depressão em julho, suicida, desesperadora, descomunal;
\item
  um período de estabilidade em agosto, que dura até agora, graças a
  Deus!
\end{itemize}

Estou torcendo para que dure muito! \emph{Dio mio}, me ajude, me ajude
muito a ficar bem legal para conseguir fazer e curtir a minha exposição
tão sonhada!

O Jardim é o curador da exposição. Na verdade, quando percebi que eu
tinha uma produção significativa do meu trabalho, pedi para alguns
artistas virem aqui avaliá"-la. O~primeiro, claro, foi o Jardim, que
acompanha o meu trabalho desde 1994 e é o meu anjo da guarda. Eu confio
completamente nele porque ele é uma das pessoas mais íntegras e honestas
que conheço. Quando acabou de ver cuidadosamente as pinturas, ele me
olhou e disse, simplesmente:

-- Agora você precisa de uma galeria.

Eu quase caí de costas porque o Jardim é também exigentíssimo com o
trabalho. Eu não imaginava, mesmo, que ele fosse dizer isso. Adorei!
Fiquei muito feliz!

Pedi também uma supervisão para o Fajardo, que eu conheço de longa data,
desde a famosa Escola Brasil. Ele foi muito bacana. Chegou aqui e me
disse: ``Me mostre tudo que você tem para me mostrar''. E~eu mostrei
aquele mar de coisas. Pra falar a verdade, até me esqueci de mostrar um
monte de trapinhos pintados que juntei durante 20 anos. Ele me
aconselhou a entelar meus trabalhos, visto que pinto sobre algodão cru
esticado, muito esticado, sobre uma mesa de 2m x 1m. Também me
aconselhou a pendurar todos os trabalhos numa mesma sala, para ter noção
do conjunto. Achei muito sensato tudo que ele disse, foi daí que nasceu
a ideia de fazer a exposição. Claro que ninguém iria me emprestar um
espaço tão grande para pendurar os trabalhos.

O Fajardo me animou a continuar o trabalho tendo a maior liberdade
possível e usando todos os instrumentos que tenha vontade de usar:
pincéis, pauzinhos, bisnagas de tinta etc. O~que me der na telha. Falou
também sobre o prazer de pintar, de fazer arte. Ele mantématé hoje o
mesmo discurso da Escola Brasil, que eu adoro, ele é uma ``síntese
ambulante'' da Escola Brasil. A~visita dele foi muito gostosa e
estimulante. Daí a um tempinho até pensei em chamá"-lo de novo porque
fiquei com algumas dúvidas, mas achei que seria um exagero. Tenho que
aprender a me virar sozinha. Pôr em prática os conselhos que ele já deu.

Pedi também a opinião da Stela Teixeira de Barros, que é uma ótima e
respeitada crítica de arte. Ela não me pareceu muito entusiasmada com as
pinturas, para falar a verdade. Criticou o uso da bisnaga como
instrumento. A~bisnaga em questão é uma dessas que vêm na mesma caixa
que a tinta para pintar cabelo. Eu esvazio a bisnaga e a encho de tinta,
de modo que ela se torna uma caneta, uma espécie de caneta com o traço
bem regular. É~essa regularidade que a Stela criticou. Mas como pinto
com as bisnagas há 30 anos e desenvolvi a maior técnica, vou continuar
usando. A~Stela gostou muito dos trapinhos pintados, disse que o meu
gesto é bonito e coerente. Disse também que quer acompanhar o meu
trabalho. Isso é o melhor de tudo.

Outra pessoa para quem pedi uma supervisão foi o Patrick Paul. Pedi para
ele fazer uma leitura espiritual do meu trabalho. Ele acha que os
pássaros de cerâmica indicam uma busca espiritual e as pedras, uma
vontade de fincar raízes na terra. Acha que devo tentar juntar essas
duas coisas, isso seria o meu trabalho espiritual, a minha meditação.
Ficou fascinado por uma pintura de 1994, onde consegui juntar essas duas
coisas. Eu o presenteei com ela, e foi divertido dar uma carona para
ele, levando"-o com a pintura até sua casa.

O Megumi e a Naoko já tinham conhecido o meu trabalho em 2011. O~Megumi
disse que o meu trabalho tem imanência material, transcendência,
espiritualidade e verdade. Dessas quatro qualidades, a que me interessa
mais, a que eu busco mais, é a verdade. A~verdade implacável. Anteontem,
eles vieram aqui para ver a minha nova produção e adoraram. Me abraçaram
muito, me elogiaram muito. Fiquei até sem graça. Me agradeceram muito
por ter eu mostrado o meu trabalho. Eles são delicadíssimos. O~Megumi
disse que gostou porque o trabalho é generoso e eu não faço concessões.
Não fico ``estetizando''. Ele acha que sou radical e direta nas minhas
pinturas. Acha que devo entrar em contato com instituições públicas como
Pinacoteca, \versal{MAM} etc. Também me disse para eu ler vários livros de
Bachelard, entre os quais só tenho \emph{A~Psicanálise do Fogo}. Mas
ainda não li. Comprei há séculos porque achava que teria a ver com as
oficinas de arte que eu dava para crianças e adultos. O~Megumi me
desvenda completamente, é impressionante! Fico até meio sem jeito quando
ele fala sobre o meu trabalho e sobre mim.

Pedi também para o Sérgio Fingermann vir ver o meu trabalho, mas não deu
certo, por minha culpa. Eu tinha marcado uma data e um horário com ele,
mas, no meio do caminho, resolvi fazer a tal exposição. Mandei os
trabalhos para o moldureiro. Conclusão: na data marcada eu não tinha
mais trabalhos comigo. Tentei remarcar a data com o Sérgio, mas não deu
certo. Acho que fui indelicada com ele. É~uma pena porque fui aluna dele
durante dois anos e meio, há uns 10 anos, e então ele já conhece o meu
trabalho. Além disso, acho que ele é um ótimo pintor e gravador e o
admiro muito. Seria muito bom para mim se ele visse o meu trabalho. Quem
sabe ele vá à exposição e dê para a gente conversar um pouco.

O Sylvio, meu irmão, foi um crítico tenaz do meu trabalho, desde o
início de 2012, e me incentivou muito. Uma pressão quase insuportável.
Funcionou, me botou pra frente. E~a exposição saiu. No final do ano eu
estava completamente exausta e esgotada de tanto trabalhar.

\begin{center}\asterisc{}\end{center}


\begin{flushright}\textbf{23/09/2012}\end{flushright}


Estou sentindo um estranhamento enorme desde o \emph{vernissage} da
exposição. É~quase uma confusão mental. Um desentendimento. Realizar
esse antigo sonho está mexendo profundamente comigo. Sinto que estou
começando uma vida nova e isso mexe muito comigo. Dá um medo enorme, que
vem junto com uma felicidade enorme por ter conseguido e por querer
conseguir mais.

No \emph{vernissage} ficou muito claro o espanto das pessoas com o meu
trabalho, com a qualidade do meu trabalho. Elas se surpreenderam muito!
A Ana disse que essa foi a melhor exposição que ela viu nos últimos
cinco anos. E~ela é artista! Disse que estava saindo alimentada, que
estava muito diferente depois de ver a exposição, que antes ela era uma
pessoa e depois outra. Era tudo o que eu gostaria de ouvir de alguém. É~exatamente assim que eu me sinto depois de ver uma boa exposição. Esse
comentário dela foi um grande presente.

Fiquei muito feliz com a presença do Megumi e da Naoko. Eles admiram
muito o meu trabalho e me dão a maior força. O~Jardim chegou super cedo
e ficou um tempão. Pediu para ser apresentado ao Sylvio e bateu um papão
com ele. Depois disso, o Sylvio me disse que o ``Jardim me adora''. Mais
tarde, a Dodora veio encontrar com o Jardim, ela estava muito bonita,
iluminada. Esse é um casal iluminado de verdade.

Adorei encontrar os meus amigos lá. Eles ficavam muito contentes e
surpresos vendo o trabalho. Da sobrinhada, foram a Heloísa Helena e o
Guilherme, filhos do Paulinho, e a Bia e o Roberto, o Duda, a Licó, o
Carlinhos com a Pila e os dois filhinhos lindos, filhos da Soninha.
Senti falta dos filhos do Rogério, não foi nenhum, nem o Lica, que é meu
afilhado! O Tiago também não foi. Não foi nenhum filho da Tuxa. E~nenhum
deles deu um telefonema, ou mandou um telegrama, qualquer coisa assim.
Fiquei sentida. Depois do \emph{vernissage}, o Joca e a Drica foram à
exposição, e fiquei feliz ao ler a mensagem deles no caderno de
assinaturas. O~apoio da família é fundamental nessas horas. Mas eles não
têm ideia do quanto é importante para a gente expor, do quanto é um
momento especial e do quanto é difícil chegar lá.

A grande surpresa da noite foi o Dr. Saito comparecer trazendo um vaso
enorme e lindo de orquídeas! Ele é meu dentista da vida inteira. E~trouxe junto toda a família de japonesinhos, fez questão de tirar
fotografia. O~afeto brota de lados que a gente nem imagina! A Regina,
secretária dele, também veio. Fiquei muito contente com a presença
deles. Outra surpresa foi receber flores do pessoal da Special Size
quando cheguei à exposição, e, além disso, as vendedoras da loja vieram.
Elas são umas graças, sempre me atendem muito bem quando eu vou lá,
naquela loja de roupas, maravilhosa, onde tudo cabe em mim. A~Quinha,
amiga de velhos tempos, também compareceu e ficou fascinada pelo
trabalho. Ela não se continha de tanto entusiasmo. Entrar no Facebook
valeu a pena porque resgatei velhos amigos que eu adoro e que foram até
lá: a Quinha, a Roberta, o Samu, a Fátima Golan, a Fu. Por causa da dor
na lombar, que reflete nas pernas em muitos momentos, eu fiquei sentada
lá fora conversando com as pessoas. Essa parte da galeria é muito
simpática, tem uma árvore grandona e parece um pátio. Eu achei também
que era melhor deixar as pessoas à vontade lá dentro, contemplando os
trabalhos. Por incrível que pareça, passei a noite toda sem fazer um
xixi, só fiz no final, antes de ir embora.

No final da exposição, ficou um grupo conversando lá fora: o Pedro e a
Paula, a Fanny Abramovich, a Germana, a Sara, eu, o Antônio e o Pierre,
a Marinês e a amiga dela, fotógrafa. Estava uma delícia. Quase não
acreditei quando vi o Pedro e a Paula chegando, há séculos que eu não
encontrava com eles. Outro casal que adorei encontrar, logo no início da
festa, foi a Bia e o Pedrão. Eles estavam deslumbrados com o trabalho e
disseram várias vezes que era o melhor trabalho que eu já tinha feito.
Senti falta da Lucinha Whitaker, embora soubesse que era aniversário
dela nesse dia. Mas ela jurou que vinha me dar um beijo. A~Preta foi
também e estava linda, fez o maior sucesso com os acompanhantes e a
minha família, todos adoraram ela. Digo para todo mundo que a Preta é a
minha segunda mãe, e ela fica toda feliz. Foi e voltou comigo, e dormiu
aqui em casa. Eu, não é por nada, também estava linda. Fiz maquiagem no
cabeleireiro, e o Eduardo fez um penteado com caracóis, todo levantado
para cima. Pra falar a verdade, fiquei um pouco com cara de madame e não
gostei muito disso. Mas fiquei bonita. Só tive que tirar os cílios antes
de ir porque estavam incomodando muito. Os olhos começaram a lacrimejar,
e a maquiagem dos olhos ia borrar. Fui com uma calça, uma camisetona
preta, decotada, e, por cima, um quimono de seda, aberto, que nunca
tinha usado. No início do \emph{vernissage} fiquei um pouco aflita por
não ver a Soninha e o Fernão, que são sempre super pontuais, mas, lá
pelas 21h30, eles chegaram, e eu sosseguei. Eles são muito importantes
para mim, e eu queria que conhecessem o meu trabalho.

Outra coisa boa que aconteceu é que vendi bem: dois quadros grandes para
a Bia, um para o Zeca e a Mazinha, um para a Soninha e o Fernão. Vendi
também três desenhos para a Bete, um para o Celso e um para o Paulinho
Portella. Ontem o Ricardo disse que tem uma pessoa interessada em
comprar mais um quadro grande, que ele reservou até terça"-feira. Fiquei
surpresa por não ter vendido aquele quadro que parece uma ``ventania
alaranjada'', eu achava que ia ser o primeiro a ser vendido. Se não
vender, é ótimo porque adoro ele e vou pendurar aqui em casa. Vendi
também várias cerâmicas. Vender é bom porque dá para recuperar o
dinheiro investido na exposição e equilibrar de novo as finanças. Quem
sabe sobre até um pouco para eu poupar para a minha sonhada viagem à
Europa, ou a Buenos Aires, que eu não conheço ainda e que é mais barato
e mais perto, vou de \versal{CVC} e pronto.

A presença dos irmãos no \emph{vernissage} foi muito importante para
mim! Foram todos: Paulinho, Rogério, Sylvio, Zeca, Tomi, Soninha e Bete.
O~Paulinho ter ido é um verdadeiro milagre, e o Tomás estar em São Paulo
e não garimpando no Pará, é outro! Foram também a Teresinha, a Sandra
Sawaya, a Maria Alice e a Maria Stela Farah. A~Rosalie Sawaya avisou que
não poderia ir devido a problemas de saúde. Já da família Barros não foi
ninguém! Não é fantástico isso?! Nessas horas o apoio da família é
fundamental, mas acho que as pessoas não sacam isso, não sacam mesmo! A
Tina passou um \emph{e"-mail} desejando sucesso, e eu adorei! Até a Fanny
Abramovich e a Ida Parente foram, e eu adorei também! Elas são amigas
muito leais! E legais! No total foram umas 90 pessoas.

Eu tenho bebido bastante depois da exposição. Bebi no final do dia na
sexta"-feira, no sábado, no domingo e estou tomando um uísque agora, ao
escrever. Na verdade, um Jack Daniel's, Old Time. Não imaginei que fosse
ficar tão perturbada com o sucesso do meu trabalho, mas estou. Na
sexta"-feira e no sábado não fui lá, dei um tempo. Estava exausta e
descansei bastante. Hoje, segunda"-feira, fui no final da tarde e fiquei
super emocionada, de cara, ao ver os meus desenhos através dos vidros,
quando estacionei o carro. Aquela parede de desenhos que o Jardim montou
ficou linda mesmo! Ficou leve e arejada, apesar de os desenhos serem
muito fortes. Nas duas paredes ao lado, o Jardim colocou as duas telas
com um desenhão quase agressivo de árvores em amarelo e preto. Uma delas
tem um fundo amarelo e um texto do Leminski escrito em vermelho. A~outra, que eu fiz sobre um lençol branco, ficou com o fundo branco. Foi
essa que a Bia comprou. Na segunda sala estão os trabalhos mais alegres
e luminosos, ``radiosos'', como diz o Sylvio. Hoje, ao ver a exposição
de novo, tive certeza de que consegui colocar o meu espiritual na arte e
fiquei muito emocionada. Também me emocionei muito, quase chorei, ao ler
o que as pessoas escreveram no livro de presença. Eu queria ficar
sozinha na galeria, mas tive que ser educada e ficar conversando com o
Ricardo. Em parte foi bom porque ele desabafou e reclamou do dia em que
gritei com ele porque os convites já tinham 10 dias de atraso e não
estavam prontos! Se eu não assumisse os convites sozinha, como assumi,
eu ia simplesmente morrer na praia! Ainda bem que deu tudo certo, se bem
que tive que gastar R\$3.000,00 com Sedex e R\$2.000,00 com a impressão
do convite numa outra gráfica, que a Helena recomendou.

Texto do catálogo da exposição: ``\versal{PAISAGENS} \versal{INTERIORES}''

``As criações de Regina Sawaya brotam de uma autêntica manifestação
interior. As paisagens que gera, seja no desenho, na pintura ou na
cerâmica, são retratos de uma alma. Isso não significa que sejam
harmoniosas. As dissonâncias presentes trazem consigo as vivências de um
ser em ebulição.

Uma dessas manifestações está no desenho. Marcado pela liberdade e pela
forma audaciosa de conduzir a linha, carrega em cada imagem o desejo de
buscar caminhos próprios, ou melhor, expressões únicas, para dar vazão a
uma voz interior irrequieta e sempre inconformada. Na pintura, esse
raciocínio se revela por meio do uso da cor. As pinceladas geralmente
largas e a sobreposição de camadas de desenhos e de construções de
fundos e de bases estabelecem um fazer pleno de sensibilidade. As
imagens sugeridas são bem menos importantes do que o processo que se
consolida no mágico ato de fazer. Esse mesmo mistério se faz presente
nas cerâmicas. Um jardim de pedras esmaltadas com cores quentes e
paisagens sutis instauram um entendimento da técnica como reino de
alternativas, em que é necessário experimentar sempre -- máxima que
caracteriza a obra de Regina Sawaya, como um todo''.

\medskip{} 
\begin{flushright}Oscar D'Ambrosio\end{flushright}

\begin{flushright}Doutorando em Educação, Arte e História da Cultura na Universidade
Mackenzie, é mestre em Artes Visuais pelo Instituto de Artes da Unesp.
Integra a Associação Internacional de Críticos de Arte (\versal{AICA} -- Seção
Brasil).\end{flushright}

\begin{center}\asterisc{}\end{center}


\begin{flushright}\textbf{05/12/2012}\end{flushright}


Só conhece o prazer de sair de uma depressão um deprimido. Estou
começando a sair da deprê. Eu percebi isso no dia em que resolvi colocar
uma rodela de tomate, uma folha de alface e uma colherzinha de maionese
\emph{light} no meu insosso sanduíche de pão integral e queijo branco.
Foi um pequeno prazer. Eu não sentia nenhum prazer há três meses. Adorei
também ir dirigindo sozinha para a análise, ontem, em vez de ir com o
Hailton (\versal{AT}) de táxi. Adorei ver o catálogo da exposição do Baselitz.
Uma maravilha! Depois comecei a ver o livrão do Monet para estudar as
pinceladas dele. Há muito tempo não curtia ver um livro de arte. E~eu
tenho um monte deles.

A exposição acabou no dia 2 de outubro, e no dia 15 eu deprimi. Me
lembro da data porque foi o dia em que tive que mandar sacrificar a
Filó. Foi uma das decisões mais difíceis da minha vida, mas ela estava
muito magra e fraquinha, devido a um tumor no intestino que não era
possível operar. Também já estava velhinha, tinha 14 anos, não conseguia
mais nem pular no sofá. Tentava e não conseguia. Eu fiquei arrasada com
a morte dela. Foi e é uma perda irreparável. Estou de luto até agora.
Lembro dela com muita saudade. Aquela coisa linda, adorável e mansa que
ela era! Meiga, amorosa, companheira. Me faz muita falta. Uma enorme
falta. Vinte dias antes, eu tinha levado a Filó ao veterinário. Ele já
queria sacrificar, mas como ele disse que ela ainda teria algumas
semanas ou até alguns meses de vida, apostei nisso. Não queria me
separar dela de jeito nenhum. Sempre disse para a Preta, minha
empregada, que eu queria morrer antes da Filó. Mas não foi assim que
aconteceu. É~uma pena que esses animaizinhos tão queridos vivam tão
menos que a gente. A~Filó viveu 14 anos, foram 14 anos de amorosidade e
ternura dela para mim. Ela foi uma grande companheira no meu cotidiano.
Sinto muita gratidão por ela. Até hoje morro de saudade dela, morro de
falta. Uma falta que é enorme.

A Filó morreu, e eu mergulhei numa profunda depressão. Na primeira
semana nem consegui sair de casa. O~Hailton foi um santo e foi ao banco
para mim, depositar e retirar dinheiro, e ao supermercado comprar
cigarros, congelados etc. Eu estava arrasada, inerte, sem vida, de luto
total. No começo da depressão senti muita tristeza, angústia e uma
fraqueza enorme. O~Del Porto receitou Wellbutrin 300mg e Pondera 20mg.
Com o tempo comecei a melhorar, mas a depressão não passava. Na segunda
consulta, quando eu já estava tomando 30mg de Pondera, o Del Porto quis
aumentar para 40mg, e eu sugeri que ele me desse o Pristiq, pois, uma
vez, tomei só 50mg desse remédio e saí da depressão em onze dias. Foi um
desastre. Desta vez, o Pristiq não funcionou bem. Aos poucos, o Del
Porto teve que subir para 100mg e depois para 200mg, a dose máxima. Eu
sentia muita tontura devido ao Pristiq. Tinha que ficar deitada a maior
parte do tempo. A~fraqueza enorme continuava e continuava. Sentia
efeitos colaterais na cabeça também. Parecia que tinham uns elásticos
dentro da minha cabeça e eles eram repuxados. Esse efeito colateral é
terrível. Eu sofri bastante com tudo isso. Tomar banho era uma tarefa
muito difícil e demorada. Eu sentava e ficava esperando alguns minutos
até ter coragem de entrar na ducha. Tinha medo de sentir fraqueza no
meio do banho. Antes de me enxugar, ficava sentada alguns minutos
também. No dia de lavar a cabeça, o banho era mais difícil ainda. Eu
ficava muito cansada no meio do banho. Deixava a porta destrancada para
poder chamar a Preta para me ajudar, caso fosse necessário. Mas nunca
foi. No café da manhã, às vezes, eu me sentia tão mal, tão frágil, que
pedia para a Preta me dar a mão. Ela ficava de pé ao meu lado, e eu
recostava a cabeça no ombro dela. Ela agradava o meu cabelo e me chamava
de meninona. Nesses momentos difíceis, a Preta me ajuda muito,
muitíssimo. Devagar, muito devagar, os efeitos colaterais foram passando
e eu fui melhorando. A~fraqueza, no entanto, não passava. Eu não
consegui ir à festa de aniversário da Bete por pura falta de energia.
Também não consegui ir naquele jantarzão de Natal que a Soninha oferece
todo final de ano para as famílias Barros e Sawaya porque estava com a
energia a zero. Graças a Deus, no dia vinte e quatro eu estava um pouco
melhor, e deu para ir passar o Natal com a Soninha, o Fernão e a
sobrinhada. No dia 31 de dezembro, à tarde, de repente, senti o meu
corpo com muita energia e parecia que eu ia ter um daqueles surtos
místicos de mania, como já tive algumas vezes antes. Consegui me
controlar, procurei ficar calma, sentei no sofá e pensei e pensei.
Concluí que a única coisa a fazer era ligar para o Del Porto, só ele
poderia me ajudar. Por sorte, ele atendeu de cara o celular. Me mandou
tomar 2,5mg de Zyprexa na mesma hora e diminuiu o Pristiq de 200mg para
100mg.

Nos dias seguintes comecei a passar muito mal durante o café da manhã.
Acordava me sentindo bem, mas, conforme ia me alimentando, a energia ia
diminuindo, diminuindo. Eu acordava ao meio"-dia nessa época,
aproveitando que estava em férias da análise. Sempre fui dorminhoca. No
final do café da manhã me sentia completamente exausta e suava muito no
rosto. A única coisa que conseguia fazer era me deitar de novo. Ficava
na cama até às 18h00, imóvel. Só descansava, não voltava a dormir. O~curioso é que a partir das 18h00 eu melhorava, até que, às 21h00, estava
me sentindo ótima. Fiquei vários dias assim, me senti completamente
perdida. Não sabia a quem recorrer. Finalmente resolvi ligar para o
santo do Del Porto. Ele me disse que deveria ser algum problema de
glicemia no sangue. Eu contei que tinha hora marcada para ir no Dr.
Carlos, gastro, no dia seguinte. Então ele insistiu para que eu fosse,
disse que o Dr. Carlos, além de gastro, é um ótimo clínico geral.

O Dr. Carlos mandou eu medir a glicemia duas horas após o café da manhã
e o almoço durante quatro dias e, também, a pressão, sentada e deitada,
uma vez por dia. Quando ele tirou a minha pressão no consultório, deu
baixa, 10 por 6. O~Hailton foi um santo e foi comprar os aparelhinhos
para eu fazer essa mensuração, visto que eu não conseguiria por só me
sentir bem após às 18h00. Mandei os relatórios para o Dr. Carlos, que me
ligou e disse que a glicemia no sangue estava muito alta, me mandou
tomar um remédio chamado Glifage (metformina). Comecei a tomar, mas o
mal"-estar após o café da manhã não passava de jeito nenhum. Eu não me
sentia mais deprimida. Depois das 18h00, quando passava a me sentir bem,
tinha vontade de desenhar, pintar e escrever. Liguei para o Del Porto
contando isso e ele diminuiu o Pristiq de 200mg para 150mg e mandou
ligar para ele uma semana depois.

A depressão passou, mas a energia não voltou. Na última consulta, o Del
Porto me deu uma dura e falou que metade da minha falta de energia é
devido ao sobrepeso. Ele fez os cálculos, como sempre faz, e me mandou
perder 30kg. Dessa vez ele foi bastante severo comigo. Cheguei a pesar
123kg, peso que nunca tive na vida. Já tinha conseguido emagrecer um
pouco e estava então com 118kg. Foi boa a dura dele porque adquiri a
consciência que carrego sempre seis pacotes de açúcar de 5kg. É~mesmo um
grande absurdo. Eu tinha conseguido, no final de agosto do ano passado,
ir a uma reunião dos Vigilantes do Peso e, por isso, tenho comigo aquele
caderninho com os pontos dos alimentos. Só que eu esqueci quantos pontos
posso fazer. Só lembro que é em torno de trinta. Não consegui voltar aos
Vigilantes do Peso até agora por pura falta de energia, mas pretendo
voltar logo. Já fiz essa dieta umas quatro vezes e sempre emagreci. Teve
uma vez que consegui eliminar 18 quilos, mas levei dois anos para obter
esse resultado. No dia em que parei para ler direito o caderninho de
pontos, tive uma crise aguda de mau humor e irritação. Percebi que teria
que comer muito menos do que imaginava. Fiquei completamente frustrada e
puta da vida, me consolei um pouco ao verificar que pelo menos uma
colher de sopa de açúcar é só um ponto. Pelo menos dá para comer de vez
em quando morangos com açúcar, e eles agora estão maravilhosos. Dá
também para a Preta fazer aquela vitamina de leite com morangos que eu
adoro tomar no café da manhã. Regime é regime. A~gente tem que ter muita
disciplina mesmo. Uns dias depois já fiquei mais conformada com a tabela
de pontos e passei a consultá"-la sempre. Ontem, porém, tive uma consulta
com a minha clínica geral e, ao me pesar, fiquei super decepcionada.
Constatei que, apesar de meus esforços no ultimo mês, só consegui manter
o peso. Nada além disso. Afinal, eu me controlei e parei de comer
castanha de caju e amendoim enquanto via televisão. Substituí essas
iguarias por um copo de iogurte que não tem a menor graça. O~perigo na
dieta é se desesperar com a decepção e voltar a comer feito uma louca.
Por enquanto estou conseguindo continuar a me controlar.

A Dra. Mônica, que é minha clínica geral, também quer que eu perca 30kg.
Como estou um pouco diabética, devido ao sobrepeso, ela receitou uma
injeção que tenho que aplicar na barriga e que terei que tomar todos os
dias, essa injeção tem como efeito colateral a diminuição do apetite. Eu
fiquei um pouco assustada, mas topei. Acho que agora topo qualquer
parada para emagrecer, estou me sentindo muito desconfortável devido ao
sobrepeso. Andar é dificílimo, assim como levantar da cama, do sofá,
entrar e sair do carro. É~um verdadeiro inferno. Fora comprar roupas
legais, o que é um verdadeiro desafio. Agora encontrei uma loja em Moema
onde pelo menos eu gosto de algumas roupas. Tenho comprado sempre lá, e
é bom porque dá para pagar em três parcelas. Só que às vezes, sem
perceber, compro umas roupas meio bregas que depois fico detestando
usar. Daí me desfaço delas. O~meu armário tem sempre vários tamanhos de
roupa, para quando estou com 100kg, para quando estou com 110kg ou
118kg. É~um esforço grande me manter razoavelmente arrumada, decente,
``correta'', como dizia a mamãe.

Logo depois que a Filó morreu, fiquei com diarreia. O~intestino é o meu
órgão sintoma. A~diarreia não passou totalmente até agora. Já são três
meses tentando curá"-la. Eu comentei isso com a Coca, minha amiga, e ela
sugeriu que eu fizesse acupuntura com o acupunturista dela, o Rafael,
que também trata com uma dieta chinesa. Perguntei ao Dr. Carlos o que
ele pensava dessa ideia, e ele achou que seria bom tentar. ``Talvez você
consiga tomar menos remédios'', ele disse. Achei a postura dele muito
aberta e passei a gostar mais ainda dele como médico. O~Rafael trata a
gente em casa, e eu já fiz três sessões de acupuntura. No final da
primeira sessão, ele disse para eu dar uma cochilada porque faria bem
para mim. Dormi profundamente por uma hora. Quando acordei, senti o meu
corpo todo vibrando de energia e o sangue pulsando nas veias. Fiquei
surpresa e até um pouco assustada, pra falar a verdade. Fiquei mais um
tempinho quieta na cama. O~curioso é que depois disso não tive nem
condições de tomar banho. A~energia estava a zero. Achei muito estranho
e desisti. Depois do almoço eu já estava me sentindo bem, faltei à
análise, mas aproveitei para fazer um monte de coisas que tinha que
fazer. O~dia rendeu. Adoro quando o meu dia rende. Eu disse para o
Rafael que o meu principal problema é a falta de energia que sempre me
persegue. De repente, ``cai o fio da tomada''. Contei para ele que todos
os meus irmãos têm muita energia, assim como papai e mamãe também
tinham. Contei também que, para conseguir montar a minha exposição, tive
que tomar muitos energéticos: Gatorade e também aqueles saquinhos com
energéticos que eu levava na bolsa para segurar a barra. Entre agosto e
setembro, eu me entupi de energéticos por causa da exposição. No que
tange à alimentação, até agora ele só mandou acrescentar gengibre. Ele
vai acrescentar outros alimentos na semana que vem. Estou fazendo duas
sessões por semana, mas o objetivo é diminuir para uma por semana,
depois para uma por 2 semanas e enfim parar. Estou com uma esperança
muito grande de que esse tratamento me ajude a ter mais energia para
viver, para desenhar, para pintar, para eu poder fazer tudo o que me
interessa fazer.

\begin{center}\asterisc{}\textbf{}\end{center}

\begin{flushright}\textbf{30/07/2013}\end{flushright}


A Preta é um capítulo à parte na minha vida, e que já dura vinte e seis
anos. Ela é uma mulata clara, simpática, apesar de séria, e anda sempre
muito estilosa. Usa sempre roupas muito caprichadas e muitos
balangandãs. Correntinhas de ouro no pescoço, brincos dourados e anéis,
e a cada dia usa um relógio diferente. Ela tem uma coleção de relógios,
que ela adora. Um badulaque novo é com ela mesma. Quando sai de casa,
ela está sempre impecável. Roupa bonita, bem lavada e bem passada.

A Preta é muito inteligente e centrada. É~uma pessoa forte, que tem
eixo, força de vontade e determinação. Tem também muita energia. Faz
tudo com muita rapidez. É~muito afetiva e gosta muito de mim. Aqui em
casa, ela lava, passa, cozinha e arruma. Vem de segunda à sexta- feira,
sempre com a maior disposição, o maior pique. É~também autoritária e, se
eu não tomar cuidado, ela acaba mandando em mim.

A Preta não é só uma empregada, é uma companheira de vida. Ela me dá
muito apoio. Ela conhece bem as fases da minha doença e sabe lidar muito
bem com elas. Quando começo a ficar em mania, ela me avisa. Ela tem
horror a que eu fique maníaca e comece a gritar com ela, a gastar muito
dinheiro e a ficar muito autoritária. Na depressão, ela me observa
quieta enquanto tomo o café da manhã. Quando estou muito angustiada, ela
fala:

-- Está muito difícil, dona Regina?

Outro dia eu estava com efeitos colaterais tão fortes que pedi para ela
me dar a mão, e ela deu e esperou passar. Foi uma ajuda e tanto. Sei que
posso contar sempre com ela, e isso é o melhor de tudo.

Quando estou bem, a gente conversa um pouco. Falamos dos produtos de
beleza da Avon que a síndica do prédio vende para nós. Falamos de
roupas, de moda, de comida, de vários assuntos.

Sem dúvida é uma bênção ter alguém como a Preta trabalhando comigo há
tanto tempo. Ela me ajuda muito a dar conta da doença, ela me apoia
sempre. Ela é uma verdadeira mãe para mim.

\begin{center}\asterisc{}\end{center}

\begin{flushright}\textbf{01/08/2013}\end{flushright}


Eu gosto muito do meu psiquiatra, que é o Del Porto. Sou paciente dele
há vinte e seis anos, fora cinco de traição, como ele brinca. Eu fui
aluna dele na cadeira de Psicopatologia na faculdade São Marcos. Já dava
para perceber que ele é um cara brilhante e que tem uma vasta cultura, o
que eu prezo muito.

O que eu mais admiro no Del Porto é a sua paixão pela psiquiatria. Ele
realmente faz o que gosta. Vibra com as drogas e os remédios novos. Vai
a muitos congressos, em vários países. Eu, como paciente, claro que
preferiria que ele ficasse no Brasil… Outro dia, na sala de
espera, uma senhora estava comentando o quanto o Del Porto trabalha. Ele
trabalha muito, já chegou a ligar às 21h30 para me orientar. A~Vani,
secretária dele, anota num caderno o nome de todos os pacientes que
ligam durante o dia para pedir que ele ligue à noite. Terminadas as
consultas do dia, o Del Porto começa a ligar. É~puxado.

A consulta presencial com o Del Porto é uma grande conversa. Através
dela, ele percebe qual é meu estado de humor para então poder me
medicar. No início da consulta, conto como tenho passado ultimamente. O~tempo todo ele faz anotações na minha ficha. Depois tira a pressão, ouve
o coração e então faz as receitas dos remédios. Ele é um ótimo ouvinte,
talvez porque, quando jovem, foi terapeuta existencial. Mas também gosta
de falar bastante durante a consulta. Ele é muito generoso e até me deu
o número do seu celular, o que já me ajudou muito muitas vezes.

Apesar de toda sua competência, o Del Porto é uma pessoa simples e
acessível. Não faz pose, apesar de ser considerado por muita gente como
o melhor psiquiatra de São Paulo. Também não faz \emph{marketing}.

Ele furou comigo na mania de 2008, a pior que eu tive. Fui internada
duas vezes. Mas errar é humano. Ninguém é perfeito. Já passou. Por cinco
vezes eu pedi uma segunda opinião e por cinco vezes eu voltei. Na última
vez, o Antônio e o Carlos Moreira me recomendaram que eu voltasse a me
tratar com o Del Porto porque ele é um ótimo psiquiatra e tem toda minha
história psiquiátrica nas mãos, o que ajuda muito, sobretudo na hora de
escolher os remédios.

Para mim, hoje, o Del Porto é um bom amigo que cuida muito bem da minha
saúde mental com uma alquimia de gramas e miligramas de remédios que só
ele sabe fazer. Adoro ele!

\begin{center}\asterisc{}\textbf{}\end{center}

\begin{flushright}\textbf{16/05/2014}\end{flushright}


O Hailton é um descendente de japonês com aparência exótica. Usa um
cabelo grisalho comprido, que vai até a cintura. Prende o cabelo num
rabo de cavalo ou numa trança, usa um minúsculo brinco dourado numa das
orelhas, carrega sempre uma mochila que não larga em canto nenhum. Ele
brinca sempre com os porteiros do prédio e com o \emph{valet} da padaria
Letícia. Quando vou sozinha à padaria, o \emph{valet} pergunta:

--- Onde está o japonês?

O Hailton é meu acompanhante há 12 anos. O~seu único defeito é não saber
dirigir, então andamos de táxi.

Ele tem uma ótima formação. Cursou psicologia na \versal{UNIP}, estudou por dois
anos Neuropsicologia na \versal{UNIFESP} e fez formação em psicanálise no Sedes.
Atualmente clinica e também trabalha como acompanhante terapêutico.

Ele é muito profissional e tem iniciativa. Um dia, quando eu estava
dirigindo o carro, dei duas cochiladas rápidas. Imediatamente ele
escreveu um \emph{e"-mail} para o Del Porto, que me mandou fazer uma
tomografia de cérebro. Eu havia sofrido uma microisquemia na zona branca
do cérebro, coisa que é comum ocorrer na minha idade.

O Hailton é muito generoso e solidário. Em uma época, eu fiquei muito
mal devido à glicemia alta. Eu sentia muitas tonturas e muito enjoo. O~médico mandou medir a glicemia e a pressão por vários dias. O~Hailton
foi comprar os dois aparelhos para mim, uma vez que eu não conseguia ir.
Na semana em que minha gatinha morreu, fiquei muito deprimida e não
consegui sair de casa, então o Hailton foi ao banco para mim e ao
supermercado comprar congelados. Ele é muito disponível, sempre pronto a
ajudar.

O Hailton me acompanha quando estou muito deprimida e não consigo ir
sozinha à analise e ao dentista. Assim que eu fico bem, ele não me
acompanha mais. Eu gosto muito dele, me sinto segura com ele. Ele é
muito discreto e meio misterioso, até hoje pouco sei da vida pessoal
dele.

\begin{center}\asterisc{}\end{center}

\begin{flushright}\textbf{17/05/2014}\end{flushright}


Eu adoro o Ricardo, que é meu acompanhante há nove anos. O~Ricardo é um
cara muito inteligente e muito sensível. Além de tudo tem senso de humor
e é um gato, com profundos olhos azuis.

Ele tem uma ótima formação: cursou Psicologia em Uberlândia, sua cidade
natal, e fez mestrado e pós"-graduação na \versal{USP}. Sua tese de mestrado foi
sobre a arte no metrô em São Paulo, ele se interessa por arte
contemporânea. É~uma ótima companhia para visitar exposições de arte.
Trabalha clinicando e atuando como acompanhante terapêutico.

O Ricardo me acompanha aos sábados à tarde e uma vez por semana, à
noite. Quando estou muito deprimida, eu nem consigo sair no sábado para
ir a um cinema com ele. Nós ficamos então aqui em casa conversando e
conversando na cozinha. Eu me sinto amparada por ele e aos poucos vou me
sentindo melhor. Quando chega a hora de ele ir embora, nunca quero que
ele vá. Os ``sábados em casa'' atravessam o tempo das depressões, neste
ano já faz três meses que não consigo sair de casa, mas agora a
depressão já passou e vou conseguir. ``Conseguir'' é para mim o verbo da
depressão. No final do dia, eu me digo: ``Hoje consegui ir à analise, ou
ir ao Del Porto , ou ir ao dentista, ou ir ao cinema, ou a uma exposição
de arte''.

O Ricardo é muito generoso e quando não consigo sair de casa para
comprar os remédios tarja preta, ele vai para mim. No último sábado, ele
foi comprar cigarros para mim, eu estava desesperada porque os meus
estavam acabando.

Uma vez, quando eu estava muito deprimida, ele chegou aqui em casa com
uma bandeja de docinhos para mim. Eu adorei! Outra vez, ele trouxe um
bolo inglês para comermos juntos com o café, e ainda teve uma vez em que
trouxe uma caixa de sorvete Chicabon. Nós dois adoramos comer e, às
vezes, quando ele me acompanha para ir à análise, depois nós nos
esbaldamos tomando um \emph{milk shake} de chocolate no Fifties. O~Ricardo é um cara muito doce, amoroso e afetivo.

Quando precisei fazer quatro sessões de eletroconvulsoterapia, o Ricardo
me acompanhou dentro da sala de cirurgia. Nas ocasiões em que fui
internada, ele me visitou várias vezes. Ele é muito generoso e delicado.
O~Ricardo é uma pessoa muito importante e querida pra mim.

\begin{center}\asterisc{}\end{center}

\begin{flushright}\textbf{20/05/2014}\end{flushright}


Eu tenho plena consciência do privilégio que tenho por contar com tantas
ajudas: psiquiatra, psicanalista, acompanhantes terapêuticos, a Preta, a
família, os amigos. Dou graças a Deus por contar com todas essas ajudas
que amenizam o meu sofrimento e me ajudam a enfrentar as crises.

Eu me lembro bem do Paulinho entrando na clínica com duas sacolas
cheias, uma com material de higiene pessoal e outra com frutas. Ele
também instalou um computador no meu quarto para eu ter com o que me
distrair e emprestou um minúsculo rádio de pilha. Eu adorei poder ouvir
música! Eu me lembro bem das longas conversas telefônicas com a Tina,
minha irmã que mora em Londres. Ela sempre me ajudando com aquela enorme
força positiva que ela tem. Lembro da Bete entrando na clínica com uma
sacola cheia de lãs e agulhas de tricô para eu ter com o que me
distrair. Lembro das inúmeras conversas que tive com o Tomás sobre a
doença, ele sempre interessado em saber que remédios eu estava tomando.
Lembro muito do Sylvio me visitando nas quintas"-feiras para discutir a
minha pintura comigo. Todas essas lembranças boas me ajudam muito nos
momentos difíceis que atravesso.

Escrevi este livro a conta"-gotas, nas fases em que estive estável nos
últimos quatro anos. Senti prazer em escrever. O~texto fluiu fácil da
minha mente. E~como toda história que começa tem um fim, aqui é o final
desta história.

\begin{center}*\end{center}



