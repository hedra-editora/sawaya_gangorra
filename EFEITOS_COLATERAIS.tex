%!TEX root=./LIVRO.tex

\textbf{EFEITOS COLATERAIS OU O OUTRO LADO DA MOEDA}

\textbf{\emph{09/09/2007}}

A carne podre, o cérebro intumescido. Dormências, tonturas, vertigens. O
mundo não para, não fixa, não tem chão. Vácuo assustador. A boca
desértica, secura absoluta, a língua, monstro paralisado, cola. A boca
sem céu da boca, sem céu... O cabelo ralo, chumaços pelo banheiro, pelo
criado-mudo, pela cozinha... Os olhos não fixam o olhar. Sou pintora...
A memória desbotada, alvejada com cândida, sem vida, inerte. O verbo não
se expressa, emudece, morre. As dores no corpo, dificuldade em andar,
levantar do sofá. Obesidade mórbida, apetite devastador. Sorvete
Häagen-Dazs, doce de leite, dois potes de cada vez. Zyprexa, Zyprexa,
Zyprexa, açúcar, açúcar e açúcar. Tonturas, tropeços, falta de
equilíbrio. É esgotante, completamente esgotante, um esforço gigantesco
viver a vida. O prazer perverso de comer custa caro. Colesterol alto,
pressão alta, diabete. Até agora não lembrei o nome do filme que citei
para a Vera ontem no almoço. O filme brasileiro que ganhou o Leão de
Ouro em Cannes e a que assisti duas vezes...

Corpo podre, vida podre, cérebro podre. Tudo nublado, tudo esquisito,
sem contorno, Zyprexa, Efexor, Carbolitium, Wellbutrin. Muito desejo,
energia flácida, não corresponde, não faço, nunca faço, poderia...
Sempre nos braços de Morfeu, de Morfeu, de Morfeu... Ele consola,
acolhe, cuida, alimenta. Até agora não lembro o filme brasileiro mais
badalado de 2007... Tontura, desespero, caos, vazio. E o raciocínio,
onde fica? Prosac, Lamictal, Lexotan, Donaren... Sem crise, mas sem
vida, torta, aleijada, pela metade, às vezes por menos da metade, às
vezes zero, às vezes menos 20, 50, 100. Depressão enorme, desamparo,
angústia, tristeza, medo insuportável, insegurança. Zyban, Risperdal...
Acompanhantes terapêuticos, ``Anjos \emph{Delivery}''. Possibilidade de
levar a vida em frente. Ir ao analista, ir ao psiquiatra, ao
supermercado, à farmácia. Sozinha não conseguiria. Eles me garantem a
sobrevivência. Eles são maravilhosos, pacientes, atentos, afetivos,
generosos e solidários.

Se benzer, rezar contra os demônios. Alfazema, arruda, guiné, sal
grosso. Mãe de santo, limpeza espiritual na casa. Roupa branca, guias,
pipoca, ritual, cantos, atabaques, dinheiro. Geodon, Trileptal,
Lamictal, \emph{jhorei}, culto aos antepassados, ofertas monetárias,
mais cultos aos antepassados. Templo maravilhoso no solo sagrado de
Guarapiranga, ``Paraíso Terrestre'', projetado pelo Sylvio, meu irmão
arquiteto.

Lá tem aquela casinha que o Sylvio fez pro altar, armazena comidas e
bebidas para os ancestrais. Eu messiânica, por causa do \emph{jhorei},
por causa da doença. Ministro Watanabe ministrou \emph{jhorei} no papai,
na mamãe e em mim, na casa da Soninha, maravilhoso.

O corpo roda, roda, roda como pião, tropeço, caio, me arrebento.
Foda-se! Seroquel, Seroquel, Seroquel. Del Porto disse que é mais leve
que o Zyprexa, só porque ele não toma... Ah, Del Porto! Amado e odiado,
sentimentos em sintonia com os efeitos colaterais, mas respeito e
admiração sempre. Dor, a depressão traz muita dor psíquica, dor
fustigante, dor imensa, dor de verdade. \emph{Tropa de Elite}, esse é o
nome do filme, lembro depois de três dias. Limbo, limbo, limbo,
desesperador.

Zyprexa, Geodon, Lamictal, Efexor.

Inflada, inchada, como um balão. A barriga cresce, o anel aperta, a
fisionomia muda. Tristeza, muita tristeza. Zyprexa, Cloridrato de
Fluoxetina, Sifrol. O intestino prende. Tamarine é fraco, só Humectol D
resolve, em último caso o velho Lactopurga. O intestino solta, pura
água, Lonium, o último grito nessa área. Conheço vários remédios há
vinte e quatro anos. Meu gastro é bom, ele receitou o Lonium. Graças a
Deus posso ter ótimos médicos.

A tartaruga, a tartaruga, agora sou dez vezes mais tartaruga. O cérebro
lento, de repente. Como um doce na doceira, para melhorar, tomo
Coca-Cola, truques de sobrevivência. Começo uma tela, meia hora depois
Red Bull para conseguir continuar. A mente quer, mas o corpo não quer. E
assim, não vai, não.

Power Gel na bolsa para alguma eventualidade energética, sempre. Tenho
de chocolate, morango , baunilha, o gosto é horrível mas... Pintar uma
tela grande, 1,50m x 1,00m? Nem pensar... Rolo de lona intocado no
ateliê, eterna expectativa de melhorar.

Sou eu quem tem que perceber os sinais de crise. É cansativo. É
exaustivo e solitário e solitário. Mas é o único caminho... Se eu não
perceber, se não for obsessiva, eu danço, danço bonito. Sinais de mania,
sinais de depressão? Eterno policiamento. A qualquer alegria mais
intensa a pergunta: Será mania?

Dor, dor e dor. Vida lascada, vida fodida, presença da morte, do não ser
dentro de mim. Aquele vazio que apesar de vazio é lúgubre, escuro,
triste.

A multidão canta alto. Divino Espírito Santo. Demônio. Evangélica,
evangélicos, Edir Macedo. Igreja Universal do Reino de Deus. Esperança
de cura, fui parar lá. Exorcismo forte, eles são bons nisso. Expulsaram
os demônios. O demônio é a doença. Caminho do sol. Muito impressionante
tudo isso. Gostei de ir lá, apesar de me sentir um E.T. Dinheiro,
dinheiro e mais dinheiro. Chantagem espiritual, o dízimo, a católica e a
messiânica também pedem. E muita gente dá. Deve haver um sentido nisso.

Doença incurável é difícil, exige humildade, exige coragem, exige fé.
Lexotan, Zyprexa, Lamictal, Donarem, Rohypnol, Carbolitium. O pastor da
Igreja Evangélica que o Santos frequenta disse que tinha um grande pombo
no meio da gente. Era o Espírito Santo! Eu adoro o Espírito Santo, na
hora do aperto é com ele mesmo. Eu adoro o Espírito Santo, na hora do
aperto é com ele mesmo.

Há um ano e meio tem a Preta, Preta, Pretinha. Pessoa maravilhosa. Cuida
de mim. Roupa bem lavada e bem passada. Até mingau de maisena na cama,
café da manhã. É tudo que eu preciso. Até banho de ervas ela faz. Vasos
de flor. Abobrinha gratinada, salada de frutas. Brincos, pingentes,
penduricalhos dourados, é ela. Impecável. Roupa bem passada. Seis horas
de condução! Cotia/São Paulo/Cotia. Afeto. Tirei cinquenta fotos dela
0,20cm x 0,30cm, fiz um álbum para ela, um \emph{book}. Ela adorou.
Chorou pitangas ao ganhar o presente. Dona Regina, este é o presente
mais bonito que ganhei na minha vida!

E teve a Dagui e o Saulo. Umbanda. Minha procura incansável e luta
cotidiana. Templo Cabocla Guaciara. No Campo Limpo, longe para chuchu.
Dagui e Saulo, pessoas adoráveis, generosas, cheias de energia. Rituais,
músicas e cânticos maravilhosos. Muita criança na gira, os atabaques
batendo, muita alegria. Brasil, meu Brasil brasileiro, eu amo esse país.
Sincretismo. Ter vaso de flor e vela para o anjo da guarda em casa. Ter
a casa muito bem limpa. Fora baganas no cinzeiro, louça lascada, conchas
e caramujos são casas vazias. Fora amuletos de outra religião que não a
sua ... Para São Cosme e Damião duas velas pequenas, doces, guaraná,
balas. Eles adoram. Ogum gosta de cerveja, melancia, ferro. Preto Velho,
de café, cana-de-açúcar, cigarro de palha. Iemanjá, colônia de alfazema,
colar de pérolas, cor cinza, sabonetes, perfumes, tudo que é de mulher.

Tuxa morreu, enlouqueci de pronto. Pela primeira vez tomei
psicotrópicos. Luto intenso, sofrido, vivido do avesso. Mania. Dançar no
Radar Tantan, muito uísque, garrafa no meu quarto, na comunidade.
Namorar, namorar e namorar. Tudo para não ver, não sentir, esquecer.
Para não doer fundo, fundíssimo.

De lá para cá, longo caminho. Infindáveis depressões, graves, menos
graves, médias. Teve aquela quando emagreci 11kg, foi a pior, 1992.
Helena e Modesto, Pinhal, ajuda imensa, generosa. Rogério, incansável
companheiro de dor. Me acudiu muito, sempre com seu coração generoso,
não tinha medo de mim, da minha loucura. Manias bravas tive, 1987, 1996,
1998 e 2008. Em 2001, engraçado, de repente os cinco irmãos homens em
casa, meus heróis! Gostei muito, foi muito importante para mim. Dra.
Cíntia, maravilhosa, não me internou. Del Porto, para variar, viajando
em algum congresso, ele buscar o melhor, o mais moderno. Ele conhece
todas as drogas, droguíssimas, isso me assegura e me acalma, apesar dos
efeitos colaterais, paradoxo. Remédios, único jeito fodido de segurar a
doença. Internação na casa da Soninha e do Fernão. Fiquei lá quase um
mês. Cheguei lá gritando e saí muito bem. Acolhimento muito importante,
obrigada, que paciência, amor e generosidade esses dois têm! Com todos
nós. Internação na Clínica Conviver, em 1998. A paranoia terrível em
relação àquela enfermeira negra. Saí em mania e ainda gastei muito
dinheiro. Agora é o oitavo ano sem mania. Em compensação, as
depressões... Às vezes acho que preferia tomar menos remédio, ter mais
fases de estabilidade, menos sofrimento com os efeitos colaterais, menos
sono, mais energia. Mais fases criativas, produtivas. Del Porto diz que
eu tenho uma `` depressão residual'' e que isso é muito difícil de
tratar. Eu que sei... Cansa. Muitas vezes cansa, exaure, enche. Arrasa.
A gente fica querendo morrer, a depressão engole meu fazer criativo,
então como produzir? O ciclo de sete anos de saúde rendeu apartamento,
Peru, Colômbia, adoro viajar. Fico puta de não conseguir ir sozinha.
Difícil arranjar companhia, muito difícil. Rendeu o Celso Vieira de
Camargo, meu analista maravilhoso, íntegro, honesto, que me dá chão. Ele
me dá equipamento mental. Terminei a análise com o Ernesto, foi difícil,
mas tive coragem. O ciclo de sete anos rendeu as aulas de cerâmica, a
exposição bem-sucedida, um monte de gente querida no \emph{vernissage},
boa venda. Adorei, trabalhei pra chuchu. O que eu mais adoro é
trabalhar. Teve também as aulas de pintura com Sérgio Fingermann, um
ótimo professor. Teve a viagem sozinha para Natal. Coragem, muita
coragem, CVC e, depois, eu por minha conta. Pousada, mar, água de coco,
tapioca. Adorei. Adorei. Adorei. Forte dos Reis Magos. Lindo. Comovente.
Pequeno. Humilde.

É foda, fodíssima viver com essa doença cruel e incurável. Ainda por
cima incurável! Precisa ter garra, precisa ter coragem. Eu tenho coragem
e encaro.
